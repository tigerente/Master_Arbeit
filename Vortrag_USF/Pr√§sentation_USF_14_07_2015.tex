\documentclass[beamer, xcolor=table]{beamer}
%\usepackage{mathpazo}
\usepackage[T1]{fontenc}
\usepackage[utf8]{inputenc}
\usepackage[ngerman]{babel}
\usepackage[babel,german=quotes]{csquotes}
\usepackage{lmodern} % for removing warning about font shape


%\usetheme{Madrid}
\usetheme{Frankfurt}
%\usetheme{Singapore}
\usecolortheme{crane}

\setbeamertemplate{footline}[frame number] % für Foliennummerierung
\beamertemplatenavigationsymbolsempty % Navigationsleiste 

\AtBeginSection{\frame{\sectionpage}}
\newtranslation[to=ngerman]{Section}{Abschnitt}
\newtranslation[to=ngerman]{Subsection}{Beispielmodell}

\usepackage{etoolbox}
\makeatletter
\patchcmd{\slideentry}{\ifnum#2>0}{\ifnum2>0}{}{\@error{unable to patch}}% replace the subsection number test with a test that always returns true
\makeatother

%\setbeameroption{show notes}

\usepackage{amsmath}
\usepackage{calc}

\usepackage{graphicx} %Zum Einbinden von Grafikdateien
\usepackage[font={footnotesize, sf}]{caption}
\usepackage{subfig}
\graphicspath{{./Abbildungen/}}

\usepackage{booktabs} %Für schönere Tabellen
\usepackage{xspace}

\usepackage{tikz} %Für Skizzen

\usepackage[bibstyle=authortitle, citestyle=authoryear, isbn=false, doi=false,
dashed=false]{biblatex}
\addbibresource{Masterarbeit.bib}

%\usepackage{vmargin} %Fürs Seitenlayout
%\setmarginsrb{2.5cm}{1.5cm}{2cm}{1.5cm}{0mm}{0.5cm}{0mm}{1cm}

%%% Eigene Befehle %%%
\newcommand{\was}[1]{\small\textit{#1}}
\newcommand{\noteS}[1]{\todo[color=green!40]{\textbf{Sascha: }#1}}
\newcommand{\noteJ}[1]{\todo[color=blue!40]{\textbf{János: }#1}}
\newcommand{\textfrac}[2]{\hspace{2pt} \frac{\text{#1}}{\text{#2}}}
\newcommand{\eqnref}[1]{\overset{(\ref{#1})}{=}} % Gleichheitszeichen mit Referenz auf die verwendete Gleichung
\newcommand{\defeq}{\vcentcolon=} %Definitions-Gleichheitszeichen
\newcommand{\eqdef}{=\vcentcolon}
\newcommand{\pfrac}[2]{\frac{\partial #1}{\partial #2}}

% Formeln:
\newcommand{\MIPS}[1][]{
  \ifthenelse {\equal {#1} {}}
  {\text{MIPS}} % if argument is blank
  {\text{MIPS}({#1})} % if an optional argument is given
}
\renewcommand{\P}[1]{P_\text{#1}}
\newcommand{\I}[1]{I_\text{#1}}
\newcommand{\itext}[1]{i_\text{#1}}
\newcommand{\T}[1]{T_\text{#1}}
\newcommand{\n}[1]{n_\text{#1}}
\newcommand{\N}[1]{N_\text{#1}}
\renewcommand{\t}[1]{t_\text{#1}}



\title{Ökologische Nachhaltigkeit durch \\ \enquote{Nutzen statt Besitzen}?}
\subtitle{{\small Entwicklung eines Modells zur Ableitung von Kriterien für die Senkung des Umweltverbrauchs durch gemeinschaftliche Produktnutzung}}
\author{Sascha, János}
\date{29.04.2015}

\begin{document}

\frame{\titlepage}


\begin{frame}{Apetitanreger}
    Grafiken, die die beiden Fallstudien darstellen
\end{frame}
\frame<beamer>{\tableofcontents[hidesubsections]}
\section{Thema, Fragestellung, Methodik}

	\begin{frame}{Thema und Fragestellung}
        Thema: 
        \begin{itemize}
            \item Gemeinschaftliche Nutzung von Produkten (Bspw. Waschmaschinen,
                Autos, Kleidung,\dots)
            \item Ökologische Vorteile solcher Nutzungsformen gegenüber
                individueller Nutzung
        \end{itemize}

        \begin{block}{Fragestellung:}
            \begin{itemize}
                \item Unter welchen Umständen kann der Umweltverbrauch eines Produktes durch gemeinschaftliche Nutzung gegenüber der individuellen Nutzung gesenkt werden?
                % \item Was sind die Mechanismen, die den Umweltverbrauch bei gemeinschaftlicher Nutzung bestimmen und wie wirken diese? 
                % \item Welche Eigenschaften des Produkts und der Nutzungsform beeinflussen die Wirkung dieser Mechanismen und wie fließen sie ein?
            \end{itemize}
        \end{block}
	\end{frame}
	
	\begin{frame}{MIPS-Konzept}
	\end{frame}

	\begin{frame}{Untersuchte Effekte}
        % \only<1>{
        %     Was ist eigentlich ein Effekt/Mechanismus?
        % }
		\begin{center}
            \scriptsize
            \only<1>{
			\begin{tabular}{p{5cm}p{5cm}}
				 \toprule
				 \multicolumn{2}{l}{\textbf{Umweltauswirkungen durch die Nutzung}}  \\
				 \textbf{positiv} & \textbf{negativ} \\
				 \midrule
				 Nutzungsintensivierung  &  Zusätzliche Transaktionen  \\
				 Einsatz langlebiger Produkte  &  zusätzlicher Ressourcenverbrauch für Langlebigkeit  \\
				 Verwendung verbrauchsarmer bzw. leistungsstarker Geräte  &  Beschleunigte Ausmusterung  \\
				 Erhöhung der Produktauslastung  & Tauschbedingter Verschleiß  \\
				 Berücksichtigung des technischen Fortschritts  &  \\
				 Wartung / Reparaturen & \\
				 \midrule
				 \multicolumn{2}{l}{\textbf{Umweltauswirkungen durch Nachfrageänderung}}  \\
				 \textbf{positiv} & \textbf{negativ} \\
				 \midrule
				 Nachfrageverringerung durch höhere Kostentransparenz  & Erleichterter Produktzugang  \\
				 Vermeidung von Fehlkäufen  & Wunsch nach Eigentum  \\
				  & Rebound-Effekt  \\
				 \bottomrule
			\end{tabular}
            \vspace{3pt}

			Mechanismen für die Umweltauswirkungen durch
            gemeinschaftliche Nutzung. Quelle: verändert nach
            \cite{scholl_marketing_2009}.}
            \only<2>{
            \small
			\begin{tabular}{p{5cm}p{5cm}}
				 \toprule
				 \multicolumn{2}{l}{\textbf{Umweltauswirkungen durch die
                 Nutzung}}  \\[5pt]
				 \textbf{positiv} & \textbf{negativ} \\
				 \midrule
				 Nutzungsintensivierung  &  Zusätzliche Transaktionen
                 \\[3pt]
                 Erhöhung der Produktauslastung  & \\[3pt]
                 Wartung / Reparaturen & \\[3pt]
				 \bottomrule
			\end{tabular}
            \vspace{3pt}

			Betrachtete Mechanismen für die Umweltauswirkungen durch gemeinschaftliche Nutzung. Quelle: verändert nach \cite{scholl_marketing_2009}.}
        \end{center}
	\end{frame}
	
	\begin{frame}{Methodik}
		\begin{center}
            \begin{itemize}
                \item<1-> Modellierung
                    \begin{itemize}
                        \item Nutzungssystem
                            \begin{itemize}
                                \item Personen 
                                \item Produkte
                                \item Organisation der Nutzung
                            \end{itemize}
                        \item Vergleich verschiedener Nutzungssysteme
                            \begin{itemize}
                                \item Effekte: Veränderung einzelner Parameter
                                    des Systems 
                                \item Wie wirken die einzelnen Parameter auf die
                                    MIPS?
                            \end{itemize}
                    \end{itemize}

                \item<2> Analyse
                    \begin{itemize}
                        \item Unter welchen Umständen verringert sich die MIPS
                            durch gemeinschaftliche Nutzung?
                        \item Grund-Annahme: Parameter ändern sich durch
                            gemeinschaftliche Nutzung in eine bestimmte Richtung
                        \item Analyse der Einzelmodelle: ein Parameter ändert
                        sich 
                    \item Modellkopplung: mehrere Parameter ändern sich	
                    \end{itemize}
            \end{itemize}
		\end{center}
	\end{frame}

\section{Modelle}
	\begin{frame}{Modell-Grundlagen}
		\begin{itemize}
			\pause
			\item Wichtige Größe: Produktnutzungsdauer $t$
			\begin{itemize}
				\item Wie lange ist ein Produkt in der Nutzung?
				\item Erreichen der technischen Lebensdauer oder der Maximalnutzungsdauer:
				$t = \min \left\{t_\text{tech}, t_{\text{max}} \right\}$
				\item Nicht fest vorgegeben, sondern von der Nutzung abhängig.
			\end{itemize}
			\pause
			\item Nicht betrachtet: Nachfrage-Änderungen
			\begin{itemize}
				\item Konstante Service-Nachfrage $S = S_D$
				\item Problem: Nicht kompatibel mit variabler Produktnutzungsdauer.
				\item Lösung: Konstanter Betrachtungszeitraum T
			\end{itemize}
		\end{itemize}
	\end{frame}

	\begin{frame}{Modell-Grundlagen}
		\begin{itemize}
			\item Produkte des Nutzungssystems
				\begin{itemize}
					\item parallel eingesetzte Produkte
					\item sequentiell eingesetzte Produkte
				\end{itemize}
		\end{itemize}
		\begin{figure}[h]
			\includegraphics<1>[height=3cm]{Produktanzahlen_1_0}
			\includegraphics<2>[height=3cm]{Produktanzahlen_1_1}
			\includegraphics<3>[height=3cm]{Produktanzahlen_1_2}
			\includegraphics<4>[height=3cm]{Produktanzahlen_1_3}
			\includegraphics<5->[height=3cm]{Produktanzahlen_2}
		\end{figure}
		\pause
		\begin{itemize}
			\item <6-> $P = p \cdot q = p \cdot \frac{T}{t}$
			\item<7-> Fragen / Anmerkungen?
		\end{itemize}
	\end{frame}

\subsection{Nutzungsintensivierung}
	\frame{\subsectionpage}
	\begin{frame}{Modellbeschreibung}
		\begin{itemize}
			\item Nutzungsintensivierung = Erhöhung der Nutzungshäufigkeit $h$
			\pause
			\item Nutzungsmenge $n$ \emph{eines} Produkts während $t$: \\ $n = h \cdot t$
			\pause
			\item \emph{Gesamt}nutzungsmenge $N$ während $T$: \\ $N = n \cdot P = h \cdot t \cdot P = h \cdot t \cdot p \cdot \frac{T}{t} = h \cdot p \cdot T$
			\pause
			\item Mindestproduktanzahl $p_{\text{min}}$: \\ $p \geq p_{\text{min}} \quad  (p, p_{\text{min}} \in \mathbb{N})$
			\pause
			\item Servicemenge $S$: \\ $S = S_D = N \cdot A \quad \Leftrightarrow \quad N = \frac{S_D}{A}$ \quad ($A$: Produktauslastung)
			\pause
			\item Nutzungshäufigkeit $h$: \\ $h = \frac{N}{p \cdot T} = \frac{S_D}{A \cdot p \cdot T} \qquad h \leq h_\text{max} := \frac{S_D}{A \cdot p_\text{min} \cdot T}$
		\end{itemize}
	\end{frame}

	\begin{frame}{Modellbeschreibung}
		\begin{itemize}
			\item Technische Lebensdauer $t_\text{tech}$: \\ $t_\text{tech} := \frac{n_{\text{max}}}{h}$
			\pause
			\item Produktnutzungsdauer $t$: \\[5pt] $t(h) = \min \left\{t_\text{tech}, t_{\text{max}} \right\} = \left\{\begin{array}{cl}  t_{\text{max}}, & \mbox{falls } h < h^* := \frac{n_{\text{max}}}{t_{\text{max}}} \\ \frac{n_{\text{max}}}{h}, & \mbox{sonst} \end{array}\right.$
			\pause
			\item Nutzungsmenge $n(h)$: \\[5pt] $n (h) = h \cdot t(h) = \left\{\begin{array}{cl}  h \cdot t_{\text{max}}, & \mbox{falls } h < h^* \\ n_{\text{max}}, & \mbox{sonst} \end{array}\right.$
			\pause
			\item Effektive Produktanzahl $P$: \\[5pt] $P (h) = \frac{N}{n(h)} = \left\{\begin{array}{cl}  \frac{S_D}{A \cdot h \cdot t_{\text{max}}}, & \mbox{falls } h < h^* \\[5pt] \frac{S_D}{A \cdot n_{\text{max}}}, & \mbox{sonst} \end{array}\right.$
		\end{itemize}
	\end{frame}

	\begin{frame}{Modellbeschreibung}
		\begin{itemize}
			\item $\text{MIPS}(h) = \frac{I}{S} = \frac{P(h) \cdot i_P + I_{\text{fix}}^h}{S} =
			\frac{I_{\text{fix}}^h}{S_D} + \left\{ \begin{array}{cl}  \frac{i_P}{h \cdot t_{\text{max}} \cdot A}, & \mbox{falls } h < h^* \\[5pt] \frac{i_P}{n_{\text{max}} \cdot A}, & \mbox{sonst} \end{array}\right.$
		\end{itemize}
		\pause
		\begin{center}
			\resizebox{0.7\linewidth}{!}{
				% Created by tikzDevice version 0.8.1 on 2015-04-03 10:57:13
% !TEX encoding = UTF-8 Unicode
\begin{tikzpicture}[x=1pt,y=1pt]
\definecolor{fillColor}{RGB}{255,255,255}
\path[use as bounding box,fill=fillColor,fill opacity=0.00] (0,0) rectangle (419.17,289.08);
\begin{scope}
\path[clip] ( 46.80, 49.20) rectangle (417.97,221.88);
\definecolor{drawColor}{RGB}{190,190,190}

\path[draw=drawColor,line width= 0.4pt,line join=round,line cap=round] (404.22,108.89) --
	(381.63,108.89) --
	(361.83,108.89) --
	(344.33,108.89) --
	(328.75,108.89) --
	(314.79,108.89) --
	(302.21,108.89) --
	(290.82,108.89) --
	(280.45,108.89) --
	(270.98,108.89) --
	(262.29,108.89) --
	(254.29,108.89) --
	(246.90,108.89) --
	(240.05,108.89) --
	(233.69,108.89) --
	(227.76,108.89) --
	(222.23,108.89) --
	(217.05,108.89) --
	(212.19,108.89) --
	(207.62,108.89) --
	(203.33,108.89) --
	(199.27,108.89) --
	(195.44,108.89) --
	(191.82,108.89) --
	(188.38,108.89) --
	(185.12,108.89) --
	(182.02,108.89) --
	(179.08,108.89) --
	(176.27,108.89) --
	(173.59,109.25) --
	(171.03,109.87) --
	(168.59,110.50) --
	(166.25,111.12) --
	(164.01,111.75) --
	(161.87,112.37) --
	(159.81,113.00) --
	(157.83,113.62) --
	(155.93,114.25) --
	(154.10,114.87) --
	(152.35,115.50) --
	(150.65,116.12) --
	(149.02,116.75) --
	(147.45,117.37) --
	(145.93,118.00) --
	(144.46,118.62) --
	(143.04,119.25) --
	(141.67,119.87) --
	(140.35,120.50) --
	(139.07,121.12) --
	(137.82,121.75) --
	(136.62,122.37) --
	(135.45,123.00) --
	(134.32,123.62) --
	(133.23,124.25) --
	(132.16,124.87) --
	(131.13,125.50) --
	(130.12,126.12) --
	(129.14,126.75) --
	(128.19,127.37) --
	(127.27,128.00) --
	(126.37,128.62) --
	(125.50,129.25) --
	(124.64,129.87) --
	(123.81,130.50) --
	(123.00,131.12) --
	(122.22,131.75) --
	(121.45,132.37) --
	(120.70,133.00) --
	(119.97,133.62) --
	(119.25,134.25) --
	(118.55,134.87) --
	(117.87,135.50) --
	(117.21,136.12) --
	(116.56,136.75) --
	(115.92,137.37) --
	(115.30,138.00) --
	(114.70,138.62) --
	(114.10,139.24) --
	(113.52,139.87) --
	(112.95,140.49) --
	(112.40,141.12) --
	(111.85,141.74) --
	(111.32,142.37) --
	(110.80,142.99) --
	(110.29,143.62) --
	(109.78,144.24) --
	(109.29,144.87) --
	(108.81,145.49) --
	(108.34,146.12) --
	(107.88,146.74) --
	(107.42,147.37) --
	(106.98,147.99) --
	(106.54,148.62) --
	(106.11,149.24) --
	(105.69,149.87) --
	(105.28,150.49) --
	(104.87,151.12) --
	(104.47,151.74) --
	(104.08,152.37) --
	(103.70,152.99) --
	(103.32,153.62) --
	(102.95,154.24) --
	(102.58,154.87) --
	(102.22,155.49) --
	(101.87,156.12) --
	(101.52,156.74) --
	(101.18,157.37) --
	(100.85,157.99) --
	(100.52,158.62) --
	(100.19,159.24) --
	( 99.87,159.87) --
	( 99.56,160.49) --
	( 99.25,161.12) --
	( 98.95,161.74) --
	( 98.65,162.37) --
	( 98.35,162.99) --
	( 98.06,163.62) --
	( 97.78,164.24) --
	( 97.49,164.87) --
	( 97.22,165.49) --
	( 96.94,166.12) --
	( 96.68,166.74) --
	( 96.41,167.37) --
	( 96.15,167.99) --
	( 95.89,168.62) --
	( 95.64,169.24) --
	( 95.39,169.87) --
	( 95.14,170.49) --
	( 94.90,171.12) --
	( 94.66,171.74) --
	( 94.42,172.37) --
	( 94.19,172.99) --
	( 93.96,173.62) --
	( 93.73,174.24) --
	( 93.51,174.86) --
	( 93.29,175.49) --
	( 93.07,176.11) --
	( 92.85,176.74) --
	( 92.64,177.36) --
	( 92.43,177.99) --
	( 92.22,178.61) --
	( 92.02,179.24) --
	( 91.82,179.86) --
	( 91.62,180.49) --
	( 91.42,181.11) --
	( 91.23,181.74) --
	( 91.04,182.36) --
	( 90.85,182.99) --
	( 90.66,183.61) --
	( 90.48,184.24) --
	( 90.29,184.86) --
	( 90.11,185.49) --
	( 89.94,186.11) --
	( 89.76,186.74) --
	( 89.59,187.36) --
	( 89.42,187.99) --
	( 89.25,188.61) --
	( 89.08,189.24) --
	( 88.91,189.86) --
	( 88.75,190.49) --
	( 88.59,191.11) --
	( 88.43,191.74) --
	( 88.27,192.36) --
	( 88.11,192.99) --
	( 87.96,193.61) --
	( 87.81,194.24) --
	( 87.65,194.86) --
	( 87.50,195.49) --
	( 87.36,196.11) --
	( 87.21,196.74) --
	( 87.07,197.36) --
	( 86.92,197.99) --
	( 86.78,198.61) --
	( 86.64,199.24) --
	( 86.50,199.86) --
	( 86.36,200.49) --
	( 86.23,201.11) --
	( 86.09,201.74) --
	( 85.96,202.36) --
	( 85.83,202.99) --
	( 85.70,203.61) --
	( 85.57,204.24) --
	( 85.44,204.86) --
	( 85.32,205.49) --
	( 85.19,206.11) --
	( 85.07,206.74) --
	( 84.95,207.36) --
	( 84.82,207.99) --
	( 84.70,208.61) --
	( 84.59,209.24) --
	( 84.47,209.86) --
	( 84.35,210.49) --
	( 84.24,211.11) --
	( 84.12,211.73) --
	( 84.01,212.36) --
	( 83.90,212.98) --
	( 83.79,213.61) --
	( 83.68,214.23) --
	( 83.57,214.86) --
	( 83.46,215.48);
\definecolor{drawColor}{RGB}{0,0,0}
\definecolor{fillColor}{RGB}{255,255,255}

\path[draw=drawColor,line width= 0.4pt,line join=round,line cap=round,fill=fillColor] (402.23,106.90) rectangle (406.21,110.89);

\path[draw=drawColor,line width= 0.4pt,line join=round,line cap=round,fill=fillColor] (230.39,106.90) rectangle (234.38,110.89);

\path[draw=drawColor,line width= 0.4pt,line join=round,line cap=round,fill=fillColor] (173.11,106.90) rectangle (177.10,110.89);

\path[draw=drawColor,line width= 0.4pt,line join=round,line cap=round,fill=fillColor] (144.47,115.78) rectangle (148.46,119.77);

\path[draw=drawColor,line width= 0.4pt,line join=round,line cap=round,fill=fillColor] (127.29,124.66) rectangle (131.28,128.65);

\path[draw=drawColor,line width= 0.4pt,line join=round,line cap=round,fill=fillColor] (115.83,133.55) rectangle (119.82,137.53);

\path[draw=drawColor,line width= 0.4pt,line join=round,line cap=round,fill=fillColor] (107.65,142.43) rectangle (111.64,146.42);

\path[draw=drawColor,line width= 0.4pt,line join=round,line cap=round,fill=fillColor] (101.51,151.31) rectangle (105.50,155.30);

\path[draw=drawColor,line width= 0.4pt,line join=round,line cap=round,fill=fillColor] ( 96.74,160.19) rectangle (100.73,164.18);

\path[draw=drawColor,line width= 0.4pt,line join=round,line cap=round,fill=fillColor] ( 92.92,169.08) rectangle ( 96.91,173.06);

\path[draw=drawColor,line width= 0.4pt,line join=round,line cap=round,fill=fillColor] ( 89.80,177.96) rectangle ( 93.78,181.95);

\path[draw=drawColor,line width= 0.4pt,line join=round,line cap=round,fill=fillColor] ( 87.19,186.84) rectangle ( 91.18,190.83);

\path[draw=drawColor,line width= 0.4pt,line join=round,line cap=round,fill=fillColor] ( 84.99,195.73) rectangle ( 88.98,199.71);

\path[draw=drawColor,line width= 0.4pt,line join=round,line cap=round,fill=fillColor] ( 83.10,204.61) rectangle ( 87.09,208.60);

\path[draw=drawColor,line width= 0.4pt,line join=round,line cap=round,fill=fillColor] ( 81.46,213.49) rectangle ( 85.45,217.48);
\end{scope}
\begin{scope}
\path[clip] (  0.00,  0.00) rectangle (419.17,289.08);
\definecolor{drawColor}{RGB}{0,0,0}

\node[text=drawColor,anchor=base,inner sep=0pt, outer sep=0pt, scale=  1.00] at (232.38,  3.60) {Nutzungsh"aufigkeit $h$ [Nutzungseinheiten/Jahr]};

\node[text=drawColor,rotate= 90.00,anchor=base,inner sep=0pt, outer sep=0pt, scale=  1.00] at (  8.40,135.54) {MIPS [kg/Service-Einheit]};
\end{scope}
\begin{scope}
\path[clip] ( 46.80, 49.20) rectangle (417.97,221.88);
\definecolor{drawColor}{RGB}{0,0,0}

\path[draw=drawColor,line width= 0.4pt,dash pattern=on 1pt off 3pt ,line join=round,line cap=round] (175.10, 49.20) -- (175.10,221.88);

\path[draw=drawColor,line width= 0.4pt,dash pattern=on 1pt off 3pt ,line join=round,line cap=round] (232.38, 49.20) -- (232.38,221.88);
\end{scope}
\begin{scope}
\path[clip] (  0.00,  0.00) rectangle (419.17,289.08);
\definecolor{drawColor}{RGB}{0,0,0}

\node[text=drawColor,anchor=base,inner sep=0pt, outer sep=0pt, scale=  1.20] at (232.38,275.88) {\bfseries Materialintensit"at pro Service-Einheit MIPS$(h)$};
\end{scope}
\begin{scope}
\path[clip] (  0.00,  0.00) rectangle (419.17,289.08);
\definecolor{drawColor}{RGB}{0,0,0}

\path[draw=drawColor,line width= 0.4pt,line join=round,line cap=round] ( 60.55, 49.20) -- (404.22, 49.20);

\path[draw=drawColor,line width= 0.4pt,line join=round,line cap=round] ( 60.55, 49.20) -- ( 60.55, 43.20);

\path[draw=drawColor,line width= 0.4pt,line join=round,line cap=round] (129.28, 49.20) -- (129.28, 43.20);

\path[draw=drawColor,line width= 0.4pt,line join=round,line cap=round] (175.10, 49.20) -- (175.10, 43.20);

\path[draw=drawColor,line width= 0.4pt,line join=round,line cap=round] (198.02, 49.20) -- (198.02, 43.20);

\path[draw=drawColor,line width= 0.4pt,line join=round,line cap=round] (232.38, 49.20) -- (232.38, 43.20);

\path[draw=drawColor,line width= 0.4pt,line join=round,line cap=round] (266.75, 49.20) -- (266.75, 43.20);

\path[draw=drawColor,line width= 0.4pt,line join=round,line cap=round] (335.48, 49.20) -- (335.48, 43.20);

\path[draw=drawColor,line width= 0.4pt,line join=round,line cap=round] (404.22, 49.20) -- (404.22, 43.20);

\node[text=drawColor,anchor=base,inner sep=0pt, outer sep=0pt, scale=  1.00] at ( 60.55, 27.60) {0};

\node[text=drawColor,anchor=base,inner sep=0pt, outer sep=0pt, scale=  1.00] at (129.28, 27.60) {200};

\node[text=drawColor,anchor=base,inner sep=0pt, outer sep=0pt, scale=  1.00] at (175.10, 27.60) {$h^*$};

\node[text=drawColor,anchor=base,inner sep=0pt, outer sep=0pt, scale=  1.00] at (198.02, 27.60) {400};

\node[text=drawColor,anchor=base,inner sep=0pt, outer sep=0pt, scale=  1.00] at (232.38, 27.60) {$h_\text{max}$};

\node[text=drawColor,anchor=base,inner sep=0pt, outer sep=0pt, scale=  1.00] at (266.75, 27.60) {600};

\node[text=drawColor,anchor=base,inner sep=0pt, outer sep=0pt, scale=  1.00] at (335.48, 27.60) {800};

\node[text=drawColor,anchor=base,inner sep=0pt, outer sep=0pt, scale=  1.00] at (404.22, 27.60) {1000};

\path[draw=drawColor,line width= 0.4pt,line join=round,line cap=round] ( 46.80, 55.60) -- ( 46.80,215.48);

\path[draw=drawColor,line width= 0.4pt,line join=round,line cap=round] ( 46.80, 55.60) -- ( 40.80, 55.60);

\path[draw=drawColor,line width= 0.4pt,line join=round,line cap=round] ( 46.80, 82.24) -- ( 40.80, 82.24);

\path[draw=drawColor,line width= 0.4pt,line join=round,line cap=round] ( 46.80,108.89) -- ( 40.80,108.89);

\path[draw=drawColor,line width= 0.4pt,line join=round,line cap=round] ( 46.80,108.89) -- ( 40.80,108.89);

\path[draw=drawColor,line width= 0.4pt,line join=round,line cap=round] ( 46.80,108.89) -- ( 40.80,108.89);

\path[draw=drawColor,line width= 0.4pt,line join=round,line cap=round] ( 46.80,117.77) -- ( 40.80,117.77);

\path[draw=drawColor,line width= 0.4pt,line join=round,line cap=round] ( 46.80,126.66) -- ( 40.80,126.66);

\path[draw=drawColor,line width= 0.4pt,line join=round,line cap=round] ( 46.80,135.54) -- ( 40.80,135.54);

\path[draw=drawColor,line width= 0.4pt,line join=round,line cap=round] ( 46.80,144.42) -- ( 40.80,144.42);

\path[draw=drawColor,line width= 0.4pt,line join=round,line cap=round] ( 46.80,153.31) -- ( 40.80,153.31);

\path[draw=drawColor,line width= 0.4pt,line join=round,line cap=round] ( 46.80,162.19) -- ( 40.80,162.19);

\path[draw=drawColor,line width= 0.4pt,line join=round,line cap=round] ( 46.80,171.07) -- ( 40.80,171.07);

\path[draw=drawColor,line width= 0.4pt,line join=round,line cap=round] ( 46.80,179.95) -- ( 40.80,179.95);

\path[draw=drawColor,line width= 0.4pt,line join=round,line cap=round] ( 46.80,188.84) -- ( 40.80,188.84);

\path[draw=drawColor,line width= 0.4pt,line join=round,line cap=round] ( 46.80,197.72) -- ( 40.80,197.72);

\path[draw=drawColor,line width= 0.4pt,line join=round,line cap=round] ( 46.80,206.60) -- ( 40.80,206.60);

\path[draw=drawColor,line width= 0.4pt,line join=round,line cap=round] ( 46.80,215.48) -- ( 40.80,215.48);

\node[text=drawColor,rotate= 90.00,anchor=base,inner sep=0pt, outer sep=0pt, scale=  1.00] at ( 32.40, 55.60) {0};

\node[text=drawColor,rotate= 90.00,anchor=base,inner sep=0pt, outer sep=0pt, scale=  1.00] at ( 32.40, 82.24) {1};

\node[text=drawColor,rotate= 90.00,anchor=base,inner sep=0pt, outer sep=0pt, scale=  1.00] at ( 32.40,108.89) {2};

\node[text=drawColor,rotate= 90.00,anchor=base,inner sep=0pt, outer sep=0pt, scale=  1.00] at ( 32.40,135.54) {3};

\node[text=drawColor,rotate= 90.00,anchor=base,inner sep=0pt, outer sep=0pt, scale=  1.00] at ( 32.40,162.19) {4};

\node[text=drawColor,rotate= 90.00,anchor=base,inner sep=0pt, outer sep=0pt, scale=  1.00] at ( 32.40,188.84) {5};

\node[text=drawColor,rotate= 90.00,anchor=base,inner sep=0pt, outer sep=0pt, scale=  1.00] at ( 32.40,215.48) {6};

\path[draw=drawColor,line width= 0.4pt,line join=round,line cap=round] ( 83.46,221.88) -- (404.22,221.88);

\path[draw=drawColor,line width= 0.4pt,line join=round,line cap=round] ( 83.46,221.88) -- ( 83.46,227.88);

\path[draw=drawColor,line width= 0.4pt,line join=round,line cap=round] ( 85.09,221.88) -- ( 85.09,227.88);

\path[draw=drawColor,line width= 0.4pt,line join=round,line cap=round] ( 86.98,221.88) -- ( 86.98,227.88);

\path[draw=drawColor,line width= 0.4pt,line join=round,line cap=round] ( 89.19,221.88) -- ( 89.19,227.88);

\path[draw=drawColor,line width= 0.4pt,line join=round,line cap=round] ( 91.79,221.88) -- ( 91.79,227.88);

\path[draw=drawColor,line width= 0.4pt,line join=round,line cap=round] ( 94.91,221.88) -- ( 94.91,227.88);

\path[draw=drawColor,line width= 0.4pt,line join=round,line cap=round] ( 98.73,221.88) -- ( 98.73,227.88);

\path[draw=drawColor,line width= 0.4pt,line join=round,line cap=round] (103.51,221.88) -- (103.51,227.88);

\path[draw=drawColor,line width= 0.4pt,line join=round,line cap=round] (109.64,221.88) -- (109.64,227.88);

\path[draw=drawColor,line width= 0.4pt,line join=round,line cap=round] (117.83,221.88) -- (117.83,227.88);

\path[draw=drawColor,line width= 0.4pt,line join=round,line cap=round] (129.28,221.88) -- (129.28,227.88);

\path[draw=drawColor,line width= 0.4pt,line join=round,line cap=round] (146.46,221.88) -- (146.46,227.88);

\path[draw=drawColor,line width= 0.4pt,line join=round,line cap=round] (175.10,221.88) -- (175.10,227.88);

\path[draw=drawColor,line width= 0.4pt,line join=round,line cap=round] (232.38,221.88) -- (232.38,227.88);

\path[draw=drawColor,line width= 0.4pt,line join=round,line cap=round] (404.22,221.88) -- (404.22,227.88);

\node[text=drawColor,anchor=base,inner sep=0pt, outer sep=0pt, scale=  1.00] at ( 83.46,236.28) {15};

\node[text=drawColor,anchor=base,inner sep=0pt, outer sep=0pt, scale=  1.00] at (103.51,236.28) {8};

\node[text=drawColor,anchor=base,inner sep=0pt, outer sep=0pt, scale=  1.00] at (117.83,236.28) {6};

\node[text=drawColor,anchor=base,inner sep=0pt, outer sep=0pt, scale=  1.00] at (146.46,236.28) {4};

\node[text=drawColor,anchor=base,inner sep=0pt, outer sep=0pt, scale=  1.00] at (175.10,236.28) {3};

\node[text=drawColor,anchor=base,inner sep=0pt, outer sep=0pt, scale=  1.00] at (232.38,236.28) {2};

\node[text=drawColor,anchor=base,inner sep=0pt, outer sep=0pt, scale=  1.00] at (404.22,236.28) {1};

\path[draw=drawColor,line width= 0.4pt,line join=round,line cap=round] ( 46.80, 49.20) --
	(417.97, 49.20) --
	(417.97,221.88) --
	( 46.80,221.88) --
	( 46.80, 49.20);

\node[text=drawColor,anchor=base,inner sep=0pt, outer sep=0pt, scale=  1.00] at (232.38,254.28) {parallele Produktanzahl $p$};
\end{scope}
\end{tikzpicture}

			}
		\end{center}
	\end{frame}

	\begin{frame}{Analyse}		
		\begin{itemize}
			\item<1-> Wie stark fällt die MIPS mit steigendem $h$ im Bereich $h$ < $h^*$?
			\item<2-> Korrespondenz von $h$ und $p$: \\ $h_1 \rightarrow h_2 = h_1 + \Delta h \qquad \Leftrightarrow \qquad p_1 \rightarrow p_2 = p_1 -1$
			\item<3-> Änderung der MIPS: \\ $\Delta \text{MIPS} = \text{MIPS}(h_2) - \text{MIPS}(h_1) = - \frac{q \cdot i_P}{S_D}$
		\end{itemize}
		\begin{center}
			\resizebox{0.5\linewidth}{!}{
				% Created by tikzDevice version 0.8.1 on 2015-04-03 10:57:13
% !TEX encoding = UTF-8 Unicode
\begin{tikzpicture}[x=1pt,y=1pt]
\definecolor{fillColor}{RGB}{255,255,255}
\path[use as bounding box,fill=fillColor,fill opacity=0.00] (0,0) rectangle (419.17,289.08);
\begin{scope}
\path[clip] ( 46.80, 49.20) rectangle (417.97,221.88);
\definecolor{drawColor}{RGB}{190,190,190}

\path[draw=drawColor,line width= 0.4pt,line join=round,line cap=round] (404.22,108.89) --
	(381.63,108.89) --
	(361.83,108.89) --
	(344.33,108.89) --
	(328.75,108.89) --
	(314.79,108.89) --
	(302.21,108.89) --
	(290.82,108.89) --
	(280.45,108.89) --
	(270.98,108.89) --
	(262.29,108.89) --
	(254.29,108.89) --
	(246.90,108.89) --
	(240.05,108.89) --
	(233.69,108.89) --
	(227.76,108.89) --
	(222.23,108.89) --
	(217.05,108.89) --
	(212.19,108.89) --
	(207.62,108.89) --
	(203.33,108.89) --
	(199.27,108.89) --
	(195.44,108.89) --
	(191.82,108.89) --
	(188.38,108.89) --
	(185.12,108.89) --
	(182.02,108.89) --
	(179.08,108.89) --
	(176.27,108.89) --
	(173.59,109.25) --
	(171.03,109.87) --
	(168.59,110.50) --
	(166.25,111.12) --
	(164.01,111.75) --
	(161.87,112.37) --
	(159.81,113.00) --
	(157.83,113.62) --
	(155.93,114.25) --
	(154.10,114.87) --
	(152.35,115.50) --
	(150.65,116.12) --
	(149.02,116.75) --
	(147.45,117.37) --
	(145.93,118.00) --
	(144.46,118.62) --
	(143.04,119.25) --
	(141.67,119.87) --
	(140.35,120.50) --
	(139.07,121.12) --
	(137.82,121.75) --
	(136.62,122.37) --
	(135.45,123.00) --
	(134.32,123.62) --
	(133.23,124.25) --
	(132.16,124.87) --
	(131.13,125.50) --
	(130.12,126.12) --
	(129.14,126.75) --
	(128.19,127.37) --
	(127.27,128.00) --
	(126.37,128.62) --
	(125.50,129.25) --
	(124.64,129.87) --
	(123.81,130.50) --
	(123.00,131.12) --
	(122.22,131.75) --
	(121.45,132.37) --
	(120.70,133.00) --
	(119.97,133.62) --
	(119.25,134.25) --
	(118.55,134.87) --
	(117.87,135.50) --
	(117.21,136.12) --
	(116.56,136.75) --
	(115.92,137.37) --
	(115.30,138.00) --
	(114.70,138.62) --
	(114.10,139.24) --
	(113.52,139.87) --
	(112.95,140.49) --
	(112.40,141.12) --
	(111.85,141.74) --
	(111.32,142.37) --
	(110.80,142.99) --
	(110.29,143.62) --
	(109.78,144.24) --
	(109.29,144.87) --
	(108.81,145.49) --
	(108.34,146.12) --
	(107.88,146.74) --
	(107.42,147.37) --
	(106.98,147.99) --
	(106.54,148.62) --
	(106.11,149.24) --
	(105.69,149.87) --
	(105.28,150.49) --
	(104.87,151.12) --
	(104.47,151.74) --
	(104.08,152.37) --
	(103.70,152.99) --
	(103.32,153.62) --
	(102.95,154.24) --
	(102.58,154.87) --
	(102.22,155.49) --
	(101.87,156.12) --
	(101.52,156.74) --
	(101.18,157.37) --
	(100.85,157.99) --
	(100.52,158.62) --
	(100.19,159.24) --
	( 99.87,159.87) --
	( 99.56,160.49) --
	( 99.25,161.12) --
	( 98.95,161.74) --
	( 98.65,162.37) --
	( 98.35,162.99) --
	( 98.06,163.62) --
	( 97.78,164.24) --
	( 97.49,164.87) --
	( 97.22,165.49) --
	( 96.94,166.12) --
	( 96.68,166.74) --
	( 96.41,167.37) --
	( 96.15,167.99) --
	( 95.89,168.62) --
	( 95.64,169.24) --
	( 95.39,169.87) --
	( 95.14,170.49) --
	( 94.90,171.12) --
	( 94.66,171.74) --
	( 94.42,172.37) --
	( 94.19,172.99) --
	( 93.96,173.62) --
	( 93.73,174.24) --
	( 93.51,174.86) --
	( 93.29,175.49) --
	( 93.07,176.11) --
	( 92.85,176.74) --
	( 92.64,177.36) --
	( 92.43,177.99) --
	( 92.22,178.61) --
	( 92.02,179.24) --
	( 91.82,179.86) --
	( 91.62,180.49) --
	( 91.42,181.11) --
	( 91.23,181.74) --
	( 91.04,182.36) --
	( 90.85,182.99) --
	( 90.66,183.61) --
	( 90.48,184.24) --
	( 90.29,184.86) --
	( 90.11,185.49) --
	( 89.94,186.11) --
	( 89.76,186.74) --
	( 89.59,187.36) --
	( 89.42,187.99) --
	( 89.25,188.61) --
	( 89.08,189.24) --
	( 88.91,189.86) --
	( 88.75,190.49) --
	( 88.59,191.11) --
	( 88.43,191.74) --
	( 88.27,192.36) --
	( 88.11,192.99) --
	( 87.96,193.61) --
	( 87.81,194.24) --
	( 87.65,194.86) --
	( 87.50,195.49) --
	( 87.36,196.11) --
	( 87.21,196.74) --
	( 87.07,197.36) --
	( 86.92,197.99) --
	( 86.78,198.61) --
	( 86.64,199.24) --
	( 86.50,199.86) --
	( 86.36,200.49) --
	( 86.23,201.11) --
	( 86.09,201.74) --
	( 85.96,202.36) --
	( 85.83,202.99) --
	( 85.70,203.61) --
	( 85.57,204.24) --
	( 85.44,204.86) --
	( 85.32,205.49) --
	( 85.19,206.11) --
	( 85.07,206.74) --
	( 84.95,207.36) --
	( 84.82,207.99) --
	( 84.70,208.61) --
	( 84.59,209.24) --
	( 84.47,209.86) --
	( 84.35,210.49) --
	( 84.24,211.11) --
	( 84.12,211.73) --
	( 84.01,212.36) --
	( 83.90,212.98) --
	( 83.79,213.61) --
	( 83.68,214.23) --
	( 83.57,214.86) --
	( 83.46,215.48);
\definecolor{drawColor}{RGB}{0,0,0}
\definecolor{fillColor}{RGB}{255,255,255}

\path[draw=drawColor,line width= 0.4pt,line join=round,line cap=round,fill=fillColor] (402.23,106.90) rectangle (406.21,110.89);

\path[draw=drawColor,line width= 0.4pt,line join=round,line cap=round,fill=fillColor] (230.39,106.90) rectangle (234.38,110.89);

\path[draw=drawColor,line width= 0.4pt,line join=round,line cap=round,fill=fillColor] (173.11,106.90) rectangle (177.10,110.89);

\path[draw=drawColor,line width= 0.4pt,line join=round,line cap=round,fill=fillColor] (144.47,115.78) rectangle (148.46,119.77);

\path[draw=drawColor,line width= 0.4pt,line join=round,line cap=round,fill=fillColor] (127.29,124.66) rectangle (131.28,128.65);

\path[draw=drawColor,line width= 0.4pt,line join=round,line cap=round,fill=fillColor] (115.83,133.55) rectangle (119.82,137.53);

\path[draw=drawColor,line width= 0.4pt,line join=round,line cap=round,fill=fillColor] (107.65,142.43) rectangle (111.64,146.42);

\path[draw=drawColor,line width= 0.4pt,line join=round,line cap=round,fill=fillColor] (101.51,151.31) rectangle (105.50,155.30);

\path[draw=drawColor,line width= 0.4pt,line join=round,line cap=round,fill=fillColor] ( 96.74,160.19) rectangle (100.73,164.18);

\path[draw=drawColor,line width= 0.4pt,line join=round,line cap=round,fill=fillColor] ( 92.92,169.08) rectangle ( 96.91,173.06);

\path[draw=drawColor,line width= 0.4pt,line join=round,line cap=round,fill=fillColor] ( 89.80,177.96) rectangle ( 93.78,181.95);

\path[draw=drawColor,line width= 0.4pt,line join=round,line cap=round,fill=fillColor] ( 87.19,186.84) rectangle ( 91.18,190.83);

\path[draw=drawColor,line width= 0.4pt,line join=round,line cap=round,fill=fillColor] ( 84.99,195.73) rectangle ( 88.98,199.71);

\path[draw=drawColor,line width= 0.4pt,line join=round,line cap=round,fill=fillColor] ( 83.10,204.61) rectangle ( 87.09,208.60);

\path[draw=drawColor,line width= 0.4pt,line join=round,line cap=round,fill=fillColor] ( 81.46,213.49) rectangle ( 85.45,217.48);
\end{scope}
\begin{scope}
\path[clip] (  0.00,  0.00) rectangle (419.17,289.08);
\definecolor{drawColor}{RGB}{0,0,0}

\node[text=drawColor,anchor=base,inner sep=0pt, outer sep=0pt, scale=  1.00] at (232.38,  3.60) {Nutzungsh"aufigkeit $h$ [Nutzungseinheiten/Jahr]};

\node[text=drawColor,rotate= 90.00,anchor=base,inner sep=0pt, outer sep=0pt, scale=  1.00] at (  8.40,135.54) {MIPS [kg/Service-Einheit]};
\end{scope}
\begin{scope}
\path[clip] ( 46.80, 49.20) rectangle (417.97,221.88);
\definecolor{drawColor}{RGB}{0,0,0}

\path[draw=drawColor,line width= 0.4pt,dash pattern=on 1pt off 3pt ,line join=round,line cap=round] (175.10, 49.20) -- (175.10,221.88);

\path[draw=drawColor,line width= 0.4pt,dash pattern=on 1pt off 3pt ,line join=round,line cap=round] (232.38, 49.20) -- (232.38,221.88);
\end{scope}
\begin{scope}
\path[clip] (  0.00,  0.00) rectangle (419.17,289.08);
\definecolor{drawColor}{RGB}{0,0,0}

\node[text=drawColor,anchor=base,inner sep=0pt, outer sep=0pt, scale=  1.20] at (232.38,275.88) {\bfseries Materialintensit"at pro Service-Einheit MIPS$(h)$};
\end{scope}
\begin{scope}
\path[clip] (  0.00,  0.00) rectangle (419.17,289.08);
\definecolor{drawColor}{RGB}{0,0,0}

\path[draw=drawColor,line width= 0.4pt,line join=round,line cap=round] ( 60.55, 49.20) -- (404.22, 49.20);

\path[draw=drawColor,line width= 0.4pt,line join=round,line cap=round] ( 60.55, 49.20) -- ( 60.55, 43.20);

\path[draw=drawColor,line width= 0.4pt,line join=round,line cap=round] (129.28, 49.20) -- (129.28, 43.20);

\path[draw=drawColor,line width= 0.4pt,line join=round,line cap=round] (175.10, 49.20) -- (175.10, 43.20);

\path[draw=drawColor,line width= 0.4pt,line join=round,line cap=round] (198.02, 49.20) -- (198.02, 43.20);

\path[draw=drawColor,line width= 0.4pt,line join=round,line cap=round] (232.38, 49.20) -- (232.38, 43.20);

\path[draw=drawColor,line width= 0.4pt,line join=round,line cap=round] (266.75, 49.20) -- (266.75, 43.20);

\path[draw=drawColor,line width= 0.4pt,line join=round,line cap=round] (335.48, 49.20) -- (335.48, 43.20);

\path[draw=drawColor,line width= 0.4pt,line join=round,line cap=round] (404.22, 49.20) -- (404.22, 43.20);

\node[text=drawColor,anchor=base,inner sep=0pt, outer sep=0pt, scale=  1.00] at ( 60.55, 27.60) {0};

\node[text=drawColor,anchor=base,inner sep=0pt, outer sep=0pt, scale=  1.00] at (129.28, 27.60) {200};

\node[text=drawColor,anchor=base,inner sep=0pt, outer sep=0pt, scale=  1.00] at (175.10, 27.60) {$h^*$};

\node[text=drawColor,anchor=base,inner sep=0pt, outer sep=0pt, scale=  1.00] at (198.02, 27.60) {400};

\node[text=drawColor,anchor=base,inner sep=0pt, outer sep=0pt, scale=  1.00] at (232.38, 27.60) {$h_\text{max}$};

\node[text=drawColor,anchor=base,inner sep=0pt, outer sep=0pt, scale=  1.00] at (266.75, 27.60) {600};

\node[text=drawColor,anchor=base,inner sep=0pt, outer sep=0pt, scale=  1.00] at (335.48, 27.60) {800};

\node[text=drawColor,anchor=base,inner sep=0pt, outer sep=0pt, scale=  1.00] at (404.22, 27.60) {1000};

\path[draw=drawColor,line width= 0.4pt,line join=round,line cap=round] ( 46.80, 55.60) -- ( 46.80,215.48);

\path[draw=drawColor,line width= 0.4pt,line join=round,line cap=round] ( 46.80, 55.60) -- ( 40.80, 55.60);

\path[draw=drawColor,line width= 0.4pt,line join=round,line cap=round] ( 46.80, 82.24) -- ( 40.80, 82.24);

\path[draw=drawColor,line width= 0.4pt,line join=round,line cap=round] ( 46.80,108.89) -- ( 40.80,108.89);

\path[draw=drawColor,line width= 0.4pt,line join=round,line cap=round] ( 46.80,108.89) -- ( 40.80,108.89);

\path[draw=drawColor,line width= 0.4pt,line join=round,line cap=round] ( 46.80,108.89) -- ( 40.80,108.89);

\path[draw=drawColor,line width= 0.4pt,line join=round,line cap=round] ( 46.80,117.77) -- ( 40.80,117.77);

\path[draw=drawColor,line width= 0.4pt,line join=round,line cap=round] ( 46.80,126.66) -- ( 40.80,126.66);

\path[draw=drawColor,line width= 0.4pt,line join=round,line cap=round] ( 46.80,135.54) -- ( 40.80,135.54);

\path[draw=drawColor,line width= 0.4pt,line join=round,line cap=round] ( 46.80,144.42) -- ( 40.80,144.42);

\path[draw=drawColor,line width= 0.4pt,line join=round,line cap=round] ( 46.80,153.31) -- ( 40.80,153.31);

\path[draw=drawColor,line width= 0.4pt,line join=round,line cap=round] ( 46.80,162.19) -- ( 40.80,162.19);

\path[draw=drawColor,line width= 0.4pt,line join=round,line cap=round] ( 46.80,171.07) -- ( 40.80,171.07);

\path[draw=drawColor,line width= 0.4pt,line join=round,line cap=round] ( 46.80,179.95) -- ( 40.80,179.95);

\path[draw=drawColor,line width= 0.4pt,line join=round,line cap=round] ( 46.80,188.84) -- ( 40.80,188.84);

\path[draw=drawColor,line width= 0.4pt,line join=round,line cap=round] ( 46.80,197.72) -- ( 40.80,197.72);

\path[draw=drawColor,line width= 0.4pt,line join=round,line cap=round] ( 46.80,206.60) -- ( 40.80,206.60);

\path[draw=drawColor,line width= 0.4pt,line join=round,line cap=round] ( 46.80,215.48) -- ( 40.80,215.48);

\node[text=drawColor,rotate= 90.00,anchor=base,inner sep=0pt, outer sep=0pt, scale=  1.00] at ( 32.40, 55.60) {0};

\node[text=drawColor,rotate= 90.00,anchor=base,inner sep=0pt, outer sep=0pt, scale=  1.00] at ( 32.40, 82.24) {1};

\node[text=drawColor,rotate= 90.00,anchor=base,inner sep=0pt, outer sep=0pt, scale=  1.00] at ( 32.40,108.89) {2};

\node[text=drawColor,rotate= 90.00,anchor=base,inner sep=0pt, outer sep=0pt, scale=  1.00] at ( 32.40,135.54) {3};

\node[text=drawColor,rotate= 90.00,anchor=base,inner sep=0pt, outer sep=0pt, scale=  1.00] at ( 32.40,162.19) {4};

\node[text=drawColor,rotate= 90.00,anchor=base,inner sep=0pt, outer sep=0pt, scale=  1.00] at ( 32.40,188.84) {5};

\node[text=drawColor,rotate= 90.00,anchor=base,inner sep=0pt, outer sep=0pt, scale=  1.00] at ( 32.40,215.48) {6};

\path[draw=drawColor,line width= 0.4pt,line join=round,line cap=round] ( 83.46,221.88) -- (404.22,221.88);

\path[draw=drawColor,line width= 0.4pt,line join=round,line cap=round] ( 83.46,221.88) -- ( 83.46,227.88);

\path[draw=drawColor,line width= 0.4pt,line join=round,line cap=round] ( 85.09,221.88) -- ( 85.09,227.88);

\path[draw=drawColor,line width= 0.4pt,line join=round,line cap=round] ( 86.98,221.88) -- ( 86.98,227.88);

\path[draw=drawColor,line width= 0.4pt,line join=round,line cap=round] ( 89.19,221.88) -- ( 89.19,227.88);

\path[draw=drawColor,line width= 0.4pt,line join=round,line cap=round] ( 91.79,221.88) -- ( 91.79,227.88);

\path[draw=drawColor,line width= 0.4pt,line join=round,line cap=round] ( 94.91,221.88) -- ( 94.91,227.88);

\path[draw=drawColor,line width= 0.4pt,line join=round,line cap=round] ( 98.73,221.88) -- ( 98.73,227.88);

\path[draw=drawColor,line width= 0.4pt,line join=round,line cap=round] (103.51,221.88) -- (103.51,227.88);

\path[draw=drawColor,line width= 0.4pt,line join=round,line cap=round] (109.64,221.88) -- (109.64,227.88);

\path[draw=drawColor,line width= 0.4pt,line join=round,line cap=round] (117.83,221.88) -- (117.83,227.88);

\path[draw=drawColor,line width= 0.4pt,line join=round,line cap=round] (129.28,221.88) -- (129.28,227.88);

\path[draw=drawColor,line width= 0.4pt,line join=round,line cap=round] (146.46,221.88) -- (146.46,227.88);

\path[draw=drawColor,line width= 0.4pt,line join=round,line cap=round] (175.10,221.88) -- (175.10,227.88);

\path[draw=drawColor,line width= 0.4pt,line join=round,line cap=round] (232.38,221.88) -- (232.38,227.88);

\path[draw=drawColor,line width= 0.4pt,line join=round,line cap=round] (404.22,221.88) -- (404.22,227.88);

\node[text=drawColor,anchor=base,inner sep=0pt, outer sep=0pt, scale=  1.00] at ( 83.46,236.28) {15};

\node[text=drawColor,anchor=base,inner sep=0pt, outer sep=0pt, scale=  1.00] at (103.51,236.28) {8};

\node[text=drawColor,anchor=base,inner sep=0pt, outer sep=0pt, scale=  1.00] at (117.83,236.28) {6};

\node[text=drawColor,anchor=base,inner sep=0pt, outer sep=0pt, scale=  1.00] at (146.46,236.28) {4};

\node[text=drawColor,anchor=base,inner sep=0pt, outer sep=0pt, scale=  1.00] at (175.10,236.28) {3};

\node[text=drawColor,anchor=base,inner sep=0pt, outer sep=0pt, scale=  1.00] at (232.38,236.28) {2};

\node[text=drawColor,anchor=base,inner sep=0pt, outer sep=0pt, scale=  1.00] at (404.22,236.28) {1};

\path[draw=drawColor,line width= 0.4pt,line join=round,line cap=round] ( 46.80, 49.20) --
	(417.97, 49.20) --
	(417.97,221.88) --
	( 46.80,221.88) --
	( 46.80, 49.20);

\node[text=drawColor,anchor=base,inner sep=0pt, outer sep=0pt, scale=  1.00] at (232.38,254.28) {parallele Produktanzahl $p$};
\end{scope}
\end{tikzpicture}

			}
		\end{center}
	\end{frame}

    \subsection{Modellkopplung}
    \section{Synthese und Diskussion}
\end{document}
