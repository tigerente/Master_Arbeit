\documentclass[beamer]{beamer}
%\usepackage{mathpazo}
\usepackage[T1]{fontenc}
\usepackage[utf8]{inputenc}
\usepackage[ngerman]{babel}
\usepackage[babel,german=quotes]{csquotes}
\usepackage{lmodern} % for removing warning about font shape

%\usetheme{Madrid}
\usetheme{Frankfurt}
%\usetheme{Singapore}
\usecolortheme{crane}

\setbeamertemplate{footline}[frame number] % für Foliennummerierung
\beamertemplatenavigationsymbolsempty % Navigationsleiste 

\AtBeginSection{\frame{\sectionpage}}
\newtranslation[to=ngerman]{Section}{Abschnitt}
\newtranslation[to=ngerman]{Subsection}{Beispielmodell}

\usepackage{etoolbox}
\makeatletter
\patchcmd{\slideentry}{\ifnum#2>0}{\ifnum2>0}{}{\@error{unable to patch}}% replace the subsection number test with a test that always returns true
\makeatother

%\setbeameroption{show notes}

\usepackage{amsmath}
\usepackage{calc}

\usepackage{graphicx} %Zum Einbinden von Grafikdateien
\usepackage[font={footnotesize, sf}]{caption}
\usepackage{subfig}
\graphicspath{{./Abbildungen/}}

\usepackage{booktabs} %Für schönere Tabellen
\usepackage{xspace}

\usepackage{tikz} %Für Skizzen

%\usepackage{vmargin} %Fürs Seitenlayout
%\setmarginsrb{2.5cm}{1.5cm}{2cm}{1.5cm}{0mm}{0.5cm}{0mm}{1cm}

\title{Ökologische Nachhaltigkeit durch \\ \enquote{Nutzen statt Besitzen}}
\subtitle{{\small Entwicklung eines Modells zur Ableitung von Kriterien für die Senkung des Umweltverbrauchs durch gemeinschaftliche Produktnutzung}}
\author{Sascha, János}
\date{29.04.2015}

\begin{document}

\frame{\titlepage}

\frame<beamer>{\tableofcontents[hidesubsections]}

\section{Thema, Fragestellung, Methodik}

	\begin{frame}{Thema und Fragestellung}
	  \begin{center}
	  	Hier steht das Thema.
	  \end{center}
	\end{frame}
	
	\begin{frame}{Untersuchte Effekte}
		\begin{center}
			Bla
		\end{center}
	\end{frame}
	
	\begin{frame}{Methodik}
		\begin{center}
			Modellierung
				Nutzungssystem
					Personen
					Service-Nachfrage
					Produkte
						…
					Organisation der Nutzung
						…
				Vergleich verschiedener Nutzungssysteme
					Wie wirken die einzelnen Faktoren auf die MIPS?
					MIPS gewährleistet Vergleichbarkeit
			
			Analyse
				Unter welchen Umständen verringert sich die MIPS durch gemeinschaftliche Nutzung?
				Grund-Annahme: Parameter ändern sich durch gemeinschaftliche Nutzung in eine bestimmte Richtung
				Partialanalyse: Nutzungssysteme, die sich in nur einem Parameter unterscheiden
				Modellkopplung: mehrere Parameter ändern sich	
		\end{center}
	\end{frame}

\section{Modelle}
	\begin{frame}{Modell-Grundlagen}
		\begin{itemize}
			\pause
			\item Wichtige Größe: Produktnutzungsdauer $t$
			\begin{itemize}
				\item Wie lange ist ein Produkt in der Nutzung?
				\item Erreichen der technischen Lebensdauer oder der Maximalnutzungsdauer:
				$	t = \min \left\{t_\text{tech}, t_{\text{max}} \right\}$
				\item Nicht fest vorgegeben, sondern von der Nutzung abhängig.
			\end{itemize}
			\pause
			\item Nicht betrachtet: Nachfrage-Änderungen
			\begin{itemize}
				\item Konstante Service-Nachfrage $S = S_D$
				\item Problem: Nicht kompatibel mit variabler Produktnutzungsdauer.
				\item Lösung: Konstanter Betrachtungszeitraum T
			\end{itemize}
		\end{itemize}
	\end{frame}
	
	\begin{frame}{Modell-Grundlagen}
		\begin{itemize}
			\item Produkte des Nutzungssystems
				\begin{itemize}
					\item parallel eingesetzte Produkte
					\item sequentiell eingesetzte Produkte
				\end{itemize}
		\end{itemize}
		\begin{figure}[h]
			\includegraphics<1>[height=3cm]{Produktanzahlen_1_0}
			\includegraphics<2>[height=3cm]{Produktanzahlen_1_1}
			\includegraphics<3>[height=3cm]{Produktanzahlen_1_2}
			\includegraphics<4>[height=3cm]{Produktanzahlen_1_3}
			\includegraphics<5->[height=3cm]{Produktanzahlen_2}
		\end{figure}
		\pause
		\begin{itemize}
			\item<6-> Fragen / Anmerkungen?
		\end{itemize}
	\end{frame}

\subsection{Nutzungsintensivierung}
	\frame{\subsectionpage}
	\begin{frame}{Modellbeschreibung}
		\begin{center}
			Bla
		\end{center}
	\end{frame}
	
	\begin{frame}{Analyse}
		\begin{center}
			Bla
			Fragen / Diskussion
		\end{center}
	\end{frame}
	
\subsection{Reparatur}
\frame{\subsectionpage}
	\begin{frame}{Modellbeschreibung}
		\begin{center}
			Bla
		\end{center}
	\end{frame}
	
	\begin{frame}{Analyse}
		\begin{center}
			Bla
			Fragen / Diskussion
		\end{center}
	\end{frame}

\section{Fallstudie}
	\begin{frame}{Fallstudie}
		\begin{center}
			Szenarien
			Datenverfügbarkeit?
		\end{center}
	\end{frame}

\end{document}
