\documentclass[beamer, xcolor=table]{beamer}
%\usepackage{mathpazo}
\usepackage[T1]{fontenc}
\usepackage[utf8]{inputenc}
\usepackage[ngerman]{babel}
\usepackage[babel,german=quotes]{csquotes}
\usepackage{lmodern} % for removing warning about font shape


%\usetheme{Madrid}
\usetheme{Frankfurt}
%\usetheme{Singapore}
\usecolortheme{crane}

\setbeamertemplate{footline}[frame number] % für Foliennummerierung
\beamertemplatenavigationsymbolsempty % Navigationsleiste 

\AtBeginSection{\frame{\sectionpage}}
\newtranslation[to=ngerman]{Section}{Abschnitt}
\newtranslation[to=ngerman]{Subsection}{Beispielmodell}

\usepackage{etoolbox}
\makeatletter
\patchcmd{\slideentry}{\ifnum#2>0}{\ifnum2>0}{}{\@error{unable to patch}}% replace the subsection number test with a test that always returns true
\makeatother

%\setbeameroption{show notes}

\usepackage{amsmath}
\usepackage{calc}

\usepackage{graphicx} %Zum Einbinden von Grafikdateien
\usepackage[font={footnotesize, sf}]{caption}
\usepackage{subfig}
\graphicspath{{./Abbildungen/}}

\usepackage{booktabs} %Für schönere Tabellen
\usepackage{xspace}

\usepackage{tikz} %Für Skizzen

\usepackage[bibstyle=authortitle, citestyle=authoryear, isbn=false, doi=false,
dashed=false]{biblatex}
\addbibresource{Masterarbeit.bib}

%\usepackage{vmargin} %Fürs Seitenlayout
%\setmarginsrb{2.5cm}{1.5cm}{2cm}{1.5cm}{0mm}{0.5cm}{0mm}{1cm}

%%% Eigene Befehle %%%
\newcommand{\was}[1]{\small\textit{#1}}
\newcommand{\noteS}[1]{\todo[color=green!40]{\textbf{Sascha: }#1}}
\newcommand{\noteJ}[1]{\todo[color=blue!40]{\textbf{János: }#1}}
\newcommand{\textfrac}[2]{\hspace{2pt} \frac{\text{#1}}{\text{#2}}}
\newcommand{\eqnref}[1]{\overset{(\ref{#1})}{=}} % Gleichheitszeichen mit Referenz auf die verwendete Gleichung
\newcommand{\defeq}{\vcentcolon=} %Definitions-Gleichheitszeichen
\newcommand{\eqdef}{=\vcentcolon}
\newcommand{\pfrac}[2]{\frac{\partial #1}{\partial #2}}

% Formeln:
\newcommand{\MIPS}[1][]{
  \ifthenelse {\equal {#1} {}}
  {\text{MIPS}} % if argument is blank
  {\text{MIPS}({#1})} % if an optional argument is given
}
\renewcommand{\P}[1]{P_\text{#1}}
\newcommand{\I}[1]{I_\text{#1}}
\newcommand{\itext}[1]{i_\text{#1}}
\newcommand{\T}[1]{T_\text{#1}}
\newcommand{\n}[1]{n_\text{#1}}
\newcommand{\N}[1]{N_\text{#1}}
\renewcommand{\t}[1]{t_\text{#1}}



\title{Ökologische Nachhaltigkeit durch \\ \enquote{Nutzen statt Besitzen}}
\subtitle{{\small Entwicklung eines Modells zur Ableitung von Kriterien für die Senkung des Umweltverbrauchs durch gemeinschaftliche Produktnutzung}}
\author{Sascha, János}
\date{29.04.2015}

\begin{document}

\frame{\titlepage}

\frame<beamer>{\tableofcontents[hidesubsections]}

\section{Thema, Fragestellung, Methodik}

	\begin{frame}{Thema und Fragestellung}
        Thema: 
        \begin{itemize}
            \item Gemeinschaftliche Nutzung von Produkten (Bspw. Waschmaschinen,
                Autos, Kleidung,\dots)
            \item Ökologische Vorteile solcher Nutzungsformen gegenüber
                individueller Nutzung
        \end{itemize}

        \begin{block}{Fragestellung:}
            \begin{itemize}
                \item Unter welchen Umständen kann der Umweltverbrauch eines Produktes durch gemeinschaftliche Nutzung gegenüber der individuellen Nutzung gesenkt werden?
                % \item Was sind die Mechanismen, die den Umweltverbrauch bei gemeinschaftlicher Nutzung bestimmen und wie wirken diese? 
                % \item Welche Eigenschaften des Produkts und der Nutzungsform beeinflussen die Wirkung dieser Mechanismen und wie fließen sie ein?
            \end{itemize}
        \end{block}
	\end{frame}
	
	\begin{frame}{Untersuchte Effekte}
        \only<1>{
            Was ist eigentlich ein Effekt/Mechanismus?
        }
		\begin{center}
            \scriptsize
            \only<2>{
			\begin{tabular}{p{5cm}p{5cm}}
				 \toprule
				 \multicolumn{2}{l}{\textbf{Umweltauswirkungen durch die Nutzung}}  \\
				 \textbf{positiv} & \textbf{negativ} \\
				 \midrule
				 Nutzungsintensivierung  &  Zusätzliche Transaktionen  \\
				 Einsatz langlebiger Produkte  &  zusätzlicher Ressourcenverbrauch für Langlebigkeit  \\
				 Verwendung verbrauchsarmer bzw. leistungsstarker Geräte  &  Beschleunigte Ausmusterung  \\
				 Maximierung der Geräteauslastung  & Tauschbedingter Verschleiß  \\
				 Berücksichtigung des technischen Fortschritts  &  \\
				 Wartung / Reparaturen & \\
				 \midrule
				 \multicolumn{2}{l}{\textbf{Umweltauswirkungen durch Nachfrageänderung}}  \\
				 \textbf{positiv} & \textbf{negativ} \\
				 \midrule
				 Nachfrageverringerung durch höhere Kostentransparenz  & Erleichterter Produktzugang  \\
				 Vermeidung von Fehlkäufen  & Wunsch nach Eigentum  \\
				  & Rebound-Effekt  \\
				 \bottomrule
			\end{tabular}
            \vspace{3pt}

			Mechanismen für die Umweltauswirkungen durch
            gemeinschaftliche Nutzung. Quelle: verändert nach
            \cite{scholl_marketing_2009}.}
            \only<3>{
            \normalsize
			\begin{tabular}{p{5cm}p{5cm}}
				 \toprule
				 \multicolumn{2}{l}{\textbf{Umweltauswirkungen durch die
                 Nutzung}}  \\[5pt]
				 \textbf{positiv} & \textbf{negativ} \\
				 \midrule
				 Nutzungsintensivierung  &  Zusätzliche Transaktionen
                 \\[3pt]
                 Maximierung der Geräteauslastung  & \\[3pt]
                 Wartung / Reparaturen & \\[3pt]
				 \bottomrule
			\end{tabular}
            \vspace{3pt}

			Betrachtete Mechanismen für die Umweltauswirkungen durch gemeinschaftliche Nutzung. Quelle: verändert nach \cite{scholl_marketing_2009}.}
        \end{center}
	\end{frame}
	
	\begin{frame}{Methodik}
		\begin{center}
            \begin{itemize}
                \item Modellierung
                    \begin{itemize}
                        \item Nutzungssystem
                            \begin{itemize}
                                \item Personen (Service-Nachfrage)
                                \item Produkte
                                \item Organisation der Nutzung
                            \end{itemize}
                        \item Vergleich verschiedener Nutzungssysteme
                            \begin{itemize}
                                \item Wie wirken die einzelnen Faktoren auf die MIPS?
                                \item MIPS gewährleistet Vergleichbarkeit
                            \end{itemize}
                    \end{itemize}

                \item Analyse
                    \begin{itemize}
                        \item Unter welchen Umständen verringert sich die MIPS
                            durch gemeinschaftliche Nutzung?
                        \item Grund-Annahme: Parameter ändern sich durch
                            gemeinschaftliche Nutzung in eine bestimmte Richtung
                        \item Partialanalyse: Nutzungssysteme, die sich in nur
                            einem Parameter unterscheiden
                        \item Modellkopplung: mehrere Parameter ändern sich	
                    \end{itemize}
            \end{itemize}
		\end{center}
	\end{frame}

\section{Modelle}
	\begin{frame}{Modell-Grundlagen}
		\begin{itemize}
			\pause
			\item Wichtige Größe: Produktnutzungsdauer $t$
			\begin{itemize}
				\item Wie lange ist ein Produkt in der Nutzung?
				\item Erreichen der technischen Lebensdauer oder der Maximalnutzungsdauer:
				$t = \min \left\{t_\text{tech}, t_{\text{max}} \right\}$
				\item Nicht fest vorgegeben, sondern von der Nutzung abhängig.
			\end{itemize}
			\pause
			\item Nicht betrachtet: Nachfrage-Änderungen
			\begin{itemize}
				\item Konstante Service-Nachfrage $S = S_D$
				\item Problem: Nicht kompatibel mit variabler Produktnutzungsdauer.
				\item Lösung: Konstanter Betrachtungszeitraum T
			\end{itemize}
		\end{itemize}
	\end{frame}
	
	\begin{frame}{Modell-Grundlagen}
		\begin{itemize}
			\item Produkte des Nutzungssystems
				\begin{itemize}
					\item parallel eingesetzte Produkte
					\item sequentiell eingesetzte Produkte
				\end{itemize}
		\end{itemize}
		\begin{figure}[h]
			\includegraphics<1>[height=3cm]{Produktanzahlen_1_0}
			\includegraphics<2>[height=3cm]{Produktanzahlen_1_1}
			\includegraphics<3>[height=3cm]{Produktanzahlen_1_2}
			\includegraphics<4>[height=3cm]{Produktanzahlen_1_3}
			\includegraphics<5->[height=3cm]{Produktanzahlen_2}
		\end{figure}
		\pause
		\begin{itemize}
			\item <6-> $P = p \cdot q = p \cdot \frac{T}{t}$
			\item<7-> Fragen / Anmerkungen?
		\end{itemize}
	\end{frame}

\subsection{Nutzungsintensivierung}
	\frame{\subsectionpage}
	\begin{frame}{Modellbeschreibung}
		\begin{itemize}
			\item Nutzungsintensivierung = Erhöhung der Nutzungshäufigkeit $h$
			\pause
			\item Nutzungsmenge $n$ \emph{eines} Produkts während $t$: \\ $n = h \cdot t$
			\pause
			\item \emph{Gesamt}nutzungsmenge $N$ während $T$: \\ $N = n \cdot P = h \cdot t \cdot P = h \cdot t \cdot p \cdot \frac{T}{t} = h \cdot p \cdot T$
			\pause
			\item Mindestproduktanzahl $p_{\text{min}}$: \\ $p \geq p_{\text{min}} \quad  (p, p_{\text{min}} \in \mathbb{N})$
			\pause
			\item Servicemenge $S$: \\ $S = S_D = N \cdot A \quad \Leftrightarrow \quad N = \frac{S_D}{A}$ \quad ($A$: Produktauslastung)
			\pause
			\item Nutzungshäufigkeit $h$: \\ $h = \frac{N}{p \cdot T} = \frac{S_D}{A \cdot p \cdot T} \qquad h \leq h_\text{max} := \frac{S_D}{A \cdot p_\text{min} \cdot T}$
		\end{itemize}
	\end{frame}
	
	\begin{frame}{Modellbeschreibung}
		\begin{itemize}
			\item Technische Lebensdauer $t_\text{tech}$: \\ $t_\text{tech} := \frac{n_{\text{max}}}{h}$
			\pause
			\item Produktnutzungsdauer $t$: \\[5pt] $t(h) = \min \left\{t_\text{tech}, t_{\text{max}} \right\} = \left\{\begin{array}{cl}  t_{\text{max}}, & \mbox{falls } h < h^* := \frac{n_{\text{max}}}{t_{\text{max}}} \\ \frac{n_{\text{max}}}{h}, & \mbox{sonst} \end{array}\right.$
			\pause
			\item Nutzungsmenge $n(h)$: \\[5pt] $n (h) = h \cdot t(h) = \left\{\begin{array}{cl}  h \cdot t_{\text{max}}, & \mbox{falls } h < h^* \\ n_{\text{max}}, & \mbox{sonst} \end{array}\right.$
			\pause
			\item Effektive Produktanzahl $P$: \\[5pt] $P (h) = \frac{N}{n(h)} = \left\{\begin{array}{cl}  \frac{S_D}{A \cdot h \cdot t_{\text{max}}}, & \mbox{falls } h < h^* \\[5pt] \frac{S_D}{A \cdot n_{\text{max}}}, & \mbox{sonst} \end{array}\right.$
		\end{itemize}
	\end{frame}
	
	\begin{frame}{Modellbeschreibung}
		\begin{itemize}
			\item $\text{MIPS}(h) = \frac{I}{S} = \frac{P(h) \cdot i_P + I_{\text{fix}}^h}{S} =
			\frac{I_{\text{fix}}^h}{S_D} + \left\{ \begin{array}{cl}  \frac{i_P}{h \cdot t_{\text{max}} \cdot A}, & \mbox{falls } h < h^* \\[5pt] \frac{i_P}{n_{\text{max}} \cdot A}, & \mbox{sonst} \end{array}\right.$
		\end{itemize}
		\pause
		\begin{center}
			\resizebox{0.7\linewidth}{!}{
				% Created by tikzDevice version 0.8.1 on 2015-04-03 10:57:13
% !TEX encoding = UTF-8 Unicode
\begin{tikzpicture}[x=1pt,y=1pt]
\definecolor{fillColor}{RGB}{255,255,255}
\path[use as bounding box,fill=fillColor,fill opacity=0.00] (0,0) rectangle (419.17,289.08);
\begin{scope}
\path[clip] ( 46.80, 49.20) rectangle (417.97,221.88);
\definecolor{drawColor}{RGB}{190,190,190}

\path[draw=drawColor,line width= 0.4pt,line join=round,line cap=round] (404.22,108.89) --
	(381.63,108.89) --
	(361.83,108.89) --
	(344.33,108.89) --
	(328.75,108.89) --
	(314.79,108.89) --
	(302.21,108.89) --
	(290.82,108.89) --
	(280.45,108.89) --
	(270.98,108.89) --
	(262.29,108.89) --
	(254.29,108.89) --
	(246.90,108.89) --
	(240.05,108.89) --
	(233.69,108.89) --
	(227.76,108.89) --
	(222.23,108.89) --
	(217.05,108.89) --
	(212.19,108.89) --
	(207.62,108.89) --
	(203.33,108.89) --
	(199.27,108.89) --
	(195.44,108.89) --
	(191.82,108.89) --
	(188.38,108.89) --
	(185.12,108.89) --
	(182.02,108.89) --
	(179.08,108.89) --
	(176.27,108.89) --
	(173.59,109.25) --
	(171.03,109.87) --
	(168.59,110.50) --
	(166.25,111.12) --
	(164.01,111.75) --
	(161.87,112.37) --
	(159.81,113.00) --
	(157.83,113.62) --
	(155.93,114.25) --
	(154.10,114.87) --
	(152.35,115.50) --
	(150.65,116.12) --
	(149.02,116.75) --
	(147.45,117.37) --
	(145.93,118.00) --
	(144.46,118.62) --
	(143.04,119.25) --
	(141.67,119.87) --
	(140.35,120.50) --
	(139.07,121.12) --
	(137.82,121.75) --
	(136.62,122.37) --
	(135.45,123.00) --
	(134.32,123.62) --
	(133.23,124.25) --
	(132.16,124.87) --
	(131.13,125.50) --
	(130.12,126.12) --
	(129.14,126.75) --
	(128.19,127.37) --
	(127.27,128.00) --
	(126.37,128.62) --
	(125.50,129.25) --
	(124.64,129.87) --
	(123.81,130.50) --
	(123.00,131.12) --
	(122.22,131.75) --
	(121.45,132.37) --
	(120.70,133.00) --
	(119.97,133.62) --
	(119.25,134.25) --
	(118.55,134.87) --
	(117.87,135.50) --
	(117.21,136.12) --
	(116.56,136.75) --
	(115.92,137.37) --
	(115.30,138.00) --
	(114.70,138.62) --
	(114.10,139.24) --
	(113.52,139.87) --
	(112.95,140.49) --
	(112.40,141.12) --
	(111.85,141.74) --
	(111.32,142.37) --
	(110.80,142.99) --
	(110.29,143.62) --
	(109.78,144.24) --
	(109.29,144.87) --
	(108.81,145.49) --
	(108.34,146.12) --
	(107.88,146.74) --
	(107.42,147.37) --
	(106.98,147.99) --
	(106.54,148.62) --
	(106.11,149.24) --
	(105.69,149.87) --
	(105.28,150.49) --
	(104.87,151.12) --
	(104.47,151.74) --
	(104.08,152.37) --
	(103.70,152.99) --
	(103.32,153.62) --
	(102.95,154.24) --
	(102.58,154.87) --
	(102.22,155.49) --
	(101.87,156.12) --
	(101.52,156.74) --
	(101.18,157.37) --
	(100.85,157.99) --
	(100.52,158.62) --
	(100.19,159.24) --
	( 99.87,159.87) --
	( 99.56,160.49) --
	( 99.25,161.12) --
	( 98.95,161.74) --
	( 98.65,162.37) --
	( 98.35,162.99) --
	( 98.06,163.62) --
	( 97.78,164.24) --
	( 97.49,164.87) --
	( 97.22,165.49) --
	( 96.94,166.12) --
	( 96.68,166.74) --
	( 96.41,167.37) --
	( 96.15,167.99) --
	( 95.89,168.62) --
	( 95.64,169.24) --
	( 95.39,169.87) --
	( 95.14,170.49) --
	( 94.90,171.12) --
	( 94.66,171.74) --
	( 94.42,172.37) --
	( 94.19,172.99) --
	( 93.96,173.62) --
	( 93.73,174.24) --
	( 93.51,174.86) --
	( 93.29,175.49) --
	( 93.07,176.11) --
	( 92.85,176.74) --
	( 92.64,177.36) --
	( 92.43,177.99) --
	( 92.22,178.61) --
	( 92.02,179.24) --
	( 91.82,179.86) --
	( 91.62,180.49) --
	( 91.42,181.11) --
	( 91.23,181.74) --
	( 91.04,182.36) --
	( 90.85,182.99) --
	( 90.66,183.61) --
	( 90.48,184.24) --
	( 90.29,184.86) --
	( 90.11,185.49) --
	( 89.94,186.11) --
	( 89.76,186.74) --
	( 89.59,187.36) --
	( 89.42,187.99) --
	( 89.25,188.61) --
	( 89.08,189.24) --
	( 88.91,189.86) --
	( 88.75,190.49) --
	( 88.59,191.11) --
	( 88.43,191.74) --
	( 88.27,192.36) --
	( 88.11,192.99) --
	( 87.96,193.61) --
	( 87.81,194.24) --
	( 87.65,194.86) --
	( 87.50,195.49) --
	( 87.36,196.11) --
	( 87.21,196.74) --
	( 87.07,197.36) --
	( 86.92,197.99) --
	( 86.78,198.61) --
	( 86.64,199.24) --
	( 86.50,199.86) --
	( 86.36,200.49) --
	( 86.23,201.11) --
	( 86.09,201.74) --
	( 85.96,202.36) --
	( 85.83,202.99) --
	( 85.70,203.61) --
	( 85.57,204.24) --
	( 85.44,204.86) --
	( 85.32,205.49) --
	( 85.19,206.11) --
	( 85.07,206.74) --
	( 84.95,207.36) --
	( 84.82,207.99) --
	( 84.70,208.61) --
	( 84.59,209.24) --
	( 84.47,209.86) --
	( 84.35,210.49) --
	( 84.24,211.11) --
	( 84.12,211.73) --
	( 84.01,212.36) --
	( 83.90,212.98) --
	( 83.79,213.61) --
	( 83.68,214.23) --
	( 83.57,214.86) --
	( 83.46,215.48);
\definecolor{drawColor}{RGB}{0,0,0}
\definecolor{fillColor}{RGB}{255,255,255}

\path[draw=drawColor,line width= 0.4pt,line join=round,line cap=round,fill=fillColor] (402.23,106.90) rectangle (406.21,110.89);

\path[draw=drawColor,line width= 0.4pt,line join=round,line cap=round,fill=fillColor] (230.39,106.90) rectangle (234.38,110.89);

\path[draw=drawColor,line width= 0.4pt,line join=round,line cap=round,fill=fillColor] (173.11,106.90) rectangle (177.10,110.89);

\path[draw=drawColor,line width= 0.4pt,line join=round,line cap=round,fill=fillColor] (144.47,115.78) rectangle (148.46,119.77);

\path[draw=drawColor,line width= 0.4pt,line join=round,line cap=round,fill=fillColor] (127.29,124.66) rectangle (131.28,128.65);

\path[draw=drawColor,line width= 0.4pt,line join=round,line cap=round,fill=fillColor] (115.83,133.55) rectangle (119.82,137.53);

\path[draw=drawColor,line width= 0.4pt,line join=round,line cap=round,fill=fillColor] (107.65,142.43) rectangle (111.64,146.42);

\path[draw=drawColor,line width= 0.4pt,line join=round,line cap=round,fill=fillColor] (101.51,151.31) rectangle (105.50,155.30);

\path[draw=drawColor,line width= 0.4pt,line join=round,line cap=round,fill=fillColor] ( 96.74,160.19) rectangle (100.73,164.18);

\path[draw=drawColor,line width= 0.4pt,line join=round,line cap=round,fill=fillColor] ( 92.92,169.08) rectangle ( 96.91,173.06);

\path[draw=drawColor,line width= 0.4pt,line join=round,line cap=round,fill=fillColor] ( 89.80,177.96) rectangle ( 93.78,181.95);

\path[draw=drawColor,line width= 0.4pt,line join=round,line cap=round,fill=fillColor] ( 87.19,186.84) rectangle ( 91.18,190.83);

\path[draw=drawColor,line width= 0.4pt,line join=round,line cap=round,fill=fillColor] ( 84.99,195.73) rectangle ( 88.98,199.71);

\path[draw=drawColor,line width= 0.4pt,line join=round,line cap=round,fill=fillColor] ( 83.10,204.61) rectangle ( 87.09,208.60);

\path[draw=drawColor,line width= 0.4pt,line join=round,line cap=round,fill=fillColor] ( 81.46,213.49) rectangle ( 85.45,217.48);
\end{scope}
\begin{scope}
\path[clip] (  0.00,  0.00) rectangle (419.17,289.08);
\definecolor{drawColor}{RGB}{0,0,0}

\node[text=drawColor,anchor=base,inner sep=0pt, outer sep=0pt, scale=  1.00] at (232.38,  3.60) {Nutzungsh"aufigkeit $h$ [Nutzungseinheiten/Jahr]};

\node[text=drawColor,rotate= 90.00,anchor=base,inner sep=0pt, outer sep=0pt, scale=  1.00] at (  8.40,135.54) {MIPS [kg/Service-Einheit]};
\end{scope}
\begin{scope}
\path[clip] ( 46.80, 49.20) rectangle (417.97,221.88);
\definecolor{drawColor}{RGB}{0,0,0}

\path[draw=drawColor,line width= 0.4pt,dash pattern=on 1pt off 3pt ,line join=round,line cap=round] (175.10, 49.20) -- (175.10,221.88);

\path[draw=drawColor,line width= 0.4pt,dash pattern=on 1pt off 3pt ,line join=round,line cap=round] (232.38, 49.20) -- (232.38,221.88);
\end{scope}
\begin{scope}
\path[clip] (  0.00,  0.00) rectangle (419.17,289.08);
\definecolor{drawColor}{RGB}{0,0,0}

\node[text=drawColor,anchor=base,inner sep=0pt, outer sep=0pt, scale=  1.20] at (232.38,275.88) {\bfseries Materialintensit"at pro Service-Einheit MIPS$(h)$};
\end{scope}
\begin{scope}
\path[clip] (  0.00,  0.00) rectangle (419.17,289.08);
\definecolor{drawColor}{RGB}{0,0,0}

\path[draw=drawColor,line width= 0.4pt,line join=round,line cap=round] ( 60.55, 49.20) -- (404.22, 49.20);

\path[draw=drawColor,line width= 0.4pt,line join=round,line cap=round] ( 60.55, 49.20) -- ( 60.55, 43.20);

\path[draw=drawColor,line width= 0.4pt,line join=round,line cap=round] (129.28, 49.20) -- (129.28, 43.20);

\path[draw=drawColor,line width= 0.4pt,line join=round,line cap=round] (175.10, 49.20) -- (175.10, 43.20);

\path[draw=drawColor,line width= 0.4pt,line join=round,line cap=round] (198.02, 49.20) -- (198.02, 43.20);

\path[draw=drawColor,line width= 0.4pt,line join=round,line cap=round] (232.38, 49.20) -- (232.38, 43.20);

\path[draw=drawColor,line width= 0.4pt,line join=round,line cap=round] (266.75, 49.20) -- (266.75, 43.20);

\path[draw=drawColor,line width= 0.4pt,line join=round,line cap=round] (335.48, 49.20) -- (335.48, 43.20);

\path[draw=drawColor,line width= 0.4pt,line join=round,line cap=round] (404.22, 49.20) -- (404.22, 43.20);

\node[text=drawColor,anchor=base,inner sep=0pt, outer sep=0pt, scale=  1.00] at ( 60.55, 27.60) {0};

\node[text=drawColor,anchor=base,inner sep=0pt, outer sep=0pt, scale=  1.00] at (129.28, 27.60) {200};

\node[text=drawColor,anchor=base,inner sep=0pt, outer sep=0pt, scale=  1.00] at (175.10, 27.60) {$h^*$};

\node[text=drawColor,anchor=base,inner sep=0pt, outer sep=0pt, scale=  1.00] at (198.02, 27.60) {400};

\node[text=drawColor,anchor=base,inner sep=0pt, outer sep=0pt, scale=  1.00] at (232.38, 27.60) {$h_\text{max}$};

\node[text=drawColor,anchor=base,inner sep=0pt, outer sep=0pt, scale=  1.00] at (266.75, 27.60) {600};

\node[text=drawColor,anchor=base,inner sep=0pt, outer sep=0pt, scale=  1.00] at (335.48, 27.60) {800};

\node[text=drawColor,anchor=base,inner sep=0pt, outer sep=0pt, scale=  1.00] at (404.22, 27.60) {1000};

\path[draw=drawColor,line width= 0.4pt,line join=round,line cap=round] ( 46.80, 55.60) -- ( 46.80,215.48);

\path[draw=drawColor,line width= 0.4pt,line join=round,line cap=round] ( 46.80, 55.60) -- ( 40.80, 55.60);

\path[draw=drawColor,line width= 0.4pt,line join=round,line cap=round] ( 46.80, 82.24) -- ( 40.80, 82.24);

\path[draw=drawColor,line width= 0.4pt,line join=round,line cap=round] ( 46.80,108.89) -- ( 40.80,108.89);

\path[draw=drawColor,line width= 0.4pt,line join=round,line cap=round] ( 46.80,108.89) -- ( 40.80,108.89);

\path[draw=drawColor,line width= 0.4pt,line join=round,line cap=round] ( 46.80,108.89) -- ( 40.80,108.89);

\path[draw=drawColor,line width= 0.4pt,line join=round,line cap=round] ( 46.80,117.77) -- ( 40.80,117.77);

\path[draw=drawColor,line width= 0.4pt,line join=round,line cap=round] ( 46.80,126.66) -- ( 40.80,126.66);

\path[draw=drawColor,line width= 0.4pt,line join=round,line cap=round] ( 46.80,135.54) -- ( 40.80,135.54);

\path[draw=drawColor,line width= 0.4pt,line join=round,line cap=round] ( 46.80,144.42) -- ( 40.80,144.42);

\path[draw=drawColor,line width= 0.4pt,line join=round,line cap=round] ( 46.80,153.31) -- ( 40.80,153.31);

\path[draw=drawColor,line width= 0.4pt,line join=round,line cap=round] ( 46.80,162.19) -- ( 40.80,162.19);

\path[draw=drawColor,line width= 0.4pt,line join=round,line cap=round] ( 46.80,171.07) -- ( 40.80,171.07);

\path[draw=drawColor,line width= 0.4pt,line join=round,line cap=round] ( 46.80,179.95) -- ( 40.80,179.95);

\path[draw=drawColor,line width= 0.4pt,line join=round,line cap=round] ( 46.80,188.84) -- ( 40.80,188.84);

\path[draw=drawColor,line width= 0.4pt,line join=round,line cap=round] ( 46.80,197.72) -- ( 40.80,197.72);

\path[draw=drawColor,line width= 0.4pt,line join=round,line cap=round] ( 46.80,206.60) -- ( 40.80,206.60);

\path[draw=drawColor,line width= 0.4pt,line join=round,line cap=round] ( 46.80,215.48) -- ( 40.80,215.48);

\node[text=drawColor,rotate= 90.00,anchor=base,inner sep=0pt, outer sep=0pt, scale=  1.00] at ( 32.40, 55.60) {0};

\node[text=drawColor,rotate= 90.00,anchor=base,inner sep=0pt, outer sep=0pt, scale=  1.00] at ( 32.40, 82.24) {1};

\node[text=drawColor,rotate= 90.00,anchor=base,inner sep=0pt, outer sep=0pt, scale=  1.00] at ( 32.40,108.89) {2};

\node[text=drawColor,rotate= 90.00,anchor=base,inner sep=0pt, outer sep=0pt, scale=  1.00] at ( 32.40,135.54) {3};

\node[text=drawColor,rotate= 90.00,anchor=base,inner sep=0pt, outer sep=0pt, scale=  1.00] at ( 32.40,162.19) {4};

\node[text=drawColor,rotate= 90.00,anchor=base,inner sep=0pt, outer sep=0pt, scale=  1.00] at ( 32.40,188.84) {5};

\node[text=drawColor,rotate= 90.00,anchor=base,inner sep=0pt, outer sep=0pt, scale=  1.00] at ( 32.40,215.48) {6};

\path[draw=drawColor,line width= 0.4pt,line join=round,line cap=round] ( 83.46,221.88) -- (404.22,221.88);

\path[draw=drawColor,line width= 0.4pt,line join=round,line cap=round] ( 83.46,221.88) -- ( 83.46,227.88);

\path[draw=drawColor,line width= 0.4pt,line join=round,line cap=round] ( 85.09,221.88) -- ( 85.09,227.88);

\path[draw=drawColor,line width= 0.4pt,line join=round,line cap=round] ( 86.98,221.88) -- ( 86.98,227.88);

\path[draw=drawColor,line width= 0.4pt,line join=round,line cap=round] ( 89.19,221.88) -- ( 89.19,227.88);

\path[draw=drawColor,line width= 0.4pt,line join=round,line cap=round] ( 91.79,221.88) -- ( 91.79,227.88);

\path[draw=drawColor,line width= 0.4pt,line join=round,line cap=round] ( 94.91,221.88) -- ( 94.91,227.88);

\path[draw=drawColor,line width= 0.4pt,line join=round,line cap=round] ( 98.73,221.88) -- ( 98.73,227.88);

\path[draw=drawColor,line width= 0.4pt,line join=round,line cap=round] (103.51,221.88) -- (103.51,227.88);

\path[draw=drawColor,line width= 0.4pt,line join=round,line cap=round] (109.64,221.88) -- (109.64,227.88);

\path[draw=drawColor,line width= 0.4pt,line join=round,line cap=round] (117.83,221.88) -- (117.83,227.88);

\path[draw=drawColor,line width= 0.4pt,line join=round,line cap=round] (129.28,221.88) -- (129.28,227.88);

\path[draw=drawColor,line width= 0.4pt,line join=round,line cap=round] (146.46,221.88) -- (146.46,227.88);

\path[draw=drawColor,line width= 0.4pt,line join=round,line cap=round] (175.10,221.88) -- (175.10,227.88);

\path[draw=drawColor,line width= 0.4pt,line join=round,line cap=round] (232.38,221.88) -- (232.38,227.88);

\path[draw=drawColor,line width= 0.4pt,line join=round,line cap=round] (404.22,221.88) -- (404.22,227.88);

\node[text=drawColor,anchor=base,inner sep=0pt, outer sep=0pt, scale=  1.00] at ( 83.46,236.28) {15};

\node[text=drawColor,anchor=base,inner sep=0pt, outer sep=0pt, scale=  1.00] at (103.51,236.28) {8};

\node[text=drawColor,anchor=base,inner sep=0pt, outer sep=0pt, scale=  1.00] at (117.83,236.28) {6};

\node[text=drawColor,anchor=base,inner sep=0pt, outer sep=0pt, scale=  1.00] at (146.46,236.28) {4};

\node[text=drawColor,anchor=base,inner sep=0pt, outer sep=0pt, scale=  1.00] at (175.10,236.28) {3};

\node[text=drawColor,anchor=base,inner sep=0pt, outer sep=0pt, scale=  1.00] at (232.38,236.28) {2};

\node[text=drawColor,anchor=base,inner sep=0pt, outer sep=0pt, scale=  1.00] at (404.22,236.28) {1};

\path[draw=drawColor,line width= 0.4pt,line join=round,line cap=round] ( 46.80, 49.20) --
	(417.97, 49.20) --
	(417.97,221.88) --
	( 46.80,221.88) --
	( 46.80, 49.20);

\node[text=drawColor,anchor=base,inner sep=0pt, outer sep=0pt, scale=  1.00] at (232.38,254.28) {parallele Produktanzahl $p$};
\end{scope}
\end{tikzpicture}

			}
		\end{center}
	\end{frame}
	
	\begin{frame}{Analyse}		
		\begin{itemize}
			\item<1-> Wie stark fällt die MIPS mit steigendem $h$ im Bereich $h$ < $h^*$?
			\item<2-> Korrespondenz von $h$ und $p$: \\ $h_1 \rightarrow h_2 = h_1 + \Delta h \qquad \Leftrightarrow \qquad p_1 \rightarrow p_2 = p_1 -1$
			\item<3-> Änderung der MIPS: \\ $\Delta \text{MIPS} = \text{MIPS}(h_2) - \text{MIPS}(h_1) = - \frac{q \cdot i_P}{S_D}$
		\end{itemize}
		\begin{center}
			\resizebox{0.5\linewidth}{!}{
				% Created by tikzDevice version 0.8.1 on 2015-04-03 10:57:13
% !TEX encoding = UTF-8 Unicode
\begin{tikzpicture}[x=1pt,y=1pt]
\definecolor{fillColor}{RGB}{255,255,255}
\path[use as bounding box,fill=fillColor,fill opacity=0.00] (0,0) rectangle (419.17,289.08);
\begin{scope}
\path[clip] ( 46.80, 49.20) rectangle (417.97,221.88);
\definecolor{drawColor}{RGB}{190,190,190}

\path[draw=drawColor,line width= 0.4pt,line join=round,line cap=round] (404.22,108.89) --
	(381.63,108.89) --
	(361.83,108.89) --
	(344.33,108.89) --
	(328.75,108.89) --
	(314.79,108.89) --
	(302.21,108.89) --
	(290.82,108.89) --
	(280.45,108.89) --
	(270.98,108.89) --
	(262.29,108.89) --
	(254.29,108.89) --
	(246.90,108.89) --
	(240.05,108.89) --
	(233.69,108.89) --
	(227.76,108.89) --
	(222.23,108.89) --
	(217.05,108.89) --
	(212.19,108.89) --
	(207.62,108.89) --
	(203.33,108.89) --
	(199.27,108.89) --
	(195.44,108.89) --
	(191.82,108.89) --
	(188.38,108.89) --
	(185.12,108.89) --
	(182.02,108.89) --
	(179.08,108.89) --
	(176.27,108.89) --
	(173.59,109.25) --
	(171.03,109.87) --
	(168.59,110.50) --
	(166.25,111.12) --
	(164.01,111.75) --
	(161.87,112.37) --
	(159.81,113.00) --
	(157.83,113.62) --
	(155.93,114.25) --
	(154.10,114.87) --
	(152.35,115.50) --
	(150.65,116.12) --
	(149.02,116.75) --
	(147.45,117.37) --
	(145.93,118.00) --
	(144.46,118.62) --
	(143.04,119.25) --
	(141.67,119.87) --
	(140.35,120.50) --
	(139.07,121.12) --
	(137.82,121.75) --
	(136.62,122.37) --
	(135.45,123.00) --
	(134.32,123.62) --
	(133.23,124.25) --
	(132.16,124.87) --
	(131.13,125.50) --
	(130.12,126.12) --
	(129.14,126.75) --
	(128.19,127.37) --
	(127.27,128.00) --
	(126.37,128.62) --
	(125.50,129.25) --
	(124.64,129.87) --
	(123.81,130.50) --
	(123.00,131.12) --
	(122.22,131.75) --
	(121.45,132.37) --
	(120.70,133.00) --
	(119.97,133.62) --
	(119.25,134.25) --
	(118.55,134.87) --
	(117.87,135.50) --
	(117.21,136.12) --
	(116.56,136.75) --
	(115.92,137.37) --
	(115.30,138.00) --
	(114.70,138.62) --
	(114.10,139.24) --
	(113.52,139.87) --
	(112.95,140.49) --
	(112.40,141.12) --
	(111.85,141.74) --
	(111.32,142.37) --
	(110.80,142.99) --
	(110.29,143.62) --
	(109.78,144.24) --
	(109.29,144.87) --
	(108.81,145.49) --
	(108.34,146.12) --
	(107.88,146.74) --
	(107.42,147.37) --
	(106.98,147.99) --
	(106.54,148.62) --
	(106.11,149.24) --
	(105.69,149.87) --
	(105.28,150.49) --
	(104.87,151.12) --
	(104.47,151.74) --
	(104.08,152.37) --
	(103.70,152.99) --
	(103.32,153.62) --
	(102.95,154.24) --
	(102.58,154.87) --
	(102.22,155.49) --
	(101.87,156.12) --
	(101.52,156.74) --
	(101.18,157.37) --
	(100.85,157.99) --
	(100.52,158.62) --
	(100.19,159.24) --
	( 99.87,159.87) --
	( 99.56,160.49) --
	( 99.25,161.12) --
	( 98.95,161.74) --
	( 98.65,162.37) --
	( 98.35,162.99) --
	( 98.06,163.62) --
	( 97.78,164.24) --
	( 97.49,164.87) --
	( 97.22,165.49) --
	( 96.94,166.12) --
	( 96.68,166.74) --
	( 96.41,167.37) --
	( 96.15,167.99) --
	( 95.89,168.62) --
	( 95.64,169.24) --
	( 95.39,169.87) --
	( 95.14,170.49) --
	( 94.90,171.12) --
	( 94.66,171.74) --
	( 94.42,172.37) --
	( 94.19,172.99) --
	( 93.96,173.62) --
	( 93.73,174.24) --
	( 93.51,174.86) --
	( 93.29,175.49) --
	( 93.07,176.11) --
	( 92.85,176.74) --
	( 92.64,177.36) --
	( 92.43,177.99) --
	( 92.22,178.61) --
	( 92.02,179.24) --
	( 91.82,179.86) --
	( 91.62,180.49) --
	( 91.42,181.11) --
	( 91.23,181.74) --
	( 91.04,182.36) --
	( 90.85,182.99) --
	( 90.66,183.61) --
	( 90.48,184.24) --
	( 90.29,184.86) --
	( 90.11,185.49) --
	( 89.94,186.11) --
	( 89.76,186.74) --
	( 89.59,187.36) --
	( 89.42,187.99) --
	( 89.25,188.61) --
	( 89.08,189.24) --
	( 88.91,189.86) --
	( 88.75,190.49) --
	( 88.59,191.11) --
	( 88.43,191.74) --
	( 88.27,192.36) --
	( 88.11,192.99) --
	( 87.96,193.61) --
	( 87.81,194.24) --
	( 87.65,194.86) --
	( 87.50,195.49) --
	( 87.36,196.11) --
	( 87.21,196.74) --
	( 87.07,197.36) --
	( 86.92,197.99) --
	( 86.78,198.61) --
	( 86.64,199.24) --
	( 86.50,199.86) --
	( 86.36,200.49) --
	( 86.23,201.11) --
	( 86.09,201.74) --
	( 85.96,202.36) --
	( 85.83,202.99) --
	( 85.70,203.61) --
	( 85.57,204.24) --
	( 85.44,204.86) --
	( 85.32,205.49) --
	( 85.19,206.11) --
	( 85.07,206.74) --
	( 84.95,207.36) --
	( 84.82,207.99) --
	( 84.70,208.61) --
	( 84.59,209.24) --
	( 84.47,209.86) --
	( 84.35,210.49) --
	( 84.24,211.11) --
	( 84.12,211.73) --
	( 84.01,212.36) --
	( 83.90,212.98) --
	( 83.79,213.61) --
	( 83.68,214.23) --
	( 83.57,214.86) --
	( 83.46,215.48);
\definecolor{drawColor}{RGB}{0,0,0}
\definecolor{fillColor}{RGB}{255,255,255}

\path[draw=drawColor,line width= 0.4pt,line join=round,line cap=round,fill=fillColor] (402.23,106.90) rectangle (406.21,110.89);

\path[draw=drawColor,line width= 0.4pt,line join=round,line cap=round,fill=fillColor] (230.39,106.90) rectangle (234.38,110.89);

\path[draw=drawColor,line width= 0.4pt,line join=round,line cap=round,fill=fillColor] (173.11,106.90) rectangle (177.10,110.89);

\path[draw=drawColor,line width= 0.4pt,line join=round,line cap=round,fill=fillColor] (144.47,115.78) rectangle (148.46,119.77);

\path[draw=drawColor,line width= 0.4pt,line join=round,line cap=round,fill=fillColor] (127.29,124.66) rectangle (131.28,128.65);

\path[draw=drawColor,line width= 0.4pt,line join=round,line cap=round,fill=fillColor] (115.83,133.55) rectangle (119.82,137.53);

\path[draw=drawColor,line width= 0.4pt,line join=round,line cap=round,fill=fillColor] (107.65,142.43) rectangle (111.64,146.42);

\path[draw=drawColor,line width= 0.4pt,line join=round,line cap=round,fill=fillColor] (101.51,151.31) rectangle (105.50,155.30);

\path[draw=drawColor,line width= 0.4pt,line join=round,line cap=round,fill=fillColor] ( 96.74,160.19) rectangle (100.73,164.18);

\path[draw=drawColor,line width= 0.4pt,line join=round,line cap=round,fill=fillColor] ( 92.92,169.08) rectangle ( 96.91,173.06);

\path[draw=drawColor,line width= 0.4pt,line join=round,line cap=round,fill=fillColor] ( 89.80,177.96) rectangle ( 93.78,181.95);

\path[draw=drawColor,line width= 0.4pt,line join=round,line cap=round,fill=fillColor] ( 87.19,186.84) rectangle ( 91.18,190.83);

\path[draw=drawColor,line width= 0.4pt,line join=round,line cap=round,fill=fillColor] ( 84.99,195.73) rectangle ( 88.98,199.71);

\path[draw=drawColor,line width= 0.4pt,line join=round,line cap=round,fill=fillColor] ( 83.10,204.61) rectangle ( 87.09,208.60);

\path[draw=drawColor,line width= 0.4pt,line join=round,line cap=round,fill=fillColor] ( 81.46,213.49) rectangle ( 85.45,217.48);
\end{scope}
\begin{scope}
\path[clip] (  0.00,  0.00) rectangle (419.17,289.08);
\definecolor{drawColor}{RGB}{0,0,0}

\node[text=drawColor,anchor=base,inner sep=0pt, outer sep=0pt, scale=  1.00] at (232.38,  3.60) {Nutzungsh"aufigkeit $h$ [Nutzungseinheiten/Jahr]};

\node[text=drawColor,rotate= 90.00,anchor=base,inner sep=0pt, outer sep=0pt, scale=  1.00] at (  8.40,135.54) {MIPS [kg/Service-Einheit]};
\end{scope}
\begin{scope}
\path[clip] ( 46.80, 49.20) rectangle (417.97,221.88);
\definecolor{drawColor}{RGB}{0,0,0}

\path[draw=drawColor,line width= 0.4pt,dash pattern=on 1pt off 3pt ,line join=round,line cap=round] (175.10, 49.20) -- (175.10,221.88);

\path[draw=drawColor,line width= 0.4pt,dash pattern=on 1pt off 3pt ,line join=round,line cap=round] (232.38, 49.20) -- (232.38,221.88);
\end{scope}
\begin{scope}
\path[clip] (  0.00,  0.00) rectangle (419.17,289.08);
\definecolor{drawColor}{RGB}{0,0,0}

\node[text=drawColor,anchor=base,inner sep=0pt, outer sep=0pt, scale=  1.20] at (232.38,275.88) {\bfseries Materialintensit"at pro Service-Einheit MIPS$(h)$};
\end{scope}
\begin{scope}
\path[clip] (  0.00,  0.00) rectangle (419.17,289.08);
\definecolor{drawColor}{RGB}{0,0,0}

\path[draw=drawColor,line width= 0.4pt,line join=round,line cap=round] ( 60.55, 49.20) -- (404.22, 49.20);

\path[draw=drawColor,line width= 0.4pt,line join=round,line cap=round] ( 60.55, 49.20) -- ( 60.55, 43.20);

\path[draw=drawColor,line width= 0.4pt,line join=round,line cap=round] (129.28, 49.20) -- (129.28, 43.20);

\path[draw=drawColor,line width= 0.4pt,line join=round,line cap=round] (175.10, 49.20) -- (175.10, 43.20);

\path[draw=drawColor,line width= 0.4pt,line join=round,line cap=round] (198.02, 49.20) -- (198.02, 43.20);

\path[draw=drawColor,line width= 0.4pt,line join=round,line cap=round] (232.38, 49.20) -- (232.38, 43.20);

\path[draw=drawColor,line width= 0.4pt,line join=round,line cap=round] (266.75, 49.20) -- (266.75, 43.20);

\path[draw=drawColor,line width= 0.4pt,line join=round,line cap=round] (335.48, 49.20) -- (335.48, 43.20);

\path[draw=drawColor,line width= 0.4pt,line join=round,line cap=round] (404.22, 49.20) -- (404.22, 43.20);

\node[text=drawColor,anchor=base,inner sep=0pt, outer sep=0pt, scale=  1.00] at ( 60.55, 27.60) {0};

\node[text=drawColor,anchor=base,inner sep=0pt, outer sep=0pt, scale=  1.00] at (129.28, 27.60) {200};

\node[text=drawColor,anchor=base,inner sep=0pt, outer sep=0pt, scale=  1.00] at (175.10, 27.60) {$h^*$};

\node[text=drawColor,anchor=base,inner sep=0pt, outer sep=0pt, scale=  1.00] at (198.02, 27.60) {400};

\node[text=drawColor,anchor=base,inner sep=0pt, outer sep=0pt, scale=  1.00] at (232.38, 27.60) {$h_\text{max}$};

\node[text=drawColor,anchor=base,inner sep=0pt, outer sep=0pt, scale=  1.00] at (266.75, 27.60) {600};

\node[text=drawColor,anchor=base,inner sep=0pt, outer sep=0pt, scale=  1.00] at (335.48, 27.60) {800};

\node[text=drawColor,anchor=base,inner sep=0pt, outer sep=0pt, scale=  1.00] at (404.22, 27.60) {1000};

\path[draw=drawColor,line width= 0.4pt,line join=round,line cap=round] ( 46.80, 55.60) -- ( 46.80,215.48);

\path[draw=drawColor,line width= 0.4pt,line join=round,line cap=round] ( 46.80, 55.60) -- ( 40.80, 55.60);

\path[draw=drawColor,line width= 0.4pt,line join=round,line cap=round] ( 46.80, 82.24) -- ( 40.80, 82.24);

\path[draw=drawColor,line width= 0.4pt,line join=round,line cap=round] ( 46.80,108.89) -- ( 40.80,108.89);

\path[draw=drawColor,line width= 0.4pt,line join=round,line cap=round] ( 46.80,108.89) -- ( 40.80,108.89);

\path[draw=drawColor,line width= 0.4pt,line join=round,line cap=round] ( 46.80,108.89) -- ( 40.80,108.89);

\path[draw=drawColor,line width= 0.4pt,line join=round,line cap=round] ( 46.80,117.77) -- ( 40.80,117.77);

\path[draw=drawColor,line width= 0.4pt,line join=round,line cap=round] ( 46.80,126.66) -- ( 40.80,126.66);

\path[draw=drawColor,line width= 0.4pt,line join=round,line cap=round] ( 46.80,135.54) -- ( 40.80,135.54);

\path[draw=drawColor,line width= 0.4pt,line join=round,line cap=round] ( 46.80,144.42) -- ( 40.80,144.42);

\path[draw=drawColor,line width= 0.4pt,line join=round,line cap=round] ( 46.80,153.31) -- ( 40.80,153.31);

\path[draw=drawColor,line width= 0.4pt,line join=round,line cap=round] ( 46.80,162.19) -- ( 40.80,162.19);

\path[draw=drawColor,line width= 0.4pt,line join=round,line cap=round] ( 46.80,171.07) -- ( 40.80,171.07);

\path[draw=drawColor,line width= 0.4pt,line join=round,line cap=round] ( 46.80,179.95) -- ( 40.80,179.95);

\path[draw=drawColor,line width= 0.4pt,line join=round,line cap=round] ( 46.80,188.84) -- ( 40.80,188.84);

\path[draw=drawColor,line width= 0.4pt,line join=round,line cap=round] ( 46.80,197.72) -- ( 40.80,197.72);

\path[draw=drawColor,line width= 0.4pt,line join=round,line cap=round] ( 46.80,206.60) -- ( 40.80,206.60);

\path[draw=drawColor,line width= 0.4pt,line join=round,line cap=round] ( 46.80,215.48) -- ( 40.80,215.48);

\node[text=drawColor,rotate= 90.00,anchor=base,inner sep=0pt, outer sep=0pt, scale=  1.00] at ( 32.40, 55.60) {0};

\node[text=drawColor,rotate= 90.00,anchor=base,inner sep=0pt, outer sep=0pt, scale=  1.00] at ( 32.40, 82.24) {1};

\node[text=drawColor,rotate= 90.00,anchor=base,inner sep=0pt, outer sep=0pt, scale=  1.00] at ( 32.40,108.89) {2};

\node[text=drawColor,rotate= 90.00,anchor=base,inner sep=0pt, outer sep=0pt, scale=  1.00] at ( 32.40,135.54) {3};

\node[text=drawColor,rotate= 90.00,anchor=base,inner sep=0pt, outer sep=0pt, scale=  1.00] at ( 32.40,162.19) {4};

\node[text=drawColor,rotate= 90.00,anchor=base,inner sep=0pt, outer sep=0pt, scale=  1.00] at ( 32.40,188.84) {5};

\node[text=drawColor,rotate= 90.00,anchor=base,inner sep=0pt, outer sep=0pt, scale=  1.00] at ( 32.40,215.48) {6};

\path[draw=drawColor,line width= 0.4pt,line join=round,line cap=round] ( 83.46,221.88) -- (404.22,221.88);

\path[draw=drawColor,line width= 0.4pt,line join=round,line cap=round] ( 83.46,221.88) -- ( 83.46,227.88);

\path[draw=drawColor,line width= 0.4pt,line join=round,line cap=round] ( 85.09,221.88) -- ( 85.09,227.88);

\path[draw=drawColor,line width= 0.4pt,line join=round,line cap=round] ( 86.98,221.88) -- ( 86.98,227.88);

\path[draw=drawColor,line width= 0.4pt,line join=round,line cap=round] ( 89.19,221.88) -- ( 89.19,227.88);

\path[draw=drawColor,line width= 0.4pt,line join=round,line cap=round] ( 91.79,221.88) -- ( 91.79,227.88);

\path[draw=drawColor,line width= 0.4pt,line join=round,line cap=round] ( 94.91,221.88) -- ( 94.91,227.88);

\path[draw=drawColor,line width= 0.4pt,line join=round,line cap=round] ( 98.73,221.88) -- ( 98.73,227.88);

\path[draw=drawColor,line width= 0.4pt,line join=round,line cap=round] (103.51,221.88) -- (103.51,227.88);

\path[draw=drawColor,line width= 0.4pt,line join=round,line cap=round] (109.64,221.88) -- (109.64,227.88);

\path[draw=drawColor,line width= 0.4pt,line join=round,line cap=round] (117.83,221.88) -- (117.83,227.88);

\path[draw=drawColor,line width= 0.4pt,line join=round,line cap=round] (129.28,221.88) -- (129.28,227.88);

\path[draw=drawColor,line width= 0.4pt,line join=round,line cap=round] (146.46,221.88) -- (146.46,227.88);

\path[draw=drawColor,line width= 0.4pt,line join=round,line cap=round] (175.10,221.88) -- (175.10,227.88);

\path[draw=drawColor,line width= 0.4pt,line join=round,line cap=round] (232.38,221.88) -- (232.38,227.88);

\path[draw=drawColor,line width= 0.4pt,line join=round,line cap=round] (404.22,221.88) -- (404.22,227.88);

\node[text=drawColor,anchor=base,inner sep=0pt, outer sep=0pt, scale=  1.00] at ( 83.46,236.28) {15};

\node[text=drawColor,anchor=base,inner sep=0pt, outer sep=0pt, scale=  1.00] at (103.51,236.28) {8};

\node[text=drawColor,anchor=base,inner sep=0pt, outer sep=0pt, scale=  1.00] at (117.83,236.28) {6};

\node[text=drawColor,anchor=base,inner sep=0pt, outer sep=0pt, scale=  1.00] at (146.46,236.28) {4};

\node[text=drawColor,anchor=base,inner sep=0pt, outer sep=0pt, scale=  1.00] at (175.10,236.28) {3};

\node[text=drawColor,anchor=base,inner sep=0pt, outer sep=0pt, scale=  1.00] at (232.38,236.28) {2};

\node[text=drawColor,anchor=base,inner sep=0pt, outer sep=0pt, scale=  1.00] at (404.22,236.28) {1};

\path[draw=drawColor,line width= 0.4pt,line join=round,line cap=round] ( 46.80, 49.20) --
	(417.97, 49.20) --
	(417.97,221.88) --
	( 46.80,221.88) --
	( 46.80, 49.20);

\node[text=drawColor,anchor=base,inner sep=0pt, outer sep=0pt, scale=  1.00] at (232.38,254.28) {parallele Produktanzahl $p$};
\end{scope}
\end{tikzpicture}

			}
		\end{center}
	\end{frame}
	
\subsection{Reparatur}
\frame{\subsectionpage}
	\begin{frame}{Modellbeschreibung}
        \begin{itemize}
            \item<1-> $t = \t{max}$
            \item<2-> $i_P = \tilde{i}_P + I_R$
            \item<3-> Einsetzen in: 
                    \begin{equation*} 
                        \text{MIPS}(h) =
                        \frac{I_{\text{fix}}^h}{S_D} + \left\{ 
                            \begin{array}{r@{\quad : \quad}l}
                                \frac{i_P}{h \cdot t_{\text{max}} \cdot A}&
                                h < h^*  \\[5pt]
                                \frac{i_P}{n_{\text{max}} \cdot A}&  \text{sonst}
                            \end{array} 
                        \right.
                    \end{equation*}
        \end{itemize}
		\begin{center}
            \begin{block}{Reparatur}<4->
                Teilmodell:
                \begin{equation*}
                    \MIPS[\t{max}, I_R] = \frac{\tilde{i}_P + I_R}{h\cdot \t{max}\cdot A} 
                    + \frac{\I{fix}^R}{S_D}
                \end{equation*}
                Mögliche Modellerweiterung:
                \begin{equation*}
                    I_R = I_R(\t{max}) = i_R \cdot
                    \left\lfloor\frac{\t{max}}{\t{tech}}\right\rfloor
                \end{equation*}
            \end{block}
		\end{center}
	\end{frame}
	
	\begin{frame}{Analyse - ohne Modellerweiterung}
		\begin{center}
            \begin{block}{Ansatz}<1->
                \begin{equation*}
                    \MIPS[\t{max}, I_R] >  \MIPS[\t{max} + \Delta \t{max}, I_R +
                    \Delta i_P]
                \end{equation*}
            \end{block}

            \begin{block}{Ergebnis}<2->
                \begin{equation*}
                    \frac{\Delta \t{max}}{\t{max}} >  \frac{\Delta i_P}{i_P}  
                \end{equation*}
            \end{block}
		\end{center}
	\end{frame}

	\begin{frame}{Analyse - mit Modellerweiterung}
          \centering
          \only<1>{
              \resizebox{\linewidth}{!}{
              % Created by tikzDevice version 0.8.1 on 2015-04-28 20:41:34
% !TEX encoding = UTF-8 Unicode
\begin{tikzpicture}[x=1pt,y=1pt]
\definecolor{fillColor}{RGB}{255,255,255}
\path[use as bounding box,fill=fillColor,fill opacity=0.00] (0,0) rectangle (325.21,231.26);
\begin{scope}
\path[clip] ( 49.20, 61.20) rectangle (300.01,182.06);
\definecolor{drawColor}{RGB}{0,0,0}

\path[draw=drawColor,line width= 0.4pt,line join=round,line cap=round] ( 58.49, 65.68) --
	( 58.72, 65.68) --
	( 58.95, 65.68) --
	( 59.19, 65.68) --
	( 59.42, 65.68) --
	( 59.65, 65.68) --
	( 59.88, 65.68) --
	( 60.12, 65.68) --
	( 60.35, 65.68) --
	( 60.58, 65.68) --
	( 60.81, 65.68) --
	( 61.05, 65.68) --
	( 61.28, 65.68) --
	( 61.51, 65.68) --
	( 61.74, 65.68) --
	( 61.98, 65.68) --
	( 62.21, 65.68) --
	( 62.44, 65.68) --
	( 62.67, 65.68) --
	( 62.91, 65.68) --
	( 63.14, 65.68) --
	( 63.37, 65.68) --
	( 63.60, 65.68) --
	( 63.84, 65.68) --
	( 64.07, 65.68) --
	( 64.30, 65.68) --
	( 64.53, 65.68) --
	( 64.77, 65.68) --
	( 65.00, 65.68) --
	( 65.23, 65.68) --
	( 65.46, 65.68) --
	( 65.70, 65.68) --
	( 65.93, 65.68) --
	( 66.16, 65.68) --
	( 66.39, 65.68) --
	( 66.63, 65.68) --
	( 66.86, 65.68) --
	( 67.09, 65.68) --
	( 67.32, 65.68) --
	( 67.56, 65.68) --
	( 67.79, 65.68) --
	( 68.02, 65.68) --
	( 68.25, 65.68) --
	( 68.49, 65.68) --
	( 68.72, 65.68) --
	( 68.95, 65.68) --
	( 69.18, 65.68) --
	( 69.42, 65.68) --
	( 69.65, 65.68) --
	( 69.88, 65.68) --
	( 70.11, 65.68) --
	( 70.35, 65.68) --
	( 70.58, 65.68) --
	( 70.81, 65.68) --
	( 71.04, 65.68) --
	( 71.28, 65.68) --
	( 71.51, 65.68) --
	( 71.74, 65.68) --
	( 71.97, 65.68) --
	( 72.21, 65.68) --
	( 72.44, 65.68) --
	( 72.67, 65.68) --
	( 72.90, 65.68) --
	( 73.13, 65.68) --
	( 73.37, 65.68) --
	( 73.60, 65.68) --
	( 73.83, 65.68) --
	( 74.06, 65.68) --
	( 74.30, 65.68) --
	( 74.53, 65.68) --
	( 74.76, 65.68) --
	( 74.99, 65.68) --
	( 75.23, 65.68) --
	( 75.46, 65.68) --
	( 75.69, 65.68) --
	( 75.92, 65.68) --
	( 76.16, 65.68) --
	( 76.39, 65.68) --
	( 76.62, 65.68) --
	( 76.85, 65.68) --
	( 77.09, 65.68) --
	( 77.32, 65.68) --
	( 77.55, 65.68) --
	( 77.78, 65.68) --
	( 78.02, 65.68) --
	( 78.25, 65.68) --
	( 78.48, 65.68) --
	( 78.71, 65.68) --
	( 78.95, 65.68) --
	( 79.18, 65.68) --
	( 79.41, 65.68) --
	( 79.64, 65.68) --
	( 79.88, 65.68) --
	( 80.11, 65.68) --
	( 80.34, 65.68) --
	( 80.57, 65.68) --
	( 80.81, 65.68) --
	( 81.04, 65.68) --
	( 81.27, 65.68) --
	( 81.50, 65.68) --
	( 81.74, 65.68) --
	( 81.97, 65.68) --
	( 82.20, 65.68) --
	( 82.43, 65.68) --
	( 82.67, 65.68) --
	( 82.90, 65.68) --
	( 83.13, 65.68) --
	( 83.36, 65.68) --
	( 83.60, 65.68) --
	( 83.83, 65.68) --
	( 84.06, 65.68) --
	( 84.29, 65.68) --
	( 84.53, 65.68) --
	( 84.76, 65.68) --
	( 84.99, 65.68) --
	( 85.22, 65.68) --
	( 85.46, 65.68) --
	( 85.69, 65.68) --
	( 85.92, 65.68) --
	( 86.15, 65.68) --
	( 86.39, 65.68) --
	( 86.62, 65.68) --
	( 86.85, 65.68) --
	( 87.08, 65.68) --
	( 87.32, 65.68) --
	( 87.55, 65.68) --
	( 87.78, 65.68) --
	( 88.01, 65.68) --
	( 88.25, 65.68) --
	( 88.48, 65.68) --
	( 88.71, 65.68) --
	( 88.94, 65.68) --
	( 89.18, 65.68) --
	( 89.41, 65.68) --
	( 89.64, 65.68) --
	( 89.87, 65.68) --
	( 90.11, 65.68) --
	( 90.34, 65.68) --
	( 90.57, 65.68) --
	( 90.80, 65.68) --
	( 91.04, 65.68) --
	( 91.27, 65.68) --
	( 91.50, 65.68) --
	( 91.73, 65.68) --
	( 91.96, 65.68) --
	( 92.20, 65.68) --
	( 92.43, 65.68) --
	( 92.66, 65.68) --
	( 92.89, 65.68) --
	( 93.13, 65.68) --
	( 93.36, 65.68) --
	( 93.59, 65.68) --
	( 93.82, 65.68) --
	( 94.06, 65.68) --
	( 94.29, 65.68) --
	( 94.52, 65.68) --
	( 94.75, 65.68) --
	( 94.99, 65.68) --
	( 95.22, 65.68) --
	( 95.45, 65.68) --
	( 95.68, 65.68) --
	( 95.92, 65.68) --
	( 96.15, 65.68) --
	( 96.38, 65.68) --
	( 96.61, 65.68) --
	( 96.85, 65.68) --
	( 97.08, 65.68) --
	( 97.31, 65.68) --
	( 97.54, 65.68) --
	( 97.78, 65.68) --
	( 98.01, 65.68) --
	( 98.24, 65.68) --
	( 98.47, 65.68) --
	( 98.71, 65.68) --
	( 98.94, 65.68) --
	( 99.17, 65.68) --
	( 99.40, 65.68) --
	( 99.64, 65.68) --
	( 99.87, 65.68) --
	(100.10, 65.68) --
	(100.33, 65.68) --
	(100.57, 65.68) --
	(100.80, 65.68) --
	(101.03, 65.68) --
	(101.26, 65.68) --
	(101.50, 65.68) --
	(101.73, 65.68) --
	(101.96, 65.68) --
	(102.19, 65.68) --
	(102.43, 65.68) --
	(102.66, 65.68) --
	(102.89, 65.68) --
	(103.12, 93.65) --
	(103.36, 93.65) --
	(103.59, 93.65) --
	(103.82, 93.65) --
	(104.05, 93.65) --
	(104.29, 93.65) --
	(104.52, 93.65) --
	(104.75, 93.65) --
	(104.98, 93.65) --
	(105.22, 93.65) --
	(105.45, 93.65) --
	(105.68, 93.65) --
	(105.91, 93.65) --
	(106.15, 93.65) --
	(106.38, 93.65) --
	(106.61, 93.65) --
	(106.84, 93.65) --
	(107.08, 93.65) --
	(107.31, 93.65) --
	(107.54, 93.65) --
	(107.77, 93.65) --
	(108.01, 93.65) --
	(108.24, 93.65) --
	(108.47, 93.65) --
	(108.70, 93.65) --
	(108.94, 93.65) --
	(109.17, 93.65) --
	(109.40, 93.65) --
	(109.63, 93.65) --
	(109.87, 93.65) --
	(110.10, 93.65) --
	(110.33, 93.65) --
	(110.56, 93.65) --
	(110.79, 93.65) --
	(111.03, 93.65) --
	(111.26, 93.65) --
	(111.49, 93.65) --
	(111.72, 93.65) --
	(111.96, 93.65) --
	(112.19, 93.65) --
	(112.42, 93.65) --
	(112.65, 93.65) --
	(112.89, 93.65) --
	(113.12, 93.65) --
	(113.35, 93.65) --
	(113.58, 93.65) --
	(113.82, 93.65) --
	(114.05, 93.65) --
	(114.28, 93.65) --
	(114.51, 93.65) --
	(114.75, 93.65) --
	(114.98, 93.65) --
	(115.21, 93.65) --
	(115.44, 93.65) --
	(115.68, 93.65) --
	(115.91, 93.65) --
	(116.14, 93.65) --
	(116.37, 93.65) --
	(116.61, 93.65) --
	(116.84, 93.65) --
	(117.07, 93.65) --
	(117.30, 93.65) --
	(117.54, 93.65) --
	(117.77, 93.65) --
	(118.00, 93.65) --
	(118.23, 93.65) --
	(118.47, 93.65) --
	(118.70, 93.65) --
	(118.93, 93.65) --
	(119.16, 93.65) --
	(119.40, 93.65) --
	(119.63, 93.65) --
	(119.86, 93.65) --
	(120.09, 93.65) --
	(120.33, 93.65) --
	(120.56, 93.65) --
	(120.79, 93.65) --
	(121.02, 93.65) --
	(121.26, 93.65) --
	(121.49, 93.65) --
	(121.72, 93.65) --
	(121.95, 93.65) --
	(122.19, 93.65) --
	(122.42, 93.65) --
	(122.65, 93.65) --
	(122.88, 93.65) --
	(123.12, 93.65) --
	(123.35, 93.65) --
	(123.58, 93.65) --
	(123.81, 93.65) --
	(124.05, 93.65) --
	(124.28, 93.65) --
	(124.51, 93.65) --
	(124.74, 93.65) --
	(124.98, 93.65) --
	(125.21, 93.65) --
	(125.44, 93.65) --
	(125.67, 93.65) --
	(125.91, 93.65) --
	(126.14, 93.65) --
	(126.37, 93.65) --
	(126.60, 93.65) --
	(126.84, 93.65) --
	(127.07, 93.65) --
	(127.30, 93.65) --
	(127.53, 93.65) --
	(127.77, 93.65) --
	(128.00, 93.65) --
	(128.23, 93.65) --
	(128.46, 93.65) --
	(128.69, 93.65) --
	(128.93, 93.65) --
	(129.16, 93.65) --
	(129.39, 93.65) --
	(129.62, 93.65) --
	(129.86, 93.65) --
	(130.09, 93.65) --
	(130.32, 93.65) --
	(130.55, 93.65) --
	(130.79, 93.65) --
	(131.02, 93.65) --
	(131.25, 93.65) --
	(131.48, 93.65) --
	(131.72, 93.65) --
	(131.95, 93.65) --
	(132.18, 93.65) --
	(132.41, 93.65) --
	(132.65, 93.65) --
	(132.88, 93.65) --
	(133.11, 93.65) --
	(133.34, 93.65) --
	(133.58, 93.65) --
	(133.81, 93.65) --
	(134.04, 93.65) --
	(134.27, 93.65) --
	(134.51, 93.65) --
	(134.74, 93.65) --
	(134.97, 93.65) --
	(135.20, 93.65) --
	(135.44, 93.65) --
	(135.67, 93.65) --
	(135.90, 93.65) --
	(136.13, 93.65) --
	(136.37, 93.65) --
	(136.60, 93.65) --
	(136.83, 93.65) --
	(137.06, 93.65) --
	(137.30, 93.65) --
	(137.53, 93.65) --
	(137.76, 93.65) --
	(137.99, 93.65) --
	(138.23, 93.65) --
	(138.46, 93.65) --
	(138.69, 93.65) --
	(138.92, 93.65) --
	(139.16, 93.65) --
	(139.39, 93.65) --
	(139.62, 93.65) --
	(139.85, 93.65) --
	(140.09, 93.65) --
	(140.32, 93.65) --
	(140.55, 93.65) --
	(140.78, 93.65) --
	(141.02, 93.65) --
	(141.25, 93.65) --
	(141.48, 93.65) --
	(141.71, 93.65) --
	(141.95, 93.65) --
	(142.18, 93.65) --
	(142.41, 93.65) --
	(142.64, 93.65) --
	(142.88, 93.65) --
	(143.11, 93.65) --
	(143.34, 93.65) --
	(143.57, 93.65) --
	(143.81, 93.65) --
	(144.04, 93.65) --
	(144.27, 93.65) --
	(144.50, 93.65) --
	(144.74, 93.65) --
	(144.97, 93.65) --
	(145.20, 93.65) --
	(145.43, 93.65) --
	(145.67, 93.65) --
	(145.90, 93.65) --
	(146.13, 93.65) --
	(146.36, 93.65) --
	(146.60, 93.65) --
	(146.83, 93.65) --
	(147.06, 93.65) --
	(147.29, 93.65) --
	(147.52, 93.65) --
	(147.76, 93.65) --
	(147.99, 93.65) --
	(148.22, 93.65) --
	(148.45, 93.65) --
	(148.69, 93.65) --
	(148.92, 93.65) --
	(149.15, 93.65) --
	(149.38, 93.65) --
	(149.62, 93.65) --
	(149.85, 93.65) --
	(150.08,121.63) --
	(150.31,121.63) --
	(150.55,121.63) --
	(150.78,121.63) --
	(151.01,121.63) --
	(151.24,121.63) --
	(151.48,121.63) --
	(151.71,121.63) --
	(151.94,121.63) --
	(152.17,121.63) --
	(152.41,121.63) --
	(152.64,121.63) --
	(152.87,121.63) --
	(153.10,121.63) --
	(153.34,121.63) --
	(153.57,121.63) --
	(153.80,121.63) --
	(154.03,121.63) --
	(154.27,121.63) --
	(154.50,121.63) --
	(154.73,121.63) --
	(154.96,121.63) --
	(155.20,121.63) --
	(155.43,121.63) --
	(155.66,121.63) --
	(155.89,121.63) --
	(156.13,121.63) --
	(156.36,121.63) --
	(156.59,121.63) --
	(156.82,121.63) --
	(157.06,121.63) --
	(157.29,121.63) --
	(157.52,121.63) --
	(157.75,121.63) --
	(157.99,121.63) --
	(158.22,121.63) --
	(158.45,121.63) --
	(158.68,121.63) --
	(158.92,121.63) --
	(159.15,121.63) --
	(159.38,121.63) --
	(159.61,121.63) --
	(159.85,121.63) --
	(160.08,121.63) --
	(160.31,121.63) --
	(160.54,121.63) --
	(160.78,121.63) --
	(161.01,121.63) --
	(161.24,121.63) --
	(161.47,121.63) --
	(161.71,121.63) --
	(161.94,121.63) --
	(162.17,121.63) --
	(162.40,121.63) --
	(162.64,121.63) --
	(162.87,121.63) --
	(163.10,121.63) --
	(163.33,121.63) --
	(163.57,121.63) --
	(163.80,121.63) --
	(164.03,121.63) --
	(164.26,121.63) --
	(164.50,121.63) --
	(164.73,121.63) --
	(164.96,121.63) --
	(165.19,121.63) --
	(165.42,121.63) --
	(165.66,121.63) --
	(165.89,121.63) --
	(166.12,121.63) --
	(166.35,121.63) --
	(166.59,121.63) --
	(166.82,121.63) --
	(167.05,121.63) --
	(167.28,121.63) --
	(167.52,121.63) --
	(167.75,121.63) --
	(167.98,121.63) --
	(168.21,121.63) --
	(168.45,121.63) --
	(168.68,121.63) --
	(168.91,121.63) --
	(169.14,121.63) --
	(169.38,121.63) --
	(169.61,121.63) --
	(169.84,121.63) --
	(170.07,121.63) --
	(170.31,121.63) --
	(170.54,121.63) --
	(170.77,121.63) --
	(171.00,121.63) --
	(171.24,121.63) --
	(171.47,121.63) --
	(171.70,121.63) --
	(171.93,121.63) --
	(172.17,121.63) --
	(172.40,121.63) --
	(172.63,121.63) --
	(172.86,121.63) --
	(173.10,121.63) --
	(173.33,121.63) --
	(173.56,121.63) --
	(173.79,121.63) --
	(174.03,121.63) --
	(174.26,121.63) --
	(174.49,121.63) --
	(174.72,121.63) --
	(174.96,121.63) --
	(175.19,121.63) --
	(175.42,121.63) --
	(175.65,121.63) --
	(175.89,121.63) --
	(176.12,121.63) --
	(176.35,121.63) --
	(176.58,121.63) --
	(176.82,121.63) --
	(177.05,121.63) --
	(177.28,121.63) --
	(177.51,121.63) --
	(177.75,121.63) --
	(177.98,121.63) --
	(178.21,121.63) --
	(178.44,121.63) --
	(178.68,121.63) --
	(178.91,121.63) --
	(179.14,121.63) --
	(179.37,121.63) --
	(179.61,121.63) --
	(179.84,121.63) --
	(180.07,121.63) --
	(180.30,121.63) --
	(180.54,121.63) --
	(180.77,121.63) --
	(181.00,121.63) --
	(181.23,121.63) --
	(181.47,121.63) --
	(181.70,121.63) --
	(181.93,121.63) --
	(182.16,121.63) --
	(182.40,121.63) --
	(182.63,121.63) --
	(182.86,121.63) --
	(183.09,121.63) --
	(183.33,121.63) --
	(183.56,121.63) --
	(183.79,121.63) --
	(184.02,121.63) --
	(184.25,121.63) --
	(184.49,121.63) --
	(184.72,121.63) --
	(184.95,121.63) --
	(185.18,121.63) --
	(185.42,121.63) --
	(185.65,121.63) --
	(185.88,121.63) --
	(186.11,121.63) --
	(186.35,121.63) --
	(186.58,121.63) --
	(186.81,121.63) --
	(187.04,121.63) --
	(187.28,121.63) --
	(187.51,121.63) --
	(187.74,121.63) --
	(187.97,121.63) --
	(188.21,121.63) --
	(188.44,121.63) --
	(188.67,121.63) --
	(188.90,121.63) --
	(189.14,121.63) --
	(189.37,121.63) --
	(189.60,121.63) --
	(189.83,121.63) --
	(190.07,121.63) --
	(190.30,121.63) --
	(190.53,121.63) --
	(190.76,121.63) --
	(191.00,121.63) --
	(191.23,121.63) --
	(191.46,121.63) --
	(191.69,121.63) --
	(191.93,121.63) --
	(192.16,121.63) --
	(192.39,121.63) --
	(192.62,121.63) --
	(192.86,121.63) --
	(193.09,121.63) --
	(193.32,121.63) --
	(193.55,121.63) --
	(193.79,121.63) --
	(194.02,121.63) --
	(194.25,121.63) --
	(194.48,121.63) --
	(194.72,121.63) --
	(194.95,121.63) --
	(195.18,121.63) --
	(195.41,121.63) --
	(195.65,121.63) --
	(195.88,121.63) --
	(196.11,121.63) --
	(196.34,121.63) --
	(196.58,121.63) --
	(196.81,121.63) --
	(197.04,149.61) --
	(197.27,149.61) --
	(197.51,149.61) --
	(197.74,149.61) --
	(197.97,149.61) --
	(198.20,149.61) --
	(198.44,149.61) --
	(198.67,149.61) --
	(198.90,149.61) --
	(199.13,149.61) --
	(199.37,149.61) --
	(199.60,149.61) --
	(199.83,149.61) --
	(200.06,149.61) --
	(200.30,149.61) --
	(200.53,149.61) --
	(200.76,149.61) --
	(200.99,149.61) --
	(201.23,149.61) --
	(201.46,149.61) --
	(201.69,149.61) --
	(201.92,149.61) --
	(202.16,149.61) --
	(202.39,149.61) --
	(202.62,149.61) --
	(202.85,149.61) --
	(203.08,149.61) --
	(203.32,149.61) --
	(203.55,149.61) --
	(203.78,149.61) --
	(204.01,149.61) --
	(204.25,149.61) --
	(204.48,149.61) --
	(204.71,149.61) --
	(204.94,149.61) --
	(205.18,149.61) --
	(205.41,149.61) --
	(205.64,149.61) --
	(205.87,149.61) --
	(206.11,149.61) --
	(206.34,149.61) --
	(206.57,149.61) --
	(206.80,149.61) --
	(207.04,149.61) --
	(207.27,149.61) --
	(207.50,149.61) --
	(207.73,149.61) --
	(207.97,149.61) --
	(208.20,149.61) --
	(208.43,149.61) --
	(208.66,149.61) --
	(208.90,149.61) --
	(209.13,149.61) --
	(209.36,149.61) --
	(209.59,149.61) --
	(209.83,149.61) --
	(210.06,149.61) --
	(210.29,149.61) --
	(210.52,149.61) --
	(210.76,149.61) --
	(210.99,149.61) --
	(211.22,149.61) --
	(211.45,149.61) --
	(211.69,149.61) --
	(211.92,149.61) --
	(212.15,149.61) --
	(212.38,149.61) --
	(212.62,149.61) --
	(212.85,149.61) --
	(213.08,149.61) --
	(213.31,149.61) --
	(213.55,149.61) --
	(213.78,149.61) --
	(214.01,149.61) --
	(214.24,149.61) --
	(214.48,149.61) --
	(214.71,149.61) --
	(214.94,149.61) --
	(215.17,149.61) --
	(215.41,149.61) --
	(215.64,149.61) --
	(215.87,149.61) --
	(216.10,149.61) --
	(216.34,149.61) --
	(216.57,149.61) --
	(216.80,149.61) --
	(217.03,149.61) --
	(217.27,149.61) --
	(217.50,149.61) --
	(217.73,149.61) --
	(217.96,149.61) --
	(218.20,149.61) --
	(218.43,149.61) --
	(218.66,149.61) --
	(218.89,149.61) --
	(219.13,149.61) --
	(219.36,149.61) --
	(219.59,149.61) --
	(219.82,149.61) --
	(220.06,149.61) --
	(220.29,149.61) --
	(220.52,149.61) --
	(220.75,149.61) --
	(220.98,149.61) --
	(221.22,149.61) --
	(221.45,149.61) --
	(221.68,149.61) --
	(221.91,149.61) --
	(222.15,149.61) --
	(222.38,149.61) --
	(222.61,149.61) --
	(222.84,149.61) --
	(223.08,149.61) --
	(223.31,149.61) --
	(223.54,149.61) --
	(223.77,149.61) --
	(224.01,149.61) --
	(224.24,149.61) --
	(224.47,149.61) --
	(224.70,149.61) --
	(224.94,149.61) --
	(225.17,149.61) --
	(225.40,149.61) --
	(225.63,149.61) --
	(225.87,149.61) --
	(226.10,149.61) --
	(226.33,149.61) --
	(226.56,149.61) --
	(226.80,149.61) --
	(227.03,149.61) --
	(227.26,149.61) --
	(227.49,149.61) --
	(227.73,149.61) --
	(227.96,149.61) --
	(228.19,149.61) --
	(228.42,149.61) --
	(228.66,149.61) --
	(228.89,149.61) --
	(229.12,149.61) --
	(229.35,149.61) --
	(229.59,149.61) --
	(229.82,149.61) --
	(230.05,149.61) --
	(230.28,149.61) --
	(230.52,149.61) --
	(230.75,149.61) --
	(230.98,149.61) --
	(231.21,149.61) --
	(231.45,149.61) --
	(231.68,149.61) --
	(231.91,149.61) --
	(232.14,149.61) --
	(232.38,149.61) --
	(232.61,149.61) --
	(232.84,149.61) --
	(233.07,149.61) --
	(233.31,149.61) --
	(233.54,149.61) --
	(233.77,149.61) --
	(234.00,149.61) --
	(234.24,149.61) --
	(234.47,149.61) --
	(234.70,149.61) --
	(234.93,149.61) --
	(235.17,149.61) --
	(235.40,149.61) --
	(235.63,149.61) --
	(235.86,149.61) --
	(236.10,149.61) --
	(236.33,149.61) --
	(236.56,149.61) --
	(236.79,149.61) --
	(237.03,149.61) --
	(237.26,149.61) --
	(237.49,149.61) --
	(237.72,149.61) --
	(237.96,149.61) --
	(238.19,149.61) --
	(238.42,149.61) --
	(238.65,149.61) --
	(238.89,149.61) --
	(239.12,149.61) --
	(239.35,149.61) --
	(239.58,149.61) --
	(239.81,149.61) --
	(240.05,149.61) --
	(240.28,149.61) --
	(240.51,149.61) --
	(240.74,149.61) --
	(240.98,149.61) --
	(241.21,149.61) --
	(241.44,149.61) --
	(241.67,149.61) --
	(241.91,149.61) --
	(242.14,149.61) --
	(242.37,149.61) --
	(242.60,149.61) --
	(242.84,149.61) --
	(243.07,149.61) --
	(243.30,149.61) --
	(243.53,149.61) --
	(243.77,149.61) --
	(244.00,177.59) --
	(244.23,177.59) --
	(244.46,177.59) --
	(244.70,177.59) --
	(244.93,177.59) --
	(245.16,177.59) --
	(245.39,177.59) --
	(245.63,177.59) --
	(245.86,177.59) --
	(246.09,177.59) --
	(246.32,177.59) --
	(246.56,177.59) --
	(246.79,177.59) --
	(247.02,177.59) --
	(247.25,177.59) --
	(247.49,177.59) --
	(247.72,177.59) --
	(247.95,177.59) --
	(248.18,177.59) --
	(248.42,177.59) --
	(248.65,177.59) --
	(248.88,177.59) --
	(249.11,177.59) --
	(249.35,177.59) --
	(249.58,177.59) --
	(249.81,177.59) --
	(250.04,177.59) --
	(250.28,177.59) --
	(250.51,177.59) --
	(250.74,177.59) --
	(250.97,177.59) --
	(251.21,177.59) --
	(251.44,177.59) --
	(251.67,177.59) --
	(251.90,177.59) --
	(252.14,177.59) --
	(252.37,177.59) --
	(252.60,177.59) --
	(252.83,177.59) --
	(253.07,177.59) --
	(253.30,177.59) --
	(253.53,177.59) --
	(253.76,177.59) --
	(254.00,177.59) --
	(254.23,177.59) --
	(254.46,177.59) --
	(254.69,177.59) --
	(254.93,177.59) --
	(255.16,177.59) --
	(255.39,177.59) --
	(255.62,177.59) --
	(255.86,177.59) --
	(256.09,177.59) --
	(256.32,177.59) --
	(256.55,177.59) --
	(256.79,177.59) --
	(257.02,177.59) --
	(257.25,177.59) --
	(257.48,177.59) --
	(257.72,177.59) --
	(257.95,177.59) --
	(258.18,177.59) --
	(258.41,177.59) --
	(258.64,177.59) --
	(258.88,177.59) --
	(259.11,177.59) --
	(259.34,177.59) --
	(259.57,177.59) --
	(259.81,177.59) --
	(260.04,177.59) --
	(260.27,177.59) --
	(260.50,177.59) --
	(260.74,177.59) --
	(260.97,177.59) --
	(261.20,177.59) --
	(261.43,177.59) --
	(261.67,177.59) --
	(261.90,177.59) --
	(262.13,177.59) --
	(262.36,177.59) --
	(262.60,177.59) --
	(262.83,177.59) --
	(263.06,177.59) --
	(263.29,177.59) --
	(263.53,177.59) --
	(263.76,177.59) --
	(263.99,177.59) --
	(264.22,177.59) --
	(264.46,177.59) --
	(264.69,177.59) --
	(264.92,177.59) --
	(265.15,177.59) --
	(265.39,177.59) --
	(265.62,177.59) --
	(265.85,177.59) --
	(266.08,177.59) --
	(266.32,177.59) --
	(266.55,177.59) --
	(266.78,177.59) --
	(267.01,177.59) --
	(267.25,177.59) --
	(267.48,177.59) --
	(267.71,177.59) --
	(267.94,177.59) --
	(268.18,177.59) --
	(268.41,177.59) --
	(268.64,177.59) --
	(268.87,177.59) --
	(269.11,177.59) --
	(269.34,177.59) --
	(269.57,177.59) --
	(269.80,177.59) --
	(270.04,177.59) --
	(270.27,177.59) --
	(270.50,177.59) --
	(270.73,177.59) --
	(270.97,177.59) --
	(271.20,177.59) --
	(271.43,177.59) --
	(271.66,177.59) --
	(271.90,177.59) --
	(272.13,177.59) --
	(272.36,177.59) --
	(272.59,177.59) --
	(272.83,177.59) --
	(273.06,177.59) --
	(273.29,177.59) --
	(273.52,177.59) --
	(273.76,177.59) --
	(273.99,177.59) --
	(274.22,177.59) --
	(274.45,177.59) --
	(274.69,177.59) --
	(274.92,177.59) --
	(275.15,177.59) --
	(275.38,177.59) --
	(275.62,177.59) --
	(275.85,177.59) --
	(276.08,177.59) --
	(276.31,177.59) --
	(276.54,177.59) --
	(276.78,177.59) --
	(277.01,177.59) --
	(277.24,177.59) --
	(277.47,177.59) --
	(277.71,177.59) --
	(277.94,177.59) --
	(278.17,177.59) --
	(278.40,177.59) --
	(278.64,177.59) --
	(278.87,177.59) --
	(279.10,177.59) --
	(279.33,177.59) --
	(279.57,177.59) --
	(279.80,177.59) --
	(280.03,177.59) --
	(280.26,177.59) --
	(280.50,177.59) --
	(280.73,177.59) --
	(280.96,177.59) --
	(281.19,177.59) --
	(281.43,177.59) --
	(281.66,177.59) --
	(281.89,177.59) --
	(282.12,177.59) --
	(282.36,177.59) --
	(282.59,177.59) --
	(282.82,177.59) --
	(283.05,177.59) --
	(283.29,177.59) --
	(283.52,177.59) --
	(283.75,177.59) --
	(283.98,177.59) --
	(284.22,177.59) --
	(284.45,177.59) --
	(284.68,177.59) --
	(284.91,177.59) --
	(285.15,177.59) --
	(285.38,177.59) --
	(285.61,177.59) --
	(285.84,177.59) --
	(286.08,177.59) --
	(286.31,177.59) --
	(286.54,177.59) --
	(286.77,177.59) --
	(287.01,177.59) --
	(287.24,177.59) --
	(287.47,177.59) --
	(287.70,177.59) --
	(287.94,177.59) --
	(288.17,177.59) --
	(288.40,177.59) --
	(288.63,177.59) --
	(288.87,177.59) --
	(289.10,177.59) --
	(289.33,177.59) --
	(289.56,177.59) --
	(289.80,177.59) --
	(290.03,177.59) --
	(290.26,177.59) --
	(290.49,177.59) --
	(290.73,177.59);
\end{scope}
\begin{scope}
\path[clip] (  0.00,  0.00) rectangle (325.21,231.26);
\definecolor{drawColor}{RGB}{0,0,0}

\path[draw=drawColor,line width= 0.4pt,line join=round,line cap=round] ( 56.14, 61.20) -- (290.73, 61.20);

\path[draw=drawColor,line width= 0.4pt,line join=round,line cap=round] ( 56.14, 61.20) -- ( 56.14, 55.20);

\path[draw=drawColor,line width= 0.4pt,line join=round,line cap=round] (103.06, 61.20) -- (103.06, 55.20);

\path[draw=drawColor,line width= 0.4pt,line join=round,line cap=round] (149.98, 61.20) -- (149.98, 55.20);

\path[draw=drawColor,line width= 0.4pt,line join=round,line cap=round] (196.89, 61.20) -- (196.89, 55.20);

\path[draw=drawColor,line width= 0.4pt,line join=round,line cap=round] (243.81, 61.20) -- (243.81, 55.20);

\path[draw=drawColor,line width= 0.4pt,line join=round,line cap=round] (290.73, 61.20) -- (290.73, 55.20);

\node[text=drawColor,anchor=base,inner sep=0pt, outer sep=0pt, scale=  1.00] at ( 56.14, 39.60) {0};

\node[text=drawColor,anchor=base,inner sep=0pt, outer sep=0pt, scale=  1.00] at (103.06, 39.60) {2};

\node[text=drawColor,anchor=base,inner sep=0pt, outer sep=0pt, scale=  1.00] at (149.98, 39.60) {4};

\node[text=drawColor,anchor=base,inner sep=0pt, outer sep=0pt, scale=  1.00] at (196.89, 39.60) {6};

\node[text=drawColor,anchor=base,inner sep=0pt, outer sep=0pt, scale=  1.00] at (243.81, 39.60) {8};

\node[text=drawColor,anchor=base,inner sep=0pt, outer sep=0pt, scale=  1.00] at (290.73, 39.60) {10};

\path[draw=drawColor,line width= 0.4pt,line join=round,line cap=round] ( 49.20, 65.68) -- ( 49.20,177.59);

\path[draw=drawColor,line width= 0.4pt,line join=round,line cap=round] ( 49.20, 65.68) -- ( 43.20, 65.68);

\path[draw=drawColor,line width= 0.4pt,line join=round,line cap=round] ( 49.20, 84.33) -- ( 43.20, 84.33);

\path[draw=drawColor,line width= 0.4pt,line join=round,line cap=round] ( 49.20,102.98) -- ( 43.20,102.98);

\path[draw=drawColor,line width= 0.4pt,line join=round,line cap=round] ( 49.20,121.63) -- ( 43.20,121.63);

\path[draw=drawColor,line width= 0.4pt,line join=round,line cap=round] ( 49.20,140.28) -- ( 43.20,140.28);

\path[draw=drawColor,line width= 0.4pt,line join=round,line cap=round] ( 49.20,158.94) -- ( 43.20,158.94);

\path[draw=drawColor,line width= 0.4pt,line join=round,line cap=round] ( 49.20,177.59) -- ( 43.20,177.59);

\node[text=drawColor,rotate= 90.00,anchor=base,inner sep=0pt, outer sep=0pt, scale=  1.00] at ( 34.80, 65.68) {0};

\node[text=drawColor,rotate= 90.00,anchor=base,inner sep=0pt, outer sep=0pt, scale=  1.00] at ( 34.80, 84.33) {2};

\node[text=drawColor,rotate= 90.00,anchor=base,inner sep=0pt, outer sep=0pt, scale=  1.00] at ( 34.80,102.98) {4};

\node[text=drawColor,rotate= 90.00,anchor=base,inner sep=0pt, outer sep=0pt, scale=  1.00] at ( 34.80,121.63) {6};

\node[text=drawColor,rotate= 90.00,anchor=base,inner sep=0pt, outer sep=0pt, scale=  1.00] at ( 34.80,140.28) {8};

\node[text=drawColor,rotate= 90.00,anchor=base,inner sep=0pt, outer sep=0pt, scale=  1.00] at ( 34.80,158.94) {10};

\node[text=drawColor,rotate= 90.00,anchor=base,inner sep=0pt, outer sep=0pt, scale=  1.00] at ( 34.80,177.59) {12};

\path[draw=drawColor,line width= 0.4pt,line join=round,line cap=round] ( 49.20, 61.20) --
	(300.01, 61.20) --
	(300.01,182.06) --
	( 49.20,182.06) --
	( 49.20, 61.20);
\end{scope}
\begin{scope}
\path[clip] (  0.00,  0.00) rectangle (325.21,231.26);
\definecolor{drawColor}{RGB}{0,0,0}

\node[text=drawColor,anchor=base,inner sep=0pt, outer sep=0pt, scale=  1.00] at (174.61, 15.60) {$t_\text{max}$};

\node[text=drawColor,rotate= 90.00,anchor=base,inner sep=0pt, outer sep=0pt, scale=  1.00] at ( 10.80,121.63) {$I_R$};
\end{scope}
\begin{scope}
\path[clip] ( 49.20, 61.20) rectangle (300.01,182.06);
\definecolor{drawColor}{RGB}{0,0,0}

\path[draw=drawColor,line width= 0.4pt,dash pattern=on 4pt off 4pt ,line join=round,line cap=round] ( 58.49, 67.08) --
	( 58.72, 67.21) --
	( 58.95, 67.35) --
	( 59.19, 67.49) --
	( 59.42, 67.63) --
	( 59.65, 67.77) --
	( 59.88, 67.91) --
	( 60.12, 68.05) --
	( 60.35, 68.18) --
	( 60.58, 68.32) --
	( 60.81, 68.46) --
	( 61.05, 68.60) --
	( 61.28, 68.74) --
	( 61.51, 68.88) --
	( 61.74, 69.02) --
	( 61.98, 69.15) --
	( 62.21, 69.29) --
	( 62.44, 69.43) --
	( 62.67, 69.57) --
	( 62.91, 69.71) --
	( 63.14, 69.85) --
	( 63.37, 69.99) --
	( 63.60, 70.13) --
	( 63.84, 70.26) --
	( 64.07, 70.40) --
	( 64.30, 70.54) --
	( 64.53, 70.68) --
	( 64.77, 70.82) --
	( 65.00, 70.96) --
	( 65.23, 71.10) --
	( 65.46, 71.23) --
	( 65.70, 71.37) --
	( 65.93, 71.51) --
	( 66.16, 71.65) --
	( 66.39, 71.79) --
	( 66.63, 71.93) --
	( 66.86, 72.07) --
	( 67.09, 72.20) --
	( 67.32, 72.34) --
	( 67.56, 72.48) --
	( 67.79, 72.62) --
	( 68.02, 72.76) --
	( 68.25, 72.90) --
	( 68.49, 73.04) --
	( 68.72, 73.17) --
	( 68.95, 73.31) --
	( 69.18, 73.45) --
	( 69.42, 73.59) --
	( 69.65, 73.73) --
	( 69.88, 73.87) --
	( 70.11, 74.01) --
	( 70.35, 74.15) --
	( 70.58, 74.28) --
	( 70.81, 74.42) --
	( 71.04, 74.56) --
	( 71.28, 74.70) --
	( 71.51, 74.84) --
	( 71.74, 74.98) --
	( 71.97, 75.12) --
	( 72.21, 75.25) --
	( 72.44, 75.39) --
	( 72.67, 75.53) --
	( 72.90, 75.67) --
	( 73.13, 75.81) --
	( 73.37, 75.95) --
	( 73.60, 76.09) --
	( 73.83, 76.22) --
	( 74.06, 76.36) --
	( 74.30, 76.50) --
	( 74.53, 76.64) --
	( 74.76, 76.78) --
	( 74.99, 76.92) --
	( 75.23, 77.06) --
	( 75.46, 77.20) --
	( 75.69, 77.33) --
	( 75.92, 77.47) --
	( 76.16, 77.61) --
	( 76.39, 77.75) --
	( 76.62, 77.89) --
	( 76.85, 78.03) --
	( 77.09, 78.17) --
	( 77.32, 78.30) --
	( 77.55, 78.44) --
	( 77.78, 78.58) --
	( 78.02, 78.72) --
	( 78.25, 78.86) --
	( 78.48, 79.00) --
	( 78.71, 79.14) --
	( 78.95, 79.27) --
	( 79.18, 79.41) --
	( 79.41, 79.55) --
	( 79.64, 79.69) --
	( 79.88, 79.83) --
	( 80.11, 79.97) --
	( 80.34, 80.11) --
	( 80.57, 80.25) --
	( 80.81, 80.38) --
	( 81.04, 80.52) --
	( 81.27, 80.66) --
	( 81.50, 80.80) --
	( 81.74, 80.94) --
	( 81.97, 81.08) --
	( 82.20, 81.22) --
	( 82.43, 81.35) --
	( 82.67, 81.49) --
	( 82.90, 81.63) --
	( 83.13, 81.77) --
	( 83.36, 81.91) --
	( 83.60, 82.05) --
	( 83.83, 82.19) --
	( 84.06, 82.32) --
	( 84.29, 82.46) --
	( 84.53, 82.60) --
	( 84.76, 82.74) --
	( 84.99, 82.88) --
	( 85.22, 83.02) --
	( 85.46, 83.16) --
	( 85.69, 83.29) --
	( 85.92, 83.43) --
	( 86.15, 83.57) --
	( 86.39, 83.71) --
	( 86.62, 83.85) --
	( 86.85, 83.99) --
	( 87.08, 84.13) --
	( 87.32, 84.27) --
	( 87.55, 84.40) --
	( 87.78, 84.54) --
	( 88.01, 84.68) --
	( 88.25, 84.82) --
	( 88.48, 84.96) --
	( 88.71, 85.10) --
	( 88.94, 85.24) --
	( 89.18, 85.37) --
	( 89.41, 85.51) --
	( 89.64, 85.65) --
	( 89.87, 85.79) --
	( 90.11, 85.93) --
	( 90.34, 86.07) --
	( 90.57, 86.21) --
	( 90.80, 86.34) --
	( 91.04, 86.48) --
	( 91.27, 86.62) --
	( 91.50, 86.76) --
	( 91.73, 86.90) --
	( 91.96, 87.04) --
	( 92.20, 87.18) --
	( 92.43, 87.32) --
	( 92.66, 87.45) --
	( 92.89, 87.59) --
	( 93.13, 87.73) --
	( 93.36, 87.87) --
	( 93.59, 88.01) --
	( 93.82, 88.15) --
	( 94.06, 88.29) --
	( 94.29, 88.42) --
	( 94.52, 88.56) --
	( 94.75, 88.70) --
	( 94.99, 88.84) --
	( 95.22, 88.98) --
	( 95.45, 89.12) --
	( 95.68, 89.26) --
	( 95.92, 89.39) --
	( 96.15, 89.53) --
	( 96.38, 89.67) --
	( 96.61, 89.81) --
	( 96.85, 89.95) --
	( 97.08, 90.09) --
	( 97.31, 90.23) --
	( 97.54, 90.36) --
	( 97.78, 90.50) --
	( 98.01, 90.64) --
	( 98.24, 90.78) --
	( 98.47, 90.92) --
	( 98.71, 91.06) --
	( 98.94, 91.20) --
	( 99.17, 91.34) --
	( 99.40, 91.47) --
	( 99.64, 91.61) --
	( 99.87, 91.75) --
	(100.10, 91.89) --
	(100.33, 92.03) --
	(100.57, 92.17) --
	(100.80, 92.31) --
	(101.03, 92.44) --
	(101.26, 92.58) --
	(101.50, 92.72) --
	(101.73, 92.86) --
	(101.96, 93.00) --
	(102.19, 93.14) --
	(102.43, 93.28) --
	(102.66, 93.41) --
	(102.89, 93.55) --
	(103.12, 93.69) --
	(103.36, 93.83) --
	(103.59, 93.97) --
	(103.82, 94.11) --
	(104.05, 94.25) --
	(104.29, 94.39) --
	(104.52, 94.52) --
	(104.75, 94.66) --
	(104.98, 94.80) --
	(105.22, 94.94) --
	(105.45, 95.08) --
	(105.68, 95.22) --
	(105.91, 95.36) --
	(106.15, 95.49) --
	(106.38, 95.63) --
	(106.61, 95.77) --
	(106.84, 95.91) --
	(107.08, 96.05) --
	(107.31, 96.19) --
	(107.54, 96.33) --
	(107.77, 96.46) --
	(108.01, 96.60) --
	(108.24, 96.74) --
	(108.47, 96.88) --
	(108.70, 97.02) --
	(108.94, 97.16) --
	(109.17, 97.30) --
	(109.40, 97.44) --
	(109.63, 97.57) --
	(109.87, 97.71) --
	(110.10, 97.85) --
	(110.33, 97.99) --
	(110.56, 98.13) --
	(110.79, 98.27) --
	(111.03, 98.41) --
	(111.26, 98.54) --
	(111.49, 98.68) --
	(111.72, 98.82) --
	(111.96, 98.96) --
	(112.19, 99.10) --
	(112.42, 99.24) --
	(112.65, 99.38) --
	(112.89, 99.51) --
	(113.12, 99.65) --
	(113.35, 99.79) --
	(113.58, 99.93) --
	(113.82,100.07) --
	(114.05,100.21) --
	(114.28,100.35) --
	(114.51,100.48) --
	(114.75,100.62) --
	(114.98,100.76) --
	(115.21,100.90) --
	(115.44,101.04) --
	(115.68,101.18) --
	(115.91,101.32) --
	(116.14,101.46) --
	(116.37,101.59) --
	(116.61,101.73) --
	(116.84,101.87) --
	(117.07,102.01) --
	(117.30,102.15) --
	(117.54,102.29) --
	(117.77,102.43) --
	(118.00,102.56) --
	(118.23,102.70) --
	(118.47,102.84) --
	(118.70,102.98) --
	(118.93,103.12) --
	(119.16,103.26) --
	(119.40,103.40) --
	(119.63,103.53) --
	(119.86,103.67) --
	(120.09,103.81) --
	(120.33,103.95) --
	(120.56,104.09) --
	(120.79,104.23) --
	(121.02,104.37) --
	(121.26,104.51) --
	(121.49,104.64) --
	(121.72,104.78) --
	(121.95,104.92) --
	(122.19,105.06) --
	(122.42,105.20) --
	(122.65,105.34) --
	(122.88,105.48) --
	(123.12,105.61) --
	(123.35,105.75) --
	(123.58,105.89) --
	(123.81,106.03) --
	(124.05,106.17) --
	(124.28,106.31) --
	(124.51,106.45) --
	(124.74,106.58) --
	(124.98,106.72) --
	(125.21,106.86) --
	(125.44,107.00) --
	(125.67,107.14) --
	(125.91,107.28) --
	(126.14,107.42) --
	(126.37,107.55) --
	(126.60,107.69) --
	(126.84,107.83) --
	(127.07,107.97) --
	(127.30,108.11) --
	(127.53,108.25) --
	(127.77,108.39) --
	(128.00,108.53) --
	(128.23,108.66) --
	(128.46,108.80) --
	(128.69,108.94) --
	(128.93,109.08) --
	(129.16,109.22) --
	(129.39,109.36) --
	(129.62,109.50) --
	(129.86,109.63) --
	(130.09,109.77) --
	(130.32,109.91) --
	(130.55,110.05) --
	(130.79,110.19) --
	(131.02,110.33) --
	(131.25,110.47) --
	(131.48,110.60) --
	(131.72,110.74) --
	(131.95,110.88) --
	(132.18,111.02) --
	(132.41,111.16) --
	(132.65,111.30) --
	(132.88,111.44) --
	(133.11,111.58) --
	(133.34,111.71) --
	(133.58,111.85) --
	(133.81,111.99) --
	(134.04,112.13) --
	(134.27,112.27) --
	(134.51,112.41) --
	(134.74,112.55) --
	(134.97,112.68) --
	(135.20,112.82) --
	(135.44,112.96) --
	(135.67,113.10) --
	(135.90,113.24) --
	(136.13,113.38) --
	(136.37,113.52) --
	(136.60,113.65) --
	(136.83,113.79) --
	(137.06,113.93) --
	(137.30,114.07) --
	(137.53,114.21) --
	(137.76,114.35) --
	(137.99,114.49) --
	(138.23,114.62) --
	(138.46,114.76) --
	(138.69,114.90) --
	(138.92,115.04) --
	(139.16,115.18) --
	(139.39,115.32) --
	(139.62,115.46) --
	(139.85,115.60) --
	(140.09,115.73) --
	(140.32,115.87) --
	(140.55,116.01) --
	(140.78,116.15) --
	(141.02,116.29) --
	(141.25,116.43) --
	(141.48,116.57) --
	(141.71,116.70) --
	(141.95,116.84) --
	(142.18,116.98) --
	(142.41,117.12) --
	(142.64,117.26) --
	(142.88,117.40) --
	(143.11,117.54) --
	(143.34,117.67) --
	(143.57,117.81) --
	(143.81,117.95) --
	(144.04,118.09) --
	(144.27,118.23) --
	(144.50,118.37) --
	(144.74,118.51) --
	(144.97,118.65) --
	(145.20,118.78) --
	(145.43,118.92) --
	(145.67,119.06) --
	(145.90,119.20) --
	(146.13,119.34) --
	(146.36,119.48) --
	(146.60,119.62) --
	(146.83,119.75) --
	(147.06,119.89) --
	(147.29,120.03) --
	(147.52,120.17) --
	(147.76,120.31) --
	(147.99,120.45) --
	(148.22,120.59) --
	(148.45,120.72) --
	(148.69,120.86) --
	(148.92,121.00) --
	(149.15,121.14) --
	(149.38,121.28) --
	(149.62,121.42) --
	(149.85,121.56) --
	(150.08,121.70) --
	(150.31,121.83) --
	(150.55,121.97) --
	(150.78,122.11) --
	(151.01,122.25) --
	(151.24,122.39) --
	(151.48,122.53) --
	(151.71,122.67) --
	(151.94,122.80) --
	(152.17,122.94) --
	(152.41,123.08) --
	(152.64,123.22) --
	(152.87,123.36) --
	(153.10,123.50) --
	(153.34,123.64) --
	(153.57,123.77) --
	(153.80,123.91) --
	(154.03,124.05) --
	(154.27,124.19) --
	(154.50,124.33) --
	(154.73,124.47) --
	(154.96,124.61) --
	(155.20,124.74) --
	(155.43,124.88) --
	(155.66,125.02) --
	(155.89,125.16) --
	(156.13,125.30) --
	(156.36,125.44) --
	(156.59,125.58) --
	(156.82,125.72) --
	(157.06,125.85) --
	(157.29,125.99) --
	(157.52,126.13) --
	(157.75,126.27) --
	(157.99,126.41) --
	(158.22,126.55) --
	(158.45,126.69) --
	(158.68,126.82) --
	(158.92,126.96) --
	(159.15,127.10) --
	(159.38,127.24) --
	(159.61,127.38) --
	(159.85,127.52) --
	(160.08,127.66) --
	(160.31,127.79) --
	(160.54,127.93) --
	(160.78,128.07) --
	(161.01,128.21) --
	(161.24,128.35) --
	(161.47,128.49) --
	(161.71,128.63) --
	(161.94,128.77) --
	(162.17,128.90) --
	(162.40,129.04) --
	(162.64,129.18) --
	(162.87,129.32) --
	(163.10,129.46) --
	(163.33,129.60) --
	(163.57,129.74) --
	(163.80,129.87) --
	(164.03,130.01) --
	(164.26,130.15) --
	(164.50,130.29) --
	(164.73,130.43) --
	(164.96,130.57) --
	(165.19,130.71) --
	(165.42,130.84) --
	(165.66,130.98) --
	(165.89,131.12) --
	(166.12,131.26) --
	(166.35,131.40) --
	(166.59,131.54) --
	(166.82,131.68) --
	(167.05,131.81) --
	(167.28,131.95) --
	(167.52,132.09) --
	(167.75,132.23) --
	(167.98,132.37) --
	(168.21,132.51) --
	(168.45,132.65) --
	(168.68,132.79) --
	(168.91,132.92) --
	(169.14,133.06) --
	(169.38,133.20) --
	(169.61,133.34) --
	(169.84,133.48) --
	(170.07,133.62) --
	(170.31,133.76) --
	(170.54,133.89) --
	(170.77,134.03) --
	(171.00,134.17) --
	(171.24,134.31) --
	(171.47,134.45) --
	(171.70,134.59) --
	(171.93,134.73) --
	(172.17,134.86) --
	(172.40,135.00) --
	(172.63,135.14) --
	(172.86,135.28) --
	(173.10,135.42) --
	(173.33,135.56) --
	(173.56,135.70) --
	(173.79,135.84) --
	(174.03,135.97) --
	(174.26,136.11) --
	(174.49,136.25) --
	(174.72,136.39) --
	(174.96,136.53) --
	(175.19,136.67) --
	(175.42,136.81) --
	(175.65,136.94) --
	(175.89,137.08) --
	(176.12,137.22) --
	(176.35,137.36) --
	(176.58,137.50) --
	(176.82,137.64) --
	(177.05,137.78) --
	(177.28,137.91) --
	(177.51,138.05) --
	(177.75,138.19) --
	(177.98,138.33) --
	(178.21,138.47) --
	(178.44,138.61) --
	(178.68,138.75) --
	(178.91,138.88) --
	(179.14,139.02) --
	(179.37,139.16) --
	(179.61,139.30) --
	(179.84,139.44) --
	(180.07,139.58) --
	(180.30,139.72) --
	(180.54,139.86) --
	(180.77,139.99) --
	(181.00,140.13) --
	(181.23,140.27) --
	(181.47,140.41) --
	(181.70,140.55) --
	(181.93,140.69) --
	(182.16,140.83) --
	(182.40,140.96) --
	(182.63,141.10) --
	(182.86,141.24) --
	(183.09,141.38) --
	(183.33,141.52) --
	(183.56,141.66) --
	(183.79,141.80) --
	(184.02,141.93) --
	(184.25,142.07) --
	(184.49,142.21) --
	(184.72,142.35) --
	(184.95,142.49) --
	(185.18,142.63) --
	(185.42,142.77) --
	(185.65,142.91) --
	(185.88,143.04) --
	(186.11,143.18) --
	(186.35,143.32) --
	(186.58,143.46) --
	(186.81,143.60) --
	(187.04,143.74) --
	(187.28,143.88) --
	(187.51,144.01) --
	(187.74,144.15) --
	(187.97,144.29) --
	(188.21,144.43) --
	(188.44,144.57) --
	(188.67,144.71) --
	(188.90,144.85) --
	(189.14,144.98) --
	(189.37,145.12) --
	(189.60,145.26) --
	(189.83,145.40) --
	(190.07,145.54) --
	(190.30,145.68) --
	(190.53,145.82) --
	(190.76,145.96) --
	(191.00,146.09) --
	(191.23,146.23) --
	(191.46,146.37) --
	(191.69,146.51) --
	(191.93,146.65) --
	(192.16,146.79) --
	(192.39,146.93) --
	(192.62,147.06) --
	(192.86,147.20) --
	(193.09,147.34) --
	(193.32,147.48) --
	(193.55,147.62) --
	(193.79,147.76) --
	(194.02,147.90) --
	(194.25,148.03) --
	(194.48,148.17) --
	(194.72,148.31) --
	(194.95,148.45) --
	(195.18,148.59) --
	(195.41,148.73) --
	(195.65,148.87) --
	(195.88,149.00) --
	(196.11,149.14) --
	(196.34,149.28) --
	(196.58,149.42) --
	(196.81,149.56) --
	(197.04,149.70) --
	(197.27,149.84) --
	(197.51,149.98) --
	(197.74,150.11) --
	(197.97,150.25) --
	(198.20,150.39) --
	(198.44,150.53) --
	(198.67,150.67) --
	(198.90,150.81) --
	(199.13,150.95) --
	(199.37,151.08) --
	(199.60,151.22) --
	(199.83,151.36) --
	(200.06,151.50) --
	(200.30,151.64) --
	(200.53,151.78) --
	(200.76,151.92) --
	(200.99,152.05) --
	(201.23,152.19) --
	(201.46,152.33) --
	(201.69,152.47) --
	(201.92,152.61) --
	(202.16,152.75) --
	(202.39,152.89) --
	(202.62,153.03) --
	(202.85,153.16) --
	(203.08,153.30) --
	(203.32,153.44) --
	(203.55,153.58) --
	(203.78,153.72) --
	(204.01,153.86) --
	(204.25,154.00) --
	(204.48,154.13) --
	(204.71,154.27) --
	(204.94,154.41) --
	(205.18,154.55) --
	(205.41,154.69) --
	(205.64,154.83) --
	(205.87,154.97) --
	(206.11,155.10) --
	(206.34,155.24) --
	(206.57,155.38) --
	(206.80,155.52) --
	(207.04,155.66) --
	(207.27,155.80) --
	(207.50,155.94) --
	(207.73,156.07) --
	(207.97,156.21) --
	(208.20,156.35) --
	(208.43,156.49) --
	(208.66,156.63) --
	(208.90,156.77) --
	(209.13,156.91) --
	(209.36,157.05) --
	(209.59,157.18) --
	(209.83,157.32) --
	(210.06,157.46) --
	(210.29,157.60) --
	(210.52,157.74) --
	(210.76,157.88) --
	(210.99,158.02) --
	(211.22,158.15) --
	(211.45,158.29) --
	(211.69,158.43) --
	(211.92,158.57) --
	(212.15,158.71) --
	(212.38,158.85) --
	(212.62,158.99) --
	(212.85,159.12) --
	(213.08,159.26) --
	(213.31,159.40) --
	(213.55,159.54) --
	(213.78,159.68) --
	(214.01,159.82) --
	(214.24,159.96) --
	(214.48,160.10) --
	(214.71,160.23) --
	(214.94,160.37) --
	(215.17,160.51) --
	(215.41,160.65) --
	(215.64,160.79) --
	(215.87,160.93) --
	(216.10,161.07) --
	(216.34,161.20) --
	(216.57,161.34) --
	(216.80,161.48) --
	(217.03,161.62) --
	(217.27,161.76) --
	(217.50,161.90) --
	(217.73,162.04) --
	(217.96,162.17) --
	(218.20,162.31) --
	(218.43,162.45) --
	(218.66,162.59) --
	(218.89,162.73) --
	(219.13,162.87) --
	(219.36,163.01) --
	(219.59,163.14) --
	(219.82,163.28) --
	(220.06,163.42) --
	(220.29,163.56) --
	(220.52,163.70) --
	(220.75,163.84) --
	(220.98,163.98) --
	(221.22,164.12) --
	(221.45,164.25) --
	(221.68,164.39) --
	(221.91,164.53) --
	(222.15,164.67) --
	(222.38,164.81) --
	(222.61,164.95) --
	(222.84,165.09) --
	(223.08,165.22) --
	(223.31,165.36) --
	(223.54,165.50) --
	(223.77,165.64) --
	(224.01,165.78) --
	(224.24,165.92) --
	(224.47,166.06) --
	(224.70,166.19) --
	(224.94,166.33) --
	(225.17,166.47) --
	(225.40,166.61) --
	(225.63,166.75) --
	(225.87,166.89) --
	(226.10,167.03) --
	(226.33,167.17) --
	(226.56,167.30) --
	(226.80,167.44) --
	(227.03,167.58) --
	(227.26,167.72) --
	(227.49,167.86) --
	(227.73,168.00) --
	(227.96,168.14) --
	(228.19,168.27) --
	(228.42,168.41) --
	(228.66,168.55) --
	(228.89,168.69) --
	(229.12,168.83) --
	(229.35,168.97) --
	(229.59,169.11) --
	(229.82,169.24) --
	(230.05,169.38) --
	(230.28,169.52) --
	(230.52,169.66) --
	(230.75,169.80) --
	(230.98,169.94) --
	(231.21,170.08) --
	(231.45,170.22) --
	(231.68,170.35) --
	(231.91,170.49) --
	(232.14,170.63) --
	(232.38,170.77) --
	(232.61,170.91) --
	(232.84,171.05) --
	(233.07,171.19) --
	(233.31,171.32) --
	(233.54,171.46) --
	(233.77,171.60) --
	(234.00,171.74) --
	(234.24,171.88) --
	(234.47,172.02) --
	(234.70,172.16) --
	(234.93,172.29) --
	(235.17,172.43) --
	(235.40,172.57) --
	(235.63,172.71) --
	(235.86,172.85) --
	(236.10,172.99) --
	(236.33,173.13) --
	(236.56,173.26) --
	(236.79,173.40) --
	(237.03,173.54) --
	(237.26,173.68) --
	(237.49,173.82) --
	(237.72,173.96) --
	(237.96,174.10) --
	(238.19,174.24) --
	(238.42,174.37) --
	(238.65,174.51) --
	(238.89,174.65) --
	(239.12,174.79) --
	(239.35,174.93) --
	(239.58,175.07) --
	(239.81,175.21) --
	(240.05,175.34) --
	(240.28,175.48) --
	(240.51,175.62) --
	(240.74,175.76) --
	(240.98,175.90) --
	(241.21,176.04) --
	(241.44,176.18) --
	(241.67,176.31) --
	(241.91,176.45) --
	(242.14,176.59) --
	(242.37,176.73) --
	(242.60,176.87) --
	(242.84,177.01) --
	(243.07,177.15) --
	(243.30,177.29) --
	(243.53,177.42) --
	(243.77,177.56) --
	(244.00,177.70) --
	(244.23,177.84) --
	(244.46,177.98) --
	(244.70,178.12) --
	(244.93,178.26) --
	(245.16,178.39) --
	(245.39,178.53) --
	(245.63,178.67) --
	(245.86,178.81) --
	(246.09,178.95) --
	(246.32,179.09) --
	(246.56,179.23) --
	(246.79,179.36) --
	(247.02,179.50) --
	(247.25,179.64) --
	(247.49,179.78) --
	(247.72,179.92) --
	(247.95,180.06) --
	(248.18,180.20) --
	(248.42,180.33) --
	(248.65,180.47) --
	(248.88,180.61) --
	(249.11,180.75) --
	(249.35,180.89) --
	(249.58,181.03) --
	(249.81,181.17) --
	(250.04,181.31) --
	(250.28,181.44) --
	(250.51,181.58) --
	(250.74,181.72) --
	(250.97,181.86) --
	(251.21,182.00) --
	(251.44,182.14) --
	(251.67,182.28) --
	(251.90,182.41) --
	(252.14,182.55) --
	(252.37,182.69) --
	(252.60,182.83) --
	(252.83,182.97) --
	(253.07,183.11) --
	(253.30,183.25) --
	(253.53,183.38) --
	(253.76,183.52) --
	(254.00,183.66) --
	(254.23,183.80) --
	(254.46,183.94) --
	(254.69,184.08) --
	(254.93,184.22) --
	(255.16,184.36) --
	(255.39,184.49) --
	(255.62,184.63) --
	(255.86,184.77) --
	(256.09,184.91) --
	(256.32,185.05) --
	(256.55,185.19) --
	(256.79,185.33) --
	(257.02,185.46) --
	(257.25,185.60) --
	(257.48,185.74) --
	(257.72,185.88) --
	(257.95,186.02) --
	(258.18,186.16) --
	(258.41,186.30) --
	(258.64,186.43) --
	(258.88,186.57) --
	(259.11,186.71) --
	(259.34,186.85) --
	(259.57,186.99) --
	(259.81,187.13) --
	(260.04,187.27) --
	(260.27,187.40) --
	(260.50,187.54) --
	(260.74,187.68) --
	(260.97,187.82) --
	(261.20,187.96) --
	(261.43,188.10) --
	(261.67,188.24) --
	(261.90,188.38) --
	(262.13,188.51) --
	(262.36,188.65) --
	(262.60,188.79) --
	(262.83,188.93) --
	(263.06,189.07) --
	(263.29,189.21) --
	(263.53,189.35) --
	(263.76,189.48) --
	(263.99,189.62) --
	(264.22,189.76) --
	(264.46,189.90) --
	(264.69,190.04) --
	(264.92,190.18) --
	(265.15,190.32) --
	(265.39,190.45) --
	(265.62,190.59) --
	(265.85,190.73) --
	(266.08,190.87) --
	(266.32,191.01) --
	(266.55,191.15) --
	(266.78,191.29) --
	(267.01,191.43) --
	(267.25,191.56) --
	(267.48,191.70) --
	(267.71,191.84) --
	(267.94,191.98) --
	(268.18,192.12) --
	(268.41,192.26) --
	(268.64,192.40) --
	(268.87,192.53) --
	(269.11,192.67) --
	(269.34,192.81) --
	(269.57,192.95) --
	(269.80,193.09) --
	(270.04,193.23) --
	(270.27,193.37) --
	(270.50,193.50) --
	(270.73,193.64) --
	(270.97,193.78) --
	(271.20,193.92) --
	(271.43,194.06) --
	(271.66,194.20) --
	(271.90,194.34) --
	(272.13,194.48) --
	(272.36,194.61) --
	(272.59,194.75) --
	(272.83,194.89) --
	(273.06,195.03) --
	(273.29,195.17) --
	(273.52,195.31) --
	(273.76,195.45) --
	(273.99,195.58) --
	(274.22,195.72) --
	(274.45,195.86) --
	(274.69,196.00) --
	(274.92,196.14) --
	(275.15,196.28) --
	(275.38,196.42) --
	(275.62,196.55) --
	(275.85,196.69) --
	(276.08,196.83) --
	(276.31,196.97) --
	(276.54,197.11) --
	(276.78,197.25) --
	(277.01,197.39) --
	(277.24,197.52) --
	(277.47,197.66) --
	(277.71,197.80) --
	(277.94,197.94) --
	(278.17,198.08) --
	(278.40,198.22) --
	(278.64,198.36) --
	(278.87,198.50) --
	(279.10,198.63) --
	(279.33,198.77) --
	(279.57,198.91) --
	(279.80,199.05) --
	(280.03,199.19) --
	(280.26,199.33) --
	(280.50,199.47) --
	(280.73,199.60) --
	(280.96,199.74) --
	(281.19,199.88) --
	(281.43,200.02) --
	(281.66,200.16) --
	(281.89,200.30) --
	(282.12,200.44) --
	(282.36,200.57) --
	(282.59,200.71) --
	(282.82,200.85) --
	(283.05,200.99) --
	(283.29,201.13) --
	(283.52,201.27) --
	(283.75,201.41) --
	(283.98,201.55) --
	(284.22,201.68) --
	(284.45,201.82) --
	(284.68,201.96) --
	(284.91,202.10) --
	(285.15,202.24) --
	(285.38,202.38) --
	(285.61,202.52) --
	(285.84,202.65) --
	(286.08,202.79) --
	(286.31,202.93) --
	(286.54,203.07) --
	(286.77,203.21) --
	(287.01,203.35) --
	(287.24,203.49) --
	(287.47,203.62) --
	(287.70,203.76) --
	(287.94,203.90) --
	(288.17,204.04) --
	(288.40,204.18) --
	(288.63,204.32) --
	(288.87,204.46) --
	(289.10,204.59) --
	(289.33,204.73) --
	(289.56,204.87) --
	(289.80,205.01) --
	(290.03,205.15) --
	(290.26,205.29) --
	(290.49,205.43) --
	(290.73,205.57);

\path[draw=drawColor,line width= 0.4pt,dash pattern=on 4pt off 4pt ,line join=round,line cap=round] ( 58.49, 39.10) --
	( 58.72, 39.24) --
	( 58.95, 39.37) --
	( 59.19, 39.51) --
	( 59.42, 39.65) --
	( 59.65, 39.79) --
	( 59.88, 39.93) --
	( 60.12, 40.07) --
	( 60.35, 40.21) --
	( 60.58, 40.35) --
	( 60.81, 40.48) --
	( 61.05, 40.62) --
	( 61.28, 40.76) --
	( 61.51, 40.90) --
	( 61.74, 41.04) --
	( 61.98, 41.18) --
	( 62.21, 41.32) --
	( 62.44, 41.45) --
	( 62.67, 41.59) --
	( 62.91, 41.73) --
	( 63.14, 41.87) --
	( 63.37, 42.01) --
	( 63.60, 42.15) --
	( 63.84, 42.29) --
	( 64.07, 42.42) --
	( 64.30, 42.56) --
	( 64.53, 42.70) --
	( 64.77, 42.84) --
	( 65.00, 42.98) --
	( 65.23, 43.12) --
	( 65.46, 43.26) --
	( 65.70, 43.40) --
	( 65.93, 43.53) --
	( 66.16, 43.67) --
	( 66.39, 43.81) --
	( 66.63, 43.95) --
	( 66.86, 44.09) --
	( 67.09, 44.23) --
	( 67.32, 44.37) --
	( 67.56, 44.50) --
	( 67.79, 44.64) --
	( 68.02, 44.78) --
	( 68.25, 44.92) --
	( 68.49, 45.06) --
	( 68.72, 45.20) --
	( 68.95, 45.34) --
	( 69.18, 45.47) --
	( 69.42, 45.61) --
	( 69.65, 45.75) --
	( 69.88, 45.89) --
	( 70.11, 46.03) --
	( 70.35, 46.17) --
	( 70.58, 46.31) --
	( 70.81, 46.44) --
	( 71.04, 46.58) --
	( 71.28, 46.72) --
	( 71.51, 46.86) --
	( 71.74, 47.00) --
	( 71.97, 47.14) --
	( 72.21, 47.28) --
	( 72.44, 47.42) --
	( 72.67, 47.55) --
	( 72.90, 47.69) --
	( 73.13, 47.83) --
	( 73.37, 47.97) --
	( 73.60, 48.11) --
	( 73.83, 48.25) --
	( 74.06, 48.39) --
	( 74.30, 48.52) --
	( 74.53, 48.66) --
	( 74.76, 48.80) --
	( 74.99, 48.94) --
	( 75.23, 49.08) --
	( 75.46, 49.22) --
	( 75.69, 49.36) --
	( 75.92, 49.49) --
	( 76.16, 49.63) --
	( 76.39, 49.77) --
	( 76.62, 49.91) --
	( 76.85, 50.05) --
	( 77.09, 50.19) --
	( 77.32, 50.33) --
	( 77.55, 50.47) --
	( 77.78, 50.60) --
	( 78.02, 50.74) --
	( 78.25, 50.88) --
	( 78.48, 51.02) --
	( 78.71, 51.16) --
	( 78.95, 51.30) --
	( 79.18, 51.44) --
	( 79.41, 51.57) --
	( 79.64, 51.71) --
	( 79.88, 51.85) --
	( 80.11, 51.99) --
	( 80.34, 52.13) --
	( 80.57, 52.27) --
	( 80.81, 52.41) --
	( 81.04, 52.54) --
	( 81.27, 52.68) --
	( 81.50, 52.82) --
	( 81.74, 52.96) --
	( 81.97, 53.10) --
	( 82.20, 53.24) --
	( 82.43, 53.38) --
	( 82.67, 53.51) --
	( 82.90, 53.65) --
	( 83.13, 53.79) --
	( 83.36, 53.93) --
	( 83.60, 54.07) --
	( 83.83, 54.21) --
	( 84.06, 54.35) --
	( 84.29, 54.49) --
	( 84.53, 54.62) --
	( 84.76, 54.76) --
	( 84.99, 54.90) --
	( 85.22, 55.04) --
	( 85.46, 55.18) --
	( 85.69, 55.32) --
	( 85.92, 55.46) --
	( 86.15, 55.59) --
	( 86.39, 55.73) --
	( 86.62, 55.87) --
	( 86.85, 56.01) --
	( 87.08, 56.15) --
	( 87.32, 56.29) --
	( 87.55, 56.43) --
	( 87.78, 56.56) --
	( 88.01, 56.70) --
	( 88.25, 56.84) --
	( 88.48, 56.98) --
	( 88.71, 57.12) --
	( 88.94, 57.26) --
	( 89.18, 57.40) --
	( 89.41, 57.54) --
	( 89.64, 57.67) --
	( 89.87, 57.81) --
	( 90.11, 57.95) --
	( 90.34, 58.09) --
	( 90.57, 58.23) --
	( 90.80, 58.37) --
	( 91.04, 58.51) --
	( 91.27, 58.64) --
	( 91.50, 58.78) --
	( 91.73, 58.92) --
	( 91.96, 59.06) --
	( 92.20, 59.20) --
	( 92.43, 59.34) --
	( 92.66, 59.48) --
	( 92.89, 59.61) --
	( 93.13, 59.75) --
	( 93.36, 59.89) --
	( 93.59, 60.03) --
	( 93.82, 60.17) --
	( 94.06, 60.31) --
	( 94.29, 60.45) --
	( 94.52, 60.58) --
	( 94.75, 60.72) --
	( 94.99, 60.86) --
	( 95.22, 61.00) --
	( 95.45, 61.14) --
	( 95.68, 61.28) --
	( 95.92, 61.42) --
	( 96.15, 61.56) --
	( 96.38, 61.69) --
	( 96.61, 61.83) --
	( 96.85, 61.97) --
	( 97.08, 62.11) --
	( 97.31, 62.25) --
	( 97.54, 62.39) --
	( 97.78, 62.53) --
	( 98.01, 62.66) --
	( 98.24, 62.80) --
	( 98.47, 62.94) --
	( 98.71, 63.08) --
	( 98.94, 63.22) --
	( 99.17, 63.36) --
	( 99.40, 63.50) --
	( 99.64, 63.63) --
	( 99.87, 63.77) --
	(100.10, 63.91) --
	(100.33, 64.05) --
	(100.57, 64.19) --
	(100.80, 64.33) --
	(101.03, 64.47) --
	(101.26, 64.61) --
	(101.50, 64.74) --
	(101.73, 64.88) --
	(101.96, 65.02) --
	(102.19, 65.16) --
	(102.43, 65.30) --
	(102.66, 65.44) --
	(102.89, 65.58) --
	(103.12, 65.71) --
	(103.36, 65.85) --
	(103.59, 65.99) --
	(103.82, 66.13) --
	(104.05, 66.27) --
	(104.29, 66.41) --
	(104.52, 66.55) --
	(104.75, 66.68) --
	(104.98, 66.82) --
	(105.22, 66.96) --
	(105.45, 67.10) --
	(105.68, 67.24) --
	(105.91, 67.38) --
	(106.15, 67.52) --
	(106.38, 67.66) --
	(106.61, 67.79) --
	(106.84, 67.93) --
	(107.08, 68.07) --
	(107.31, 68.21) --
	(107.54, 68.35) --
	(107.77, 68.49) --
	(108.01, 68.63) --
	(108.24, 68.76) --
	(108.47, 68.90) --
	(108.70, 69.04) --
	(108.94, 69.18) --
	(109.17, 69.32) --
	(109.40, 69.46) --
	(109.63, 69.60) --
	(109.87, 69.73) --
	(110.10, 69.87) --
	(110.33, 70.01) --
	(110.56, 70.15) --
	(110.79, 70.29) --
	(111.03, 70.43) --
	(111.26, 70.57) --
	(111.49, 70.70) --
	(111.72, 70.84) --
	(111.96, 70.98) --
	(112.19, 71.12) --
	(112.42, 71.26) --
	(112.65, 71.40) --
	(112.89, 71.54) --
	(113.12, 71.68) --
	(113.35, 71.81) --
	(113.58, 71.95) --
	(113.82, 72.09) --
	(114.05, 72.23) --
	(114.28, 72.37) --
	(114.51, 72.51) --
	(114.75, 72.65) --
	(114.98, 72.78) --
	(115.21, 72.92) --
	(115.44, 73.06) --
	(115.68, 73.20) --
	(115.91, 73.34) --
	(116.14, 73.48) --
	(116.37, 73.62) --
	(116.61, 73.75) --
	(116.84, 73.89) --
	(117.07, 74.03) --
	(117.30, 74.17) --
	(117.54, 74.31) --
	(117.77, 74.45) --
	(118.00, 74.59) --
	(118.23, 74.73) --
	(118.47, 74.86) --
	(118.70, 75.00) --
	(118.93, 75.14) --
	(119.16, 75.28) --
	(119.40, 75.42) --
	(119.63, 75.56) --
	(119.86, 75.70) --
	(120.09, 75.83) --
	(120.33, 75.97) --
	(120.56, 76.11) --
	(120.79, 76.25) --
	(121.02, 76.39) --
	(121.26, 76.53) --
	(121.49, 76.67) --
	(121.72, 76.80) --
	(121.95, 76.94) --
	(122.19, 77.08) --
	(122.42, 77.22) --
	(122.65, 77.36) --
	(122.88, 77.50) --
	(123.12, 77.64) --
	(123.35, 77.77) --
	(123.58, 77.91) --
	(123.81, 78.05) --
	(124.05, 78.19) --
	(124.28, 78.33) --
	(124.51, 78.47) --
	(124.74, 78.61) --
	(124.98, 78.75) --
	(125.21, 78.88) --
	(125.44, 79.02) --
	(125.67, 79.16) --
	(125.91, 79.30) --
	(126.14, 79.44) --
	(126.37, 79.58) --
	(126.60, 79.72) --
	(126.84, 79.85) --
	(127.07, 79.99) --
	(127.30, 80.13) --
	(127.53, 80.27) --
	(127.77, 80.41) --
	(128.00, 80.55) --
	(128.23, 80.69) --
	(128.46, 80.82) --
	(128.69, 80.96) --
	(128.93, 81.10) --
	(129.16, 81.24) --
	(129.39, 81.38) --
	(129.62, 81.52) --
	(129.86, 81.66) --
	(130.09, 81.80) --
	(130.32, 81.93) --
	(130.55, 82.07) --
	(130.79, 82.21) --
	(131.02, 82.35) --
	(131.25, 82.49) --
	(131.48, 82.63) --
	(131.72, 82.77) --
	(131.95, 82.90) --
	(132.18, 83.04) --
	(132.41, 83.18) --
	(132.65, 83.32) --
	(132.88, 83.46) --
	(133.11, 83.60) --
	(133.34, 83.74) --
	(133.58, 83.87) --
	(133.81, 84.01) --
	(134.04, 84.15) --
	(134.27, 84.29) --
	(134.51, 84.43) --
	(134.74, 84.57) --
	(134.97, 84.71) --
	(135.20, 84.85) --
	(135.44, 84.98) --
	(135.67, 85.12) --
	(135.90, 85.26) --
	(136.13, 85.40) --
	(136.37, 85.54) --
	(136.60, 85.68) --
	(136.83, 85.82) --
	(137.06, 85.95) --
	(137.30, 86.09) --
	(137.53, 86.23) --
	(137.76, 86.37) --
	(137.99, 86.51) --
	(138.23, 86.65) --
	(138.46, 86.79) --
	(138.69, 86.92) --
	(138.92, 87.06) --
	(139.16, 87.20) --
	(139.39, 87.34) --
	(139.62, 87.48) --
	(139.85, 87.62) --
	(140.09, 87.76) --
	(140.32, 87.89) --
	(140.55, 88.03) --
	(140.78, 88.17) --
	(141.02, 88.31) --
	(141.25, 88.45) --
	(141.48, 88.59) --
	(141.71, 88.73) --
	(141.95, 88.87) --
	(142.18, 89.00) --
	(142.41, 89.14) --
	(142.64, 89.28) --
	(142.88, 89.42) --
	(143.11, 89.56) --
	(143.34, 89.70) --
	(143.57, 89.84) --
	(143.81, 89.97) --
	(144.04, 90.11) --
	(144.27, 90.25) --
	(144.50, 90.39) --
	(144.74, 90.53) --
	(144.97, 90.67) --
	(145.20, 90.81) --
	(145.43, 90.94) --
	(145.67, 91.08) --
	(145.90, 91.22) --
	(146.13, 91.36) --
	(146.36, 91.50) --
	(146.60, 91.64) --
	(146.83, 91.78) --
	(147.06, 91.92) --
	(147.29, 92.05) --
	(147.52, 92.19) --
	(147.76, 92.33) --
	(147.99, 92.47) --
	(148.22, 92.61) --
	(148.45, 92.75) --
	(148.69, 92.89) --
	(148.92, 93.02) --
	(149.15, 93.16) --
	(149.38, 93.30) --
	(149.62, 93.44) --
	(149.85, 93.58) --
	(150.08, 93.72) --
	(150.31, 93.86) --
	(150.55, 93.99) --
	(150.78, 94.13) --
	(151.01, 94.27) --
	(151.24, 94.41) --
	(151.48, 94.55) --
	(151.71, 94.69) --
	(151.94, 94.83) --
	(152.17, 94.96) --
	(152.41, 95.10) --
	(152.64, 95.24) --
	(152.87, 95.38) --
	(153.10, 95.52) --
	(153.34, 95.66) --
	(153.57, 95.80) --
	(153.80, 95.94) --
	(154.03, 96.07) --
	(154.27, 96.21) --
	(154.50, 96.35) --
	(154.73, 96.49) --
	(154.96, 96.63) --
	(155.20, 96.77) --
	(155.43, 96.91) --
	(155.66, 97.04) --
	(155.89, 97.18) --
	(156.13, 97.32) --
	(156.36, 97.46) --
	(156.59, 97.60) --
	(156.82, 97.74) --
	(157.06, 97.88) --
	(157.29, 98.01) --
	(157.52, 98.15) --
	(157.75, 98.29) --
	(157.99, 98.43) --
	(158.22, 98.57) --
	(158.45, 98.71) --
	(158.68, 98.85) --
	(158.92, 98.99) --
	(159.15, 99.12) --
	(159.38, 99.26) --
	(159.61, 99.40) --
	(159.85, 99.54) --
	(160.08, 99.68) --
	(160.31, 99.82) --
	(160.54, 99.96) --
	(160.78,100.09) --
	(161.01,100.23) --
	(161.24,100.37) --
	(161.47,100.51) --
	(161.71,100.65) --
	(161.94,100.79) --
	(162.17,100.93) --
	(162.40,101.06) --
	(162.64,101.20) --
	(162.87,101.34) --
	(163.10,101.48) --
	(163.33,101.62) --
	(163.57,101.76) --
	(163.80,101.90) --
	(164.03,102.03) --
	(164.26,102.17) --
	(164.50,102.31) --
	(164.73,102.45) --
	(164.96,102.59) --
	(165.19,102.73) --
	(165.42,102.87) --
	(165.66,103.01) --
	(165.89,103.14) --
	(166.12,103.28) --
	(166.35,103.42) --
	(166.59,103.56) --
	(166.82,103.70) --
	(167.05,103.84) --
	(167.28,103.98) --
	(167.52,104.11) --
	(167.75,104.25) --
	(167.98,104.39) --
	(168.21,104.53) --
	(168.45,104.67) --
	(168.68,104.81) --
	(168.91,104.95) --
	(169.14,105.08) --
	(169.38,105.22) --
	(169.61,105.36) --
	(169.84,105.50) --
	(170.07,105.64) --
	(170.31,105.78) --
	(170.54,105.92) --
	(170.77,106.06) --
	(171.00,106.19) --
	(171.24,106.33) --
	(171.47,106.47) --
	(171.70,106.61) --
	(171.93,106.75) --
	(172.17,106.89) --
	(172.40,107.03) --
	(172.63,107.16) --
	(172.86,107.30) --
	(173.10,107.44) --
	(173.33,107.58) --
	(173.56,107.72) --
	(173.79,107.86) --
	(174.03,108.00) --
	(174.26,108.13) --
	(174.49,108.27) --
	(174.72,108.41) --
	(174.96,108.55) --
	(175.19,108.69) --
	(175.42,108.83) --
	(175.65,108.97) --
	(175.89,109.11) --
	(176.12,109.24) --
	(176.35,109.38) --
	(176.58,109.52) --
	(176.82,109.66) --
	(177.05,109.80) --
	(177.28,109.94) --
	(177.51,110.08) --
	(177.75,110.21) --
	(177.98,110.35) --
	(178.21,110.49) --
	(178.44,110.63) --
	(178.68,110.77) --
	(178.91,110.91) --
	(179.14,111.05) --
	(179.37,111.18) --
	(179.61,111.32) --
	(179.84,111.46) --
	(180.07,111.60) --
	(180.30,111.74) --
	(180.54,111.88) --
	(180.77,112.02) --
	(181.00,112.15) --
	(181.23,112.29) --
	(181.47,112.43) --
	(181.70,112.57) --
	(181.93,112.71) --
	(182.16,112.85) --
	(182.40,112.99) --
	(182.63,113.13) --
	(182.86,113.26) --
	(183.09,113.40) --
	(183.33,113.54) --
	(183.56,113.68) --
	(183.79,113.82) --
	(184.02,113.96) --
	(184.25,114.10) --
	(184.49,114.23) --
	(184.72,114.37) --
	(184.95,114.51) --
	(185.18,114.65) --
	(185.42,114.79) --
	(185.65,114.93) --
	(185.88,115.07) --
	(186.11,115.20) --
	(186.35,115.34) --
	(186.58,115.48) --
	(186.81,115.62) --
	(187.04,115.76) --
	(187.28,115.90) --
	(187.51,116.04) --
	(187.74,116.18) --
	(187.97,116.31) --
	(188.21,116.45) --
	(188.44,116.59) --
	(188.67,116.73) --
	(188.90,116.87) --
	(189.14,117.01) --
	(189.37,117.15) --
	(189.60,117.28) --
	(189.83,117.42) --
	(190.07,117.56) --
	(190.30,117.70) --
	(190.53,117.84) --
	(190.76,117.98) --
	(191.00,118.12) --
	(191.23,118.25) --
	(191.46,118.39) --
	(191.69,118.53) --
	(191.93,118.67) --
	(192.16,118.81) --
	(192.39,118.95) --
	(192.62,119.09) --
	(192.86,119.22) --
	(193.09,119.36) --
	(193.32,119.50) --
	(193.55,119.64) --
	(193.79,119.78) --
	(194.02,119.92) --
	(194.25,120.06) --
	(194.48,120.20) --
	(194.72,120.33) --
	(194.95,120.47) --
	(195.18,120.61) --
	(195.41,120.75) --
	(195.65,120.89) --
	(195.88,121.03) --
	(196.11,121.17) --
	(196.34,121.30) --
	(196.58,121.44) --
	(196.81,121.58) --
	(197.04,121.72) --
	(197.27,121.86) --
	(197.51,122.00) --
	(197.74,122.14) --
	(197.97,122.27) --
	(198.20,122.41) --
	(198.44,122.55) --
	(198.67,122.69) --
	(198.90,122.83) --
	(199.13,122.97) --
	(199.37,123.11) --
	(199.60,123.25) --
	(199.83,123.38) --
	(200.06,123.52) --
	(200.30,123.66) --
	(200.53,123.80) --
	(200.76,123.94) --
	(200.99,124.08) --
	(201.23,124.22) --
	(201.46,124.35) --
	(201.69,124.49) --
	(201.92,124.63) --
	(202.16,124.77) --
	(202.39,124.91) --
	(202.62,125.05) --
	(202.85,125.19) --
	(203.08,125.32) --
	(203.32,125.46) --
	(203.55,125.60) --
	(203.78,125.74) --
	(204.01,125.88) --
	(204.25,126.02) --
	(204.48,126.16) --
	(204.71,126.29) --
	(204.94,126.43) --
	(205.18,126.57) --
	(205.41,126.71) --
	(205.64,126.85) --
	(205.87,126.99) --
	(206.11,127.13) --
	(206.34,127.27) --
	(206.57,127.40) --
	(206.80,127.54) --
	(207.04,127.68) --
	(207.27,127.82) --
	(207.50,127.96) --
	(207.73,128.10) --
	(207.97,128.24) --
	(208.20,128.37) --
	(208.43,128.51) --
	(208.66,128.65) --
	(208.90,128.79) --
	(209.13,128.93) --
	(209.36,129.07) --
	(209.59,129.21) --
	(209.83,129.34) --
	(210.06,129.48) --
	(210.29,129.62) --
	(210.52,129.76) --
	(210.76,129.90) --
	(210.99,130.04) --
	(211.22,130.18) --
	(211.45,130.32) --
	(211.69,130.45) --
	(211.92,130.59) --
	(212.15,130.73) --
	(212.38,130.87) --
	(212.62,131.01) --
	(212.85,131.15) --
	(213.08,131.29) --
	(213.31,131.42) --
	(213.55,131.56) --
	(213.78,131.70) --
	(214.01,131.84) --
	(214.24,131.98) --
	(214.48,132.12) --
	(214.71,132.26) --
	(214.94,132.39) --
	(215.17,132.53) --
	(215.41,132.67) --
	(215.64,132.81) --
	(215.87,132.95) --
	(216.10,133.09) --
	(216.34,133.23) --
	(216.57,133.37) --
	(216.80,133.50) --
	(217.03,133.64) --
	(217.27,133.78) --
	(217.50,133.92) --
	(217.73,134.06) --
	(217.96,134.20) --
	(218.20,134.34) --
	(218.43,134.47) --
	(218.66,134.61) --
	(218.89,134.75) --
	(219.13,134.89) --
	(219.36,135.03) --
	(219.59,135.17) --
	(219.82,135.31) --
	(220.06,135.44) --
	(220.29,135.58) --
	(220.52,135.72) --
	(220.75,135.86) --
	(220.98,136.00) --
	(221.22,136.14) --
	(221.45,136.28) --
	(221.68,136.41) --
	(221.91,136.55) --
	(222.15,136.69) --
	(222.38,136.83) --
	(222.61,136.97) --
	(222.84,137.11) --
	(223.08,137.25) --
	(223.31,137.39) --
	(223.54,137.52) --
	(223.77,137.66) --
	(224.01,137.80) --
	(224.24,137.94) --
	(224.47,138.08) --
	(224.70,138.22) --
	(224.94,138.36) --
	(225.17,138.49) --
	(225.40,138.63) --
	(225.63,138.77) --
	(225.87,138.91) --
	(226.10,139.05) --
	(226.33,139.19) --
	(226.56,139.33) --
	(226.80,139.46) --
	(227.03,139.60) --
	(227.26,139.74) --
	(227.49,139.88) --
	(227.73,140.02) --
	(227.96,140.16) --
	(228.19,140.30) --
	(228.42,140.44) --
	(228.66,140.57) --
	(228.89,140.71) --
	(229.12,140.85) --
	(229.35,140.99) --
	(229.59,141.13) --
	(229.82,141.27) --
	(230.05,141.41) --
	(230.28,141.54) --
	(230.52,141.68) --
	(230.75,141.82) --
	(230.98,141.96) --
	(231.21,142.10) --
	(231.45,142.24) --
	(231.68,142.38) --
	(231.91,142.51) --
	(232.14,142.65) --
	(232.38,142.79) --
	(232.61,142.93) --
	(232.84,143.07) --
	(233.07,143.21) --
	(233.31,143.35) --
	(233.54,143.48) --
	(233.77,143.62) --
	(234.00,143.76) --
	(234.24,143.90) --
	(234.47,144.04) --
	(234.70,144.18) --
	(234.93,144.32) --
	(235.17,144.46) --
	(235.40,144.59) --
	(235.63,144.73) --
	(235.86,144.87) --
	(236.10,145.01) --
	(236.33,145.15) --
	(236.56,145.29) --
	(236.79,145.43) --
	(237.03,145.56) --
	(237.26,145.70) --
	(237.49,145.84) --
	(237.72,145.98) --
	(237.96,146.12) --
	(238.19,146.26) --
	(238.42,146.40) --
	(238.65,146.53) --
	(238.89,146.67) --
	(239.12,146.81) --
	(239.35,146.95) --
	(239.58,147.09) --
	(239.81,147.23) --
	(240.05,147.37) --
	(240.28,147.51) --
	(240.51,147.64) --
	(240.74,147.78) --
	(240.98,147.92) --
	(241.21,148.06) --
	(241.44,148.20) --
	(241.67,148.34) --
	(241.91,148.48) --
	(242.14,148.61) --
	(242.37,148.75) --
	(242.60,148.89) --
	(242.84,149.03) --
	(243.07,149.17) --
	(243.30,149.31) --
	(243.53,149.45) --
	(243.77,149.58) --
	(244.00,149.72) --
	(244.23,149.86) --
	(244.46,150.00) --
	(244.70,150.14) --
	(244.93,150.28) --
	(245.16,150.42) --
	(245.39,150.55) --
	(245.63,150.69) --
	(245.86,150.83) --
	(246.09,150.97) --
	(246.32,151.11) --
	(246.56,151.25) --
	(246.79,151.39) --
	(247.02,151.53) --
	(247.25,151.66) --
	(247.49,151.80) --
	(247.72,151.94) --
	(247.95,152.08) --
	(248.18,152.22) --
	(248.42,152.36) --
	(248.65,152.50) --
	(248.88,152.63) --
	(249.11,152.77) --
	(249.35,152.91) --
	(249.58,153.05) --
	(249.81,153.19) --
	(250.04,153.33) --
	(250.28,153.47) --
	(250.51,153.60) --
	(250.74,153.74) --
	(250.97,153.88) --
	(251.21,154.02) --
	(251.44,154.16) --
	(251.67,154.30) --
	(251.90,154.44) --
	(252.14,154.58) --
	(252.37,154.71) --
	(252.60,154.85) --
	(252.83,154.99) --
	(253.07,155.13) --
	(253.30,155.27) --
	(253.53,155.41) --
	(253.76,155.55) --
	(254.00,155.68) --
	(254.23,155.82) --
	(254.46,155.96) --
	(254.69,156.10) --
	(254.93,156.24) --
	(255.16,156.38) --
	(255.39,156.52) --
	(255.62,156.65) --
	(255.86,156.79) --
	(256.09,156.93) --
	(256.32,157.07) --
	(256.55,157.21) --
	(256.79,157.35) --
	(257.02,157.49) --
	(257.25,157.63) --
	(257.48,157.76) --
	(257.72,157.90) --
	(257.95,158.04) --
	(258.18,158.18) --
	(258.41,158.32) --
	(258.64,158.46) --
	(258.88,158.60) --
	(259.11,158.73) --
	(259.34,158.87) --
	(259.57,159.01) --
	(259.81,159.15) --
	(260.04,159.29) --
	(260.27,159.43) --
	(260.50,159.57) --
	(260.74,159.70) --
	(260.97,159.84) --
	(261.20,159.98) --
	(261.43,160.12) --
	(261.67,160.26) --
	(261.90,160.40) --
	(262.13,160.54) --
	(262.36,160.67) --
	(262.60,160.81) --
	(262.83,160.95) --
	(263.06,161.09) --
	(263.29,161.23) --
	(263.53,161.37) --
	(263.76,161.51) --
	(263.99,161.65) --
	(264.22,161.78) --
	(264.46,161.92) --
	(264.69,162.06) --
	(264.92,162.20) --
	(265.15,162.34) --
	(265.39,162.48) --
	(265.62,162.62) --
	(265.85,162.75) --
	(266.08,162.89) --
	(266.32,163.03) --
	(266.55,163.17) --
	(266.78,163.31) --
	(267.01,163.45) --
	(267.25,163.59) --
	(267.48,163.72) --
	(267.71,163.86) --
	(267.94,164.00) --
	(268.18,164.14) --
	(268.41,164.28) --
	(268.64,164.42) --
	(268.87,164.56) --
	(269.11,164.70) --
	(269.34,164.83) --
	(269.57,164.97) --
	(269.80,165.11) --
	(270.04,165.25) --
	(270.27,165.39) --
	(270.50,165.53) --
	(270.73,165.67) --
	(270.97,165.80) --
	(271.20,165.94) --
	(271.43,166.08) --
	(271.66,166.22) --
	(271.90,166.36) --
	(272.13,166.50) --
	(272.36,166.64) --
	(272.59,166.77) --
	(272.83,166.91) --
	(273.06,167.05) --
	(273.29,167.19) --
	(273.52,167.33) --
	(273.76,167.47) --
	(273.99,167.61) --
	(274.22,167.74) --
	(274.45,167.88) --
	(274.69,168.02) --
	(274.92,168.16) --
	(275.15,168.30) --
	(275.38,168.44) --
	(275.62,168.58) --
	(275.85,168.72) --
	(276.08,168.85) --
	(276.31,168.99) --
	(276.54,169.13) --
	(276.78,169.27) --
	(277.01,169.41) --
	(277.24,169.55) --
	(277.47,169.69) --
	(277.71,169.82) --
	(277.94,169.96) --
	(278.17,170.10) --
	(278.40,170.24) --
	(278.64,170.38) --
	(278.87,170.52) --
	(279.10,170.66) --
	(279.33,170.79) --
	(279.57,170.93) --
	(279.80,171.07) --
	(280.03,171.21) --
	(280.26,171.35) --
	(280.50,171.49) --
	(280.73,171.63) --
	(280.96,171.77) --
	(281.19,171.90) --
	(281.43,172.04) --
	(281.66,172.18) --
	(281.89,172.32) --
	(282.12,172.46) --
	(282.36,172.60) --
	(282.59,172.74) --
	(282.82,172.87) --
	(283.05,173.01) --
	(283.29,173.15) --
	(283.52,173.29) --
	(283.75,173.43) --
	(283.98,173.57) --
	(284.22,173.71) --
	(284.45,173.84) --
	(284.68,173.98) --
	(284.91,174.12) --
	(285.15,174.26) --
	(285.38,174.40) --
	(285.61,174.54) --
	(285.84,174.68) --
	(286.08,174.81) --
	(286.31,174.95) --
	(286.54,175.09) --
	(286.77,175.23) --
	(287.01,175.37) --
	(287.24,175.51) --
	(287.47,175.65) --
	(287.70,175.79) --
	(287.94,175.92) --
	(288.17,176.06) --
	(288.40,176.20) --
	(288.63,176.34) --
	(288.87,176.48) --
	(289.10,176.62) --
	(289.33,176.76) --
	(289.56,176.89) --
	(289.80,177.03) --
	(290.03,177.17) --
	(290.26,177.31) --
	(290.49,177.45) --
	(290.73,177.59);
\end{scope}
\end{tikzpicture}
\path[draw=drawColor,line width= 0.4pt,line join=round,line cap=round] ( 49.20, 62.60) -- (300.01, 62.60);
\definecolor{drawColor}{RGB}{0,0,0}

\path[draw=drawColor,line width= 0.4pt,line join=round,line cap=round] (207.02, 58.97) -- (217.82, 58.97);

\path[draw=drawColor,line width= 0.4pt,dash pattern=on 4pt off 4pt ,line join=round,line cap=round] (207.02, 44.57) -- (217.82, 44.57);
\definecolor{drawColor}{RGB}{190,190,190}

\path[draw=drawColor,line width= 0.4pt,line join=round,line cap=round] (207.02, 30.17) -- (217.82, 30.17);
\definecolor{drawColor}{RGB}{0,0,0}

\node[text=drawColor,anchor=base west,inner sep=0pt, outer sep=0pt, scale=  0.60] at (223.22, 56.90) {MIPS};

\node[text=drawColor,anchor=base west,inner sep=0pt, outer sep=0pt, scale=  0.60] at (223.22, 42.50) {$\text{MIPS}^\text{max}\text{ und }\text{MIPS}^\text{min}$};

\node[text=drawColor,anchor=base west,inner sep=0pt, outer sep=0pt, scale=  0.60] at (223.22, 28.10) {$\lim\limits_{t_\text{max}\rightarrow\infty} \text{MIPS}$};
\end{scope}
\end{tikzpicture}
}
          }
          \only<2>{
              \resizebox{\linewidth}{!}{
              % Created by tikzDevice version 0.8.1 on 2015-04-28 22:25:47
% !TEX encoding = UTF-8 Unicode
\begin{tikzpicture}[x=1pt,y=1pt]
\definecolor{fillColor}{RGB}{255,255,255}
\path[use as bounding box,fill=fillColor,fill opacity=0.00] (0,0) rectangle (397.48,289.08);
\begin{scope}
\path[clip] ( 49.20, 61.20) rectangle (372.28,239.88);
\definecolor{drawColor}{RGB}{0,0,0}

\path[draw=drawColor,line width= 0.4pt,line join=round,line cap=round] ( 76.87,289.08) --
	( 77.04,287.14) --
	( 77.34,283.73) --
	( 77.64,280.41) --
	( 77.94,277.20) --
	( 78.23,274.08) --
	( 78.53,271.06) --
	( 78.83,268.12) --
	( 79.13,265.27) --
	( 79.43,262.49) --
	( 79.73,259.80) --
	( 80.03,257.17) --
	( 80.33,254.62) --
	( 80.63,252.14) --
	( 80.93,249.72) --
	( 81.23,247.36) --
	( 81.53,245.07) --
	( 81.83,242.83) --
	( 82.13,240.65) --
	( 82.43,238.52) --
	( 82.73,236.44) --
	( 83.03,234.42) --
	( 83.33,232.44) --
	( 83.63,230.51) --
	( 83.92,228.62) --
	( 84.22,226.78) --
	( 84.52,224.98) --
	( 84.82,223.22) --
	( 85.12,221.49) --
	( 85.42,219.81) --
	( 85.72,218.16) --
	( 86.02,216.55) --
	( 86.32,214.97) --
	( 86.62,213.43) --
	( 86.92,211.92) --
	( 87.22,210.43) --
	( 87.52,208.98) --
	( 87.82,207.56) --
	( 88.12,206.17) --
	( 88.42,204.80) --
	( 88.72,203.46) --
	( 89.02,202.15) --
	( 89.31,200.86) --
	( 89.61,199.60) --
	( 89.91,198.36) --
	( 90.21,197.14) --
	( 90.51,195.95) --
	( 90.81,194.78) --
	( 91.11,193.63) --
	( 91.41,192.50) --
	( 91.71,191.39) --
	( 92.01,190.30) --
	( 92.31,189.23) --
	( 92.61,188.17) --
	( 92.91,187.14) --
	( 93.21,186.12) --
	( 93.51,185.12) --
	( 93.81,184.14) --
	( 94.11,183.17) --
	( 94.41,182.22) --
	( 94.70,181.29) --
	( 95.00,180.37) --
	( 95.30,179.46) --
	( 95.60,178.57) --
	( 95.90,177.70) --
	( 96.20,176.83) --
	( 96.50,175.99) --
	( 96.80,175.15) --
	( 97.10,174.33) --
	( 97.40,173.52) --
	( 97.70,172.72) --
	( 98.00,171.93) --
	( 98.30,171.16) --
	( 98.60,170.39) --
	( 98.90,169.64) --
	( 99.20,168.90) --
	( 99.50,168.17) --
	( 99.80,167.45) --
	(100.09,166.74) --
	(100.39,166.04) --
	(100.69,165.36) --
	(100.99,164.68) --
	(101.29,164.01) --
	(101.59,163.34) --
	(101.89,162.69) --
	(102.19,162.05) --
	(102.49,161.41) --
	(102.79,160.79) --
	(103.09,160.17) --
	(103.39,159.56) --
	(103.69,158.96) --
	(103.99,158.37) --
	(104.29,157.78) --
	(104.59,157.20) --
	(104.89,156.63) --
	(105.19,156.07) --
	(105.49,155.51) --
	(105.78,154.96) --
	(106.08,154.42) --
	(106.38,153.88) --
	(106.68,153.35) --
	(106.98,152.83) --
	(107.28,152.31) --
	(107.58,151.80) --
	(107.88,151.30) --
	(108.18,150.80) --
	(108.48,150.31) --
	(108.78,149.82) --
	(109.08,149.34) --
	(109.38,148.87) --
	(109.68,148.40) --
	(109.98,147.94) --
	(110.28,147.48) --
	(110.58,147.02) --
	(110.88,146.58) --
	(111.17,146.13) --
	(111.47,145.69) --
	(111.77,145.26) --
	(112.07,144.83) --
	(112.37,144.41) --
	(112.67,143.99) --
	(112.97,143.57) --
	(113.27,143.16) --
	(113.57,142.76) --
	(113.87,142.36) --
	(114.17,141.96) --
	(114.47,141.57) --
	(114.77,141.18) --
	(115.07,140.80) --
	(115.37,140.41) --
	(115.67,140.04) --
	(115.97,139.67) --
	(116.27,139.30) --
	(116.56,138.93) --
	(116.86,138.57) --
	(117.16,138.21) --
	(117.46,137.86) --
	(117.76,137.51) --
	(118.06,137.16) --
	(118.36,136.82) --
	(118.66,177.51) --
	(118.96,176.97) --
	(119.26,176.44) --
	(119.56,175.91) --
	(119.86,175.39) --
	(120.16,174.87) --
	(120.46,174.35) --
	(120.76,173.85) --
	(121.06,173.34) --
	(121.36,172.84) --
	(121.66,172.35) --
	(121.95,171.86) --
	(122.25,171.38) --
	(122.55,170.90) --
	(122.85,170.42) --
	(123.15,169.95) --
	(123.45,169.48) --
	(123.75,169.02) --
	(124.05,168.56) --
	(124.35,168.11) --
	(124.65,167.66) --
	(124.95,167.21) --
	(125.25,166.77) --
	(125.55,166.33) --
	(125.85,165.89) --
	(126.15,165.46) --
	(126.45,165.04) --
	(126.75,164.61) --
	(127.05,164.19) --
	(127.35,163.78) --
	(127.64,163.37) --
	(127.94,162.96) --
	(128.24,162.55) --
	(128.54,162.15) --
	(128.84,161.75) --
	(129.14,161.36) --
	(129.44,160.97) --
	(129.74,160.58) --
	(130.04,160.19) --
	(130.34,159.81) --
	(130.64,159.43) --
	(130.94,159.06) --
	(131.24,158.68) --
	(131.54,158.31) --
	(131.84,157.95) --
	(132.14,157.58) --
	(132.44,157.22) --
	(132.74,156.86) --
	(133.03,156.51) --
	(133.33,156.16) --
	(133.63,155.81) --
	(133.93,155.46) --
	(134.23,155.12) --
	(134.53,154.78) --
	(134.83,154.44) --
	(135.13,154.10) --
	(135.43,153.77) --
	(135.73,153.44) --
	(136.03,153.11) --
	(136.33,152.78) --
	(136.63,152.46) --
	(136.93,152.14) --
	(137.23,151.82) --
	(137.53,151.51) --
	(137.83,151.19) --
	(138.13,150.88) --
	(138.42,150.57) --
	(138.72,150.27) --
	(139.02,149.96) --
	(139.32,149.66) --
	(139.62,149.36) --
	(139.92,149.06) --
	(140.22,148.77) --
	(140.52,148.47) --
	(140.82,148.18) --
	(141.12,147.89) --
	(141.42,147.61) --
	(141.72,147.32) --
	(142.02,147.04) --
	(142.32,146.76) --
	(142.62,146.48) --
	(142.92,146.20) --
	(143.22,145.93) --
	(143.52,145.65) --
	(143.81,145.38) --
	(144.11,145.11) --
	(144.41,144.85) --
	(144.71,144.58) --
	(145.01,144.32) --
	(145.31,144.06) --
	(145.61,143.80) --
	(145.91,143.54) --
	(146.21,143.28) --
	(146.51,143.03) --
	(146.81,142.77) --
	(147.11,142.52) --
	(147.41,142.27) --
	(147.71,142.02) --
	(148.01,141.78) --
	(148.31,141.53) --
	(148.61,141.29) --
	(148.91,141.05) --
	(149.21,140.81) --
	(149.50,140.57) --
	(149.80,140.33) --
	(150.10,140.10) --
	(150.40,139.86) --
	(150.70,139.63) --
	(151.00,139.40) --
	(151.30,139.17) --
	(151.60,138.94) --
	(151.90,138.72) --
	(152.20,138.49) --
	(152.50,138.27) --
	(152.80,138.05) --
	(153.10,137.83) --
	(153.40,137.61) --
	(153.70,137.39) --
	(154.00,137.17) --
	(154.30,136.96) --
	(154.60,136.74) --
	(154.89,136.53) --
	(155.19,136.32) --
	(155.49,136.11) --
	(155.79,135.90) --
	(156.09,135.69) --
	(156.39,135.49) --
	(156.69,135.28) --
	(156.99,135.08) --
	(157.29,134.88) --
	(157.59,134.68) --
	(157.89,134.48) --
	(158.19,134.28) --
	(158.49,134.08) --
	(158.79,133.88) --
	(159.09,133.69) --
	(159.39,133.50) --
	(159.69,133.30) --
	(159.99,133.11) --
	(160.28,132.92) --
	(160.58,132.73) --
	(160.88,132.54) --
	(161.18,132.35) --
	(161.48,132.17) --
	(161.78,131.98) --
	(162.08,131.80) --
	(162.38,131.62) --
	(162.68,131.43) --
	(162.98,131.25) --
	(163.28,131.07) --
	(163.58,130.89) --
	(163.88,130.72) --
	(164.18,130.54) --
	(164.48,130.36) --
	(164.78,130.19) --
	(165.08,130.02) --
	(165.38,129.84) --
	(165.67,129.67) --
	(165.97,129.50) --
	(166.27,129.33) --
	(166.57,129.16) --
	(166.87,128.99) --
	(167.17,128.82) --
	(167.47,128.66) --
	(167.77,128.49) --
	(168.07,128.33) --
	(168.37,128.16) --
	(168.67,128.00) --
	(168.97,127.84) --
	(169.27,127.68) --
	(169.57,127.52) --
	(169.87,127.36) --
	(170.17,127.20) --
	(170.47,127.04) --
	(170.77,126.89) --
	(171.07,126.73) --
	(171.36,126.58) --
	(171.66,126.42) --
	(171.96,126.27) --
	(172.26,126.12) --
	(172.56,125.96) --
	(172.86,125.81) --
	(173.16,125.66) --
	(173.46,125.51) --
	(173.76,125.36) --
	(174.06,125.22) --
	(174.36,125.07) --
	(174.66,124.92) --
	(174.96,124.78) --
	(175.26,124.63) --
	(175.56,124.49) --
	(175.86,124.34) --
	(176.16,124.20) --
	(176.46,124.06) --
	(176.75,123.92) --
	(177.05,123.78) --
	(177.35,123.64) --
	(177.65,123.50) --
	(177.95,123.36) --
	(178.25,123.22) --
	(178.55,123.09) --
	(178.85,122.95) --
	(179.15,143.33) --
	(179.45,143.15) --
	(179.75,142.96) --
	(180.05,142.78) --
	(180.35,142.60) --
	(180.65,142.41) --
	(180.95,142.23) --
	(181.25,142.05) --
	(181.55,141.87) --
	(181.85,141.69) --
	(182.14,141.52) --
	(182.44,141.34) --
	(182.74,141.16) --
	(183.04,140.99) --
	(183.34,140.81) --
	(183.64,140.64) --
	(183.94,140.47) --
	(184.24,140.30) --
	(184.54,140.12) --
	(184.84,139.95) --
	(185.14,139.79) --
	(185.44,139.62) --
	(185.74,139.45) --
	(186.04,139.28) --
	(186.34,139.12) --
	(186.64,138.95) --
	(186.94,138.78) --
	(187.24,138.62) --
	(187.53,138.46) --
	(187.83,138.30) --
	(188.13,138.13) --
	(188.43,137.97) --
	(188.73,137.81) --
	(189.03,137.65) --
	(189.33,137.49) --
	(189.63,137.34) --
	(189.93,137.18) --
	(190.23,137.02) --
	(190.53,136.87) --
	(190.83,136.71) --
	(191.13,136.56) --
	(191.43,136.40) --
	(191.73,136.25) --
	(192.03,136.10) --
	(192.33,135.95) --
	(192.63,135.79) --
	(192.93,135.64) --
	(193.22,135.49) --
	(193.52,135.34) --
	(193.82,135.20) --
	(194.12,135.05) --
	(194.42,134.90) --
	(194.72,134.75) --
	(195.02,134.61) --
	(195.32,134.46) --
	(195.62,134.32) --
	(195.92,134.18) --
	(196.22,134.03) --
	(196.52,133.89) --
	(196.82,133.75) --
	(197.12,133.61) --
	(197.42,133.46) --
	(197.72,133.32) --
	(198.02,133.19) --
	(198.32,133.05) --
	(198.61,132.91) --
	(198.91,132.77) --
	(199.21,132.63) --
	(199.51,132.50) --
	(199.81,132.36) --
	(200.11,132.22) --
	(200.41,132.09) --
	(200.71,131.95) --
	(201.01,131.82) --
	(201.31,131.69) --
	(201.61,131.55) --
	(201.91,131.42) --
	(202.21,131.29) --
	(202.51,131.16) --
	(202.81,131.03) --
	(203.11,130.90) --
	(203.41,130.77) --
	(203.71,130.64) --
	(204.00,130.51) --
	(204.30,130.38) --
	(204.60,130.26) --
	(204.90,130.13) --
	(205.20,130.00) --
	(205.50,129.88) --
	(205.80,129.75) --
	(206.10,129.63) --
	(206.40,129.50) --
	(206.70,129.38) --
	(207.00,129.26) --
	(207.30,129.13) --
	(207.60,129.01) --
	(207.90,128.89) --
	(208.20,128.77) --
	(208.50,128.65) --
	(208.80,128.53) --
	(209.10,128.41) --
	(209.39,128.29) --
	(209.69,128.17) --
	(209.99,128.05) --
	(210.29,127.93) --
	(210.59,127.81) --
	(210.89,127.70) --
	(211.19,127.58) --
	(211.49,127.46) --
	(211.79,127.35) --
	(212.09,127.23) --
	(212.39,127.12) --
	(212.69,127.00) --
	(212.99,126.89) --
	(213.29,126.78) --
	(213.59,126.66) --
	(213.89,126.55) --
	(214.19,126.44) --
	(214.49,126.33) --
	(214.79,126.22) --
	(215.08,126.11) --
	(215.38,126.00) --
	(215.68,125.88) --
	(215.98,125.78) --
	(216.28,125.67) --
	(216.58,125.56) --
	(216.88,125.45) --
	(217.18,125.34) --
	(217.48,125.23) --
	(217.78,125.13) --
	(218.08,125.02) --
	(218.38,124.91) --
	(218.68,124.81) --
	(218.98,124.70) --
	(219.28,124.60) --
	(219.58,124.49) --
	(219.88,124.39) --
	(220.18,124.28) --
	(220.47,124.18) --
	(220.77,124.08) --
	(221.07,123.97) --
	(221.37,123.87) --
	(221.67,123.77) --
	(221.97,123.67) --
	(222.27,123.57) --
	(222.57,123.46) --
	(222.87,123.36) --
	(223.17,123.26) --
	(223.47,123.16) --
	(223.77,123.06) --
	(224.07,122.96) --
	(224.37,122.87) --
	(224.67,122.77) --
	(224.97,122.67) --
	(225.27,122.57) --
	(225.57,122.47) --
	(225.86,122.38) --
	(226.16,122.28) --
	(226.46,122.18) --
	(226.76,122.09) --
	(227.06,121.99) --
	(227.36,121.90) --
	(227.66,121.80) --
	(227.96,121.71) --
	(228.26,121.61) --
	(228.56,121.52) --
	(228.86,121.42) --
	(229.16,121.33) --
	(229.46,121.24) --
	(229.76,121.15) --
	(230.06,121.05) --
	(230.36,120.96) --
	(230.66,120.87) --
	(230.96,120.78) --
	(231.25,120.69) --
	(231.55,120.60) --
	(231.85,120.51) --
	(232.15,120.41) --
	(232.45,120.32) --
	(232.75,120.24) --
	(233.05,120.15) --
	(233.35,120.06) --
	(233.65,119.97) --
	(233.95,119.88) --
	(234.25,119.79) --
	(234.55,119.70) --
	(234.85,119.62) --
	(235.15,119.53) --
	(235.45,119.44) --
	(235.75,119.36) --
	(236.05,119.27) --
	(236.35,119.18) --
	(236.65,119.10) --
	(236.94,119.01) --
	(237.24,118.93) --
	(237.54,118.84) --
	(237.84,118.76) --
	(238.14,118.67) --
	(238.44,118.59) --
	(238.74,118.51) --
	(239.04,118.42) --
	(239.34,118.34) --
	(239.64,131.94) --
	(239.94,131.83) --
	(240.24,131.73) --
	(240.54,131.62) --
	(240.84,131.52) --
	(241.14,131.42) --
	(241.44,131.31) --
	(241.74,131.21) --
	(242.04,131.11) --
	(242.33,131.00) --
	(242.63,130.90) --
	(242.93,130.80) --
	(243.23,130.70) --
	(243.53,130.60) --
	(243.83,130.50) --
	(244.13,130.40) --
	(244.43,130.30) --
	(244.73,130.20) --
	(245.03,130.10) --
	(245.33,130.00) --
	(245.63,129.90) --
	(245.93,129.80) --
	(246.23,129.70) --
	(246.53,129.60) --
	(246.83,129.51) --
	(247.13,129.41) --
	(247.43,129.31) --
	(247.72,129.22) --
	(248.02,129.12) --
	(248.32,129.02) --
	(248.62,128.93) --
	(248.92,128.83) --
	(249.22,128.74) --
	(249.52,128.64) --
	(249.82,128.55) --
	(250.12,128.45) --
	(250.42,128.36) --
	(250.72,128.26) --
	(251.02,128.17) --
	(251.32,128.08) --
	(251.62,127.99) --
	(251.92,127.89) --
	(252.22,127.80) --
	(252.52,127.71) --
	(252.82,127.62) --
	(253.11,127.53) --
	(253.41,127.43) --
	(253.71,127.34) --
	(254.01,127.25) --
	(254.31,127.16) --
	(254.61,127.07) --
	(254.91,126.98) --
	(255.21,126.89) --
	(255.51,126.80) --
	(255.81,126.71) --
	(256.11,126.63) --
	(256.41,126.54) --
	(256.71,126.45) --
	(257.01,126.36) --
	(257.31,126.27) --
	(257.61,126.19) --
	(257.91,126.10) --
	(258.21,126.01) --
	(258.51,125.93) --
	(258.80,125.84) --
	(259.10,125.75) --
	(259.40,125.67) --
	(259.70,125.58) --
	(260.00,125.50) --
	(260.30,125.41) --
	(260.60,125.33) --
	(260.90,125.24) --
	(261.20,125.16) --
	(261.50,125.07) --
	(261.80,124.99) --
	(262.10,124.91) --
	(262.40,124.82) --
	(262.70,124.74) --
	(263.00,124.66) --
	(263.30,124.58) --
	(263.60,124.49) --
	(263.90,124.41) --
	(264.19,124.33) --
	(264.49,124.25) --
	(264.79,124.17) --
	(265.09,124.09) --
	(265.39,124.00) --
	(265.69,123.92) --
	(265.99,123.84) --
	(266.29,123.76) --
	(266.59,123.68) --
	(266.89,123.60) --
	(267.19,123.52) --
	(267.49,123.44) --
	(267.79,123.37) --
	(268.09,123.29) --
	(268.39,123.21) --
	(268.69,123.13) --
	(268.99,123.05) --
	(269.29,122.97) --
	(269.58,122.90) --
	(269.88,122.82) --
	(270.18,122.74) --
	(270.48,122.66) --
	(270.78,122.59) --
	(271.08,122.51) --
	(271.38,122.43) --
	(271.68,122.36) --
	(271.98,122.28) --
	(272.28,122.21) --
	(272.58,122.13) --
	(272.88,122.06) --
	(273.18,121.98) --
	(273.48,121.91) --
	(273.78,121.83) --
	(274.08,121.76) --
	(274.38,121.68) --
	(274.68,121.61) --
	(274.98,121.53) --
	(275.27,121.46) --
	(275.57,121.39) --
	(275.87,121.31) --
	(276.17,121.24) --
	(276.47,121.17) --
	(276.77,121.09) --
	(277.07,121.02) --
	(277.37,120.95) --
	(277.67,120.88) --
	(277.97,120.81) --
	(278.27,120.73) --
	(278.57,120.66) --
	(278.87,120.59) --
	(279.17,120.52) --
	(279.47,120.45) --
	(279.77,120.38) --
	(280.07,120.31) --
	(280.37,120.24) --
	(280.66,120.17) --
	(280.96,120.10) --
	(281.26,120.03) --
	(281.56,119.96) --
	(281.86,119.89) --
	(282.16,119.82) --
	(282.46,119.75) --
	(282.76,119.68) --
	(283.06,119.61) --
	(283.36,119.54) --
	(283.66,119.48) --
	(283.96,119.41) --
	(284.26,119.34) --
	(284.56,119.27) --
	(284.86,119.20) --
	(285.16,119.14) --
	(285.46,119.07) --
	(285.76,119.00) --
	(286.05,118.94) --
	(286.35,118.87) --
	(286.65,118.80) --
	(286.95,118.74) --
	(287.25,118.67) --
	(287.55,118.60) --
	(287.85,118.54) --
	(288.15,118.47) --
	(288.45,118.41) --
	(288.75,118.34) --
	(289.05,118.28) --
	(289.35,118.21) --
	(289.65,118.15) --
	(289.95,118.08) --
	(290.25,118.02) --
	(290.55,117.95) --
	(290.85,117.89) --
	(291.15,117.82) --
	(291.44,117.76) --
	(291.74,117.70) --
	(292.04,117.63) --
	(292.34,117.57) --
	(292.64,117.51) --
	(292.94,117.44) --
	(293.24,117.38) --
	(293.54,117.32) --
	(293.84,117.26) --
	(294.14,117.19) --
	(294.44,117.13) --
	(294.74,117.07) --
	(295.04,117.01) --
	(295.34,116.95) --
	(295.64,116.88) --
	(295.94,116.82) --
	(296.24,116.76) --
	(296.54,116.70) --
	(296.84,116.64) --
	(297.13,116.58) --
	(297.43,116.52) --
	(297.73,116.46) --
	(298.03,116.40) --
	(298.33,116.34) --
	(298.63,116.28) --
	(298.93,116.22) --
	(299.23,116.16) --
	(299.53,116.10) --
	(299.83,116.04) --
	(300.13,126.24) --
	(300.43,126.17) --
	(300.73,126.10) --
	(301.03,126.02) --
	(301.33,125.95) --
	(301.63,125.88) --
	(301.93,125.81) --
	(302.23,125.74) --
	(302.52,125.67) --
	(302.82,125.60) --
	(303.12,125.53) --
	(303.42,125.46) --
	(303.72,125.39) --
	(304.02,125.32) --
	(304.32,125.25) --
	(304.62,125.18) --
	(304.92,125.11) --
	(305.22,125.04) --
	(305.52,124.97) --
	(305.82,124.90) --
	(306.12,124.84) --
	(306.42,124.77) --
	(306.72,124.70) --
	(307.02,124.63) --
	(307.32,124.56) --
	(307.62,124.49) --
	(307.91,124.43) --
	(308.21,124.36) --
	(308.51,124.29) --
	(308.81,124.23) --
	(309.11,124.16) --
	(309.41,124.09) --
	(309.71,124.02) --
	(310.01,123.96) --
	(310.31,123.89) --
	(310.61,123.83) --
	(310.91,123.76) --
	(311.21,123.69) --
	(311.51,123.63) --
	(311.81,123.56) --
	(312.11,123.50) --
	(312.41,123.43) --
	(312.71,123.37) --
	(313.01,123.30) --
	(313.30,123.24) --
	(313.60,123.17) --
	(313.90,123.11) --
	(314.20,123.04) --
	(314.50,122.98) --
	(314.80,122.92) --
	(315.10,122.85) --
	(315.40,122.79) --
	(315.70,122.72) --
	(316.00,122.66) --
	(316.30,122.60) --
	(316.60,122.53) --
	(316.90,122.47) --
	(317.20,122.41) --
	(317.50,122.35) --
	(317.80,122.28) --
	(318.10,122.22) --
	(318.40,122.16) --
	(318.70,122.10) --
	(318.99,122.03) --
	(319.29,121.97) --
	(319.59,121.91) --
	(319.89,121.85) --
	(320.19,121.79) --
	(320.49,121.73) --
	(320.79,121.67) --
	(321.09,121.60) --
	(321.39,121.54) --
	(321.69,121.48) --
	(321.99,121.42) --
	(322.29,121.36) --
	(322.59,121.30) --
	(322.89,121.24) --
	(323.19,121.18) --
	(323.49,121.12) --
	(323.79,121.06) --
	(324.09,121.00) --
	(324.38,120.94) --
	(324.68,120.88) --
	(324.98,120.82) --
	(325.28,120.76) --
	(325.58,120.71) --
	(325.88,120.65) --
	(326.18,120.59) --
	(326.48,120.53) --
	(326.78,120.47) --
	(327.08,120.41) --
	(327.38,120.35) --
	(327.68,120.30) --
	(327.98,120.24) --
	(328.28,120.18) --
	(328.58,120.12) --
	(328.88,120.07) --
	(329.18,120.01) --
	(329.48,119.95) --
	(329.77,119.89) --
	(330.07,119.84) --
	(330.37,119.78) --
	(330.67,119.72) --
	(330.97,119.67) --
	(331.27,119.61) --
	(331.57,119.55) --
	(331.87,119.50) --
	(332.17,119.44) --
	(332.47,119.38) --
	(332.77,119.33) --
	(333.07,119.27) --
	(333.37,119.22) --
	(333.67,119.16) --
	(333.97,119.11) --
	(334.27,119.05) --
	(334.57,119.00) --
	(334.87,118.94) --
	(335.16,118.89) --
	(335.46,118.83) --
	(335.76,118.78) --
	(336.06,118.72) --
	(336.36,118.67) --
	(336.66,118.61) --
	(336.96,118.56) --
	(337.26,118.50) --
	(337.56,118.45) --
	(337.86,118.40) --
	(338.16,118.34) --
	(338.46,118.29) --
	(338.76,118.23) --
	(339.06,118.18) --
	(339.36,118.13) --
	(339.66,118.07) --
	(339.96,118.02) --
	(340.26,117.97) --
	(340.56,117.92) --
	(340.85,117.86) --
	(341.15,117.81) --
	(341.45,117.76) --
	(341.75,117.71) --
	(342.05,117.65) --
	(342.35,117.60) --
	(342.65,117.55) --
	(342.95,117.50) --
	(343.25,117.44) --
	(343.55,117.39) --
	(343.85,117.34) --
	(344.15,117.29) --
	(344.45,117.24) --
	(344.75,117.19) --
	(345.05,117.14) --
	(345.35,117.08) --
	(345.65,117.03) --
	(345.95,116.98) --
	(346.24,116.93) --
	(346.54,116.88) --
	(346.84,116.83) --
	(347.14,116.78) --
	(347.44,116.73) --
	(347.74,116.68) --
	(348.04,116.63) --
	(348.34,116.58) --
	(348.64,116.53) --
	(348.94,116.48) --
	(349.24,116.43) --
	(349.54,116.38) --
	(349.84,116.33) --
	(350.14,116.28) --
	(350.44,116.23) --
	(350.74,116.18) --
	(351.04,116.13) --
	(351.34,116.08) --
	(351.63,116.04) --
	(351.93,115.99) --
	(352.23,115.94) --
	(352.53,115.89) --
	(352.83,115.84) --
	(353.13,115.79) --
	(353.43,115.74) --
	(353.73,115.70) --
	(354.03,115.65) --
	(354.33,115.60) --
	(354.63,115.55) --
	(354.93,115.50) --
	(355.23,115.46) --
	(355.53,115.41) --
	(355.83,115.36) --
	(356.13,115.31) --
	(356.43,115.26) --
	(356.73,115.22) --
	(357.02,115.17) --
	(357.32,115.12) --
	(357.62,115.08) --
	(357.92,115.03) --
	(358.22,114.98) --
	(358.52,114.94) --
	(358.82,114.89) --
	(359.12,114.84) --
	(359.42,114.80) --
	(359.72,114.75) --
	(360.02,114.70) --
	(360.32,114.66);
\end{scope}
\begin{scope}
\path[clip] (  0.00,  0.00) rectangle (397.48,289.08);
\definecolor{drawColor}{RGB}{0,0,0}

\path[draw=drawColor,line width= 0.4pt,line join=round,line cap=round] ( 58.14, 61.20) -- (360.32, 61.20);

\path[draw=drawColor,line width= 0.4pt,line join=round,line cap=round] ( 58.14, 61.20) -- ( 58.14, 55.20);

\path[draw=drawColor,line width= 0.4pt,line join=round,line cap=round] (118.58, 61.20) -- (118.58, 55.20);

\path[draw=drawColor,line width= 0.4pt,line join=round,line cap=round] (179.01, 61.20) -- (179.01, 55.20);

\path[draw=drawColor,line width= 0.4pt,line join=round,line cap=round] (239.45, 61.20) -- (239.45, 55.20);

\path[draw=drawColor,line width= 0.4pt,line join=round,line cap=round] (299.88, 61.20) -- (299.88, 55.20);

\path[draw=drawColor,line width= 0.4pt,line join=round,line cap=round] (360.32, 61.20) -- (360.32, 55.20);

\node[text=drawColor,anchor=base,inner sep=0pt, outer sep=0pt, scale=  1.00] at ( 58.14, 39.60) {0};

\node[text=drawColor,anchor=base,inner sep=0pt, outer sep=0pt, scale=  1.00] at (118.58, 39.60) {2};

\node[text=drawColor,anchor=base,inner sep=0pt, outer sep=0pt, scale=  1.00] at (179.01, 39.60) {4};

\node[text=drawColor,anchor=base,inner sep=0pt, outer sep=0pt, scale=  1.00] at (239.45, 39.60) {6};

\node[text=drawColor,anchor=base,inner sep=0pt, outer sep=0pt, scale=  1.00] at (299.88, 39.60) {8};

\node[text=drawColor,anchor=base,inner sep=0pt, outer sep=0pt, scale=  1.00] at (360.32, 39.60) {10};

\path[draw=drawColor,line width= 0.4pt,line join=round,line cap=round] ( 49.20, 67.82) -- ( 49.20,232.17);

\path[draw=drawColor,line width= 0.4pt,line join=round,line cap=round] ( 49.20, 67.82) -- ( 43.20, 67.82);

\path[draw=drawColor,line width= 0.4pt,line join=round,line cap=round] ( 49.20, 95.21) -- ( 43.20, 95.21);

\path[draw=drawColor,line width= 0.4pt,line join=round,line cap=round] ( 49.20,122.60) -- ( 43.20,122.60);

\path[draw=drawColor,line width= 0.4pt,line join=round,line cap=round] ( 49.20,149.99) -- ( 43.20,149.99);

\path[draw=drawColor,line width= 0.4pt,line join=round,line cap=round] ( 49.20,177.38) -- ( 43.20,177.38);

\path[draw=drawColor,line width= 0.4pt,line join=round,line cap=round] ( 49.20,204.78) -- ( 43.20,204.78);

\path[draw=drawColor,line width= 0.4pt,line join=round,line cap=round] ( 49.20,232.17) -- ( 43.20,232.17);

\node[text=drawColor,rotate= 90.00,anchor=base,inner sep=0pt, outer sep=0pt, scale=  1.00] at ( 34.80, 67.82) {0};

\node[text=drawColor,rotate= 90.00,anchor=base,inner sep=0pt, outer sep=0pt, scale=  1.00] at ( 34.80, 95.21) {1};

\node[text=drawColor,rotate= 90.00,anchor=base,inner sep=0pt, outer sep=0pt, scale=  1.00] at ( 34.80,122.60) {2};

\node[text=drawColor,rotate= 90.00,anchor=base,inner sep=0pt, outer sep=0pt, scale=  1.00] at ( 34.80,149.99) {3};

\node[text=drawColor,rotate= 90.00,anchor=base,inner sep=0pt, outer sep=0pt, scale=  1.00] at ( 34.80,177.38) {4};

\node[text=drawColor,rotate= 90.00,anchor=base,inner sep=0pt, outer sep=0pt, scale=  1.00] at ( 34.80,204.78) {5};

\node[text=drawColor,rotate= 90.00,anchor=base,inner sep=0pt, outer sep=0pt, scale=  1.00] at ( 34.80,232.17) {6};

\path[draw=drawColor,line width= 0.4pt,line join=round,line cap=round] ( 49.20, 61.20) --
	(372.28, 61.20) --
	(372.28,239.88) --
	( 49.20,239.88) --
	( 49.20, 61.20);
\end{scope}
\begin{scope}
\path[clip] (  0.00,  0.00) rectangle (397.48,289.08);
\definecolor{drawColor}{RGB}{0,0,0}

\node[text=drawColor,anchor=base,inner sep=0pt, outer sep=0pt, scale=  1.20] at (210.74,260.34) {\bfseries $\tilde{i}_P > i_R$};

\node[text=drawColor,anchor=base,inner sep=0pt, outer sep=0pt, scale=  1.00] at (210.74, 27.60) {$t_\text{max}$};

\node[text=drawColor,rotate= 90.00,anchor=base,inner sep=0pt, outer sep=0pt, scale=  1.00] at ( 22.80,150.54) {MIPS};
\end{scope}
\begin{scope}
\path[clip] ( 49.20, 61.20) rectangle (372.28,239.88);
\definecolor{drawColor}{RGB}{0,0,0}

\path[draw=drawColor,line width= 0.4pt,dash pattern=on 4pt off 4pt ,line join=round,line cap=round] ( 81.15,289.08) --
	( 81.23,288.45) --
	( 81.53,286.16) --
	( 81.83,283.92) --
	( 82.13,281.74) --
	( 82.43,279.61) --
	( 82.73,277.53) --
	( 83.03,275.51) --
	( 83.33,273.53) --
	( 83.63,271.60) --
	( 83.92,269.71) --
	( 84.22,267.87) --
	( 84.52,266.06) --
	( 84.82,264.30) --
	( 85.12,262.58) --
	( 85.42,260.90) --
	( 85.72,259.25) --
	( 86.02,257.64) --
	( 86.32,256.06) --
	( 86.62,254.52) --
	( 86.92,253.00) --
	( 87.22,251.52) --
	( 87.52,250.07) --
	( 87.82,248.65) --
	( 88.12,247.26) --
	( 88.42,245.89) --
	( 88.72,244.55) --
	( 89.02,243.24) --
	( 89.31,241.95) --
	( 89.61,240.69) --
	( 89.91,239.45) --
	( 90.21,238.23) --
	( 90.51,237.04) --
	( 90.81,235.86) --
	( 91.11,234.71) --
	( 91.41,233.58) --
	( 91.71,232.47) --
	( 92.01,231.38) --
	( 92.31,230.31) --
	( 92.61,229.26) --
	( 92.91,228.23) --
	( 93.21,227.21) --
	( 93.51,226.21) --
	( 93.81,225.23) --
	( 94.11,224.26) --
	( 94.41,223.31) --
	( 94.70,222.38) --
	( 95.00,221.46) --
	( 95.30,220.55) --
	( 95.60,219.66) --
	( 95.90,218.78) --
	( 96.20,217.92) --
	( 96.50,217.07) --
	( 96.80,216.24) --
	( 97.10,215.41) --
	( 97.40,214.60) --
	( 97.70,213.81) --
	( 98.00,213.02) --
	( 98.30,212.25) --
	( 98.60,211.48) --
	( 98.90,210.73) --
	( 99.20,209.99) --
	( 99.50,209.26) --
	( 99.80,208.54) --
	(100.09,207.83) --
	(100.39,207.13) --
	(100.69,206.44) --
	(100.99,205.76) --
	(101.29,205.09) --
	(101.59,204.43) --
	(101.89,203.78) --
	(102.19,203.14) --
	(102.49,202.50) --
	(102.79,201.88) --
	(103.09,201.26) --
	(103.39,200.65) --
	(103.69,200.05) --
	(103.99,199.45) --
	(104.29,198.87) --
	(104.59,198.29) --
	(104.89,197.72) --
	(105.19,197.15) --
	(105.49,196.60) --
	(105.78,196.05) --
	(106.08,195.51) --
	(106.38,194.97) --
	(106.68,194.44) --
	(106.98,193.92) --
	(107.28,193.40) --
	(107.58,192.89) --
	(107.88,192.39) --
	(108.18,191.89) --
	(108.48,191.40) --
	(108.78,190.91) --
	(109.08,190.43) --
	(109.38,189.96) --
	(109.68,189.49) --
	(109.98,189.02) --
	(110.28,188.56) --
	(110.58,188.11) --
	(110.88,187.66) --
	(111.17,187.22) --
	(111.47,186.78) --
	(111.77,186.35) --
	(112.07,185.92) --
	(112.37,185.50) --
	(112.67,185.08) --
	(112.97,184.66) --
	(113.27,184.25) --
	(113.57,183.85) --
	(113.87,183.44) --
	(114.17,183.05) --
	(114.47,182.66) --
	(114.77,182.27) --
	(115.07,181.88) --
	(115.37,181.50) --
	(115.67,181.13) --
	(115.97,180.75) --
	(116.27,180.38) --
	(116.56,180.02) --
	(116.86,179.66) --
	(117.16,179.30) --
	(117.46,178.95) --
	(117.76,178.60) --
	(118.06,178.25) --
	(118.36,177.91) --
	(118.66,177.57) --
	(118.96,177.23) --
	(119.26,176.89) --
	(119.56,176.56) --
	(119.86,176.24) --
	(120.16,175.91) --
	(120.46,175.59) --
	(120.76,175.28) --
	(121.06,174.96) --
	(121.36,174.65) --
	(121.66,174.34) --
	(121.95,174.03) --
	(122.25,173.73) --
	(122.55,173.43) --
	(122.85,173.13) --
	(123.15,172.84) --
	(123.45,172.55) --
	(123.75,172.26) --
	(124.05,171.97) --
	(124.35,171.69) --
	(124.65,171.41) --
	(124.95,171.13) --
	(125.25,170.85) --
	(125.55,170.58) --
	(125.85,170.31) --
	(126.15,170.04) --
	(126.45,169.77) --
	(126.75,169.51) --
	(127.05,169.24) --
	(127.35,168.98) --
	(127.64,168.73) --
	(127.94,168.47) --
	(128.24,168.22) --
	(128.54,167.97) --
	(128.84,167.72) --
	(129.14,167.47) --
	(129.44,167.22) --
	(129.74,166.98) --
	(130.04,166.74) --
	(130.34,166.50) --
	(130.64,166.27) --
	(130.94,166.03) --
	(131.24,165.80) --
	(131.54,165.57) --
	(131.84,165.34) --
	(132.14,165.11) --
	(132.44,164.89) --
	(132.74,164.66) --
	(133.03,164.44) --
	(133.33,164.22) --
	(133.63,164.00) --
	(133.93,163.78) --
	(134.23,163.57) --
	(134.53,163.36) --
	(134.83,163.15) --
	(135.13,162.94) --
	(135.43,162.73) --
	(135.73,162.52) --
	(136.03,162.32) --
	(136.33,162.11) --
	(136.63,161.91) --
	(136.93,161.71) --
	(137.23,161.51) --
	(137.53,161.31) --
	(137.83,161.12) --
	(138.13,160.92) --
	(138.42,160.73) --
	(138.72,160.54) --
	(139.02,160.35) --
	(139.32,160.16) --
	(139.62,159.97) --
	(139.92,159.79) --
	(140.22,159.60) --
	(140.52,159.42) --
	(140.82,159.24) --
	(141.12,159.06) --
	(141.42,158.88) --
	(141.72,158.70) --
	(142.02,158.52) --
	(142.32,158.35) --
	(142.62,158.17) --
	(142.92,158.00) --
	(143.22,157.83) --
	(143.52,157.66) --
	(143.81,157.49) --
	(144.11,157.32) --
	(144.41,157.15) --
	(144.71,156.98) --
	(145.01,156.82) --
	(145.31,156.66) --
	(145.61,156.49) --
	(145.91,156.33) --
	(146.21,156.17) --
	(146.51,156.01) --
	(146.81,155.85) --
	(147.11,155.70) --
	(147.41,155.54) --
	(147.71,155.39) --
	(148.01,155.23) --
	(148.31,155.08) --
	(148.61,154.93) --
	(148.91,154.78) --
	(149.21,154.63) --
	(149.50,154.48) --
	(149.80,154.33) --
	(150.10,154.18) --
	(150.40,154.04) --
	(150.70,153.89) --
	(151.00,153.75) --
	(151.30,153.60) --
	(151.60,153.46) --
	(151.90,153.32) --
	(152.20,153.18) --
	(152.50,153.04) --
	(152.80,152.90) --
	(153.10,152.76) --
	(153.40,152.63) --
	(153.70,152.49) --
	(154.00,152.35) --
	(154.30,152.22) --
	(154.60,152.09) --
	(154.89,151.95) --
	(155.19,151.82) --
	(155.49,151.69) --
	(155.79,151.56) --
	(156.09,151.43) --
	(156.39,151.30) --
	(156.69,151.17) --
	(156.99,151.05) --
	(157.29,150.92) --
	(157.59,150.79) --
	(157.89,150.67) --
	(158.19,150.55) --
	(158.49,150.42) --
	(158.79,150.30) --
	(159.09,150.18) --
	(159.39,150.06) --
	(159.69,149.94) --
	(159.99,149.82) --
	(160.28,149.70) --
	(160.58,149.58) --
	(160.88,149.46) --
	(161.18,149.34) --
	(161.48,149.23) --
	(161.78,149.11) --
	(162.08,149.00) --
	(162.38,148.88) --
	(162.68,148.77) --
	(162.98,148.66) --
	(163.28,148.54) --
	(163.58,148.43) --
	(163.88,148.32) --
	(164.18,148.21) --
	(164.48,148.10) --
	(164.78,147.99) --
	(165.08,147.88) --
	(165.38,147.77) --
	(165.67,147.67) --
	(165.97,147.56) --
	(166.27,147.45) --
	(166.57,147.35) --
	(166.87,147.24) --
	(167.17,147.14) --
	(167.47,147.03) --
	(167.77,146.93) --
	(168.07,146.83) --
	(168.37,146.72) --
	(168.67,146.62) --
	(168.97,146.52) --
	(169.27,146.42) --
	(169.57,146.32) --
	(169.87,146.22) --
	(170.17,146.12) --
	(170.47,146.02) --
	(170.77,145.93) --
	(171.07,145.83) --
	(171.36,145.73) --
	(171.66,145.64) --
	(171.96,145.54) --
	(172.26,145.44) --
	(172.56,145.35) --
	(172.86,145.25) --
	(173.16,145.16) --
	(173.46,145.07) --
	(173.76,144.97) --
	(174.06,144.88) --
	(174.36,144.79) --
	(174.66,144.70) --
	(174.96,144.61) --
	(175.26,144.52) --
	(175.56,144.43) --
	(175.86,144.34) --
	(176.16,144.25) --
	(176.46,144.16) --
	(176.75,144.07) --
	(177.05,143.98) --
	(177.35,143.90) --
	(177.65,143.81) --
	(177.95,143.72) --
	(178.25,143.64) --
	(178.55,143.55) --
	(178.85,143.46) --
	(179.15,143.38) --
	(179.45,143.30) --
	(179.75,143.21) --
	(180.05,143.13) --
	(180.35,143.04) --
	(180.65,142.96) --
	(180.95,142.88) --
	(181.25,142.80) --
	(181.55,142.72) --
	(181.85,142.63) --
	(182.14,142.55) --
	(182.44,142.47) --
	(182.74,142.39) --
	(183.04,142.31) --
	(183.34,142.23) --
	(183.64,142.16) --
	(183.94,142.08) --
	(184.24,142.00) --
	(184.54,141.92) --
	(184.84,141.84) --
	(185.14,141.77) --
	(185.44,141.69) --
	(185.74,141.61) --
	(186.04,141.54) --
	(186.34,141.46) --
	(186.64,141.39) --
	(186.94,141.31) --
	(187.24,141.24) --
	(187.53,141.16) --
	(187.83,141.09) --
	(188.13,141.02) --
	(188.43,140.94) --
	(188.73,140.87) --
	(189.03,140.80) --
	(189.33,140.73) --
	(189.63,140.65) --
	(189.93,140.58) --
	(190.23,140.51) --
	(190.53,140.44) --
	(190.83,140.37) --
	(191.13,140.30) --
	(191.43,140.23) --
	(191.73,140.16) --
	(192.03,140.09) --
	(192.33,140.02) --
	(192.63,139.95) --
	(192.93,139.88) --
	(193.22,139.82) --
	(193.52,139.75) --
	(193.82,139.68) --
	(194.12,139.61) --
	(194.42,139.55) --
	(194.72,139.48) --
	(195.02,139.41) --
	(195.32,139.35) --
	(195.62,139.28) --
	(195.92,139.22) --
	(196.22,139.15) --
	(196.52,139.09) --
	(196.82,139.02) --
	(197.12,138.96) --
	(197.42,138.89) --
	(197.72,138.83) --
	(198.02,138.77) --
	(198.32,138.70) --
	(198.61,138.64) --
	(198.91,138.58) --
	(199.21,138.52) --
	(199.51,138.45) --
	(199.81,138.39) --
	(200.11,138.33) --
	(200.41,138.27) --
	(200.71,138.21) --
	(201.01,138.15) --
	(201.31,138.09) --
	(201.61,138.03) --
	(201.91,137.97) --
	(202.21,137.91) --
	(202.51,137.85) --
	(202.81,137.79) --
	(203.11,137.73) --
	(203.41,137.67) --
	(203.71,137.61) --
	(204.00,137.55) --
	(204.30,137.49) --
	(204.60,137.44) --
	(204.90,137.38) --
	(205.20,137.32) --
	(205.50,137.26) --
	(205.80,137.21) --
	(206.10,137.15) --
	(206.40,137.09) --
	(206.70,137.04) --
	(207.00,136.98) --
	(207.30,136.93) --
	(207.60,136.87) --
	(207.90,136.81) --
	(208.20,136.76) --
	(208.50,136.70) --
	(208.80,136.65) --
	(209.10,136.60) --
	(209.39,136.54) --
	(209.69,136.49) --
	(209.99,136.43) --
	(210.29,136.38) --
	(210.59,136.33) --
	(210.89,136.27) --
	(211.19,136.22) --
	(211.49,136.17) --
	(211.79,136.11) --
	(212.09,136.06) --
	(212.39,136.01) --
	(212.69,135.96) --
	(212.99,135.91) --
	(213.29,135.85) --
	(213.59,135.80) --
	(213.89,135.75) --
	(214.19,135.70) --
	(214.49,135.65) --
	(214.79,135.60) --
	(215.08,135.55) --
	(215.38,135.50) --
	(215.68,135.45) --
	(215.98,135.40) --
	(216.28,135.35) --
	(216.58,135.30) --
	(216.88,135.25) --
	(217.18,135.20) --
	(217.48,135.15) --
	(217.78,135.10) --
	(218.08,135.06) --
	(218.38,135.01) --
	(218.68,134.96) --
	(218.98,134.91) --
	(219.28,134.86) --
	(219.58,134.82) --
	(219.88,134.77) --
	(220.18,134.72) --
	(220.47,134.67) --
	(220.77,134.63) --
	(221.07,134.58) --
	(221.37,134.53) --
	(221.67,134.49) --
	(221.97,134.44) --
	(222.27,134.39) --
	(222.57,134.35) --
	(222.87,134.30) --
	(223.17,134.26) --
	(223.47,134.21) --
	(223.77,134.17) --
	(224.07,134.12) --
	(224.37,134.08) --
	(224.67,134.03) --
	(224.97,133.99) --
	(225.27,133.94) --
	(225.57,133.90) --
	(225.86,133.85) --
	(226.16,133.81) --
	(226.46,133.77) --
	(226.76,133.72) --
	(227.06,133.68) --
	(227.36,133.64) --
	(227.66,133.59) --
	(227.96,133.55) --
	(228.26,133.51) --
	(228.56,133.46) --
	(228.86,133.42) --
	(229.16,133.38) --
	(229.46,133.34) --
	(229.76,133.29) --
	(230.06,133.25) --
	(230.36,133.21) --
	(230.66,133.17) --
	(230.96,133.13) --
	(231.25,133.09) --
	(231.55,133.04) --
	(231.85,133.00) --
	(232.15,132.96) --
	(232.45,132.92) --
	(232.75,132.88) --
	(233.05,132.84) --
	(233.35,132.80) --
	(233.65,132.76) --
	(233.95,132.72) --
	(234.25,132.68) --
	(234.55,132.64) --
	(234.85,132.60) --
	(235.15,132.56) --
	(235.45,132.52) --
	(235.75,132.48) --
	(236.05,132.44) --
	(236.35,132.40) --
	(236.65,132.36) --
	(236.94,132.32) --
	(237.24,132.29) --
	(237.54,132.25) --
	(237.84,132.21) --
	(238.14,132.17) --
	(238.44,132.13) --
	(238.74,132.09) --
	(239.04,132.06) --
	(239.34,132.02) --
	(239.64,131.98) --
	(239.94,131.94) --
	(240.24,131.91) --
	(240.54,131.87) --
	(240.84,131.83) --
	(241.14,131.79) --
	(241.44,131.76) --
	(241.74,131.72) --
	(242.04,131.68) --
	(242.33,131.65) --
	(242.63,131.61) --
	(242.93,131.57) --
	(243.23,131.54) --
	(243.53,131.50) --
	(243.83,131.47) --
	(244.13,131.43) --
	(244.43,131.39) --
	(244.73,131.36) --
	(245.03,131.32) --
	(245.33,131.29) --
	(245.63,131.25) --
	(245.93,131.22) --
	(246.23,131.18) --
	(246.53,131.15) --
	(246.83,131.11) --
	(247.13,131.08) --
	(247.43,131.04) --
	(247.72,131.01) --
	(248.02,130.97) --
	(248.32,130.94) --
	(248.62,130.91) --
	(248.92,130.87) --
	(249.22,130.84) --
	(249.52,130.80) --
	(249.82,130.77) --
	(250.12,130.74) --
	(250.42,130.70) --
	(250.72,130.67) --
	(251.02,130.64) --
	(251.32,130.60) --
	(251.62,130.57) --
	(251.92,130.54) --
	(252.22,130.50) --
	(252.52,130.47) --
	(252.82,130.44) --
	(253.11,130.41) --
	(253.41,130.37) --
	(253.71,130.34) --
	(254.01,130.31) --
	(254.31,130.28) --
	(254.61,130.24) --
	(254.91,130.21) --
	(255.21,130.18) --
	(255.51,130.15) --
	(255.81,130.12) --
	(256.11,130.08) --
	(256.41,130.05) --
	(256.71,130.02) --
	(257.01,129.99) --
	(257.31,129.96) --
	(257.61,129.93) --
	(257.91,129.90) --
	(258.21,129.87) --
	(258.51,129.83) --
	(258.80,129.80) --
	(259.10,129.77) --
	(259.40,129.74) --
	(259.70,129.71) --
	(260.00,129.68) --
	(260.30,129.65) --
	(260.60,129.62) --
	(260.90,129.59) --
	(261.20,129.56) --
	(261.50,129.53) --
	(261.80,129.50) --
	(262.10,129.47) --
	(262.40,129.44) --
	(262.70,129.41) --
	(263.00,129.38) --
	(263.30,129.35) --
	(263.60,129.32) --
	(263.90,129.29) --
	(264.19,129.26) --
	(264.49,129.23) --
	(264.79,129.21) --
	(265.09,129.18) --
	(265.39,129.15) --
	(265.69,129.12) --
	(265.99,129.09) --
	(266.29,129.06) --
	(266.59,129.03) --
	(266.89,129.00) --
	(267.19,128.98) --
	(267.49,128.95) --
	(267.79,128.92) --
	(268.09,128.89) --
	(268.39,128.86) --
	(268.69,128.84) --
	(268.99,128.81) --
	(269.29,128.78) --
	(269.58,128.75) --
	(269.88,128.72) --
	(270.18,128.70) --
	(270.48,128.67) --
	(270.78,128.64) --
	(271.08,128.61) --
	(271.38,128.59) --
	(271.68,128.56) --
	(271.98,128.53) --
	(272.28,128.51) --
	(272.58,128.48) --
	(272.88,128.45) --
	(273.18,128.42) --
	(273.48,128.40) --
	(273.78,128.37) --
	(274.08,128.34) --
	(274.38,128.32) --
	(274.68,128.29) --
	(274.98,128.27) --
	(275.27,128.24) --
	(275.57,128.21) --
	(275.87,128.19) --
	(276.17,128.16) --
	(276.47,128.13) --
	(276.77,128.11) --
	(277.07,128.08) --
	(277.37,128.06) --
	(277.67,128.03) --
	(277.97,128.01) --
	(278.27,127.98) --
	(278.57,127.95) --
	(278.87,127.93) --
	(279.17,127.90) --
	(279.47,127.88) --
	(279.77,127.85) --
	(280.07,127.83) --
	(280.37,127.80) --
	(280.66,127.78) --
	(280.96,127.75) --
	(281.26,127.73) --
	(281.56,127.70) --
	(281.86,127.68) --
	(282.16,127.65) --
	(282.46,127.63) --
	(282.76,127.60) --
	(283.06,127.58) --
	(283.36,127.55) --
	(283.66,127.53) --
	(283.96,127.51) --
	(284.26,127.48) --
	(284.56,127.46) --
	(284.86,127.43) --
	(285.16,127.41) --
	(285.46,127.39) --
	(285.76,127.36) --
	(286.05,127.34) --
	(286.35,127.31) --
	(286.65,127.29) --
	(286.95,127.27) --
	(287.25,127.24) --
	(287.55,127.22) --
	(287.85,127.20) --
	(288.15,127.17) --
	(288.45,127.15) --
	(288.75,127.13) --
	(289.05,127.10) --
	(289.35,127.08) --
	(289.65,127.06) --
	(289.95,127.03) --
	(290.25,127.01) --
	(290.55,126.99) --
	(290.85,126.96) --
	(291.15,126.94) --
	(291.44,126.92) --
	(291.74,126.90) --
	(292.04,126.87) --
	(292.34,126.85) --
	(292.64,126.83) --
	(292.94,126.80) --
	(293.24,126.78) --
	(293.54,126.76) --
	(293.84,126.74) --
	(294.14,126.72) --
	(294.44,126.69) --
	(294.74,126.67) --
	(295.04,126.65) --
	(295.34,126.63) --
	(295.64,126.60) --
	(295.94,126.58) --
	(296.24,126.56) --
	(296.54,126.54) --
	(296.84,126.52) --
	(297.13,126.50) --
	(297.43,126.47) --
	(297.73,126.45) --
	(298.03,126.43) --
	(298.33,126.41) --
	(298.63,126.39) --
	(298.93,126.37) --
	(299.23,126.34) --
	(299.53,126.32) --
	(299.83,126.30) --
	(300.13,126.28) --
	(300.43,126.26) --
	(300.73,126.24) --
	(301.03,126.22) --
	(301.33,126.20) --
	(301.63,126.18) --
	(301.93,126.16) --
	(302.23,126.13) --
	(302.52,126.11) --
	(302.82,126.09) --
	(303.12,126.07) --
	(303.42,126.05) --
	(303.72,126.03) --
	(304.02,126.01) --
	(304.32,125.99) --
	(304.62,125.97) --
	(304.92,125.95) --
	(305.22,125.93) --
	(305.52,125.91) --
	(305.82,125.89) --
	(306.12,125.87) --
	(306.42,125.85) --
	(306.72,125.83) --
	(307.02,125.81) --
	(307.32,125.79) --
	(307.62,125.77) --
	(307.91,125.75) --
	(308.21,125.73) --
	(308.51,125.71) --
	(308.81,125.69) --
	(309.11,125.67) --
	(309.41,125.65) --
	(309.71,125.63) --
	(310.01,125.61) --
	(310.31,125.59) --
	(310.61,125.57) --
	(310.91,125.55) --
	(311.21,125.53) --
	(311.51,125.51) --
	(311.81,125.49) --
	(312.11,125.47) --
	(312.41,125.46) --
	(312.71,125.44) --
	(313.01,125.42) --
	(313.30,125.40) --
	(313.60,125.38) --
	(313.90,125.36) --
	(314.20,125.34) --
	(314.50,125.32) --
	(314.80,125.30) --
	(315.10,125.28) --
	(315.40,125.27) --
	(315.70,125.25) --
	(316.00,125.23) --
	(316.30,125.21) --
	(316.60,125.19) --
	(316.90,125.17) --
	(317.20,125.15) --
	(317.50,125.14) --
	(317.80,125.12) --
	(318.10,125.10) --
	(318.40,125.08) --
	(318.70,125.06) --
	(318.99,125.04) --
	(319.29,125.03) --
	(319.59,125.01) --
	(319.89,124.99) --
	(320.19,124.97) --
	(320.49,124.95) --
	(320.79,124.94) --
	(321.09,124.92) --
	(321.39,124.90) --
	(321.69,124.88) --
	(321.99,124.86) --
	(322.29,124.85) --
	(322.59,124.83) --
	(322.89,124.81) --
	(323.19,124.79) --
	(323.49,124.78) --
	(323.79,124.76) --
	(324.09,124.74) --
	(324.38,124.72) --
	(324.68,124.71) --
	(324.98,124.69) --
	(325.28,124.67) --
	(325.58,124.65) --
	(325.88,124.64) --
	(326.18,124.62) --
	(326.48,124.60) --
	(326.78,124.58) --
	(327.08,124.57) --
	(327.38,124.55) --
	(327.68,124.53) --
	(327.98,124.52) --
	(328.28,124.50) --
	(328.58,124.48) --
	(328.88,124.47) --
	(329.18,124.45) --
	(329.48,124.43) --
	(329.77,124.41) --
	(330.07,124.40) --
	(330.37,124.38) --
	(330.67,124.36) --
	(330.97,124.35) --
	(331.27,124.33) --
	(331.57,124.31) --
	(331.87,124.30) --
	(332.17,124.28) --
	(332.47,124.26) --
	(332.77,124.25) --
	(333.07,124.23) --
	(333.37,124.22) --
	(333.67,124.20) --
	(333.97,124.18) --
	(334.27,124.17) --
	(334.57,124.15) --
	(334.87,124.13) --
	(335.16,124.12) --
	(335.46,124.10) --
	(335.76,124.09) --
	(336.06,124.07) --
	(336.36,124.05) --
	(336.66,124.04) --
	(336.96,124.02) --
	(337.26,124.01) --
	(337.56,123.99) --
	(337.86,123.97) --
	(338.16,123.96) --
	(338.46,123.94) --
	(338.76,123.93) --
	(339.06,123.91) --
	(339.36,123.90) --
	(339.66,123.88) --
	(339.96,123.86) --
	(340.26,123.85) --
	(340.56,123.83) --
	(340.85,123.82) --
	(341.15,123.80) --
	(341.45,123.79) --
	(341.75,123.77) --
	(342.05,123.76) --
	(342.35,123.74) --
	(342.65,123.73) --
	(342.95,123.71) --
	(343.25,123.69) --
	(343.55,123.68) --
	(343.85,123.66) --
	(344.15,123.65) --
	(344.45,123.63) --
	(344.75,123.62) --
	(345.05,123.60) --
	(345.35,123.59) --
	(345.65,123.57) --
	(345.95,123.56) --
	(346.24,123.54) --
	(346.54,123.53) --
	(346.84,123.51) --
	(347.14,123.50) --
	(347.44,123.48) --
	(347.74,123.47) --
	(348.04,123.45) --
	(348.34,123.44) --
	(348.64,123.43) --
	(348.94,123.41) --
	(349.24,123.40) --
	(349.54,123.38) --
	(349.84,123.37) --
	(350.14,123.35) --
	(350.44,123.34) --
	(350.74,123.32) --
	(351.04,123.31) --
	(351.34,123.29) --
	(351.63,123.28) --
	(351.93,123.27) --
	(352.23,123.25) --
	(352.53,123.24) --
	(352.83,123.22) --
	(353.13,123.21) --
	(353.43,123.19) --
	(353.73,123.18) --
	(354.03,123.17) --
	(354.33,123.15) --
	(354.63,123.14) --
	(354.93,123.12) --
	(355.23,123.11) --
	(355.53,123.10) --
	(355.83,123.08) --
	(356.13,123.07) --
	(356.43,123.05) --
	(356.73,123.04) --
	(357.02,123.03) --
	(357.32,123.01) --
	(357.62,123.00) --
	(357.92,122.98) --
	(358.22,122.97) --
	(358.52,122.96) --
	(358.82,122.94) --
	(359.12,122.93) --
	(359.42,122.92) --
	(359.72,122.90) --
	(360.02,122.89) --
	(360.32,122.87);

\path[draw=drawColor,line width= 0.4pt,dash pattern=on 4pt off 4pt ,line join=round,line cap=round] ( 67.35,289.08) --
	( 67.45,286.98) --
	( 67.75,281.44) --
	( 68.05,276.24) --
	( 68.35,271.34) --
	( 68.65,266.72) --
	( 68.95,262.35) --
	( 69.25,258.22) --
	( 69.55,254.31) --
	( 69.85,250.60) --
	( 70.15,247.07) --
	( 70.45,243.71) --
	( 70.75,240.52) --
	( 71.05,237.47) --
	( 71.35,234.56) --
	( 71.65,231.78) --
	( 71.95,229.12) --
	( 72.25,226.57) --
	( 72.55,224.13) --
	( 72.84,221.79) --
	( 73.14,219.54) --
	( 73.44,217.38) --
	( 73.74,215.30) --
	( 74.04,213.30) --
	( 74.34,211.38) --
	( 74.64,209.52) --
	( 74.94,207.73) --
	( 75.24,206.01) --
	( 75.54,204.34) --
	( 75.84,202.73) --
	( 76.14,201.17) --
	( 76.44,199.67) --
	( 76.74,198.21) --
	( 77.04,196.80) --
	( 77.34,195.43) --
	( 77.64,194.11) --
	( 77.94,192.82) --
	( 78.23,191.58) --
	( 78.53,190.37) --
	( 78.83,189.19) --
	( 79.13,188.05) --
	( 79.43,186.94) --
	( 79.73,185.86) --
	( 80.03,184.81) --
	( 80.33,183.79) --
	( 80.63,182.80) --
	( 80.93,181.83) --
	( 81.23,180.89) --
	( 81.53,179.97) --
	( 81.83,179.07) --
	( 82.13,178.20) --
	( 82.43,177.35) --
	( 82.73,176.52) --
	( 83.03,175.71) --
	( 83.33,174.92) --
	( 83.63,174.15) --
	( 83.92,173.39) --
	( 84.22,172.65) --
	( 84.52,171.93) --
	( 84.82,171.23) --
	( 85.12,170.54) --
	( 85.42,169.87) --
	( 85.72,169.21) --
	( 86.02,168.56) --
	( 86.32,167.93) --
	( 86.62,167.31) --
	( 86.92,166.71) --
	( 87.22,166.12) --
	( 87.52,165.54) --
	( 87.82,164.97) --
	( 88.12,164.41) --
	( 88.42,163.86) --
	( 88.72,163.33) --
	( 89.02,162.80) --
	( 89.31,162.29) --
	( 89.61,161.78) --
	( 89.91,161.29) --
	( 90.21,160.80) --
	( 90.51,160.32) --
	( 90.81,159.85) --
	( 91.11,159.39) --
	( 91.41,158.94) --
	( 91.71,158.50) --
	( 92.01,158.06) --
	( 92.31,157.63) --
	( 92.61,157.21) --
	( 92.91,156.80) --
	( 93.21,156.39) --
	( 93.51,155.99) --
	( 93.81,155.60) --
	( 94.11,155.21) --
	( 94.41,154.83) --
	( 94.70,154.46) --
	( 95.00,154.09) --
	( 95.30,153.73) --
	( 95.60,153.37) --
	( 95.90,153.02) --
	( 96.20,152.68) --
	( 96.50,152.34) --
	( 96.80,152.00) --
	( 97.10,151.67) --
	( 97.40,151.35) --
	( 97.70,151.03) --
	( 98.00,150.72) --
	( 98.30,150.41) --
	( 98.60,150.10) --
	( 98.90,149.80) --
	( 99.20,149.50) --
	( 99.50,149.21) --
	( 99.80,148.92) --
	(100.09,148.64) --
	(100.39,148.36) --
	(100.69,148.08) --
	(100.99,147.81) --
	(101.29,147.54) --
	(101.59,147.28) --
	(101.89,147.02) --
	(102.19,146.76) --
	(102.49,146.51) --
	(102.79,146.26) --
	(103.09,146.01) --
	(103.39,145.77) --
	(103.69,145.53) --
	(103.99,145.29) --
	(104.29,145.05) --
	(104.59,144.82) --
	(104.89,144.59) --
	(105.19,144.37) --
	(105.49,144.15) --
	(105.78,143.93) --
	(106.08,143.71) --
	(106.38,143.50) --
	(106.68,143.28) --
	(106.98,143.07) --
	(107.28,142.87) --
	(107.58,142.66) --
	(107.88,142.46) --
	(108.18,142.26) --
	(108.48,142.07) --
	(108.78,141.87) --
	(109.08,141.68) --
	(109.38,141.49) --
	(109.68,141.30) --
	(109.98,141.12) --
	(110.28,140.93) --
	(110.58,140.75) --
	(110.88,140.57) --
	(111.17,140.40) --
	(111.47,140.22) --
	(111.77,140.05) --
	(112.07,139.88) --
	(112.37,139.71) --
	(112.67,139.54) --
	(112.97,139.37) --
	(113.27,139.21) --
	(113.57,139.05) --
	(113.87,138.89) --
	(114.17,138.73) --
	(114.47,138.57) --
	(114.77,138.41) --
	(115.07,138.26) --
	(115.37,138.11) --
	(115.67,137.96) --
	(115.97,137.81) --
	(116.27,137.66) --
	(116.56,137.51) --
	(116.86,137.37) --
	(117.16,137.23) --
	(117.46,137.09) --
	(117.76,136.95) --
	(118.06,136.81) --
	(118.36,136.67) --
	(118.66,136.53) --
	(118.96,136.40) --
	(119.26,136.27) --
	(119.56,136.13) --
	(119.86,136.00) --
	(120.16,135.87) --
	(120.46,135.74) --
	(120.76,135.62) --
	(121.06,135.49) --
	(121.36,135.37) --
	(121.66,135.24) --
	(121.95,135.12) --
	(122.25,135.00) --
	(122.55,134.88) --
	(122.85,134.76) --
	(123.15,134.64) --
	(123.45,134.53) --
	(123.75,134.41) --
	(124.05,134.30) --
	(124.35,134.18) --
	(124.65,134.07) --
	(124.95,133.96) --
	(125.25,133.85) --
	(125.55,133.74) --
	(125.85,133.63) --
	(126.15,133.52) --
	(126.45,133.42) --
	(126.75,133.31) --
	(127.05,133.20) --
	(127.35,133.10) --
	(127.64,133.00) --
	(127.94,132.90) --
	(128.24,132.79) --
	(128.54,132.69) --
	(128.84,132.59) --
	(129.14,132.50) --
	(129.44,132.40) --
	(129.74,132.30) --
	(130.04,132.20) --
	(130.34,132.11) --
	(130.64,132.01) --
	(130.94,131.92) --
	(131.24,131.83) --
	(131.54,131.73) --
	(131.84,131.64) --
	(132.14,131.55) --
	(132.44,131.46) --
	(132.74,131.37) --
	(133.03,131.28) --
	(133.33,131.20) --
	(133.63,131.11) --
	(133.93,131.02) --
	(134.23,130.94) --
	(134.53,130.85) --
	(134.83,130.77) --
	(135.13,130.68) --
	(135.43,130.60) --
	(135.73,130.52) --
	(136.03,130.43) --
	(136.33,130.35) --
	(136.63,130.27) --
	(136.93,130.19) --
	(137.23,130.11) --
	(137.53,130.03) --
	(137.83,129.95) --
	(138.13,129.88) --
	(138.42,129.80) --
	(138.72,129.72) --
	(139.02,129.65) --
	(139.32,129.57) --
	(139.62,129.50) --
	(139.92,129.42) --
	(140.22,129.35) --
	(140.52,129.27) --
	(140.82,129.20) --
	(141.12,129.13) --
	(141.42,129.06) --
	(141.72,128.99) --
	(142.02,128.92) --
	(142.32,128.85) --
	(142.62,128.78) --
	(142.92,128.71) --
	(143.22,128.64) --
	(143.52,128.57) --
	(143.81,128.50) --
	(144.11,128.43) --
	(144.41,128.37) --
	(144.71,128.30) --
	(145.01,128.24) --
	(145.31,128.17) --
	(145.61,128.10) --
	(145.91,128.04) --
	(146.21,127.98) --
	(146.51,127.91) --
	(146.81,127.85) --
	(147.11,127.79) --
	(147.41,127.72) --
	(147.71,127.66) --
	(148.01,127.60) --
	(148.31,127.54) --
	(148.61,127.48) --
	(148.91,127.42) --
	(149.21,127.36) --
	(149.50,127.30) --
	(149.80,127.24) --
	(150.10,127.18) --
	(150.40,127.12) --
	(150.70,127.06) --
	(151.00,127.01) --
	(151.30,126.95) --
	(151.60,126.89) --
	(151.90,126.84) --
	(152.20,126.78) --
	(152.50,126.72) --
	(152.80,126.67) --
	(153.10,126.61) --
	(153.40,126.56) --
	(153.70,126.50) --
	(154.00,126.45) --
	(154.30,126.40) --
	(154.60,126.34) --
	(154.89,126.29) --
	(155.19,126.24) --
	(155.49,126.18) --
	(155.79,126.13) --
	(156.09,126.08) --
	(156.39,126.03) --
	(156.69,125.98) --
	(156.99,125.93) --
	(157.29,125.88) --
	(157.59,125.83) --
	(157.89,125.78) --
	(158.19,125.73) --
	(158.49,125.68) --
	(158.79,125.63) --
	(159.09,125.58) --
	(159.39,125.53) --
	(159.69,125.48) --
	(159.99,125.43) --
	(160.28,125.39) --
	(160.58,125.34) --
	(160.88,125.29) --
	(161.18,125.24) --
	(161.48,125.20) --
	(161.78,125.15) --
	(162.08,125.11) --
	(162.38,125.06) --
	(162.68,125.01) --
	(162.98,124.97) --
	(163.28,124.92) --
	(163.58,124.88) --
	(163.88,124.84) --
	(164.18,124.79) --
	(164.48,124.75) --
	(164.78,124.70) --
	(165.08,124.66) --
	(165.38,124.62) --
	(165.67,124.57) --
	(165.97,124.53) --
	(166.27,124.49) --
	(166.57,124.45) --
	(166.87,124.40) --
	(167.17,124.36) --
	(167.47,124.32) --
	(167.77,124.28) --
	(168.07,124.24) --
	(168.37,124.20) --
	(168.67,124.16) --
	(168.97,124.12) --
	(169.27,124.08) --
	(169.57,124.04) --
	(169.87,124.00) --
	(170.17,123.96) --
	(170.47,123.92) --
	(170.77,123.88) --
	(171.07,123.84) --
	(171.36,123.80) --
	(171.66,123.76) --
	(171.96,123.72) --
	(172.26,123.68) --
	(172.56,123.65) --
	(172.86,123.61) --
	(173.16,123.57) --
	(173.46,123.53) --
	(173.76,123.50) --
	(174.06,123.46) --
	(174.36,123.42) --
	(174.66,123.39) --
	(174.96,123.35) --
	(175.26,123.31) --
	(175.56,123.28) --
	(175.86,123.24) --
	(176.16,123.21) --
	(176.46,123.17) --
	(176.75,123.14) --
	(177.05,123.10) --
	(177.35,123.07) --
	(177.65,123.03) --
	(177.95,123.00) --
	(178.25,122.96) --
	(178.55,122.93) --
	(178.85,122.89) --
	(179.15,122.86) --
	(179.45,122.83) --
	(179.75,122.79) --
	(180.05,122.76) --
	(180.35,122.73) --
	(180.65,122.69) --
	(180.95,122.66) --
	(181.25,122.63) --
	(181.55,122.59) --
	(181.85,122.56) --
	(182.14,122.53) --
	(182.44,122.50) --
	(182.74,122.46) --
	(183.04,122.43) --
	(183.34,122.40) --
	(183.64,122.37) --
	(183.94,122.34) --
	(184.24,122.31) --
	(184.54,122.28) --
	(184.84,122.24) --
	(185.14,122.21) --
	(185.44,122.18) --
	(185.74,122.15) --
	(186.04,122.12) --
	(186.34,122.09) --
	(186.64,122.06) --
	(186.94,122.03) --
	(187.24,122.00) --
	(187.53,121.97) --
	(187.83,121.94) --
	(188.13,121.91) --
	(188.43,121.88) --
	(188.73,121.86) --
	(189.03,121.83) --
	(189.33,121.80) --
	(189.63,121.77) --
	(189.93,121.74) --
	(190.23,121.71) --
	(190.53,121.68) --
	(190.83,121.66) --
	(191.13,121.63) --
	(191.43,121.60) --
	(191.73,121.57) --
	(192.03,121.54) --
	(192.33,121.52) --
	(192.63,121.49) --
	(192.93,121.46) --
	(193.22,121.43) --
	(193.52,121.41) --
	(193.82,121.38) --
	(194.12,121.35) --
	(194.42,121.33) --
	(194.72,121.30) --
	(195.02,121.27) --
	(195.32,121.25) --
	(195.62,121.22) --
	(195.92,121.19) --
	(196.22,121.17) --
	(196.52,121.14) --
	(196.82,121.12) --
	(197.12,121.09) --
	(197.42,121.06) --
	(197.72,121.04) --
	(198.02,121.01) --
	(198.32,120.99) --
	(198.61,120.96) --
	(198.91,120.94) --
	(199.21,120.91) --
	(199.51,120.89) --
	(199.81,120.86) --
	(200.11,120.84) --
	(200.41,120.81) --
	(200.71,120.79) --
	(201.01,120.77) --
	(201.31,120.74) --
	(201.61,120.72) --
	(201.91,120.69) --
	(202.21,120.67) --
	(202.51,120.65) --
	(202.81,120.62) --
	(203.11,120.60) --
	(203.41,120.57) --
	(203.71,120.55) --
	(204.00,120.53) --
	(204.30,120.50) --
	(204.60,120.48) --
	(204.90,120.46) --
	(205.20,120.44) --
	(205.50,120.41) --
	(205.80,120.39) --
	(206.10,120.37) --
	(206.40,120.34) --
	(206.70,120.32) --
	(207.00,120.30) --
	(207.30,120.28) --
	(207.60,120.26) --
	(207.90,120.23) --
	(208.20,120.21) --
	(208.50,120.19) --
	(208.80,120.17) --
	(209.10,120.15) --
	(209.39,120.12) --
	(209.69,120.10) --
	(209.99,120.08) --
	(210.29,120.06) --
	(210.59,120.04) --
	(210.89,120.02) --
	(211.19,120.00) --
	(211.49,119.97) --
	(211.79,119.95) --
	(212.09,119.93) --
	(212.39,119.91) --
	(212.69,119.89) --
	(212.99,119.87) --
	(213.29,119.85) --
	(213.59,119.83) --
	(213.89,119.81) --
	(214.19,119.79) --
	(214.49,119.77) --
	(214.79,119.75) --
	(215.08,119.73) --
	(215.38,119.71) --
	(215.68,119.69) --
	(215.98,119.67) --
	(216.28,119.65) --
	(216.58,119.63) --
	(216.88,119.61) --
	(217.18,119.59) --
	(217.48,119.57) --
	(217.78,119.55) --
	(218.08,119.53) --
	(218.38,119.51) --
	(218.68,119.49) --
	(218.98,119.47) --
	(219.28,119.45) --
	(219.58,119.43) --
	(219.88,119.41) --
	(220.18,119.40) --
	(220.47,119.38) --
	(220.77,119.36) --
	(221.07,119.34) --
	(221.37,119.32) --
	(221.67,119.30) --
	(221.97,119.28) --
	(222.27,119.26) --
	(222.57,119.25) --
	(222.87,119.23) --
	(223.17,119.21) --
	(223.47,119.19) --
	(223.77,119.17) --
	(224.07,119.16) --
	(224.37,119.14) --
	(224.67,119.12) --
	(224.97,119.10) --
	(225.27,119.08) --
	(225.57,119.07) --
	(225.86,119.05) --
	(226.16,119.03) --
	(226.46,119.01) --
	(226.76,119.00) --
	(227.06,118.98) --
	(227.36,118.96) --
	(227.66,118.94) --
	(227.96,118.93) --
	(228.26,118.91) --
	(228.56,118.89) --
	(228.86,118.88) --
	(229.16,118.86) --
	(229.46,118.84) --
	(229.76,118.82) --
	(230.06,118.81) --
	(230.36,118.79) --
	(230.66,118.77) --
	(230.96,118.76) --
	(231.25,118.74) --
	(231.55,118.73) --
	(231.85,118.71) --
	(232.15,118.69) --
	(232.45,118.68) --
	(232.75,118.66) --
	(233.05,118.64) --
	(233.35,118.63) --
	(233.65,118.61) --
	(233.95,118.59) --
	(234.25,118.58) --
	(234.55,118.56) --
	(234.85,118.55) --
	(235.15,118.53) --
	(235.45,118.52) --
	(235.75,118.50) --
	(236.05,118.48) --
	(236.35,118.47) --
	(236.65,118.45) --
	(236.94,118.44) --
	(237.24,118.42) --
	(237.54,118.41) --
	(237.84,118.39) --
	(238.14,118.38) --
	(238.44,118.36) --
	(238.74,118.35) --
	(239.04,118.33) --
	(239.34,118.31) --
	(239.64,118.30) --
	(239.94,118.28) --
	(240.24,118.27) --
	(240.54,118.25) --
	(240.84,118.24) --
	(241.14,118.23) --
	(241.44,118.21) --
	(241.74,118.20) --
	(242.04,118.18) --
	(242.33,118.17) --
	(242.63,118.15) --
	(242.93,118.14) --
	(243.23,118.12) --
	(243.53,118.11) --
	(243.83,118.09) --
	(244.13,118.08) --
	(244.43,118.07) --
	(244.73,118.05) --
	(245.03,118.04) --
	(245.33,118.02) --
	(245.63,118.01) --
	(245.93,117.99) --
	(246.23,117.98) --
	(246.53,117.97) --
	(246.83,117.95) --
	(247.13,117.94) --
	(247.43,117.92) --
	(247.72,117.91) --
	(248.02,117.90) --
	(248.32,117.88) --
	(248.62,117.87) --
	(248.92,117.86) --
	(249.22,117.84) --
	(249.52,117.83) --
	(249.82,117.82) --
	(250.12,117.80) --
	(250.42,117.79) --
	(250.72,117.78) --
	(251.02,117.76) --
	(251.32,117.75) --
	(251.62,117.74) --
	(251.92,117.72) --
	(252.22,117.71) --
	(252.52,117.70) --
	(252.82,117.68) --
	(253.11,117.67) --
	(253.41,117.66) --
	(253.71,117.64) --
	(254.01,117.63) --
	(254.31,117.62) --
	(254.61,117.60) --
	(254.91,117.59) --
	(255.21,117.58) --
	(255.51,117.57) --
	(255.81,117.55) --
	(256.11,117.54) --
	(256.41,117.53) --
	(256.71,117.52) --
	(257.01,117.50) --
	(257.31,117.49) --
	(257.61,117.48) --
	(257.91,117.47) --
	(258.21,117.45) --
	(258.51,117.44) --
	(258.80,117.43) --
	(259.10,117.42) --
	(259.40,117.40) --
	(259.70,117.39) --
	(260.00,117.38) --
	(260.30,117.37) --
	(260.60,117.36) --
	(260.90,117.34) --
	(261.20,117.33) --
	(261.50,117.32) --
	(261.80,117.31) --
	(262.10,117.30) --
	(262.40,117.28) --
	(262.70,117.27) --
	(263.00,117.26) --
	(263.30,117.25) --
	(263.60,117.24) --
	(263.90,117.22) --
	(264.19,117.21) --
	(264.49,117.20) --
	(264.79,117.19) --
	(265.09,117.18) --
	(265.39,117.17) --
	(265.69,117.15) --
	(265.99,117.14) --
	(266.29,117.13) --
	(266.59,117.12) --
	(266.89,117.11) --
	(267.19,117.10) --
	(267.49,117.09) --
	(267.79,117.08) --
	(268.09,117.06) --
	(268.39,117.05) --
	(268.69,117.04) --
	(268.99,117.03) --
	(269.29,117.02) --
	(269.58,117.01) --
	(269.88,117.00) --
	(270.18,116.99) --
	(270.48,116.97) --
	(270.78,116.96) --
	(271.08,116.95) --
	(271.38,116.94) --
	(271.68,116.93) --
	(271.98,116.92) --
	(272.28,116.91) --
	(272.58,116.90) --
	(272.88,116.89) --
	(273.18,116.88) --
	(273.48,116.87) --
	(273.78,116.86) --
	(274.08,116.85) --
	(274.38,116.83) --
	(274.68,116.82) --
	(274.98,116.81) --
	(275.27,116.80) --
	(275.57,116.79) --
	(275.87,116.78) --
	(276.17,116.77) --
	(276.47,116.76) --
	(276.77,116.75) --
	(277.07,116.74) --
	(277.37,116.73) --
	(277.67,116.72) --
	(277.97,116.71) --
	(278.27,116.70) --
	(278.57,116.69) --
	(278.87,116.68) --
	(279.17,116.67) --
	(279.47,116.66) --
	(279.77,116.65) --
	(280.07,116.64) --
	(280.37,116.63) --
	(280.66,116.62) --
	(280.96,116.61) --
	(281.26,116.60) --
	(281.56,116.59) --
	(281.86,116.58) --
	(282.16,116.57) --
	(282.46,116.56) --
	(282.76,116.55) --
	(283.06,116.54) --
	(283.36,116.53) --
	(283.66,116.52) --
	(283.96,116.51) --
	(284.26,116.50) --
	(284.56,116.49) --
	(284.86,116.48) --
	(285.16,116.47) --
	(285.46,116.46) --
	(285.76,116.45) --
	(286.05,116.44) --
	(286.35,116.43) --
	(286.65,116.42) --
	(286.95,116.41) --
	(287.25,116.40) --
	(287.55,116.39) --
	(287.85,116.39) --
	(288.15,116.38) --
	(288.45,116.37) --
	(288.75,116.36) --
	(289.05,116.35) --
	(289.35,116.34) --
	(289.65,116.33) --
	(289.95,116.32) --
	(290.25,116.31) --
	(290.55,116.30) --
	(290.85,116.29) --
	(291.15,116.28) --
	(291.44,116.27) --
	(291.74,116.27) --
	(292.04,116.26) --
	(292.34,116.25) --
	(292.64,116.24) --
	(292.94,116.23) --
	(293.24,116.22) --
	(293.54,116.21) --
	(293.84,116.20) --
	(294.14,116.19) --
	(294.44,116.18) --
	(294.74,116.18) --
	(295.04,116.17) --
	(295.34,116.16) --
	(295.64,116.15) --
	(295.94,116.14) --
	(296.24,116.13) --
	(296.54,116.12) --
	(296.84,116.11) --
	(297.13,116.11) --
	(297.43,116.10) --
	(297.73,116.09) --
	(298.03,116.08) --
	(298.33,116.07) --
	(298.63,116.06) --
	(298.93,116.05) --
	(299.23,116.05) --
	(299.53,116.04) --
	(299.83,116.03) --
	(300.13,116.02) --
	(300.43,116.01) --
	(300.73,116.00) --
	(301.03,115.99) --
	(301.33,115.99) --
	(301.63,115.98) --
	(301.93,115.97) --
	(302.23,115.96) --
	(302.52,115.95) --
	(302.82,115.94) --
	(303.12,115.94) --
	(303.42,115.93) --
	(303.72,115.92) --
	(304.02,115.91) --
	(304.32,115.90) --
	(304.62,115.90) --
	(304.92,115.89) --
	(305.22,115.88) --
	(305.52,115.87) --
	(305.82,115.86) --
	(306.12,115.85) --
	(306.42,115.85) --
	(306.72,115.84) --
	(307.02,115.83) --
	(307.32,115.82) --
	(307.62,115.81) --
	(307.91,115.81) --
	(308.21,115.80) --
	(308.51,115.79) --
	(308.81,115.78) --
	(309.11,115.77) --
	(309.41,115.77) --
	(309.71,115.76) --
	(310.01,115.75) --
	(310.31,115.74) --
	(310.61,115.74) --
	(310.91,115.73) --
	(311.21,115.72) --
	(311.51,115.71) --
	(311.81,115.70) --
	(312.11,115.70) --
	(312.41,115.69) --
	(312.71,115.68) --
	(313.01,115.67) --
	(313.30,115.67) --
	(313.60,115.66) --
	(313.90,115.65) --
	(314.20,115.64) --
	(314.50,115.64) --
	(314.80,115.63) --
	(315.10,115.62) --
	(315.40,115.61) --
	(315.70,115.61) --
	(316.00,115.60) --
	(316.30,115.59) --
	(316.60,115.58) --
	(316.90,115.58) --
	(317.20,115.57) --
	(317.50,115.56) --
	(317.80,115.55) --
	(318.10,115.55) --
	(318.40,115.54) --
	(318.70,115.53) --
	(318.99,115.53) --
	(319.29,115.52) --
	(319.59,115.51) --
	(319.89,115.50) --
	(320.19,115.50) --
	(320.49,115.49) --
	(320.79,115.48) --
	(321.09,115.47) --
	(321.39,115.47) --
	(321.69,115.46) --
	(321.99,115.45) --
	(322.29,115.45) --
	(322.59,115.44) --
	(322.89,115.43) --
	(323.19,115.42) --
	(323.49,115.42) --
	(323.79,115.41) --
	(324.09,115.40) --
	(324.38,115.40) --
	(324.68,115.39) --
	(324.98,115.38) --
	(325.28,115.38) --
	(325.58,115.37) --
	(325.88,115.36) --
	(326.18,115.35) --
	(326.48,115.35) --
	(326.78,115.34) --
	(327.08,115.33) --
	(327.38,115.33) --
	(327.68,115.32) --
	(327.98,115.31) --
	(328.28,115.31) --
	(328.58,115.30) --
	(328.88,115.29) --
	(329.18,115.29) --
	(329.48,115.28) --
	(329.77,115.27) --
	(330.07,115.27) --
	(330.37,115.26) --
	(330.67,115.25) --
	(330.97,115.25) --
	(331.27,115.24) --
	(331.57,115.23) --
	(331.87,115.23) --
	(332.17,115.22) --
	(332.47,115.21) --
	(332.77,115.21) --
	(333.07,115.20) --
	(333.37,115.19) --
	(333.67,115.19) --
	(333.97,115.18) --
	(334.27,115.17) --
	(334.57,115.17) --
	(334.87,115.16) --
	(335.16,115.15) --
	(335.46,115.15) --
	(335.76,115.14) --
	(336.06,115.14) --
	(336.36,115.13) --
	(336.66,115.12) --
	(336.96,115.12) --
	(337.26,115.11) --
	(337.56,115.10) --
	(337.86,115.10) --
	(338.16,115.09) --
	(338.46,115.08) --
	(338.76,115.08) --
	(339.06,115.07) --
	(339.36,115.07) --
	(339.66,115.06) --
	(339.96,115.05) --
	(340.26,115.05) --
	(340.56,115.04) --
	(340.85,115.03) --
	(341.15,115.03) --
	(341.45,115.02) --
	(341.75,115.02) --
	(342.05,115.01) --
	(342.35,115.00) --
	(342.65,115.00) --
	(342.95,114.99) --
	(343.25,114.99) --
	(343.55,114.98) --
	(343.85,114.97) --
	(344.15,114.97) --
	(344.45,114.96) --
	(344.75,114.95) --
	(345.05,114.95) --
	(345.35,114.94) --
	(345.65,114.94) --
	(345.95,114.93) --
	(346.24,114.92) --
	(346.54,114.92) --
	(346.84,114.91) --
	(347.14,114.91) --
	(347.44,114.90) --
	(347.74,114.90) --
	(348.04,114.89) --
	(348.34,114.88) --
	(348.64,114.88) --
	(348.94,114.87) --
	(349.24,114.87) --
	(349.54,114.86) --
	(349.84,114.85) --
	(350.14,114.85) --
	(350.44,114.84) --
	(350.74,114.84) --
	(351.04,114.83) --
	(351.34,114.83) --
	(351.63,114.82) --
	(351.93,114.81) --
	(352.23,114.81) --
	(352.53,114.80) --
	(352.83,114.80) --
	(353.13,114.79) --
	(353.43,114.78) --
	(353.73,114.78) --
	(354.03,114.77) --
	(354.33,114.77) --
	(354.63,114.76) --
	(354.93,114.76) --
	(355.23,114.75) --
	(355.53,114.75) --
	(355.83,114.74) --
	(356.13,114.73) --
	(356.43,114.73) --
	(356.73,114.72) --
	(357.02,114.72) --
	(357.32,114.71) --
	(357.62,114.71) --
	(357.92,114.70) --
	(358.22,114.70) --
	(358.52,114.69) --
	(358.82,114.68) --
	(359.12,114.68) --
	(359.42,114.67) --
	(359.72,114.67) --
	(360.02,114.66) --
	(360.32,114.66);
\definecolor{drawColor}{RGB}{190,190,190}

\path[draw=drawColor,line width= 0.4pt,line join=round,line cap=round] ( 49.20,109.18) -- (372.28,109.18);
\definecolor{drawColor}{RGB}{0,0,0}

\path[draw=drawColor,line width= 0.4pt,line join=round,line cap=round] (275.67,223.75) -- (286.47,223.75);

\path[draw=drawColor,line width= 0.4pt,dash pattern=on 4pt off 4pt ,line join=round,line cap=round] (275.67,209.35) -- (286.47,209.35);
\definecolor{drawColor}{RGB}{190,190,190}

\path[draw=drawColor,line width= 0.4pt,line join=round,line cap=round] (275.67,194.95) -- (286.47,194.95);
\definecolor{drawColor}{RGB}{0,0,0}

\node[text=drawColor,anchor=base west,inner sep=0pt, outer sep=0pt, scale=  0.60] at (291.87,221.68) {MIPS};

\node[text=drawColor,anchor=base west,inner sep=0pt, outer sep=0pt, scale=  0.60] at (291.87,207.28) {$\text{MIPS}^\text{max}\text{ und }\text{MIPS}^\text{min}$};

\node[text=drawColor,anchor=base west,inner sep=0pt, outer sep=0pt, scale=  0.60] at (291.87,192.88) {$\lim\limits_{t_\text{max}\rightarrow\infty} \text{MIPS}$};
\end{scope}
\end{tikzpicture}
}
          }
          \only<3>{
              \resizebox{\linewidth}{!}{
              % Created by tikzDevice version 0.8.1 on 2015-04-28 22:25:50
% !TEX encoding = UTF-8 Unicode
\begin{tikzpicture}[x=1pt,y=1pt]
\definecolor{fillColor}{RGB}{255,255,255}
\path[use as bounding box,fill=fillColor,fill opacity=0.00] (0,0) rectangle (397.48,289.08);
\begin{scope}
\path[clip] ( 49.20, 61.20) rectangle (372.28,239.88);
\definecolor{drawColor}{RGB}{190,190,190}

\path[draw=drawColor,line width= 0.4pt,line join=round,line cap=round] ( 49.20,109.18) -- (372.28,109.18);
\definecolor{drawColor}{RGB}{0,0,0}

\path[draw=drawColor,line width= 0.4pt,line join=round,line cap=round] ( 66.20,289.08) --
	( 66.26,287.40) --
	( 66.56,279.59) --
	( 66.86,272.31) --
	( 67.16,265.52) --
	( 67.45,259.16) --
	( 67.75,253.20) --
	( 68.05,247.61) --
	( 68.35,242.34) --
	( 68.65,237.37) --
	( 68.95,232.67) --
	( 69.25,228.23) --
	( 69.55,224.02) --
	( 69.85,220.03) --
	( 70.15,216.24) --
	( 70.45,212.63) --
	( 70.75,209.19) --
	( 71.05,205.91) --
	( 71.35,202.78) --
	( 71.65,199.79) --
	( 71.95,196.93) --
	( 72.25,194.19) --
	( 72.55,191.57) --
	( 72.84,189.05) --
	( 73.14,186.63) --
	( 73.44,184.31) --
	( 73.74,182.07) --
	( 74.04,179.92) --
	( 74.34,177.85) --
	( 74.64,175.86) --
	( 74.94,173.93) --
	( 75.24,172.08) --
	( 75.54,170.28) --
	( 75.84,168.55) --
	( 76.14,166.88) --
	( 76.44,165.26) --
	( 76.74,163.69) --
	( 77.04,162.17) --
	( 77.34,160.70) --
	( 77.64,159.28) --
	( 77.94,157.89) --
	( 78.23,156.55) --
	( 78.53,155.25) --
	( 78.83,153.99) --
	( 79.13,152.76) --
	( 79.43,151.57) --
	( 79.73,150.41) --
	( 80.03,149.28) --
	( 80.33,148.18) --
	( 80.63,147.11) --
	( 80.93,146.07) --
	( 81.23,145.06) --
	( 81.53,144.07) --
	( 81.83,143.11) --
	( 82.13,142.17) --
	( 82.43,141.25) --
	( 82.73,140.36) --
	( 83.03,139.49) --
	( 83.33,138.64) --
	( 83.63,137.81) --
	( 83.92,137.00) --
	( 84.22,136.20) --
	( 84.52,135.43) --
	( 84.82,134.67) --
	( 85.12,133.93) --
	( 85.42,133.21) --
	( 85.72,132.50) --
	( 86.02,131.80) --
	( 86.32,131.12) --
	( 86.62,130.46) --
	( 86.92,129.81) --
	( 87.22,129.17) --
	( 87.52,128.55) --
	( 87.82,127.94) --
	( 88.12,127.34) --
	( 88.42,126.75) --
	( 88.72,126.17) --
	( 89.02,125.61) --
	( 89.31,125.05) --
	( 89.61,124.51) --
	( 89.91,123.98) --
	( 90.21,123.45) --
	( 90.51,122.94) --
	( 90.81,122.44) --
	( 91.11,121.94) --
	( 91.41,121.45) --
	( 91.71,120.98) --
	( 92.01,120.51) --
	( 92.31,120.05) --
	( 92.61,119.59) --
	( 92.91,119.15) --
	( 93.21,118.71) --
	( 93.51,118.28) --
	( 93.81,117.86) --
	( 94.11,117.44) --
	( 94.41,117.03) --
	( 94.70,116.63) --
	( 95.00,116.24) --
	( 95.30,115.85) --
	( 95.60,115.46) --
	( 95.90,115.09) --
	( 96.20,114.72) --
	( 96.50,114.35) --
	( 96.80,113.99) --
	( 97.10,113.64) --
	( 97.40,113.29) --
	( 97.70,112.95) --
	( 98.00,112.61) --
	( 98.30,112.27) --
	( 98.60,111.95) --
	( 98.90,111.62) --
	( 99.20,111.30) --
	( 99.50,110.99) --
	( 99.80,110.68) --
	(100.09,110.38) --
	(100.39,110.07) --
	(100.69,109.78) --
	(100.99,109.49) --
	(101.29,109.20) --
	(101.59,108.91) --
	(101.89,108.63) --
	(102.19,108.36) --
	(102.49,108.08) --
	(102.79,107.81) --
	(103.09,107.55) --
	(103.39,107.29) --
	(103.69,107.03) --
	(103.99,106.77) --
	(104.29,106.52) --
	(104.59,106.27) --
	(104.89,106.03) --
	(105.19,105.78) --
	(105.49,105.54) --
	(105.78,105.31) --
	(106.08,105.07) --
	(106.38,104.84) --
	(106.68,104.62) --
	(106.98,104.39) --
	(107.28,104.17) --
	(107.58,103.95) --
	(107.88,103.73) --
	(108.18,103.52) --
	(108.48,103.31) --
	(108.78,103.10) --
	(109.08,102.89) --
	(109.38,102.69) --
	(109.68,102.48) --
	(109.98,102.28) --
	(110.28,102.09) --
	(110.58,101.89) --
	(110.88,101.70) --
	(111.17,101.51) --
	(111.47,101.32) --
	(111.77,101.13) --
	(112.07,100.95) --
	(112.37,100.77) --
	(112.67,100.59) --
	(112.97,100.41) --
	(113.27,100.23) --
	(113.57,100.06) --
	(113.87, 99.88) --
	(114.17, 99.71) --
	(114.47, 99.55) --
	(114.77, 99.38) --
	(115.07, 99.21) --
	(115.37, 99.05) --
	(115.67, 98.89) --
	(115.97, 98.73) --
	(116.27, 98.57) --
	(116.56, 98.41) --
	(116.86, 98.26) --
	(117.16, 98.10) --
	(117.46, 97.95) --
	(117.76, 97.80) --
	(118.06, 97.65) --
	(118.36, 97.50) --
	(118.66,138.54) --
	(118.96,138.20) --
	(119.26,137.85) --
	(119.56,137.51) --
	(119.86,137.17) --
	(120.16,136.84) --
	(120.46,136.51) --
	(120.76,136.18) --
	(121.06,135.85) --
	(121.36,135.53) --
	(121.66,135.21) --
	(121.95,134.90) --
	(122.25,134.59) --
	(122.55,134.28) --
	(122.85,133.97) --
	(123.15,133.66) --
	(123.45,133.36) --
	(123.75,133.06) --
	(124.05,132.77) --
	(124.35,132.48) --
	(124.65,132.18) --
	(124.95,131.90) --
	(125.25,131.61) --
	(125.55,131.33) --
	(125.85,131.05) --
	(126.15,130.77) --
	(126.45,130.49) --
	(126.75,130.22) --
	(127.05,129.95) --
	(127.35,129.68) --
	(127.64,129.42) --
	(127.94,129.15) --
	(128.24,128.89) --
	(128.54,128.63) --
	(128.84,128.37) --
	(129.14,128.12) --
	(129.44,127.87) --
	(129.74,127.62) --
	(130.04,127.37) --
	(130.34,127.12) --
	(130.64,126.88) --
	(130.94,126.63) --
	(131.24,126.39) --
	(131.54,126.16) --
	(131.84,125.92) --
	(132.14,125.68) --
	(132.44,125.45) --
	(132.74,125.22) --
	(133.03,124.99) --
	(133.33,124.76) --
	(133.63,124.54) --
	(133.93,124.31) --
	(134.23,124.09) --
	(134.53,123.87) --
	(134.83,123.65) --
	(135.13,123.44) --
	(135.43,123.22) --
	(135.73,123.01) --
	(136.03,122.80) --
	(136.33,122.59) --
	(136.63,122.38) --
	(136.93,122.17) --
	(137.23,121.97) --
	(137.53,121.76) --
	(137.83,121.56) --
	(138.13,121.36) --
	(138.42,121.16) --
	(138.72,120.96) --
	(139.02,120.77) --
	(139.32,120.57) --
	(139.62,120.38) --
	(139.92,120.19) --
	(140.22,120.00) --
	(140.52,119.81) --
	(140.82,119.62) --
	(141.12,119.43) --
	(141.42,119.25) --
	(141.72,119.06) --
	(142.02,118.88) --
	(142.32,118.70) --
	(142.62,118.52) --
	(142.92,118.34) --
	(143.22,118.16) --
	(143.52,117.99) --
	(143.81,117.81) --
	(144.11,117.64) --
	(144.41,117.47) --
	(144.71,117.29) --
	(145.01,117.12) --
	(145.31,116.96) --
	(145.61,116.79) --
	(145.91,116.62) --
	(146.21,116.45) --
	(146.51,116.29) --
	(146.81,116.13) --
	(147.11,115.97) --
	(147.41,115.80) --
	(147.71,115.64) --
	(148.01,115.48) --
	(148.31,115.33) --
	(148.61,115.17) --
	(148.91,115.01) --
	(149.21,114.86) --
	(149.50,114.71) --
	(149.80,114.55) --
	(150.10,114.40) --
	(150.40,114.25) --
	(150.70,114.10) --
	(151.00,113.95) --
	(151.30,113.80) --
	(151.60,113.66) --
	(151.90,113.51) --
	(152.20,113.37) --
	(152.50,113.22) --
	(152.80,113.08) --
	(153.10,112.94) --
	(153.40,112.79) --
	(153.70,112.65) --
	(154.00,112.51) --
	(154.30,112.38) --
	(154.60,112.24) --
	(154.89,112.10) --
	(155.19,111.96) --
	(155.49,111.83) --
	(155.79,111.69) --
	(156.09,111.56) --
	(156.39,111.43) --
	(156.69,111.29) --
	(156.99,111.16) --
	(157.29,111.03) --
	(157.59,110.90) --
	(157.89,110.77) --
	(158.19,110.65) --
	(158.49,110.52) --
	(158.79,110.39) --
	(159.09,110.27) --
	(159.39,110.14) --
	(159.69,110.02) --
	(159.99,109.89) --
	(160.28,109.77) --
	(160.58,109.65) --
	(160.88,109.53) --
	(161.18,109.40) --
	(161.48,109.28) --
	(161.78,109.16) --
	(162.08,109.05) --
	(162.38,108.93) --
	(162.68,108.81) --
	(162.98,108.69) --
	(163.28,108.58) --
	(163.58,108.46) --
	(163.88,108.35) --
	(164.18,108.23) --
	(164.48,108.12) --
	(164.78,108.01) --
	(165.08,107.89) --
	(165.38,107.78) --
	(165.67,107.67) --
	(165.97,107.56) --
	(166.27,107.45) --
	(166.57,107.34) --
	(166.87,107.23) --
	(167.17,107.13) --
	(167.47,107.02) --
	(167.77,106.91) --
	(168.07,106.81) --
	(168.37,106.70) --
	(168.67,106.60) --
	(168.97,106.49) --
	(169.27,106.39) --
	(169.57,106.28) --
	(169.87,106.18) --
	(170.17,106.08) --
	(170.47,105.98) --
	(170.77,105.88) --
	(171.07,105.78) --
	(171.36,105.68) --
	(171.66,105.58) --
	(171.96,105.48) --
	(172.26,105.38) --
	(172.56,105.28) --
	(172.86,105.18) --
	(173.16,105.09) --
	(173.46,104.99) --
	(173.76,104.89) --
	(174.06,104.80) --
	(174.36,104.70) --
	(174.66,104.61) --
	(174.96,104.51) --
	(175.26,104.42) --
	(175.56,104.33) --
	(175.86,104.24) --
	(176.16,104.14) --
	(176.46,104.05) --
	(176.75,103.96) --
	(177.05,103.87) --
	(177.35,103.78) --
	(177.65,103.69) --
	(177.95,103.60) --
	(178.25,103.51) --
	(178.55,103.42) --
	(178.85,103.33) --
	(179.15,123.85) --
	(179.45,123.71) --
	(179.75,123.57) --
	(180.05,123.43) --
	(180.35,123.30) --
	(180.65,123.16) --
	(180.95,123.03) --
	(181.25,122.89) --
	(181.55,122.76) --
	(181.85,122.63) --
	(182.14,122.50) --
	(182.44,122.36) --
	(182.74,122.23) --
	(183.04,122.10) --
	(183.34,121.97) --
	(183.64,121.84) --
	(183.94,121.72) --
	(184.24,121.59) --
	(184.54,121.46) --
	(184.84,121.33) --
	(185.14,121.21) --
	(185.44,121.08) --
	(185.74,120.96) --
	(186.04,120.83) --
	(186.34,120.71) --
	(186.64,120.59) --
	(186.94,120.47) --
	(187.24,120.34) --
	(187.53,120.22) --
	(187.83,120.10) --
	(188.13,119.98) --
	(188.43,119.86) --
	(188.73,119.74) --
	(189.03,119.62) --
	(189.33,119.51) --
	(189.63,119.39) --
	(189.93,119.27) --
	(190.23,119.16) --
	(190.53,119.04) --
	(190.83,118.92) --
	(191.13,118.81) --
	(191.43,118.70) --
	(191.73,118.58) --
	(192.03,118.47) --
	(192.33,118.36) --
	(192.63,118.24) --
	(192.93,118.13) --
	(193.22,118.02) --
	(193.52,117.91) --
	(193.82,117.80) --
	(194.12,117.69) --
	(194.42,117.58) --
	(194.72,117.47) --
	(195.02,117.36) --
	(195.32,117.25) --
	(195.62,117.15) --
	(195.92,117.04) --
	(196.22,116.93) --
	(196.52,116.83) --
	(196.82,116.72) --
	(197.12,116.62) --
	(197.42,116.51) --
	(197.72,116.41) --
	(198.02,116.30) --
	(198.32,116.20) --
	(198.61,116.10) --
	(198.91,116.00) --
	(199.21,115.89) --
	(199.51,115.79) --
	(199.81,115.69) --
	(200.11,115.59) --
	(200.41,115.49) --
	(200.71,115.39) --
	(201.01,115.29) --
	(201.31,115.19) --
	(201.61,115.09) --
	(201.91,114.99) --
	(202.21,114.90) --
	(202.51,114.80) --
	(202.81,114.70) --
	(203.11,114.61) --
	(203.41,114.51) --
	(203.71,114.41) --
	(204.00,114.32) --
	(204.30,114.22) --
	(204.60,114.13) --
	(204.90,114.03) --
	(205.20,113.94) --
	(205.50,113.85) --
	(205.80,113.75) --
	(206.10,113.66) --
	(206.40,113.57) --
	(206.70,113.48) --
	(207.00,113.39) --
	(207.30,113.29) --
	(207.60,113.20) --
	(207.90,113.11) --
	(208.20,113.02) --
	(208.50,112.93) --
	(208.80,112.84) --
	(209.10,112.75) --
	(209.39,112.67) --
	(209.69,112.58) --
	(209.99,112.49) --
	(210.29,112.40) --
	(210.59,112.31) --
	(210.89,112.23) --
	(211.19,112.14) --
	(211.49,112.05) --
	(211.79,111.97) --
	(212.09,111.88) --
	(212.39,111.80) --
	(212.69,111.71) --
	(212.99,111.63) --
	(213.29,111.54) --
	(213.59,111.46) --
	(213.89,111.38) --
	(214.19,111.29) --
	(214.49,111.21) --
	(214.79,111.13) --
	(215.08,111.04) --
	(215.38,110.96) --
	(215.68,110.88) --
	(215.98,110.80) --
	(216.28,110.72) --
	(216.58,110.64) --
	(216.88,110.56) --
	(217.18,110.48) --
	(217.48,110.40) --
	(217.78,110.32) --
	(218.08,110.24) --
	(218.38,110.16) --
	(218.68,110.08) --
	(218.98,110.00) --
	(219.28,109.92) --
	(219.58,109.84) --
	(219.88,109.77) --
	(220.18,109.69) --
	(220.47,109.61) --
	(220.77,109.54) --
	(221.07,109.46) --
	(221.37,109.38) --
	(221.67,109.31) --
	(221.97,109.23) --
	(222.27,109.16) --
	(222.57,109.08) --
	(222.87,109.01) --
	(223.17,108.93) --
	(223.47,108.86) --
	(223.77,108.78) --
	(224.07,108.71) --
	(224.37,108.64) --
	(224.67,108.56) --
	(224.97,108.49) --
	(225.27,108.42) --
	(225.57,108.35) --
	(225.86,108.27) --
	(226.16,108.20) --
	(226.46,108.13) --
	(226.76,108.06) --
	(227.06,107.99) --
	(227.36,107.92) --
	(227.66,107.85) --
	(227.96,107.78) --
	(228.26,107.71) --
	(228.56,107.64) --
	(228.86,107.57) --
	(229.16,107.50) --
	(229.46,107.43) --
	(229.76,107.36) --
	(230.06,107.29) --
	(230.36,107.22) --
	(230.66,107.15) --
	(230.96,107.08) --
	(231.25,107.02) --
	(231.55,106.95) --
	(231.85,106.88) --
	(232.15,106.82) --
	(232.45,106.75) --
	(232.75,106.68) --
	(233.05,106.62) --
	(233.35,106.55) --
	(233.65,106.48) --
	(233.95,106.42) --
	(234.25,106.35) --
	(234.55,106.29) --
	(234.85,106.22) --
	(235.15,106.16) --
	(235.45,106.09) --
	(235.75,106.03) --
	(236.05,105.96) --
	(236.35,105.90) --
	(236.65,105.84) --
	(236.94,105.77) --
	(237.24,105.71) --
	(237.54,105.65) --
	(237.84,105.58) --
	(238.14,105.52) --
	(238.44,105.46) --
	(238.74,105.40) --
	(239.04,105.34) --
	(239.34,105.27) --
	(239.64,118.95) --
	(239.94,118.86) --
	(240.24,118.78) --
	(240.54,118.69) --
	(240.84,118.61) --
	(241.14,118.53) --
	(241.44,118.45) --
	(241.74,118.36) --
	(242.04,118.28) --
	(242.33,118.20) --
	(242.63,118.12) --
	(242.93,118.04) --
	(243.23,117.95) --
	(243.53,117.87) --
	(243.83,117.79) --
	(244.13,117.71) --
	(244.43,117.63) --
	(244.73,117.55) --
	(245.03,117.47) --
	(245.33,117.39) --
	(245.63,117.32) --
	(245.93,117.24) --
	(246.23,117.16) --
	(246.53,117.08) --
	(246.83,117.00) --
	(247.13,116.92) --
	(247.43,116.85) --
	(247.72,116.77) --
	(248.02,116.69) --
	(248.32,116.62) --
	(248.62,116.54) --
	(248.92,116.46) --
	(249.22,116.39) --
	(249.52,116.31) --
	(249.82,116.24) --
	(250.12,116.16) --
	(250.42,116.09) --
	(250.72,116.01) --
	(251.02,115.94) --
	(251.32,115.86) --
	(251.62,115.79) --
	(251.92,115.71) --
	(252.22,115.64) --
	(252.52,115.57) --
	(252.82,115.49) --
	(253.11,115.42) --
	(253.41,115.35) --
	(253.71,115.27) --
	(254.01,115.20) --
	(254.31,115.13) --
	(254.61,115.06) --
	(254.91,114.99) --
	(255.21,114.91) --
	(255.51,114.84) --
	(255.81,114.77) --
	(256.11,114.70) --
	(256.41,114.63) --
	(256.71,114.56) --
	(257.01,114.49) --
	(257.31,114.42) --
	(257.61,114.35) --
	(257.91,114.28) --
	(258.21,114.21) --
	(258.51,114.14) --
	(258.80,114.07) --
	(259.10,114.00) --
	(259.40,113.94) --
	(259.70,113.87) --
	(260.00,113.80) --
	(260.30,113.73) --
	(260.60,113.66) --
	(260.90,113.60) --
	(261.20,113.53) --
	(261.50,113.46) --
	(261.80,113.39) --
	(262.10,113.33) --
	(262.40,113.26) --
	(262.70,113.20) --
	(263.00,113.13) --
	(263.30,113.06) --
	(263.60,113.00) --
	(263.90,112.93) --
	(264.19,112.87) --
	(264.49,112.80) --
	(264.79,112.74) --
	(265.09,112.67) --
	(265.39,112.61) --
	(265.69,112.54) --
	(265.99,112.48) --
	(266.29,112.41) --
	(266.59,112.35) --
	(266.89,112.29) --
	(267.19,112.22) --
	(267.49,112.16) --
	(267.79,112.10) --
	(268.09,112.03) --
	(268.39,111.97) --
	(268.69,111.91) --
	(268.99,111.85) --
	(269.29,111.78) --
	(269.58,111.72) --
	(269.88,111.66) --
	(270.18,111.60) --
	(270.48,111.54) --
	(270.78,111.47) --
	(271.08,111.41) --
	(271.38,111.35) --
	(271.68,111.29) --
	(271.98,111.23) --
	(272.28,111.17) --
	(272.58,111.11) --
	(272.88,111.05) --
	(273.18,110.99) --
	(273.48,110.93) --
	(273.78,110.87) --
	(274.08,110.81) --
	(274.38,110.75) --
	(274.68,110.69) --
	(274.98,110.63) --
	(275.27,110.57) --
	(275.57,110.51) --
	(275.87,110.46) --
	(276.17,110.40) --
	(276.47,110.34) --
	(276.77,110.28) --
	(277.07,110.22) --
	(277.37,110.17) --
	(277.67,110.11) --
	(277.97,110.05) --
	(278.27,109.99) --
	(278.57,109.94) --
	(278.87,109.88) --
	(279.17,109.82) --
	(279.47,109.77) --
	(279.77,109.71) --
	(280.07,109.65) --
	(280.37,109.60) --
	(280.66,109.54) --
	(280.96,109.48) --
	(281.26,109.43) --
	(281.56,109.37) --
	(281.86,109.32) --
	(282.16,109.26) --
	(282.46,109.21) --
	(282.76,109.15) --
	(283.06,109.10) --
	(283.36,109.04) --
	(283.66,108.99) --
	(283.96,108.93) --
	(284.26,108.88) --
	(284.56,108.83) --
	(284.86,108.77) --
	(285.16,108.72) --
	(285.46,108.66) --
	(285.76,108.61) --
	(286.05,108.56) --
	(286.35,108.50) --
	(286.65,108.45) --
	(286.95,108.40) --
	(287.25,108.34) --
	(287.55,108.29) --
	(287.85,108.24) --
	(288.15,108.19) --
	(288.45,108.13) --
	(288.75,108.08) --
	(289.05,108.03) --
	(289.35,107.98) --
	(289.65,107.93) --
	(289.95,107.87) --
	(290.25,107.82) --
	(290.55,107.77) --
	(290.85,107.72) --
	(291.15,107.67) --
	(291.44,107.62) --
	(291.74,107.57) --
	(292.04,107.52) --
	(292.34,107.47) --
	(292.64,107.42) --
	(292.94,107.37) --
	(293.24,107.31) --
	(293.54,107.26) --
	(293.84,107.21) --
	(294.14,107.17) --
	(294.44,107.12) --
	(294.74,107.07) --
	(295.04,107.02) --
	(295.34,106.97) --
	(295.64,106.92) --
	(295.94,106.87) --
	(296.24,106.82) --
	(296.54,106.77) --
	(296.84,106.72) --
	(297.13,106.67) --
	(297.43,106.63) --
	(297.73,106.58) --
	(298.03,106.53) --
	(298.33,106.48) --
	(298.63,106.43) --
	(298.93,106.38) --
	(299.23,106.34) --
	(299.53,106.29) --
	(299.83,106.24) --
	(300.13,116.49) --
	(300.43,116.43) --
	(300.73,116.37) --
	(301.03,116.31) --
	(301.33,116.26) --
	(301.63,116.20) --
	(301.93,116.14) --
	(302.23,116.08) --
	(302.52,116.02) --
	(302.82,115.96) --
	(303.12,115.90) --
	(303.42,115.84) --
	(303.72,115.78) --
	(304.02,115.73) --
	(304.32,115.67) --
	(304.62,115.61) --
	(304.92,115.55) --
	(305.22,115.49) --
	(305.52,115.44) --
	(305.82,115.38) --
	(306.12,115.32) --
	(306.42,115.26) --
	(306.72,115.21) --
	(307.02,115.15) --
	(307.32,115.09) --
	(307.62,115.04) --
	(307.91,114.98) --
	(308.21,114.92) --
	(308.51,114.87) --
	(308.81,114.81) --
	(309.11,114.76) --
	(309.41,114.70) --
	(309.71,114.64) --
	(310.01,114.59) --
	(310.31,114.53) --
	(310.61,114.48) --
	(310.91,114.42) --
	(311.21,114.37) --
	(311.51,114.31) --
	(311.81,114.26) --
	(312.11,114.20) --
	(312.41,114.15) --
	(312.71,114.10) --
	(313.01,114.04) --
	(313.30,113.99) --
	(313.60,113.93) --
	(313.90,113.88) --
	(314.20,113.83) --
	(314.50,113.77) --
	(314.80,113.72) --
	(315.10,113.67) --
	(315.40,113.61) --
	(315.70,113.56) --
	(316.00,113.51) --
	(316.30,113.45) --
	(316.60,113.40) --
	(316.90,113.35) --
	(317.20,113.30) --
	(317.50,113.24) --
	(317.80,113.19) --
	(318.10,113.14) --
	(318.40,113.09) --
	(318.70,113.03) --
	(318.99,112.98) --
	(319.29,112.93) --
	(319.59,112.88) --
	(319.89,112.83) --
	(320.19,112.78) --
	(320.49,112.73) --
	(320.79,112.67) --
	(321.09,112.62) --
	(321.39,112.57) --
	(321.69,112.52) --
	(321.99,112.47) --
	(322.29,112.42) --
	(322.59,112.37) --
	(322.89,112.32) --
	(323.19,112.27) --
	(323.49,112.22) --
	(323.79,112.17) --
	(324.09,112.12) --
	(324.38,112.07) --
	(324.68,112.02) --
	(324.98,111.97) --
	(325.28,111.92) --
	(325.58,111.87) --
	(325.88,111.82) --
	(326.18,111.77) --
	(326.48,111.73) --
	(326.78,111.68) --
	(327.08,111.63) --
	(327.38,111.58) --
	(327.68,111.53) --
	(327.98,111.48) --
	(328.28,111.43) --
	(328.58,111.39) --
	(328.88,111.34) --
	(329.18,111.29) --
	(329.48,111.24) --
	(329.77,111.19) --
	(330.07,111.15) --
	(330.37,111.10) --
	(330.67,111.05) --
	(330.97,111.00) --
	(331.27,110.96) --
	(331.57,110.91) --
	(331.87,110.86) --
	(332.17,110.82) --
	(332.47,110.77) --
	(332.77,110.72) --
	(333.07,110.68) --
	(333.37,110.63) --
	(333.67,110.58) --
	(333.97,110.54) --
	(334.27,110.49) --
	(334.57,110.45) --
	(334.87,110.40) --
	(335.16,110.35) --
	(335.46,110.31) --
	(335.76,110.26) --
	(336.06,110.22) --
	(336.36,110.17) --
	(336.66,110.13) --
	(336.96,110.08) --
	(337.26,110.03) --
	(337.56,109.99) --
	(337.86,109.94) --
	(338.16,109.90) --
	(338.46,109.85) --
	(338.76,109.81) --
	(339.06,109.77) --
	(339.36,109.72) --
	(339.66,109.68) --
	(339.96,109.63) --
	(340.26,109.59) --
	(340.56,109.54) --
	(340.85,109.50) --
	(341.15,109.46) --
	(341.45,109.41) --
	(341.75,109.37) --
	(342.05,109.32) --
	(342.35,109.28) --
	(342.65,109.24) --
	(342.95,109.19) --
	(343.25,109.15) --
	(343.55,109.11) --
	(343.85,109.06) --
	(344.15,109.02) --
	(344.45,108.98) --
	(344.75,108.93) --
	(345.05,108.89) --
	(345.35,108.85) --
	(345.65,108.81) --
	(345.95,108.76) --
	(346.24,108.72) --
	(346.54,108.68) --
	(346.84,108.64) --
	(347.14,108.60) --
	(347.44,108.55) --
	(347.74,108.51) --
	(348.04,108.47) --
	(348.34,108.43) --
	(348.64,108.39) --
	(348.94,108.34) --
	(349.24,108.30) --
	(349.54,108.26) --
	(349.84,108.22) --
	(350.14,108.18) --
	(350.44,108.14) --
	(350.74,108.10) --
	(351.04,108.05) --
	(351.34,108.01) --
	(351.63,107.97) --
	(351.93,107.93) --
	(352.23,107.89) --
	(352.53,107.85) --
	(352.83,107.81) --
	(353.13,107.77) --
	(353.43,107.73) --
	(353.73,107.69) --
	(354.03,107.65) --
	(354.33,107.61) --
	(354.63,107.57) --
	(354.93,107.53) --
	(355.23,107.49) --
	(355.53,107.45) --
	(355.83,107.41) --
	(356.13,107.37) --
	(356.43,107.33) --
	(356.73,107.29) --
	(357.02,107.25) --
	(357.32,107.21) --
	(357.62,107.17) --
	(357.92,107.13) --
	(358.22,107.09) --
	(358.52,107.05) --
	(358.82,107.02) --
	(359.12,106.98) --
	(359.42,106.94) --
	(359.72,106.90) --
	(360.02,106.86) --
	(360.32,106.82);
\end{scope}
\begin{scope}
\path[clip] (  0.00,  0.00) rectangle (397.48,289.08);
\definecolor{drawColor}{RGB}{0,0,0}

\path[draw=drawColor,line width= 0.4pt,line join=round,line cap=round] ( 58.14, 61.20) -- (360.32, 61.20);

\path[draw=drawColor,line width= 0.4pt,line join=round,line cap=round] ( 58.14, 61.20) -- ( 58.14, 55.20);

\path[draw=drawColor,line width= 0.4pt,line join=round,line cap=round] (118.58, 61.20) -- (118.58, 55.20);

\path[draw=drawColor,line width= 0.4pt,line join=round,line cap=round] (179.01, 61.20) -- (179.01, 55.20);

\path[draw=drawColor,line width= 0.4pt,line join=round,line cap=round] (239.45, 61.20) -- (239.45, 55.20);

\path[draw=drawColor,line width= 0.4pt,line join=round,line cap=round] (299.88, 61.20) -- (299.88, 55.20);

\path[draw=drawColor,line width= 0.4pt,line join=round,line cap=round] (360.32, 61.20) -- (360.32, 55.20);

\node[text=drawColor,anchor=base,inner sep=0pt, outer sep=0pt, scale=  1.00] at ( 58.14, 39.60) {0};

\node[text=drawColor,anchor=base,inner sep=0pt, outer sep=0pt, scale=  1.00] at (118.58, 39.60) {2};

\node[text=drawColor,anchor=base,inner sep=0pt, outer sep=0pt, scale=  1.00] at (179.01, 39.60) {4};

\node[text=drawColor,anchor=base,inner sep=0pt, outer sep=0pt, scale=  1.00] at (239.45, 39.60) {6};

\node[text=drawColor,anchor=base,inner sep=0pt, outer sep=0pt, scale=  1.00] at (299.88, 39.60) {8};

\node[text=drawColor,anchor=base,inner sep=0pt, outer sep=0pt, scale=  1.00] at (360.32, 39.60) {10};

\path[draw=drawColor,line width= 0.4pt,line join=round,line cap=round] ( 49.20, 67.82) -- ( 49.20,232.79);

\path[draw=drawColor,line width= 0.4pt,line join=round,line cap=round] ( 49.20, 67.82) -- ( 43.20, 67.82);

\path[draw=drawColor,line width= 0.4pt,line join=round,line cap=round] ( 49.20, 91.39) -- ( 43.20, 91.39);

\path[draw=drawColor,line width= 0.4pt,line join=round,line cap=round] ( 49.20,114.95) -- ( 43.20,114.95);

\path[draw=drawColor,line width= 0.4pt,line join=round,line cap=round] ( 49.20,138.52) -- ( 43.20,138.52);

\path[draw=drawColor,line width= 0.4pt,line join=round,line cap=round] ( 49.20,162.09) -- ( 43.20,162.09);

\path[draw=drawColor,line width= 0.4pt,line join=round,line cap=round] ( 49.20,185.66) -- ( 43.20,185.66);

\path[draw=drawColor,line width= 0.4pt,line join=round,line cap=round] ( 49.20,209.22) -- ( 43.20,209.22);

\path[draw=drawColor,line width= 0.4pt,line join=round,line cap=round] ( 49.20,232.79) -- ( 43.20,232.79);

\node[text=drawColor,rotate= 90.00,anchor=base,inner sep=0pt, outer sep=0pt, scale=  1.00] at ( 34.80, 67.82) {0};

\node[text=drawColor,rotate= 90.00,anchor=base,inner sep=0pt, outer sep=0pt, scale=  1.00] at ( 34.80, 91.39) {2};

\node[text=drawColor,rotate= 90.00,anchor=base,inner sep=0pt, outer sep=0pt, scale=  1.00] at ( 34.80,114.95) {4};

\node[text=drawColor,rotate= 90.00,anchor=base,inner sep=0pt, outer sep=0pt, scale=  1.00] at ( 34.80,138.52) {6};

\node[text=drawColor,rotate= 90.00,anchor=base,inner sep=0pt, outer sep=0pt, scale=  1.00] at ( 34.80,162.09) {8};

\node[text=drawColor,rotate= 90.00,anchor=base,inner sep=0pt, outer sep=0pt, scale=  1.00] at ( 34.80,185.66) {10};

\node[text=drawColor,rotate= 90.00,anchor=base,inner sep=0pt, outer sep=0pt, scale=  1.00] at ( 34.80,209.22) {12};

\node[text=drawColor,rotate= 90.00,anchor=base,inner sep=0pt, outer sep=0pt, scale=  1.00] at ( 34.80,232.79) {14};

\path[draw=drawColor,line width= 0.4pt,line join=round,line cap=round] ( 49.20, 61.20) --
	(372.28, 61.20) --
	(372.28,239.88) --
	( 49.20,239.88) --
	( 49.20, 61.20);
\end{scope}
\begin{scope}
\path[clip] (  0.00,  0.00) rectangle (397.48,289.08);
\definecolor{drawColor}{RGB}{0,0,0}

\node[text=drawColor,anchor=base,inner sep=0pt, outer sep=0pt, scale=  1.20] at (210.74,260.34) {\bfseries $\tilde{i}_P < i_R$};

\node[text=drawColor,anchor=base,inner sep=0pt, outer sep=0pt, scale=  1.00] at (210.74, 27.60) {$t_\text{max}$};

\node[text=drawColor,rotate= 90.00,anchor=base,inner sep=0pt, outer sep=0pt, scale=  1.00] at ( 22.80,150.54) {MIPS};
\end{scope}
\begin{scope}
\path[clip] ( 49.20, 61.20) rectangle (372.28,239.88);
\definecolor{drawColor}{RGB}{0,0,0}

\path[draw=drawColor,line width= 0.4pt,dash pattern=on 4pt off 4pt ,line join=round,line cap=round] ( 68.04,289.08) --
	( 68.05,288.85) --
	( 68.35,283.58) --
	( 68.65,278.61) --
	( 68.95,273.91) --
	( 69.25,269.47) --
	( 69.55,265.26) --
	( 69.85,261.27) --
	( 70.15,257.48) --
	( 70.45,253.87) --
	( 70.75,250.43) --
	( 71.05,247.15) --
	( 71.35,244.02) --
	( 71.65,241.03) --
	( 71.95,238.17) --
	( 72.25,235.43) --
	( 72.55,232.81) --
	( 72.84,230.29) --
	( 73.14,227.87) --
	( 73.44,225.55) --
	( 73.74,223.32) --
	( 74.04,221.17) --
	( 74.34,219.09) --
	( 74.64,217.10) --
	( 74.94,215.18) --
	( 75.24,213.32) --
	( 75.54,211.53) --
	( 75.84,209.79) --
	( 76.14,208.12) --
	( 76.44,206.50) --
	( 76.74,204.93) --
	( 77.04,203.42) --
	( 77.34,201.94) --
	( 77.64,200.52) --
	( 77.94,199.14) --
	( 78.23,197.80) --
	( 78.53,196.50) --
	( 78.83,195.23) --
	( 79.13,194.00) --
	( 79.43,192.81) --
	( 79.73,191.65) --
	( 80.03,190.52) --
	( 80.33,189.42) --
	( 80.63,188.36) --
	( 80.93,187.32) --
	( 81.23,186.30) --
	( 81.53,185.31) --
	( 81.83,184.35) --
	( 82.13,183.41) --
	( 82.43,182.50) --
	( 82.73,181.60) --
	( 83.03,180.73) --
	( 83.33,179.88) --
	( 83.63,179.05) --
	( 83.92,178.24) --
	( 84.22,177.45) --
	( 84.52,176.67) --
	( 84.82,175.91) --
	( 85.12,175.17) --
	( 85.42,174.45) --
	( 85.72,173.74) --
	( 86.02,173.05) --
	( 86.32,172.37) --
	( 86.62,171.70) --
	( 86.92,171.05) --
	( 87.22,170.41) --
	( 87.52,169.79) --
	( 87.82,169.18) --
	( 88.12,168.58) --
	( 88.42,167.99) --
	( 88.72,167.42) --
	( 89.02,166.85) --
	( 89.31,166.30) --
	( 89.61,165.75) --
	( 89.91,165.22) --
	( 90.21,164.70) --
	( 90.51,164.18) --
	( 90.81,163.68) --
	( 91.11,163.18) --
	( 91.41,162.70) --
	( 91.71,162.22) --
	( 92.01,161.75) --
	( 92.31,161.29) --
	( 92.61,160.84) --
	( 92.91,160.39) --
	( 93.21,159.96) --
	( 93.51,159.53) --
	( 93.81,159.10) --
	( 94.11,158.69) --
	( 94.41,158.28) --
	( 94.70,157.88) --
	( 95.00,157.48) --
	( 95.30,157.09) --
	( 95.60,156.71) --
	( 95.90,156.33) --
	( 96.20,155.96) --
	( 96.50,155.59) --
	( 96.80,155.24) --
	( 97.10,154.88) --
	( 97.40,154.53) --
	( 97.70,154.19) --
	( 98.00,153.85) --
	( 98.30,153.52) --
	( 98.60,153.19) --
	( 98.90,152.87) --
	( 99.20,152.55) --
	( 99.50,152.23) --
	( 99.80,151.92) --
	(100.09,151.62) --
	(100.39,151.32) --
	(100.69,151.02) --
	(100.99,150.73) --
	(101.29,150.44) --
	(101.59,150.16) --
	(101.89,149.88) --
	(102.19,149.60) --
	(102.49,149.33) --
	(102.79,149.06) --
	(103.09,148.79) --
	(103.39,148.53) --
	(103.69,148.27) --
	(103.99,148.02) --
	(104.29,147.76) --
	(104.59,147.51) --
	(104.89,147.27) --
	(105.19,147.03) --
	(105.49,146.79) --
	(105.78,146.55) --
	(106.08,146.32) --
	(106.38,146.09) --
	(106.68,145.86) --
	(106.98,145.63) --
	(107.28,145.41) --
	(107.58,145.19) --
	(107.88,144.98) --
	(108.18,144.76) --
	(108.48,144.55) --
	(108.78,144.34) --
	(109.08,144.13) --
	(109.38,143.93) --
	(109.68,143.73) --
	(109.98,143.53) --
	(110.28,143.33) --
	(110.58,143.14) --
	(110.88,142.94) --
	(111.17,142.75) --
	(111.47,142.56) --
	(111.77,142.38) --
	(112.07,142.19) --
	(112.37,142.01) --
	(112.67,141.83) --
	(112.97,141.65) --
	(113.27,141.48) --
	(113.57,141.30) --
	(113.87,141.13) --
	(114.17,140.96) --
	(114.47,140.79) --
	(114.77,140.62) --
	(115.07,140.46) --
	(115.37,140.29) --
	(115.67,140.13) --
	(115.97,139.97) --
	(116.27,139.81) --
	(116.56,139.65) --
	(116.86,139.50) --
	(117.16,139.34) --
	(117.46,139.19) --
	(117.76,139.04) --
	(118.06,138.89) --
	(118.36,138.74) --
	(118.66,138.60) --
	(118.96,138.45) --
	(119.26,138.31) --
	(119.56,138.17) --
	(119.86,138.03) --
	(120.16,137.89) --
	(120.46,137.75) --
	(120.76,137.61) --
	(121.06,137.48) --
	(121.36,137.34) --
	(121.66,137.21) --
	(121.95,137.08) --
	(122.25,136.95) --
	(122.55,136.82) --
	(122.85,136.69) --
	(123.15,136.57) --
	(123.45,136.44) --
	(123.75,136.32) --
	(124.05,136.19) --
	(124.35,136.07) --
	(124.65,135.95) --
	(124.95,135.83) --
	(125.25,135.71) --
	(125.55,135.59) --
	(125.85,135.48) --
	(126.15,135.36) --
	(126.45,135.25) --
	(126.75,135.13) --
	(127.05,135.02) --
	(127.35,134.91) --
	(127.64,134.80) --
	(127.94,134.69) --
	(128.24,134.58) --
	(128.54,134.47) --
	(128.84,134.36) --
	(129.14,134.26) --
	(129.44,134.15) --
	(129.74,134.05) --
	(130.04,133.94) --
	(130.34,133.84) --
	(130.64,133.74) --
	(130.94,133.64) --
	(131.24,133.54) --
	(131.54,133.44) --
	(131.84,133.34) --
	(132.14,133.24) --
	(132.44,133.14) --
	(132.74,133.05) --
	(133.03,132.95) --
	(133.33,132.86) --
	(133.63,132.76) --
	(133.93,132.67) --
	(134.23,132.58) --
	(134.53,132.49) --
	(134.83,132.40) --
	(135.13,132.30) --
	(135.43,132.22) --
	(135.73,132.13) --
	(136.03,132.04) --
	(136.33,131.95) --
	(136.63,131.86) --
	(136.93,131.78) --
	(137.23,131.69) --
	(137.53,131.61) --
	(137.83,131.52) --
	(138.13,131.44) --
	(138.42,131.36) --
	(138.72,131.27) --
	(139.02,131.19) --
	(139.32,131.11) --
	(139.62,131.03) --
	(139.92,130.95) --
	(140.22,130.87) --
	(140.52,130.79) --
	(140.82,130.71) --
	(141.12,130.64) --
	(141.42,130.56) --
	(141.72,130.48) --
	(142.02,130.41) --
	(142.32,130.33) --
	(142.62,130.26) --
	(142.92,130.18) --
	(143.22,130.11) --
	(143.52,130.03) --
	(143.81,129.96) --
	(144.11,129.89) --
	(144.41,129.82) --
	(144.71,129.74) --
	(145.01,129.67) --
	(145.31,129.60) --
	(145.61,129.53) --
	(145.91,129.46) --
	(146.21,129.40) --
	(146.51,129.33) --
	(146.81,129.26) --
	(147.11,129.19) --
	(147.41,129.12) --
	(147.71,129.06) --
	(148.01,128.99) --
	(148.31,128.93) --
	(148.61,128.86) --
	(148.91,128.79) --
	(149.21,128.73) --
	(149.50,128.67) --
	(149.80,128.60) --
	(150.10,128.54) --
	(150.40,128.48) --
	(150.70,128.41) --
	(151.00,128.35) --
	(151.30,128.29) --
	(151.60,128.23) --
	(151.90,128.17) --
	(152.20,128.11) --
	(152.50,128.05) --
	(152.80,127.99) --
	(153.10,127.93) --
	(153.40,127.87) --
	(153.70,127.81) --
	(154.00,127.75) --
	(154.30,127.70) --
	(154.60,127.64) --
	(154.89,127.58) --
	(155.19,127.52) --
	(155.49,127.47) --
	(155.79,127.41) --
	(156.09,127.36) --
	(156.39,127.30) --
	(156.69,127.25) --
	(156.99,127.19) --
	(157.29,127.14) --
	(157.59,127.08) --
	(157.89,127.03) --
	(158.19,126.97) --
	(158.49,126.92) --
	(158.79,126.87) --
	(159.09,126.82) --
	(159.39,126.76) --
	(159.69,126.71) --
	(159.99,126.66) --
	(160.28,126.61) --
	(160.58,126.56) --
	(160.88,126.51) --
	(161.18,126.46) --
	(161.48,126.41) --
	(161.78,126.36) --
	(162.08,126.31) --
	(162.38,126.26) --
	(162.68,126.21) --
	(162.98,126.16) --
	(163.28,126.11) --
	(163.58,126.07) --
	(163.88,126.02) --
	(164.18,125.97) --
	(164.48,125.92) --
	(164.78,125.88) --
	(165.08,125.83) --
	(165.38,125.78) --
	(165.67,125.74) --
	(165.97,125.69) --
	(166.27,125.64) --
	(166.57,125.60) --
	(166.87,125.55) --
	(167.17,125.51) --
	(167.47,125.46) --
	(167.77,125.42) --
	(168.07,125.38) --
	(168.37,125.33) --
	(168.67,125.29) --
	(168.97,125.24) --
	(169.27,125.20) --
	(169.57,125.16) --
	(169.87,125.11) --
	(170.17,125.07) --
	(170.47,125.03) --
	(170.77,124.99) --
	(171.07,124.95) --
	(171.36,124.90) --
	(171.66,124.86) --
	(171.96,124.82) --
	(172.26,124.78) --
	(172.56,124.74) --
	(172.86,124.70) --
	(173.16,124.66) --
	(173.46,124.62) --
	(173.76,124.58) --
	(174.06,124.54) --
	(174.36,124.50) --
	(174.66,124.46) --
	(174.96,124.42) --
	(175.26,124.38) --
	(175.56,124.34) --
	(175.86,124.30) --
	(176.16,124.27) --
	(176.46,124.23) --
	(176.75,124.19) --
	(177.05,124.15) --
	(177.35,124.11) --
	(177.65,124.08) --
	(177.95,124.04) --
	(178.25,124.00) --
	(178.55,123.97) --
	(178.85,123.93) --
	(179.15,123.89) --
	(179.45,123.86) --
	(179.75,123.82) --
	(180.05,123.78) --
	(180.35,123.75) --
	(180.65,123.71) --
	(180.95,123.68) --
	(181.25,123.64) --
	(181.55,123.61) --
	(181.85,123.57) --
	(182.14,123.54) --
	(182.44,123.50) --
	(182.74,123.47) --
	(183.04,123.43) --
	(183.34,123.40) --
	(183.64,123.37) --
	(183.94,123.33) --
	(184.24,123.30) --
	(184.54,123.26) --
	(184.84,123.23) --
	(185.14,123.20) --
	(185.44,123.17) --
	(185.74,123.13) --
	(186.04,123.10) --
	(186.34,123.07) --
	(186.64,123.03) --
	(186.94,123.00) --
	(187.24,122.97) --
	(187.53,122.94) --
	(187.83,122.91) --
	(188.13,122.88) --
	(188.43,122.84) --
	(188.73,122.81) --
	(189.03,122.78) --
	(189.33,122.75) --
	(189.63,122.72) --
	(189.93,122.69) --
	(190.23,122.66) --
	(190.53,122.63) --
	(190.83,122.60) --
	(191.13,122.57) --
	(191.43,122.54) --
	(191.73,122.51) --
	(192.03,122.48) --
	(192.33,122.45) --
	(192.63,122.42) --
	(192.93,122.39) --
	(193.22,122.36) --
	(193.52,122.33) --
	(193.82,122.30) --
	(194.12,122.27) --
	(194.42,122.24) --
	(194.72,122.21) --
	(195.02,122.19) --
	(195.32,122.16) --
	(195.62,122.13) --
	(195.92,122.10) --
	(196.22,122.07) --
	(196.52,122.05) --
	(196.82,122.02) --
	(197.12,121.99) --
	(197.42,121.96) --
	(197.72,121.93) --
	(198.02,121.91) --
	(198.32,121.88) --
	(198.61,121.85) --
	(198.91,121.83) --
	(199.21,121.80) --
	(199.51,121.77) --
	(199.81,121.75) --
	(200.11,121.72) --
	(200.41,121.69) --
	(200.71,121.67) --
	(201.01,121.64) --
	(201.31,121.61) --
	(201.61,121.59) --
	(201.91,121.56) --
	(202.21,121.54) --
	(202.51,121.51) --
	(202.81,121.49) --
	(203.11,121.46) --
	(203.41,121.44) --
	(203.71,121.41) --
	(204.00,121.38) --
	(204.30,121.36) --
	(204.60,121.34) --
	(204.90,121.31) --
	(205.20,121.29) --
	(205.50,121.26) --
	(205.80,121.24) --
	(206.10,121.21) --
	(206.40,121.19) --
	(206.70,121.16) --
	(207.00,121.14) --
	(207.30,121.12) --
	(207.60,121.09) --
	(207.90,121.07) --
	(208.20,121.04) --
	(208.50,121.02) --
	(208.80,121.00) --
	(209.10,120.97) --
	(209.39,120.95) --
	(209.69,120.93) --
	(209.99,120.90) --
	(210.29,120.88) --
	(210.59,120.86) --
	(210.89,120.83) --
	(211.19,120.81) --
	(211.49,120.79) --
	(211.79,120.77) --
	(212.09,120.74) --
	(212.39,120.72) --
	(212.69,120.70) --
	(212.99,120.68) --
	(213.29,120.65) --
	(213.59,120.63) --
	(213.89,120.61) --
	(214.19,120.59) --
	(214.49,120.57) --
	(214.79,120.54) --
	(215.08,120.52) --
	(215.38,120.50) --
	(215.68,120.48) --
	(215.98,120.46) --
	(216.28,120.44) --
	(216.58,120.42) --
	(216.88,120.39) --
	(217.18,120.37) --
	(217.48,120.35) --
	(217.78,120.33) --
	(218.08,120.31) --
	(218.38,120.29) --
	(218.68,120.27) --
	(218.98,120.25) --
	(219.28,120.23) --
	(219.58,120.21) --
	(219.88,120.19) --
	(220.18,120.17) --
	(220.47,120.15) --
	(220.77,120.13) --
	(221.07,120.11) --
	(221.37,120.09) --
	(221.67,120.07) --
	(221.97,120.05) --
	(222.27,120.03) --
	(222.57,120.01) --
	(222.87,119.99) --
	(223.17,119.97) --
	(223.47,119.95) --
	(223.77,119.93) --
	(224.07,119.91) --
	(224.37,119.89) --
	(224.67,119.87) --
	(224.97,119.85) --
	(225.27,119.83) --
	(225.57,119.81) --
	(225.86,119.79) --
	(226.16,119.78) --
	(226.46,119.76) --
	(226.76,119.74) --
	(227.06,119.72) --
	(227.36,119.70) --
	(227.66,119.68) --
	(227.96,119.66) --
	(228.26,119.64) --
	(228.56,119.63) --
	(228.86,119.61) --
	(229.16,119.59) --
	(229.46,119.57) --
	(229.76,119.55) --
	(230.06,119.54) --
	(230.36,119.52) --
	(230.66,119.50) --
	(230.96,119.48) --
	(231.25,119.46) --
	(231.55,119.45) --
	(231.85,119.43) --
	(232.15,119.41) --
	(232.45,119.39) --
	(232.75,119.38) --
	(233.05,119.36) --
	(233.35,119.34) --
	(233.65,119.32) --
	(233.95,119.31) --
	(234.25,119.29) --
	(234.55,119.27) --
	(234.85,119.25) --
	(235.15,119.24) --
	(235.45,119.22) --
	(235.75,119.20) --
	(236.05,119.19) --
	(236.35,119.17) --
	(236.65,119.15) --
	(236.94,119.14) --
	(237.24,119.12) --
	(237.54,119.10) --
	(237.84,119.09) --
	(238.14,119.07) --
	(238.44,119.05) --
	(238.74,119.04) --
	(239.04,119.02) --
	(239.34,119.00) --
	(239.64,118.99) --
	(239.94,118.97) --
	(240.24,118.96) --
	(240.54,118.94) --
	(240.84,118.92) --
	(241.14,118.91) --
	(241.44,118.89) --
	(241.74,118.88) --
	(242.04,118.86) --
	(242.33,118.84) --
	(242.63,118.83) --
	(242.93,118.81) --
	(243.23,118.80) --
	(243.53,118.78) --
	(243.83,118.77) --
	(244.13,118.75) --
	(244.43,118.74) --
	(244.73,118.72) --
	(245.03,118.71) --
	(245.33,118.69) --
	(245.63,118.68) --
	(245.93,118.66) --
	(246.23,118.64) --
	(246.53,118.63) --
	(246.83,118.61) --
	(247.13,118.60) --
	(247.43,118.58) --
	(247.72,118.57) --
	(248.02,118.56) --
	(248.32,118.54) --
	(248.62,118.53) --
	(248.92,118.51) --
	(249.22,118.50) --
	(249.52,118.48) --
	(249.82,118.47) --
	(250.12,118.45) --
	(250.42,118.44) --
	(250.72,118.42) --
	(251.02,118.41) --
	(251.32,118.40) --
	(251.62,118.38) --
	(251.92,118.37) --
	(252.22,118.35) --
	(252.52,118.34) --
	(252.82,118.32) --
	(253.11,118.31) --
	(253.41,118.30) --
	(253.71,118.28) --
	(254.01,118.27) --
	(254.31,118.25) --
	(254.61,118.24) --
	(254.91,118.23) --
	(255.21,118.21) --
	(255.51,118.20) --
	(255.81,118.19) --
	(256.11,118.17) --
	(256.41,118.16) --
	(256.71,118.15) --
	(257.01,118.13) --
	(257.31,118.12) --
	(257.61,118.10) --
	(257.91,118.09) --
	(258.21,118.08) --
	(258.51,118.06) --
	(258.80,118.05) --
	(259.10,118.04) --
	(259.40,118.03) --
	(259.70,118.01) --
	(260.00,118.00) --
	(260.30,117.99) --
	(260.60,117.97) --
	(260.90,117.96) --
	(261.20,117.95) --
	(261.50,117.93) --
	(261.80,117.92) --
	(262.10,117.91) --
	(262.40,117.90) --
	(262.70,117.88) --
	(263.00,117.87) --
	(263.30,117.86) --
	(263.60,117.84) --
	(263.90,117.83) --
	(264.19,117.82) --
	(264.49,117.81) --
	(264.79,117.79) --
	(265.09,117.78) --
	(265.39,117.77) --
	(265.69,117.76) --
	(265.99,117.74) --
	(266.29,117.73) --
	(266.59,117.72) --
	(266.89,117.71) --
	(267.19,117.70) --
	(267.49,117.68) --
	(267.79,117.67) --
	(268.09,117.66) --
	(268.39,117.65) --
	(268.69,117.64) --
	(268.99,117.62) --
	(269.29,117.61) --
	(269.58,117.60) --
	(269.88,117.59) --
	(270.18,117.58) --
	(270.48,117.56) --
	(270.78,117.55) --
	(271.08,117.54) --
	(271.38,117.53) --
	(271.68,117.52) --
	(271.98,117.50) --
	(272.28,117.49) --
	(272.58,117.48) --
	(272.88,117.47) --
	(273.18,117.46) --
	(273.48,117.45) --
	(273.78,117.44) --
	(274.08,117.42) --
	(274.38,117.41) --
	(274.68,117.40) --
	(274.98,117.39) --
	(275.27,117.38) --
	(275.57,117.37) --
	(275.87,117.36) --
	(276.17,117.34) --
	(276.47,117.33) --
	(276.77,117.32) --
	(277.07,117.31) --
	(277.37,117.30) --
	(277.67,117.29) --
	(277.97,117.28) --
	(278.27,117.27) --
	(278.57,117.26) --
	(278.87,117.25) --
	(279.17,117.23) --
	(279.47,117.22) --
	(279.77,117.21) --
	(280.07,117.20) --
	(280.37,117.19) --
	(280.66,117.18) --
	(280.96,117.17) --
	(281.26,117.16) --
	(281.56,117.15) --
	(281.86,117.14) --
	(282.16,117.13) --
	(282.46,117.12) --
	(282.76,117.11) --
	(283.06,117.09) --
	(283.36,117.08) --
	(283.66,117.07) --
	(283.96,117.06) --
	(284.26,117.05) --
	(284.56,117.04) --
	(284.86,117.03) --
	(285.16,117.02) --
	(285.46,117.01) --
	(285.76,117.00) --
	(286.05,116.99) --
	(286.35,116.98) --
	(286.65,116.97) --
	(286.95,116.96) --
	(287.25,116.95) --
	(287.55,116.94) --
	(287.85,116.93) --
	(288.15,116.92) --
	(288.45,116.91) --
	(288.75,116.90) --
	(289.05,116.89) --
	(289.35,116.88) --
	(289.65,116.87) --
	(289.95,116.86) --
	(290.25,116.85) --
	(290.55,116.84) --
	(290.85,116.83) --
	(291.15,116.82) --
	(291.44,116.81) --
	(291.74,116.80) --
	(292.04,116.79) --
	(292.34,116.78) --
	(292.64,116.77) --
	(292.94,116.76) --
	(293.24,116.75) --
	(293.54,116.74) --
	(293.84,116.73) --
	(294.14,116.72) --
	(294.44,116.71) --
	(294.74,116.70) --
	(295.04,116.69) --
	(295.34,116.68) --
	(295.64,116.68) --
	(295.94,116.67) --
	(296.24,116.66) --
	(296.54,116.65) --
	(296.84,116.64) --
	(297.13,116.63) --
	(297.43,116.62) --
	(297.73,116.61) --
	(298.03,116.60) --
	(298.33,116.59) --
	(298.63,116.58) --
	(298.93,116.57) --
	(299.23,116.56) --
	(299.53,116.55) --
	(299.83,116.55) --
	(300.13,116.54) --
	(300.43,116.53) --
	(300.73,116.52) --
	(301.03,116.51) --
	(301.33,116.50) --
	(301.63,116.49) --
	(301.93,116.48) --
	(302.23,116.47) --
	(302.52,116.46) --
	(302.82,116.46) --
	(303.12,116.45) --
	(303.42,116.44) --
	(303.72,116.43) --
	(304.02,116.42) --
	(304.32,116.41) --
	(304.62,116.40) --
	(304.92,116.39) --
	(305.22,116.38) --
	(305.52,116.38) --
	(305.82,116.37) --
	(306.12,116.36) --
	(306.42,116.35) --
	(306.72,116.34) --
	(307.02,116.33) --
	(307.32,116.32) --
	(307.62,116.32) --
	(307.91,116.31) --
	(308.21,116.30) --
	(308.51,116.29) --
	(308.81,116.28) --
	(309.11,116.27) --
	(309.41,116.26) --
	(309.71,116.26) --
	(310.01,116.25) --
	(310.31,116.24) --
	(310.61,116.23) --
	(310.91,116.22) --
	(311.21,116.21) --
	(311.51,116.21) --
	(311.81,116.20) --
	(312.11,116.19) --
	(312.41,116.18) --
	(312.71,116.17) --
	(313.01,116.16) --
	(313.30,116.16) --
	(313.60,116.15) --
	(313.90,116.14) --
	(314.20,116.13) --
	(314.50,116.12) --
	(314.80,116.12) --
	(315.10,116.11) --
	(315.40,116.10) --
	(315.70,116.09) --
	(316.00,116.08) --
	(316.30,116.08) --
	(316.60,116.07) --
	(316.90,116.06) --
	(317.20,116.05) --
	(317.50,116.04) --
	(317.80,116.04) --
	(318.10,116.03) --
	(318.40,116.02) --
	(318.70,116.01) --
	(318.99,116.00) --
	(319.29,116.00) --
	(319.59,115.99) --
	(319.89,115.98) --
	(320.19,115.97) --
	(320.49,115.97) --
	(320.79,115.96) --
	(321.09,115.95) --
	(321.39,115.94) --
	(321.69,115.93) --
	(321.99,115.93) --
	(322.29,115.92) --
	(322.59,115.91) --
	(322.89,115.90) --
	(323.19,115.90) --
	(323.49,115.89) --
	(323.79,115.88) --
	(324.09,115.87) --
	(324.38,115.87) --
	(324.68,115.86) --
	(324.98,115.85) --
	(325.28,115.84) --
	(325.58,115.84) --
	(325.88,115.83) --
	(326.18,115.82) --
	(326.48,115.81) --
	(326.78,115.81) --
	(327.08,115.80) --
	(327.38,115.79) --
	(327.68,115.78) --
	(327.98,115.78) --
	(328.28,115.77) --
	(328.58,115.76) --
	(328.88,115.76) --
	(329.18,115.75) --
	(329.48,115.74) --
	(329.77,115.73) --
	(330.07,115.73) --
	(330.37,115.72) --
	(330.67,115.71) --
	(330.97,115.70) --
	(331.27,115.70) --
	(331.57,115.69) --
	(331.87,115.68) --
	(332.17,115.68) --
	(332.47,115.67) --
	(332.77,115.66) --
	(333.07,115.65) --
	(333.37,115.65) --
	(333.67,115.64) --
	(333.97,115.63) --
	(334.27,115.63) --
	(334.57,115.62) --
	(334.87,115.61) --
	(335.16,115.61) --
	(335.46,115.60) --
	(335.76,115.59) --
	(336.06,115.59) --
	(336.36,115.58) --
	(336.66,115.57) --
	(336.96,115.56) --
	(337.26,115.56) --
	(337.56,115.55) --
	(337.86,115.54) --
	(338.16,115.54) --
	(338.46,115.53) --
	(338.76,115.52) --
	(339.06,115.52) --
	(339.36,115.51) --
	(339.66,115.50) --
	(339.96,115.50) --
	(340.26,115.49) --
	(340.56,115.48) --
	(340.85,115.48) --
	(341.15,115.47) --
	(341.45,115.46) --
	(341.75,115.46) --
	(342.05,115.45) --
	(342.35,115.44) --
	(342.65,115.44) --
	(342.95,115.43) --
	(343.25,115.42) --
	(343.55,115.42) --
	(343.85,115.41) --
	(344.15,115.40) --
	(344.45,115.40) --
	(344.75,115.39) --
	(345.05,115.38) --
	(345.35,115.38) --
	(345.65,115.37) --
	(345.95,115.37) --
	(346.24,115.36) --
	(346.54,115.35) --
	(346.84,115.35) --
	(347.14,115.34) --
	(347.44,115.33) --
	(347.74,115.33) --
	(348.04,115.32) --
	(348.34,115.31) --
	(348.64,115.31) --
	(348.94,115.30) --
	(349.24,115.30) --
	(349.54,115.29) --
	(349.84,115.28) --
	(350.14,115.28) --
	(350.44,115.27) --
	(350.74,115.26) --
	(351.04,115.26) --
	(351.34,115.25) --
	(351.63,115.25) --
	(351.93,115.24) --
	(352.23,115.23) --
	(352.53,115.23) --
	(352.83,115.22) --
	(353.13,115.21) --
	(353.43,115.21) --
	(353.73,115.20) --
	(354.03,115.20) --
	(354.33,115.19) --
	(354.63,115.18) --
	(354.93,115.18) --
	(355.23,115.17) --
	(355.53,115.17) --
	(355.83,115.16) --
	(356.13,115.15) --
	(356.43,115.15) --
	(356.73,115.14) --
	(357.02,115.14) --
	(357.32,115.13) --
	(357.62,115.12) --
	(357.92,115.12) --
	(358.22,115.11) --
	(358.52,115.11) --
	(358.82,115.10) --
	(359.12,115.09) --
	(359.42,115.09) --
	(359.72,115.08) --
	(360.02,115.08) --
	(360.32,115.07);

\path[draw=drawColor,line width= 0.4pt,dash pattern=on 4pt off 4pt ,line join=round,line cap=round] ( 64.67,  0.00) --
	( 64.76,  1.52) --
	( 65.06,  6.19) --
	( 65.36, 10.46) --
	( 65.66, 14.40) --
	( 65.96, 18.03) --
	( 66.26, 21.39) --
	( 66.56, 24.52) --
	( 66.86, 27.43) --
	( 67.16, 30.15) --
	( 67.45, 32.69) --
	( 67.75, 35.07) --
	( 68.05, 37.31) --
	( 68.35, 39.42) --
	( 68.65, 41.41) --
	( 68.95, 43.28) --
	( 69.25, 45.06) --
	( 69.55, 46.74) --
	( 69.85, 48.34) --
	( 70.15, 49.86) --
	( 70.45, 51.30) --
	( 70.75, 52.68) --
	( 71.05, 53.99) --
	( 71.35, 55.24) --
	( 71.65, 56.44) --
	( 71.95, 57.58) --
	( 72.25, 58.68) --
	( 72.55, 59.73) --
	( 72.84, 60.73) --
	( 73.14, 61.70) --
	( 73.44, 62.63) --
	( 73.74, 63.52) --
	( 74.04, 64.38) --
	( 74.34, 65.21) --
	( 74.64, 66.01) --
	( 74.94, 66.78) --
	( 75.24, 67.52) --
	( 75.54, 68.24) --
	( 75.84, 68.93) --
	( 76.14, 69.60) --
	( 76.44, 70.25) --
	( 76.74, 70.88) --
	( 77.04, 71.48) --
	( 77.34, 72.07) --
	( 77.64, 72.64) --
	( 77.94, 73.20) --
	( 78.23, 73.73) --
	( 78.53, 74.25) --
	( 78.83, 74.76) --
	( 79.13, 75.25) --
	( 79.43, 75.73) --
	( 79.73, 76.19) --
	( 80.03, 76.64) --
	( 80.33, 77.08) --
	( 80.63, 77.51) --
	( 80.93, 77.92) --
	( 81.23, 78.33) --
	( 81.53, 78.72) --
	( 81.83, 79.11) --
	( 82.13, 79.49) --
	( 82.43, 79.85) --
	( 82.73, 80.21) --
	( 83.03, 80.56) --
	( 83.33, 80.90) --
	( 83.63, 81.23) --
	( 83.92, 81.55) --
	( 84.22, 81.87) --
	( 84.52, 82.18) --
	( 84.82, 82.48) --
	( 85.12, 82.78) --
	( 85.42, 83.07) --
	( 85.72, 83.35) --
	( 86.02, 83.63) --
	( 86.32, 83.90) --
	( 86.62, 84.17) --
	( 86.92, 84.43) --
	( 87.22, 84.68) --
	( 87.52, 84.93) --
	( 87.82, 85.18) --
	( 88.12, 85.42) --
	( 88.42, 85.65) --
	( 88.72, 85.88) --
	( 89.02, 86.11) --
	( 89.31, 86.33) --
	( 89.61, 86.55) --
	( 89.91, 86.76) --
	( 90.21, 86.97) --
	( 90.51, 87.18) --
	( 90.81, 87.38) --
	( 91.11, 87.58) --
	( 91.41, 87.77) --
	( 91.71, 87.96) --
	( 92.01, 88.15) --
	( 92.31, 88.33) --
	( 92.61, 88.52) --
	( 92.91, 88.69) --
	( 93.21, 88.87) --
	( 93.51, 89.04) --
	( 93.81, 89.21) --
	( 94.11, 89.38) --
	( 94.41, 89.54) --
	( 94.70, 89.70) --
	( 95.00, 89.86) --
	( 95.30, 90.01) --
	( 95.60, 90.17) --
	( 95.90, 90.32) --
	( 96.20, 90.47) --
	( 96.50, 90.61) --
	( 96.80, 90.76) --
	( 97.10, 90.90) --
	( 97.40, 91.04) --
	( 97.70, 91.17) --
	( 98.00, 91.31) --
	( 98.30, 91.44) --
	( 98.60, 91.57) --
	( 98.90, 91.70) --
	( 99.20, 91.83) --
	( 99.50, 91.96) --
	( 99.80, 92.08) --
	(100.09, 92.20) --
	(100.39, 92.32) --
	(100.69, 92.44) --
	(100.99, 92.56) --
	(101.29, 92.67) --
	(101.59, 92.79) --
	(101.89, 92.90) --
	(102.19, 93.01) --
	(102.49, 93.12) --
	(102.79, 93.23) --
	(103.09, 93.33) --
	(103.39, 93.44) --
	(103.69, 93.54) --
	(103.99, 93.64) --
	(104.29, 93.75) --
	(104.59, 93.84) --
	(104.89, 93.94) --
	(105.19, 94.04) --
	(105.49, 94.14) --
	(105.78, 94.23) --
	(106.08, 94.32) --
	(106.38, 94.42) --
	(106.68, 94.51) --
	(106.98, 94.60) --
	(107.28, 94.69) --
	(107.58, 94.77) --
	(107.88, 94.86) --
	(108.18, 94.95) --
	(108.48, 95.03) --
	(108.78, 95.11) --
	(109.08, 95.20) --
	(109.38, 95.28) --
	(109.68, 95.36) --
	(109.98, 95.44) --
	(110.28, 95.52) --
	(110.58, 95.60) --
	(110.88, 95.67) --
	(111.17, 95.75) --
	(111.47, 95.83) --
	(111.77, 95.90) --
	(112.07, 95.97) --
	(112.37, 96.05) --
	(112.67, 96.12) --
	(112.97, 96.19) --
	(113.27, 96.26) --
	(113.57, 96.33) --
	(113.87, 96.40) --
	(114.17, 96.47) --
	(114.47, 96.54) --
	(114.77, 96.60) --
	(115.07, 96.67) --
	(115.37, 96.73) --
	(115.67, 96.80) --
	(115.97, 96.86) --
	(116.27, 96.93) --
	(116.56, 96.99) --
	(116.86, 97.05) --
	(117.16, 97.11) --
	(117.46, 97.17) --
	(117.76, 97.23) --
	(118.06, 97.29) --
	(118.36, 97.35) --
	(118.66, 97.41) --
	(118.96, 97.47) --
	(119.26, 97.53) --
	(119.56, 97.58) --
	(119.86, 97.64) --
	(120.16, 97.70) --
	(120.46, 97.75) --
	(120.76, 97.80) --
	(121.06, 97.86) --
	(121.36, 97.91) --
	(121.66, 97.97) --
	(121.95, 98.02) --
	(122.25, 98.07) --
	(122.55, 98.12) --
	(122.85, 98.17) --
	(123.15, 98.22) --
	(123.45, 98.27) --
	(123.75, 98.32) --
	(124.05, 98.37) --
	(124.35, 98.42) --
	(124.65, 98.47) --
	(124.95, 98.52) --
	(125.25, 98.57) --
	(125.55, 98.61) --
	(125.85, 98.66) --
	(126.15, 98.71) --
	(126.45, 98.75) --
	(126.75, 98.80) --
	(127.05, 98.84) --
	(127.35, 98.89) --
	(127.64, 98.93) --
	(127.94, 98.98) --
	(128.24, 99.02) --
	(128.54, 99.06) --
	(128.84, 99.11) --
	(129.14, 99.15) --
	(129.44, 99.19) --
	(129.74, 99.23) --
	(130.04, 99.27) --
	(130.34, 99.31) --
	(130.64, 99.36) --
	(130.94, 99.40) --
	(131.24, 99.44) --
	(131.54, 99.48) --
	(131.84, 99.52) --
	(132.14, 99.55) --
	(132.44, 99.59) --
	(132.74, 99.63) --
	(133.03, 99.67) --
	(133.33, 99.71) --
	(133.63, 99.75) --
	(133.93, 99.78) --
	(134.23, 99.82) --
	(134.53, 99.86) --
	(134.83, 99.89) --
	(135.13, 99.93) --
	(135.43, 99.96) --
	(135.73,100.00) --
	(136.03,100.04) --
	(136.33,100.07) --
	(136.63,100.11) --
	(136.93,100.14) --
	(137.23,100.17) --
	(137.53,100.21) --
	(137.83,100.24) --
	(138.13,100.27) --
	(138.42,100.31) --
	(138.72,100.34) --
	(139.02,100.37) --
	(139.32,100.41) --
	(139.62,100.44) --
	(139.92,100.47) --
	(140.22,100.50) --
	(140.52,100.53) --
	(140.82,100.57) --
	(141.12,100.60) --
	(141.42,100.63) --
	(141.72,100.66) --
	(142.02,100.69) --
	(142.32,100.72) --
	(142.62,100.75) --
	(142.92,100.78) --
	(143.22,100.81) --
	(143.52,100.84) --
	(143.81,100.87) --
	(144.11,100.90) --
	(144.41,100.92) --
	(144.71,100.95) --
	(145.01,100.98) --
	(145.31,101.01) --
	(145.61,101.04) --
	(145.91,101.06) --
	(146.21,101.09) --
	(146.51,101.12) --
	(146.81,101.15) --
	(147.11,101.17) --
	(147.41,101.20) --
	(147.71,101.23) --
	(148.01,101.25) --
	(148.31,101.28) --
	(148.61,101.31) --
	(148.91,101.33) --
	(149.21,101.36) --
	(149.50,101.38) --
	(149.80,101.41) --
	(150.10,101.43) --
	(150.40,101.46) --
	(150.70,101.48) --
	(151.00,101.51) --
	(151.30,101.53) --
	(151.60,101.56) --
	(151.90,101.58) --
	(152.20,101.61) --
	(152.50,101.63) --
	(152.80,101.66) --
	(153.10,101.68) --
	(153.40,101.70) --
	(153.70,101.73) --
	(154.00,101.75) --
	(154.30,101.77) --
	(154.60,101.80) --
	(154.89,101.82) --
	(155.19,101.84) --
	(155.49,101.86) --
	(155.79,101.89) --
	(156.09,101.91) --
	(156.39,101.93) --
	(156.69,101.95) --
	(156.99,101.97) --
	(157.29,102.00) --
	(157.59,102.02) --
	(157.89,102.04) --
	(158.19,102.06) --
	(158.49,102.08) --
	(158.79,102.10) --
	(159.09,102.12) --
	(159.39,102.14) --
	(159.69,102.17) --
	(159.99,102.19) --
	(160.28,102.21) --
	(160.58,102.23) --
	(160.88,102.25) --
	(161.18,102.27) --
	(161.48,102.29) --
	(161.78,102.31) --
	(162.08,102.33) --
	(162.38,102.35) --
	(162.68,102.37) --
	(162.98,102.39) --
	(163.28,102.41) --
	(163.58,102.42) --
	(163.88,102.44) --
	(164.18,102.46) --
	(164.48,102.48) --
	(164.78,102.50) --
	(165.08,102.52) --
	(165.38,102.54) --
	(165.67,102.56) --
	(165.97,102.57) --
	(166.27,102.59) --
	(166.57,102.61) --
	(166.87,102.63) --
	(167.17,102.65) --
	(167.47,102.66) --
	(167.77,102.68) --
	(168.07,102.70) --
	(168.37,102.72) --
	(168.67,102.74) --
	(168.97,102.75) --
	(169.27,102.77) --
	(169.57,102.79) --
	(169.87,102.80) --
	(170.17,102.82) --
	(170.47,102.84) --
	(170.77,102.86) --
	(171.07,102.87) --
	(171.36,102.89) --
	(171.66,102.91) --
	(171.96,102.92) --
	(172.26,102.94) --
	(172.56,102.95) --
	(172.86,102.97) --
	(173.16,102.99) --
	(173.46,103.00) --
	(173.76,103.02) --
	(174.06,103.04) --
	(174.36,103.05) --
	(174.66,103.07) --
	(174.96,103.08) --
	(175.26,103.10) --
	(175.56,103.11) --
	(175.86,103.13) --
	(176.16,103.14) --
	(176.46,103.16) --
	(176.75,103.17) --
	(177.05,103.19) --
	(177.35,103.20) --
	(177.65,103.22) --
	(177.95,103.23) --
	(178.25,103.25) --
	(178.55,103.26) --
	(178.85,103.28) --
	(179.15,103.29) --
	(179.45,103.31) --
	(179.75,103.32) --
	(180.05,103.34) --
	(180.35,103.35) --
	(180.65,103.37) --
	(180.95,103.38) --
	(181.25,103.39) --
	(181.55,103.41) --
	(181.85,103.42) --
	(182.14,103.44) --
	(182.44,103.45) --
	(182.74,103.46) --
	(183.04,103.48) --
	(183.34,103.49) --
	(183.64,103.50) --
	(183.94,103.52) --
	(184.24,103.53) --
	(184.54,103.54) --
	(184.84,103.56) --
	(185.14,103.57) --
	(185.44,103.58) --
	(185.74,103.60) --
	(186.04,103.61) --
	(186.34,103.62) --
	(186.64,103.64) --
	(186.94,103.65) --
	(187.24,103.66) --
	(187.53,103.67) --
	(187.83,103.69) --
	(188.13,103.70) --
	(188.43,103.71) --
	(188.73,103.73) --
	(189.03,103.74) --
	(189.33,103.75) --
	(189.63,103.76) --
	(189.93,103.78) --
	(190.23,103.79) --
	(190.53,103.80) --
	(190.83,103.81) --
	(191.13,103.82) --
	(191.43,103.84) --
	(191.73,103.85) --
	(192.03,103.86) --
	(192.33,103.87) --
	(192.63,103.88) --
	(192.93,103.90) --
	(193.22,103.91) --
	(193.52,103.92) --
	(193.82,103.93) --
	(194.12,103.94) --
	(194.42,103.95) --
	(194.72,103.96) --
	(195.02,103.98) --
	(195.32,103.99) --
	(195.62,104.00) --
	(195.92,104.01) --
	(196.22,104.02) --
	(196.52,104.03) --
	(196.82,104.04) --
	(197.12,104.05) --
	(197.42,104.07) --
	(197.72,104.08) --
	(198.02,104.09) --
	(198.32,104.10) --
	(198.61,104.11) --
	(198.91,104.12) --
	(199.21,104.13) --
	(199.51,104.14) --
	(199.81,104.15) --
	(200.11,104.16) --
	(200.41,104.17) --
	(200.71,104.18) --
	(201.01,104.19) --
	(201.31,104.20) --
	(201.61,104.21) --
	(201.91,104.23) --
	(202.21,104.24) --
	(202.51,104.25) --
	(202.81,104.26) --
	(203.11,104.27) --
	(203.41,104.28) --
	(203.71,104.29) --
	(204.00,104.30) --
	(204.30,104.31) --
	(204.60,104.32) --
	(204.90,104.33) --
	(205.20,104.34) --
	(205.50,104.35) --
	(205.80,104.36) --
	(206.10,104.37) --
	(206.40,104.38) --
	(206.70,104.39) --
	(207.00,104.39) --
	(207.30,104.40) --
	(207.60,104.41) --
	(207.90,104.42) --
	(208.20,104.43) --
	(208.50,104.44) --
	(208.80,104.45) --
	(209.10,104.46) --
	(209.39,104.47) --
	(209.69,104.48) --
	(209.99,104.49) --
	(210.29,104.50) --
	(210.59,104.51) --
	(210.89,104.52) --
	(211.19,104.53) --
	(211.49,104.53) --
	(211.79,104.54) --
	(212.09,104.55) --
	(212.39,104.56) --
	(212.69,104.57) --
	(212.99,104.58) --
	(213.29,104.59) --
	(213.59,104.60) --
	(213.89,104.61) --
	(214.19,104.62) --
	(214.49,104.62) --
	(214.79,104.63) --
	(215.08,104.64) --
	(215.38,104.65) --
	(215.68,104.66) --
	(215.98,104.67) --
	(216.28,104.68) --
	(216.58,104.68) --
	(216.88,104.69) --
	(217.18,104.70) --
	(217.48,104.71) --
	(217.78,104.72) --
	(218.08,104.73) --
	(218.38,104.73) --
	(218.68,104.74) --
	(218.98,104.75) --
	(219.28,104.76) --
	(219.58,104.77) --
	(219.88,104.78) --
	(220.18,104.78) --
	(220.47,104.79) --
	(220.77,104.80) --
	(221.07,104.81) --
	(221.37,104.82) --
	(221.67,104.82) --
	(221.97,104.83) --
	(222.27,104.84) --
	(222.57,104.85) --
	(222.87,104.86) --
	(223.17,104.86) --
	(223.47,104.87) --
	(223.77,104.88) --
	(224.07,104.89) --
	(224.37,104.89) --
	(224.67,104.90) --
	(224.97,104.91) --
	(225.27,104.92) --
	(225.57,104.93) --
	(225.86,104.93) --
	(226.16,104.94) --
	(226.46,104.95) --
	(226.76,104.96) --
	(227.06,104.96) --
	(227.36,104.97) --
	(227.66,104.98) --
	(227.96,104.99) --
	(228.26,104.99) --
	(228.56,105.00) --
	(228.86,105.01) --
	(229.16,105.01) --
	(229.46,105.02) --
	(229.76,105.03) --
	(230.06,105.04) --
	(230.36,105.04) --
	(230.66,105.05) --
	(230.96,105.06) --
	(231.25,105.07) --
	(231.55,105.07) --
	(231.85,105.08) --
	(232.15,105.09) --
	(232.45,105.09) --
	(232.75,105.10) --
	(233.05,105.11) --
	(233.35,105.11) --
	(233.65,105.12) --
	(233.95,105.13) --
	(234.25,105.13) --
	(234.55,105.14) --
	(234.85,105.15) --
	(235.15,105.16) --
	(235.45,105.16) --
	(235.75,105.17) --
	(236.05,105.18) --
	(236.35,105.18) --
	(236.65,105.19) --
	(236.94,105.20) --
	(237.24,105.20) --
	(237.54,105.21) --
	(237.84,105.22) --
	(238.14,105.22) --
	(238.44,105.23) --
	(238.74,105.24) --
	(239.04,105.24) --
	(239.34,105.25) --
	(239.64,105.26) --
	(239.94,105.26) --
	(240.24,105.27) --
	(240.54,105.27) --
	(240.84,105.28) --
	(241.14,105.29) --
	(241.44,105.29) --
	(241.74,105.30) --
	(242.04,105.31) --
	(242.33,105.31) --
	(242.63,105.32) --
	(242.93,105.33) --
	(243.23,105.33) --
	(243.53,105.34) --
	(243.83,105.34) --
	(244.13,105.35) --
	(244.43,105.36) --
	(244.73,105.36) --
	(245.03,105.37) --
	(245.33,105.37) --
	(245.63,105.38) --
	(245.93,105.39) --
	(246.23,105.39) --
	(246.53,105.40) --
	(246.83,105.40) --
	(247.13,105.41) --
	(247.43,105.42) --
	(247.72,105.42) --
	(248.02,105.43) --
	(248.32,105.43) --
	(248.62,105.44) --
	(248.92,105.45) --
	(249.22,105.45) --
	(249.52,105.46) --
	(249.82,105.46) --
	(250.12,105.47) --
	(250.42,105.48) --
	(250.72,105.48) --
	(251.02,105.49) --
	(251.32,105.49) --
	(251.62,105.50) --
	(251.92,105.50) --
	(252.22,105.51) --
	(252.52,105.52) --
	(252.82,105.52) --
	(253.11,105.53) --
	(253.41,105.53) --
	(253.71,105.54) --
	(254.01,105.54) --
	(254.31,105.55) --
	(254.61,105.55) --
	(254.91,105.56) --
	(255.21,105.57) --
	(255.51,105.57) --
	(255.81,105.58) --
	(256.11,105.58) --
	(256.41,105.59) --
	(256.71,105.59) --
	(257.01,105.60) --
	(257.31,105.60) --
	(257.61,105.61) --
	(257.91,105.61) --
	(258.21,105.62) --
	(258.51,105.62) --
	(258.80,105.63) --
	(259.10,105.64) --
	(259.40,105.64) --
	(259.70,105.65) --
	(260.00,105.65) --
	(260.30,105.66) --
	(260.60,105.66) --
	(260.90,105.67) --
	(261.20,105.67) --
	(261.50,105.68) --
	(261.80,105.68) --
	(262.10,105.69) --
	(262.40,105.69) --
	(262.70,105.70) --
	(263.00,105.70) --
	(263.30,105.71) --
	(263.60,105.71) --
	(263.90,105.72) --
	(264.19,105.72) --
	(264.49,105.73) --
	(264.79,105.73) --
	(265.09,105.74) --
	(265.39,105.74) --
	(265.69,105.75) --
	(265.99,105.75) --
	(266.29,105.76) --
	(266.59,105.76) --
	(266.89,105.77) --
	(267.19,105.77) --
	(267.49,105.78) --
	(267.79,105.78) --
	(268.09,105.79) --
	(268.39,105.79) --
	(268.69,105.80) --
	(268.99,105.80) --
	(269.29,105.81) --
	(269.58,105.81) --
	(269.88,105.82) --
	(270.18,105.82) --
	(270.48,105.83) --
	(270.78,105.83) --
	(271.08,105.83) --
	(271.38,105.84) --
	(271.68,105.84) --
	(271.98,105.85) --
	(272.28,105.85) --
	(272.58,105.86) --
	(272.88,105.86) --
	(273.18,105.87) --
	(273.48,105.87) --
	(273.78,105.88) --
	(274.08,105.88) --
	(274.38,105.89) --
	(274.68,105.89) --
	(274.98,105.89) --
	(275.27,105.90) --
	(275.57,105.90) --
	(275.87,105.91) --
	(276.17,105.91) --
	(276.47,105.92) --
	(276.77,105.92) --
	(277.07,105.93) --
	(277.37,105.93) --
	(277.67,105.93) --
	(277.97,105.94) --
	(278.27,105.94) --
	(278.57,105.95) --
	(278.87,105.95) --
	(279.17,105.96) --
	(279.47,105.96) --
	(279.77,105.97) --
	(280.07,105.97) --
	(280.37,105.97) --
	(280.66,105.98) --
	(280.96,105.98) --
	(281.26,105.99) --
	(281.56,105.99) --
	(281.86,106.00) --
	(282.16,106.00) --
	(282.46,106.00) --
	(282.76,106.01) --
	(283.06,106.01) --
	(283.36,106.02) --
	(283.66,106.02) --
	(283.96,106.03) --
	(284.26,106.03) --
	(284.56,106.03) --
	(284.86,106.04) --
	(285.16,106.04) --
	(285.46,106.05) --
	(285.76,106.05) --
	(286.05,106.05) --
	(286.35,106.06) --
	(286.65,106.06) --
	(286.95,106.07) --
	(287.25,106.07) --
	(287.55,106.07) --
	(287.85,106.08) --
	(288.15,106.08) --
	(288.45,106.09) --
	(288.75,106.09) --
	(289.05,106.09) --
	(289.35,106.10) --
	(289.65,106.10) --
	(289.95,106.11) --
	(290.25,106.11) --
	(290.55,106.11) --
	(290.85,106.12) --
	(291.15,106.12) --
	(291.44,106.13) --
	(291.74,106.13) --
	(292.04,106.13) --
	(292.34,106.14) --
	(292.64,106.14) --
	(292.94,106.15) --
	(293.24,106.15) --
	(293.54,106.15) --
	(293.84,106.16) --
	(294.14,106.16) --
	(294.44,106.17) --
	(294.74,106.17) --
	(295.04,106.17) --
	(295.34,106.18) --
	(295.64,106.18) --
	(295.94,106.18) --
	(296.24,106.19) --
	(296.54,106.19) --
	(296.84,106.20) --
	(297.13,106.20) --
	(297.43,106.20) --
	(297.73,106.21) --
	(298.03,106.21) --
	(298.33,106.21) --
	(298.63,106.22) --
	(298.93,106.22) --
	(299.23,106.22) --
	(299.53,106.23) --
	(299.83,106.23) --
	(300.13,106.24) --
	(300.43,106.24) --
	(300.73,106.24) --
	(301.03,106.25) --
	(301.33,106.25) --
	(301.63,106.25) --
	(301.93,106.26) --
	(302.23,106.26) --
	(302.52,106.26) --
	(302.82,106.27) --
	(303.12,106.27) --
	(303.42,106.28) --
	(303.72,106.28) --
	(304.02,106.28) --
	(304.32,106.29) --
	(304.62,106.29) --
	(304.92,106.29) --
	(305.22,106.30) --
	(305.52,106.30) --
	(305.82,106.30) --
	(306.12,106.31) --
	(306.42,106.31) --
	(306.72,106.31) --
	(307.02,106.32) --
	(307.32,106.32) --
	(307.62,106.32) --
	(307.91,106.33) --
	(308.21,106.33) --
	(308.51,106.33) --
	(308.81,106.34) --
	(309.11,106.34) --
	(309.41,106.34) --
	(309.71,106.35) --
	(310.01,106.35) --
	(310.31,106.35) --
	(310.61,106.36) --
	(310.91,106.36) --
	(311.21,106.36) --
	(311.51,106.37) --
	(311.81,106.37) --
	(312.11,106.37) --
	(312.41,106.38) --
	(312.71,106.38) --
	(313.01,106.38) --
	(313.30,106.39) --
	(313.60,106.39) --
	(313.90,106.39) --
	(314.20,106.40) --
	(314.50,106.40) --
	(314.80,106.40) --
	(315.10,106.41) --
	(315.40,106.41) --
	(315.70,106.41) --
	(316.00,106.42) --
	(316.30,106.42) --
	(316.60,106.42) --
	(316.90,106.43) --
	(317.20,106.43) --
	(317.50,106.43) --
	(317.80,106.44) --
	(318.10,106.44) --
	(318.40,106.44) --
	(318.70,106.45) --
	(318.99,106.45) --
	(319.29,106.45) --
	(319.59,106.46) --
	(319.89,106.46) --
	(320.19,106.46) --
	(320.49,106.46) --
	(320.79,106.47) --
	(321.09,106.47) --
	(321.39,106.47) --
	(321.69,106.48) --
	(321.99,106.48) --
	(322.29,106.48) --
	(322.59,106.49) --
	(322.89,106.49) --
	(323.19,106.49) --
	(323.49,106.49) --
	(323.79,106.50) --
	(324.09,106.50) --
	(324.38,106.50) --
	(324.68,106.51) --
	(324.98,106.51) --
	(325.28,106.51) --
	(325.58,106.52) --
	(325.88,106.52) --
	(326.18,106.52) --
	(326.48,106.52) --
	(326.78,106.53) --
	(327.08,106.53) --
	(327.38,106.53) --
	(327.68,106.54) --
	(327.98,106.54) --
	(328.28,106.54) --
	(328.58,106.55) --
	(328.88,106.55) --
	(329.18,106.55) --
	(329.48,106.55) --
	(329.77,106.56) --
	(330.07,106.56) --
	(330.37,106.56) --
	(330.67,106.57) --
	(330.97,106.57) --
	(331.27,106.57) --
	(331.57,106.57) --
	(331.87,106.58) --
	(332.17,106.58) --
	(332.47,106.58) --
	(332.77,106.59) --
	(333.07,106.59) --
	(333.37,106.59) --
	(333.67,106.59) --
	(333.97,106.60) --
	(334.27,106.60) --
	(334.57,106.60) --
	(334.87,106.61) --
	(335.16,106.61) --
	(335.46,106.61) --
	(335.76,106.61) --
	(336.06,106.62) --
	(336.36,106.62) --
	(336.66,106.62) --
	(336.96,106.62) --
	(337.26,106.63) --
	(337.56,106.63) --
	(337.86,106.63) --
	(338.16,106.64) --
	(338.46,106.64) --
	(338.76,106.64) --
	(339.06,106.64) --
	(339.36,106.65) --
	(339.66,106.65) --
	(339.96,106.65) --
	(340.26,106.65) --
	(340.56,106.66) --
	(340.85,106.66) --
	(341.15,106.66) --
	(341.45,106.67) --
	(341.75,106.67) --
	(342.05,106.67) --
	(342.35,106.67) --
	(342.65,106.68) --
	(342.95,106.68) --
	(343.25,106.68) --
	(343.55,106.68) --
	(343.85,106.69) --
	(344.15,106.69) --
	(344.45,106.69) --
	(344.75,106.69) --
	(345.05,106.70) --
	(345.35,106.70) --
	(345.65,106.70) --
	(345.95,106.70) --
	(346.24,106.71) --
	(346.54,106.71) --
	(346.84,106.71) --
	(347.14,106.71) --
	(347.44,106.72) --
	(347.74,106.72) --
	(348.04,106.72) --
	(348.34,106.72) --
	(348.64,106.73) --
	(348.94,106.73) --
	(349.24,106.73) --
	(349.54,106.73) --
	(349.84,106.74) --
	(350.14,106.74) --
	(350.44,106.74) --
	(350.74,106.74) --
	(351.04,106.75) --
	(351.34,106.75) --
	(351.63,106.75) --
	(351.93,106.75) --
	(352.23,106.76) --
	(352.53,106.76) --
	(352.83,106.76) --
	(353.13,106.76) --
	(353.43,106.77) --
	(353.73,106.77) --
	(354.03,106.77) --
	(354.33,106.77) --
	(354.63,106.78) --
	(354.93,106.78) --
	(355.23,106.78) --
	(355.53,106.78) --
	(355.83,106.79) --
	(356.13,106.79) --
	(356.43,106.79) --
	(356.73,106.79) --
	(357.02,106.80) --
	(357.32,106.80) --
	(357.62,106.80) --
	(357.92,106.80) --
	(358.22,106.81) --
	(358.52,106.81) --
	(358.82,106.81) --
	(359.12,106.81) --
	(359.42,106.82) --
	(359.72,106.82) --
	(360.02,106.82) --
	(360.32,106.82);
\definecolor{drawColor}{RGB}{190,190,190}

\path[draw=drawColor,line width= 0.4pt,line join=round,line cap=round] ( 49.20,109.18) -- (372.28,109.18);
\definecolor{drawColor}{RGB}{0,0,0}

\path[draw=drawColor,line width= 0.4pt,line join=round,line cap=round] (275.67,223.75) -- (286.47,223.75);

\path[draw=drawColor,line width= 0.4pt,dash pattern=on 4pt off 4pt ,line join=round,line cap=round] (275.67,209.35) -- (286.47,209.35);
\definecolor{drawColor}{RGB}{190,190,190}

\path[draw=drawColor,line width= 0.4pt,line join=round,line cap=round] (275.67,194.95) -- (286.47,194.95);
\definecolor{drawColor}{RGB}{0,0,0}

\node[text=drawColor,anchor=base west,inner sep=0pt, outer sep=0pt, scale=  0.60] at (291.87,221.68) {MIPS};

\node[text=drawColor,anchor=base west,inner sep=0pt, outer sep=0pt, scale=  0.60] at (291.87,207.28) {$\text{MIPS}^\text{max}\text{ und }\text{MIPS}^\text{min}$};

\node[text=drawColor,anchor=base west,inner sep=0pt, outer sep=0pt, scale=  0.60] at (291.87,192.88) {$\lim\limits_{t_\text{max}\rightarrow\infty} \text{MIPS}$};
\end{scope}
\end{tikzpicture}
}
          }
          % % Created by tikzDevice version 0.8.1 on 2015-04-20 18:17:42
% !TEX encoding = UTF-8 Unicode
\begin{tikzpicture}[x=1pt,y=1pt]
\definecolor{fillColor}{RGB}{255,255,255}
\path[use as bounding box,fill=fillColor,fill opacity=0.00] (0,0) rectangle (397.48,542.02);
\begin{scope}
\path[clip] ( 49.20,332.21) rectangle (372.28,492.82);
\definecolor{drawColor}{RGB}{0,0,0}

\path[draw=drawColor,line width= 0.4pt,line join=round,line cap=round] ( 76.41,542.02) --
	( 76.44,541.76) --
	( 76.74,538.48) --
	( 77.04,535.31) --
	( 77.34,532.24) --
	( 77.64,529.26) --
	( 77.94,526.37) --
	( 78.23,523.57) --
	( 78.53,520.85) --
	( 78.83,518.21) --
	( 79.13,515.65) --
	( 79.43,513.15) --
	( 79.73,510.73) --
	( 80.03,508.37) --
	( 80.33,506.08) --
	( 80.63,503.84) --
	( 80.93,501.67) --
	( 81.23,499.55) --
	( 81.53,497.49) --
	( 81.83,495.48) --
	( 82.13,493.52) --
	( 82.43,491.60) --
	( 82.73,489.74) --
	( 83.03,487.92) --
	( 83.33,486.14) --
	( 83.63,484.40) --
	( 83.92,482.71) --
	( 84.22,481.05) --
	( 84.52,479.43) --
	( 84.82,477.85) --
	( 85.12,476.30) --
	( 85.42,474.79) --
	( 85.72,473.30) --
	( 86.02,471.86) --
	( 86.32,470.44) --
	( 86.62,469.05) --
	( 86.92,467.69) --
	( 87.22,466.36) --
	( 87.52,465.05) --
	( 87.82,463.77) --
	( 88.12,462.52) --
	( 88.42,461.29) --
	( 88.72,460.09) --
	( 89.02,458.91) --
	( 89.31,457.75) --
	( 89.61,456.62) --
	( 89.91,455.50) --
	( 90.21,454.41) --
	( 90.51,453.34) --
	( 90.81,452.28) --
	( 91.11,451.25) --
	( 91.41,450.23) --
	( 91.71,449.24) --
	( 92.01,448.26) --
	( 92.31,447.29) --
	( 92.61,446.35) --
	( 92.91,445.42) --
	( 93.21,444.50) --
	( 93.51,443.60) --
	( 93.81,442.72) --
	( 94.11,441.85) --
	( 94.41,441.00) --
	( 94.70,440.16) --
	( 95.00,439.33) --
	( 95.30,438.52) --
	( 95.60,437.72) --
	( 95.90,436.93) --
	( 96.20,436.15) --
	( 96.50,435.39) --
	( 96.80,434.64) --
	( 97.10,433.90) --
	( 97.40,433.17) --
	( 97.70,432.45) --
	( 98.00,431.75) --
	( 98.30,431.05) --
	( 98.60,430.37) --
	( 98.90,429.69) --
	( 99.20,429.02) --
	( 99.50,428.37) --
	( 99.80,427.72) --
	(100.09,427.08) --
	(100.39,426.46) --
	(100.69,425.84) --
	(100.99,425.22) --
	(101.29,424.62) --
	(101.59,424.03) --
	(101.89,423.44) --
	(102.19,422.86) --
	(102.49,422.29) --
	(102.79,421.73) --
	(103.09,421.18) --
	(103.39,420.63) --
	(103.69,420.09) --
	(103.99,419.55) --
	(104.29,419.03) --
	(104.59,418.51) --
	(104.89,417.99) --
	(105.19,417.49) --
	(105.49,416.99) --
	(105.78,416.49) --
	(106.08,416.01) --
	(106.38,415.52) --
	(106.68,415.05) --
	(106.98,414.58) --
	(107.28,414.11) --
	(107.58,413.66) --
	(107.88,413.20) --
	(108.18,412.75) --
	(108.48,412.31) --
	(108.78,411.88) --
	(109.08,411.44) --
	(109.38,411.02) --
	(109.68,410.59) --
	(109.98,410.18) --
	(110.28,409.77) --
	(110.58,409.36) --
	(110.88,408.95) --
	(111.17,408.56) --
	(111.47,408.16) --
	(111.77,407.77) --
	(112.07,407.39) --
	(112.37,407.01) --
	(112.67,406.63) --
	(112.97,406.26) --
	(113.27,405.89) --
	(113.57,405.52) --
	(113.87,405.16) --
	(114.17,404.81) --
	(114.47,404.45) --
	(114.77,404.10) --
	(115.07,403.76) --
	(115.37,403.42) --
	(115.67,403.08) --
	(115.97,402.74) --
	(116.27,402.41) --
	(116.56,402.08) --
	(116.86,401.76) --
	(117.16,401.44) --
	(117.46,401.12) --
	(117.76,400.80) --
	(118.06,400.49) --
	(118.36,400.18) --
	(118.66,436.76) --
	(118.96,436.28) --
	(119.26,435.80) --
	(119.56,435.32) --
	(119.86,434.85) --
	(120.16,434.39) --
	(120.46,433.93) --
	(120.76,433.47) --
	(121.06,433.02) --
	(121.36,432.57) --
	(121.66,432.12) --
	(121.95,431.68) --
	(122.25,431.25) --
	(122.55,430.82) --
	(122.85,430.39) --
	(123.15,429.97) --
	(123.45,429.55) --
	(123.75,429.13) --
	(124.05,428.72) --
	(124.35,428.31) --
	(124.65,427.90) --
	(124.95,427.50) --
	(125.25,427.11) --
	(125.55,426.71) --
	(125.85,426.32) --
	(126.15,425.93) --
	(126.45,425.55) --
	(126.75,425.17) --
	(127.05,424.79) --
	(127.35,424.42) --
	(127.64,424.05) --
	(127.94,423.68) --
	(128.24,423.32) --
	(128.54,422.96) --
	(128.84,422.60) --
	(129.14,422.24) --
	(129.44,421.89) --
	(129.74,421.54) --
	(130.04,421.19) --
	(130.34,420.85) --
	(130.64,420.51) --
	(130.94,420.17) --
	(131.24,419.84) --
	(131.54,419.51) --
	(131.84,419.18) --
	(132.14,418.85) --
	(132.44,418.52) --
	(132.74,418.20) --
	(133.03,417.88) --
	(133.33,417.57) --
	(133.63,417.25) --
	(133.93,416.94) --
	(134.23,416.63) --
	(134.53,416.33) --
	(134.83,416.02) --
	(135.13,415.72) --
	(135.43,415.42) --
	(135.73,415.12) --
	(136.03,414.83) --
	(136.33,414.54) --
	(136.63,414.25) --
	(136.93,413.96) --
	(137.23,413.67) --
	(137.53,413.39) --
	(137.83,413.11) --
	(138.13,412.83) --
	(138.42,412.55) --
	(138.72,412.27) --
	(139.02,412.00) --
	(139.32,411.73) --
	(139.62,411.46) --
	(139.92,411.19) --
	(140.22,410.93) --
	(140.52,410.66) --
	(140.82,410.40) --
	(141.12,410.14) --
	(141.42,409.88) --
	(141.72,409.63) --
	(142.02,409.37) --
	(142.32,409.12) --
	(142.62,408.87) --
	(142.92,408.62) --
	(143.22,408.37) --
	(143.52,408.13) --
	(143.81,407.88) --
	(144.11,407.64) --
	(144.41,407.40) --
	(144.71,407.16) --
	(145.01,406.93) --
	(145.31,406.69) --
	(145.61,406.46) --
	(145.91,406.22) --
	(146.21,405.99) --
	(146.51,405.76) --
	(146.81,405.54) --
	(147.11,405.31) --
	(147.41,405.09) --
	(147.71,404.86) --
	(148.01,404.64) --
	(148.31,404.42) --
	(148.61,404.20) --
	(148.91,403.99) --
	(149.21,403.77) --
	(149.50,403.56) --
	(149.80,403.34) --
	(150.10,403.13) --
	(150.40,402.92) --
	(150.70,402.71) --
	(151.00,402.51) --
	(151.30,402.30) --
	(151.60,402.10) --
	(151.90,401.89) --
	(152.20,401.69) --
	(152.50,401.49) --
	(152.80,401.29) --
	(153.10,401.09) --
	(153.40,400.89) --
	(153.70,400.70) --
	(154.00,400.50) --
	(154.30,400.31) --
	(154.60,400.12) --
	(154.89,399.93) --
	(155.19,399.74) --
	(155.49,399.55) --
	(155.79,399.36) --
	(156.09,399.17) --
	(156.39,398.99) --
	(156.69,398.81) --
	(156.99,398.62) --
	(157.29,398.44) --
	(157.59,398.26) --
	(157.89,398.08) --
	(158.19,397.90) --
	(158.49,397.72) --
	(158.79,397.55) --
	(159.09,397.37) --
	(159.39,397.20) --
	(159.69,397.02) --
	(159.99,396.85) --
	(160.28,396.68) --
	(160.58,396.51) --
	(160.88,396.34) --
	(161.18,396.17) --
	(161.48,396.01) --
	(161.78,395.84) --
	(162.08,395.67) --
	(162.38,395.51) --
	(162.68,395.35) --
	(162.98,395.18) --
	(163.28,395.02) --
	(163.58,394.86) --
	(163.88,394.70) --
	(164.18,394.54) --
	(164.48,394.38) --
	(164.78,394.23) --
	(165.08,394.07) --
	(165.38,393.91) --
	(165.67,393.76) --
	(165.97,393.61) --
	(166.27,393.45) --
	(166.57,393.30) --
	(166.87,393.15) --
	(167.17,393.00) --
	(167.47,392.85) --
	(167.77,392.70) --
	(168.07,392.55) --
	(168.37,392.41) --
	(168.67,392.26) --
	(168.97,392.11) --
	(169.27,391.97) --
	(169.57,391.83) --
	(169.87,391.68) --
	(170.17,391.54) --
	(170.47,391.40) --
	(170.77,391.26) --
	(171.07,391.12) --
	(171.36,390.98) --
	(171.66,390.84) --
	(171.96,390.70) --
	(172.26,390.56) --
	(172.56,390.43) --
	(172.86,390.29) --
	(173.16,390.16) --
	(173.46,390.02) --
	(173.76,389.89) --
	(174.06,389.76) --
	(174.36,389.62) --
	(174.66,389.49) --
	(174.96,389.36) --
	(175.26,389.23) --
	(175.56,389.10) --
	(175.86,388.97) --
	(176.16,388.84) --
	(176.46,388.72) --
	(176.75,388.59) --
	(177.05,388.46) --
	(177.35,388.34) --
	(177.65,388.21) --
	(177.95,388.09) --
	(178.25,387.96) --
	(178.55,387.84) --
	(178.85,387.72) --
	(179.15,406.04) --
	(179.45,405.87) --
	(179.75,405.71) --
	(180.05,405.54) --
	(180.35,405.38) --
	(180.65,405.21) --
	(180.95,405.05) --
	(181.25,404.89) --
	(181.55,404.73) --
	(181.85,404.57) --
	(182.14,404.41) --
	(182.44,404.25) --
	(182.74,404.09) --
	(183.04,403.93) --
	(183.34,403.78) --
	(183.64,403.62) --
	(183.94,403.46) --
	(184.24,403.31) --
	(184.54,403.16) --
	(184.84,403.00) --
	(185.14,402.85) --
	(185.44,402.70) --
	(185.74,402.55) --
	(186.04,402.40) --
	(186.34,402.25) --
	(186.64,402.10) --
	(186.94,401.95) --
	(187.24,401.80) --
	(187.53,401.66) --
	(187.83,401.51) --
	(188.13,401.37) --
	(188.43,401.22) --
	(188.73,401.08) --
	(189.03,400.93) --
	(189.33,400.79) --
	(189.63,400.65) --
	(189.93,400.51) --
	(190.23,400.37) --
	(190.53,400.23) --
	(190.83,400.09) --
	(191.13,399.95) --
	(191.43,399.81) --
	(191.73,399.67) --
	(192.03,399.54) --
	(192.33,399.40) --
	(192.63,399.26) --
	(192.93,399.13) --
	(193.22,398.99) --
	(193.52,398.86) --
	(193.82,398.73) --
	(194.12,398.59) --
	(194.42,398.46) --
	(194.72,398.33) --
	(195.02,398.20) --
	(195.32,398.07) --
	(195.62,397.94) --
	(195.92,397.81) --
	(196.22,397.68) --
	(196.52,397.55) --
	(196.82,397.42) --
	(197.12,397.30) --
	(197.42,397.17) --
	(197.72,397.04) --
	(198.02,396.92) --
	(198.32,396.79) --
	(198.61,396.67) --
	(198.91,396.55) --
	(199.21,396.42) --
	(199.51,396.30) --
	(199.81,396.18) --
	(200.11,396.05) --
	(200.41,395.93) --
	(200.71,395.81) --
	(201.01,395.69) --
	(201.31,395.57) --
	(201.61,395.45) --
	(201.91,395.33) --
	(202.21,395.22) --
	(202.51,395.10) --
	(202.81,394.98) --
	(203.11,394.86) --
	(203.41,394.75) --
	(203.71,394.63) --
	(204.00,394.52) --
	(204.30,394.40) --
	(204.60,394.29) --
	(204.90,394.17) --
	(205.20,394.06) --
	(205.50,393.95) --
	(205.80,393.83) --
	(206.10,393.72) --
	(206.40,393.61) --
	(206.70,393.50) --
	(207.00,393.39) --
	(207.30,393.28) --
	(207.60,393.17) --
	(207.90,393.06) --
	(208.20,392.95) --
	(208.50,392.84) --
	(208.80,392.73) --
	(209.10,392.62) --
	(209.39,392.52) --
	(209.69,392.41) --
	(209.99,392.30) --
	(210.29,392.20) --
	(210.59,392.09) --
	(210.89,391.99) --
	(211.19,391.88) --
	(211.49,391.78) --
	(211.79,391.67) --
	(212.09,391.57) --
	(212.39,391.47) --
	(212.69,391.36) --
	(212.99,391.26) --
	(213.29,391.16) --
	(213.59,391.06) --
	(213.89,390.96) --
	(214.19,390.86) --
	(214.49,390.75) --
	(214.79,390.65) --
	(215.08,390.56) --
	(215.38,390.46) --
	(215.68,390.36) --
	(215.98,390.26) --
	(216.28,390.16) --
	(216.58,390.06) --
	(216.88,389.96) --
	(217.18,389.87) --
	(217.48,389.77) --
	(217.78,389.67) --
	(218.08,389.58) --
	(218.38,389.48) --
	(218.68,389.39) --
	(218.98,389.29) --
	(219.28,389.20) --
	(219.58,389.10) --
	(219.88,389.01) --
	(220.18,388.92) --
	(220.47,388.82) --
	(220.77,388.73) --
	(221.07,388.64) --
	(221.37,388.55) --
	(221.67,388.45) --
	(221.97,388.36) --
	(222.27,388.27) --
	(222.57,388.18) --
	(222.87,388.09) --
	(223.17,388.00) --
	(223.47,387.91) --
	(223.77,387.82) --
	(224.07,387.73) --
	(224.37,387.64) --
	(224.67,387.55) --
	(224.97,387.47) --
	(225.27,387.38) --
	(225.57,387.29) --
	(225.86,387.20) --
	(226.16,387.12) --
	(226.46,387.03) --
	(226.76,386.94) --
	(227.06,386.86) --
	(227.36,386.77) --
	(227.66,386.69) --
	(227.96,386.60) --
	(228.26,386.52) --
	(228.56,386.43) --
	(228.86,386.35) --
	(229.16,386.26) --
	(229.46,386.18) --
	(229.76,386.10) --
	(230.06,386.01) --
	(230.36,385.93) --
	(230.66,385.85) --
	(230.96,385.77) --
	(231.25,385.68) --
	(231.55,385.60) --
	(231.85,385.52) --
	(232.15,385.44) --
	(232.45,385.36) --
	(232.75,385.28) --
	(233.05,385.20) --
	(233.35,385.12) --
	(233.65,385.04) --
	(233.95,384.96) --
	(234.25,384.88) --
	(234.55,384.80) --
	(234.85,384.72) --
	(235.15,384.64) --
	(235.45,384.57) --
	(235.75,384.49) --
	(236.05,384.41) --
	(236.35,384.33) --
	(236.65,384.26) --
	(236.94,384.18) --
	(237.24,384.10) --
	(237.54,384.03) --
	(237.84,383.95) --
	(238.14,383.87) --
	(238.44,383.80) --
	(238.74,383.72) --
	(239.04,383.65) --
	(239.34,383.57) --
	(239.64,395.80) --
	(239.94,395.70) --
	(240.24,395.61) --
	(240.54,395.52) --
	(240.84,395.42) --
	(241.14,395.33) --
	(241.44,395.24) --
	(241.74,395.14) --
	(242.04,395.05) --
	(242.33,394.96) --
	(242.63,394.87) --
	(242.93,394.77) --
	(243.23,394.68) --
	(243.53,394.59) --
	(243.83,394.50) --
	(244.13,394.41) --
	(244.43,394.32) --
	(244.73,394.23) --
	(245.03,394.14) --
	(245.33,394.05) --
	(245.63,393.96) --
	(245.93,393.88) --
	(246.23,393.79) --
	(246.53,393.70) --
	(246.83,393.61) --
	(247.13,393.52) --
	(247.43,393.44) --
	(247.72,393.35) --
	(248.02,393.26) --
	(248.32,393.18) --
	(248.62,393.09) --
	(248.92,393.01) --
	(249.22,392.92) --
	(249.52,392.83) --
	(249.82,392.75) --
	(250.12,392.66) --
	(250.42,392.58) --
	(250.72,392.50) --
	(251.02,392.41) --
	(251.32,392.33) --
	(251.62,392.24) --
	(251.92,392.16) --
	(252.22,392.08) --
	(252.52,392.00) --
	(252.82,391.91) --
	(253.11,391.83) --
	(253.41,391.75) --
	(253.71,391.67) --
	(254.01,391.59) --
	(254.31,391.50) --
	(254.61,391.42) --
	(254.91,391.34) --
	(255.21,391.26) --
	(255.51,391.18) --
	(255.81,391.10) --
	(256.11,391.02) --
	(256.41,390.94) --
	(256.71,390.86) --
	(257.01,390.79) --
	(257.31,390.71) --
	(257.61,390.63) --
	(257.91,390.55) --
	(258.21,390.47) --
	(258.51,390.39) --
	(258.80,390.32) --
	(259.10,390.24) --
	(259.40,390.16) --
	(259.70,390.09) --
	(260.00,390.01) --
	(260.30,389.93) --
	(260.60,389.86) --
	(260.90,389.78) --
	(261.20,389.70) --
	(261.50,389.63) --
	(261.80,389.55) --
	(262.10,389.48) --
	(262.40,389.40) --
	(262.70,389.33) --
	(263.00,389.25) --
	(263.30,389.18) --
	(263.60,389.11) --
	(263.90,389.03) --
	(264.19,388.96) --
	(264.49,388.89) --
	(264.79,388.81) --
	(265.09,388.74) --
	(265.39,388.67) --
	(265.69,388.59) --
	(265.99,388.52) --
	(266.29,388.45) --
	(266.59,388.38) --
	(266.89,388.31) --
	(267.19,388.23) --
	(267.49,388.16) --
	(267.79,388.09) --
	(268.09,388.02) --
	(268.39,387.95) --
	(268.69,387.88) --
	(268.99,387.81) --
	(269.29,387.74) --
	(269.58,387.67) --
	(269.88,387.60) --
	(270.18,387.53) --
	(270.48,387.46) --
	(270.78,387.39) --
	(271.08,387.32) --
	(271.38,387.25) --
	(271.68,387.19) --
	(271.98,387.12) --
	(272.28,387.05) --
	(272.58,386.98) --
	(272.88,386.91) --
	(273.18,386.85) --
	(273.48,386.78) --
	(273.78,386.71) --
	(274.08,386.65) --
	(274.38,386.58) --
	(274.68,386.51) --
	(274.98,386.45) --
	(275.27,386.38) --
	(275.57,386.31) --
	(275.87,386.25) --
	(276.17,386.18) --
	(276.47,386.12) --
	(276.77,386.05) --
	(277.07,385.99) --
	(277.37,385.92) --
	(277.67,385.86) --
	(277.97,385.79) --
	(278.27,385.73) --
	(278.57,385.66) --
	(278.87,385.60) --
	(279.17,385.53) --
	(279.47,385.47) --
	(279.77,385.41) --
	(280.07,385.34) --
	(280.37,385.28) --
	(280.66,385.22) --
	(280.96,385.15) --
	(281.26,385.09) --
	(281.56,385.03) --
	(281.86,384.97) --
	(282.16,384.90) --
	(282.46,384.84) --
	(282.76,384.78) --
	(283.06,384.72) --
	(283.36,384.66) --
	(283.66,384.60) --
	(283.96,384.53) --
	(284.26,384.47) --
	(284.56,384.41) --
	(284.86,384.35) --
	(285.16,384.29) --
	(285.46,384.23) --
	(285.76,384.17) --
	(286.05,384.11) --
	(286.35,384.05) --
	(286.65,383.99) --
	(286.95,383.93) --
	(287.25,383.87) --
	(287.55,383.81) --
	(287.85,383.75) --
	(288.15,383.69) --
	(288.45,383.63) --
	(288.75,383.58) --
	(289.05,383.52) --
	(289.35,383.46) --
	(289.65,383.40) --
	(289.95,383.34) --
	(290.25,383.28) --
	(290.55,383.23) --
	(290.85,383.17) --
	(291.15,383.11) --
	(291.44,383.05) --
	(291.74,383.00) --
	(292.04,382.94) --
	(292.34,382.88) --
	(292.64,382.83) --
	(292.94,382.77) --
	(293.24,382.71) --
	(293.54,382.66) --
	(293.84,382.60) --
	(294.14,382.54) --
	(294.44,382.49) --
	(294.74,382.43) --
	(295.04,382.38) --
	(295.34,382.32) --
	(295.64,382.27) --
	(295.94,382.21) --
	(296.24,382.16) --
	(296.54,382.10) --
	(296.84,382.05) --
	(297.13,381.99) --
	(297.43,381.94) --
	(297.73,381.88) --
	(298.03,381.83) --
	(298.33,381.77) --
	(298.63,381.72) --
	(298.93,381.67) --
	(299.23,381.61) --
	(299.53,381.56) --
	(299.83,381.51) --
	(300.13,390.68) --
	(300.43,390.61) --
	(300.73,390.55) --
	(301.03,390.48) --
	(301.33,390.42) --
	(301.63,390.35) --
	(301.93,390.29) --
	(302.23,390.23) --
	(302.52,390.16) --
	(302.82,390.10) --
	(303.12,390.04) --
	(303.42,389.97) --
	(303.72,389.91) --
	(304.02,389.85) --
	(304.32,389.79) --
	(304.62,389.72) --
	(304.92,389.66) --
	(305.22,389.60) --
	(305.52,389.54) --
	(305.82,389.47) --
	(306.12,389.41) --
	(306.42,389.35) --
	(306.72,389.29) --
	(307.02,389.23) --
	(307.32,389.17) --
	(307.62,389.11) --
	(307.91,389.05) --
	(308.21,388.99) --
	(308.51,388.93) --
	(308.81,388.86) --
	(309.11,388.80) --
	(309.41,388.74) --
	(309.71,388.68) --
	(310.01,388.62) --
	(310.31,388.57) --
	(310.61,388.51) --
	(310.91,388.45) --
	(311.21,388.39) --
	(311.51,388.33) --
	(311.81,388.27) --
	(312.11,388.21) --
	(312.41,388.15) --
	(312.71,388.09) --
	(313.01,388.03) --
	(313.30,387.98) --
	(313.60,387.92) --
	(313.90,387.86) --
	(314.20,387.80) --
	(314.50,387.74) --
	(314.80,387.69) --
	(315.10,387.63) --
	(315.40,387.57) --
	(315.70,387.52) --
	(316.00,387.46) --
	(316.30,387.40) --
	(316.60,387.34) --
	(316.90,387.29) --
	(317.20,387.23) --
	(317.50,387.18) --
	(317.80,387.12) --
	(318.10,387.06) --
	(318.40,387.01) --
	(318.70,386.95) --
	(318.99,386.90) --
	(319.29,386.84) --
	(319.59,386.78) --
	(319.89,386.73) --
	(320.19,386.67) --
	(320.49,386.62) --
	(320.79,386.56) --
	(321.09,386.51) --
	(321.39,386.45) --
	(321.69,386.40) --
	(321.99,386.35) --
	(322.29,386.29) --
	(322.59,386.24) --
	(322.89,386.18) --
	(323.19,386.13) --
	(323.49,386.07) --
	(323.79,386.02) --
	(324.09,385.97) --
	(324.38,385.91) --
	(324.68,385.86) --
	(324.98,385.81) --
	(325.28,385.75) --
	(325.58,385.70) --
	(325.88,385.65) --
	(326.18,385.60) --
	(326.48,385.54) --
	(326.78,385.49) --
	(327.08,385.44) --
	(327.38,385.39) --
	(327.68,385.33) --
	(327.98,385.28) --
	(328.28,385.23) --
	(328.58,385.18) --
	(328.88,385.13) --
	(329.18,385.07) --
	(329.48,385.02) --
	(329.77,384.97) --
	(330.07,384.92) --
	(330.37,384.87) --
	(330.67,384.82) --
	(330.97,384.77) --
	(331.27,384.72) --
	(331.57,384.67) --
	(331.87,384.61) --
	(332.17,384.56) --
	(332.47,384.51) --
	(332.77,384.46) --
	(333.07,384.41) --
	(333.37,384.36) --
	(333.67,384.31) --
	(333.97,384.26) --
	(334.27,384.21) --
	(334.57,384.16) --
	(334.87,384.11) --
	(335.16,384.06) --
	(335.46,384.02) --
	(335.76,383.97) --
	(336.06,383.92) --
	(336.36,383.87) --
	(336.66,383.82) --
	(336.96,383.77) --
	(337.26,383.72) --
	(337.56,383.67) --
	(337.86,383.63) --
	(338.16,383.58) --
	(338.46,383.53) --
	(338.76,383.48) --
	(339.06,383.43) --
	(339.36,383.38) --
	(339.66,383.34) --
	(339.96,383.29) --
	(340.26,383.24) --
	(340.56,383.19) --
	(340.85,383.15) --
	(341.15,383.10) --
	(341.45,383.05) --
	(341.75,383.00) --
	(342.05,382.96) --
	(342.35,382.91) --
	(342.65,382.86) --
	(342.95,382.82) --
	(343.25,382.77) --
	(343.55,382.72) --
	(343.85,382.68) --
	(344.15,382.63) --
	(344.45,382.58) --
	(344.75,382.54) --
	(345.05,382.49) --
	(345.35,382.45) --
	(345.65,382.40) --
	(345.95,382.35) --
	(346.24,382.31) --
	(346.54,382.26) --
	(346.84,382.22) --
	(347.14,382.17) --
	(347.44,382.13) --
	(347.74,382.08) --
	(348.04,382.04) --
	(348.34,381.99) --
	(348.64,381.95) --
	(348.94,381.90) --
	(349.24,381.86) --
	(349.54,381.81) --
	(349.84,381.77) --
	(350.14,381.72) --
	(350.44,381.68) --
	(350.74,381.64) --
	(351.04,381.59) --
	(351.34,381.55) --
	(351.63,381.50) --
	(351.93,381.46) --
	(352.23,381.42) --
	(352.53,381.37) --
	(352.83,381.33) --
	(353.13,381.28) --
	(353.43,381.24) --
	(353.73,381.20) --
	(354.03,381.15) --
	(354.33,381.11) --
	(354.63,381.07) --
	(354.93,381.02) --
	(355.23,380.98) --
	(355.53,380.94) --
	(355.83,380.90) --
	(356.13,380.85) --
	(356.43,380.81) --
	(356.73,380.77) --
	(357.02,380.73) --
	(357.32,380.68) --
	(357.62,380.64) --
	(357.92,380.60) --
	(358.22,380.56) --
	(358.52,380.51) --
	(358.82,380.47) --
	(359.12,380.43) --
	(359.42,380.39) --
	(359.72,380.35) --
	(360.02,380.31) --
	(360.32,380.26);
\end{scope}
\begin{scope}
\path[clip] (  0.00,  0.00) rectangle (397.48,542.02);
\definecolor{drawColor}{RGB}{0,0,0}

\path[draw=drawColor,line width= 0.4pt,line join=round,line cap=round] ( 58.14,332.21) -- (360.32,332.21);

\path[draw=drawColor,line width= 0.4pt,line join=round,line cap=round] ( 58.14,332.21) -- ( 58.14,326.21);

\path[draw=drawColor,line width= 0.4pt,line join=round,line cap=round] (118.58,332.21) -- (118.58,326.21);

\path[draw=drawColor,line width= 0.4pt,line join=round,line cap=round] (179.01,332.21) -- (179.01,326.21);

\path[draw=drawColor,line width= 0.4pt,line join=round,line cap=round] (239.45,332.21) -- (239.45,326.21);

\path[draw=drawColor,line width= 0.4pt,line join=round,line cap=round] (299.88,332.21) -- (299.88,326.21);

\path[draw=drawColor,line width= 0.4pt,line join=round,line cap=round] (360.32,332.21) -- (360.32,326.21);

\node[text=drawColor,anchor=base,inner sep=0pt, outer sep=0pt, scale=  1.00] at ( 58.14,310.61) {0};

\node[text=drawColor,anchor=base,inner sep=0pt, outer sep=0pt, scale=  1.00] at (118.58,310.61) {2};

\node[text=drawColor,anchor=base,inner sep=0pt, outer sep=0pt, scale=  1.00] at (179.01,310.61) {4};

\node[text=drawColor,anchor=base,inner sep=0pt, outer sep=0pt, scale=  1.00] at (239.45,310.61) {6};

\node[text=drawColor,anchor=base,inner sep=0pt, outer sep=0pt, scale=  1.00] at (299.88,310.61) {8};

\node[text=drawColor,anchor=base,inner sep=0pt, outer sep=0pt, scale=  1.00] at (360.32,310.61) {10};

\path[draw=drawColor,line width= 0.4pt,line join=round,line cap=round] ( 49.20,338.16) -- ( 49.20,485.89);

\path[draw=drawColor,line width= 0.4pt,line join=round,line cap=round] ( 49.20,338.16) -- ( 43.20,338.16);

\path[draw=drawColor,line width= 0.4pt,line join=round,line cap=round] ( 49.20,362.78) -- ( 43.20,362.78);

\path[draw=drawColor,line width= 0.4pt,line join=round,line cap=round] ( 49.20,387.40) -- ( 43.20,387.40);

\path[draw=drawColor,line width= 0.4pt,line join=round,line cap=round] ( 49.20,412.03) -- ( 43.20,412.03);

\path[draw=drawColor,line width= 0.4pt,line join=round,line cap=round] ( 49.20,436.65) -- ( 43.20,436.65);

\path[draw=drawColor,line width= 0.4pt,line join=round,line cap=round] ( 49.20,461.27) -- ( 43.20,461.27);

\path[draw=drawColor,line width= 0.4pt,line join=round,line cap=round] ( 49.20,485.89) -- ( 43.20,485.89);

\node[text=drawColor,rotate= 90.00,anchor=base,inner sep=0pt, outer sep=0pt, scale=  1.00] at ( 34.80,338.16) {0};

\node[text=drawColor,rotate= 90.00,anchor=base,inner sep=0pt, outer sep=0pt, scale=  1.00] at ( 34.80,362.78) {1};

\node[text=drawColor,rotate= 90.00,anchor=base,inner sep=0pt, outer sep=0pt, scale=  1.00] at ( 34.80,387.40) {2};

\node[text=drawColor,rotate= 90.00,anchor=base,inner sep=0pt, outer sep=0pt, scale=  1.00] at ( 34.80,412.03) {3};

\node[text=drawColor,rotate= 90.00,anchor=base,inner sep=0pt, outer sep=0pt, scale=  1.00] at ( 34.80,436.65) {4};

\node[text=drawColor,rotate= 90.00,anchor=base,inner sep=0pt, outer sep=0pt, scale=  1.00] at ( 34.80,461.27) {5};

\node[text=drawColor,rotate= 90.00,anchor=base,inner sep=0pt, outer sep=0pt, scale=  1.00] at ( 34.80,485.89) {6};

\path[draw=drawColor,line width= 0.4pt,line join=round,line cap=round] ( 49.20,332.21) --
	(372.28,332.21) --
	(372.28,492.82) --
	( 49.20,492.82) --
	( 49.20,332.21);
\end{scope}
\begin{scope}
\path[clip] (  0.00,271.01) rectangle (397.48,542.02);
\definecolor{drawColor}{RGB}{0,0,0}

\node[text=drawColor,anchor=base,inner sep=0pt, outer sep=0pt, scale=  1.20] at (210.74,513.28) {\bfseries $\tilde{i}_P > i_R$};

\node[text=drawColor,anchor=base,inner sep=0pt, outer sep=0pt, scale=  1.00] at (210.74,298.61) {$t_\text{max}$};

\node[text=drawColor,rotate= 90.00,anchor=base,inner sep=0pt, outer sep=0pt, scale=  1.00] at ( 22.80,412.52) {MIPS};
\end{scope}
\begin{scope}
\path[clip] ( 49.20,332.21) rectangle (372.28,492.82);
\definecolor{drawColor}{RGB}{0,0,0}

\path[draw=drawColor,line width= 0.4pt,dash pattern=on 4pt off 4pt ,line join=round,line cap=round] ( 80.46,542.02) --
	( 80.63,540.78) --
	( 80.93,538.60) --
	( 81.23,536.48) --
	( 81.53,534.42) --
	( 81.83,532.41) --
	( 82.13,530.45) --
	( 82.43,528.54) --
	( 82.73,526.67) --
	( 83.03,524.85) --
	( 83.33,523.07) --
	( 83.63,521.33) --
	( 83.92,519.64) --
	( 84.22,517.98) --
	( 84.52,516.36) --
	( 84.82,514.78) --
	( 85.12,513.23) --
	( 85.42,511.72) --
	( 85.72,510.24) --
	( 86.02,508.79) --
	( 86.32,507.37) --
	( 86.62,505.98) --
	( 86.92,504.62) --
	( 87.22,503.29) --
	( 87.52,501.99) --
	( 87.82,500.71) --
	( 88.12,499.45) --
	( 88.42,498.23) --
	( 88.72,497.02) --
	( 89.02,495.84) --
	( 89.31,494.69) --
	( 89.61,493.55) --
	( 89.91,492.44) --
	( 90.21,491.34) --
	( 90.51,490.27) --
	( 90.81,489.22) --
	( 91.11,488.18) --
	( 91.41,487.17) --
	( 91.71,486.17) --
	( 92.01,485.19) --
	( 92.31,484.22) --
	( 92.61,483.28) --
	( 92.91,482.35) --
	( 93.21,481.44) --
	( 93.51,480.54) --
	( 93.81,479.65) --
	( 94.11,478.78) --
	( 94.41,477.93) --
	( 94.70,477.09) --
	( 95.00,476.26) --
	( 95.30,475.45) --
	( 95.60,474.65) --
	( 95.90,473.86) --
	( 96.20,473.09) --
	( 96.50,472.32) --
	( 96.80,471.57) --
	( 97.10,470.83) --
	( 97.40,470.10) --
	( 97.70,469.39) --
	( 98.00,468.68) --
	( 98.30,467.98) --
	( 98.60,467.30) --
	( 98.90,466.62) --
	( 99.20,465.96) --
	( 99.50,465.30) --
	( 99.80,464.65) --
	(100.09,464.02) --
	(100.39,463.39) --
	(100.69,462.77) --
	(100.99,462.16) --
	(101.29,461.55) --
	(101.59,460.96) --
	(101.89,460.37) --
	(102.19,459.80) --
	(102.49,459.23) --
	(102.79,458.66) --
	(103.09,458.11) --
	(103.39,457.56) --
	(103.69,457.02) --
	(103.99,456.49) --
	(104.29,455.96) --
	(104.59,455.44) --
	(104.89,454.93) --
	(105.19,454.42) --
	(105.49,453.92) --
	(105.78,453.43) --
	(106.08,452.94) --
	(106.38,452.46) --
	(106.68,451.98) --
	(106.98,451.51) --
	(107.28,451.05) --
	(107.58,450.59) --
	(107.88,450.14) --
	(108.18,449.69) --
	(108.48,449.25) --
	(108.78,448.81) --
	(109.08,448.38) --
	(109.38,447.95) --
	(109.68,447.53) --
	(109.98,447.11) --
	(110.28,446.70) --
	(110.58,446.29) --
	(110.88,445.89) --
	(111.17,445.49) --
	(111.47,445.10) --
	(111.77,444.71) --
	(112.07,444.32) --
	(112.37,443.94) --
	(112.67,443.56) --
	(112.97,443.19) --
	(113.27,442.82) --
	(113.57,442.46) --
	(113.87,442.10) --
	(114.17,441.74) --
	(114.47,441.39) --
	(114.77,441.04) --
	(115.07,440.69) --
	(115.37,440.35) --
	(115.67,440.01) --
	(115.97,439.68) --
	(116.27,439.34) --
	(116.56,439.02) --
	(116.86,438.69) --
	(117.16,438.37) --
	(117.46,438.05) --
	(117.76,437.74) --
	(118.06,437.43) --
	(118.36,437.12) --
	(118.66,436.81) --
	(118.96,436.51) --
	(119.26,436.21) --
	(119.56,435.91) --
	(119.86,435.62) --
	(120.16,435.33) --
	(120.46,435.04) --
	(120.76,434.75) --
	(121.06,434.47) --
	(121.36,434.19) --
	(121.66,433.91) --
	(121.95,433.64) --
	(122.25,433.37) --
	(122.55,433.10) --
	(122.85,432.83) --
	(123.15,432.56) --
	(123.45,432.30) --
	(123.75,432.04) --
	(124.05,431.78) --
	(124.35,431.53) --
	(124.65,431.28) --
	(124.95,431.02) --
	(125.25,430.78) --
	(125.55,430.53) --
	(125.85,430.29) --
	(126.15,430.04) --
	(126.45,429.80) --
	(126.75,429.57) --
	(127.05,429.33) --
	(127.35,429.10) --
	(127.64,428.87) --
	(127.94,428.64) --
	(128.24,428.41) --
	(128.54,428.18) --
	(128.84,427.96) --
	(129.14,427.74) --
	(129.44,427.52) --
	(129.74,427.30) --
	(130.04,427.08) --
	(130.34,426.87) --
	(130.64,426.65) --
	(130.94,426.44) --
	(131.24,426.23) --
	(131.54,426.03) --
	(131.84,425.82) --
	(132.14,425.62) --
	(132.44,425.41) --
	(132.74,425.21) --
	(133.03,425.01) --
	(133.33,424.82) --
	(133.63,424.62) --
	(133.93,424.42) --
	(134.23,424.23) --
	(134.53,424.04) --
	(134.83,423.85) --
	(135.13,423.66) --
	(135.43,423.47) --
	(135.73,423.29) --
	(136.03,423.10) --
	(136.33,422.92) --
	(136.63,422.74) --
	(136.93,422.56) --
	(137.23,422.38) --
	(137.53,422.20) --
	(137.83,422.03) --
	(138.13,421.85) --
	(138.42,421.68) --
	(138.72,421.51) --
	(139.02,421.33) --
	(139.32,421.17) --
	(139.62,421.00) --
	(139.92,420.83) --
	(140.22,420.66) --
	(140.52,420.50) --
	(140.82,420.34) --
	(141.12,420.17) --
	(141.42,420.01) --
	(141.72,419.85) --
	(142.02,419.69) --
	(142.32,419.53) --
	(142.62,419.38) --
	(142.92,419.22) --
	(143.22,419.07) --
	(143.52,418.91) --
	(143.81,418.76) --
	(144.11,418.61) --
	(144.41,418.46) --
	(144.71,418.31) --
	(145.01,418.16) --
	(145.31,418.02) --
	(145.61,417.87) --
	(145.91,417.73) --
	(146.21,417.58) --
	(146.51,417.44) --
	(146.81,417.30) --
	(147.11,417.15) --
	(147.41,417.01) --
	(147.71,416.88) --
	(148.01,416.74) --
	(148.31,416.60) --
	(148.61,416.46) --
	(148.91,416.33) --
	(149.21,416.19) --
	(149.50,416.06) --
	(149.80,415.93) --
	(150.10,415.79) --
	(150.40,415.66) --
	(150.70,415.53) --
	(151.00,415.40) --
	(151.30,415.27) --
	(151.60,415.14) --
	(151.90,415.02) --
	(152.20,414.89) --
	(152.50,414.77) --
	(152.80,414.64) --
	(153.10,414.52) --
	(153.40,414.39) --
	(153.70,414.27) --
	(154.00,414.15) --
	(154.30,414.03) --
	(154.60,413.91) --
	(154.89,413.79) --
	(155.19,413.67) --
	(155.49,413.55) --
	(155.79,413.44) --
	(156.09,413.32) --
	(156.39,413.20) --
	(156.69,413.09) --
	(156.99,412.97) --
	(157.29,412.86) --
	(157.59,412.75) --
	(157.89,412.64) --
	(158.19,412.52) --
	(158.49,412.41) --
	(158.79,412.30) --
	(159.09,412.19) --
	(159.39,412.08) --
	(159.69,411.98) --
	(159.99,411.87) --
	(160.28,411.76) --
	(160.58,411.65) --
	(160.88,411.55) --
	(161.18,411.44) --
	(161.48,411.34) --
	(161.78,411.23) --
	(162.08,411.13) --
	(162.38,411.03) --
	(162.68,410.93) --
	(162.98,410.82) --
	(163.28,410.72) --
	(163.58,410.62) --
	(163.88,410.52) --
	(164.18,410.42) --
	(164.48,410.32) --
	(164.78,410.23) --
	(165.08,410.13) --
	(165.38,410.03) --
	(165.67,409.94) --
	(165.97,409.84) --
	(166.27,409.74) --
	(166.57,409.65) --
	(166.87,409.55) --
	(167.17,409.46) --
	(167.47,409.37) --
	(167.77,409.27) --
	(168.07,409.18) --
	(168.37,409.09) --
	(168.67,409.00) --
	(168.97,408.91) --
	(169.27,408.82) --
	(169.57,408.73) --
	(169.87,408.64) --
	(170.17,408.55) --
	(170.47,408.46) --
	(170.77,408.37) --
	(171.07,408.28) --
	(171.36,408.20) --
	(171.66,408.11) --
	(171.96,408.02) --
	(172.26,407.94) --
	(172.56,407.85) --
	(172.86,407.77) --
	(173.16,407.68) --
	(173.46,407.60) --
	(173.76,407.52) --
	(174.06,407.43) --
	(174.36,407.35) --
	(174.66,407.27) --
	(174.96,407.19) --
	(175.26,407.10) --
	(175.56,407.02) --
	(175.86,406.94) --
	(176.16,406.86) --
	(176.46,406.78) --
	(176.75,406.70) --
	(177.05,406.62) --
	(177.35,406.55) --
	(177.65,406.47) --
	(177.95,406.39) --
	(178.25,406.31) --
	(178.55,406.24) --
	(178.85,406.16) --
	(179.15,406.08) --
	(179.45,406.01) --
	(179.75,405.93) --
	(180.05,405.86) --
	(180.35,405.78) --
	(180.65,405.71) --
	(180.95,405.63) --
	(181.25,405.56) --
	(181.55,405.49) --
	(181.85,405.41) --
	(182.14,405.34) --
	(182.44,405.27) --
	(182.74,405.20) --
	(183.04,405.12) --
	(183.34,405.05) --
	(183.64,404.98) --
	(183.94,404.91) --
	(184.24,404.84) --
	(184.54,404.77) --
	(184.84,404.70) --
	(185.14,404.63) --
	(185.44,404.56) --
	(185.74,404.50) --
	(186.04,404.43) --
	(186.34,404.36) --
	(186.64,404.29) --
	(186.94,404.22) --
	(187.24,404.16) --
	(187.53,404.09) --
	(187.83,404.02) --
	(188.13,403.96) --
	(188.43,403.89) --
	(188.73,403.83) --
	(189.03,403.76) --
	(189.33,403.70) --
	(189.63,403.63) --
	(189.93,403.57) --
	(190.23,403.50) --
	(190.53,403.44) --
	(190.83,403.38) --
	(191.13,403.31) --
	(191.43,403.25) --
	(191.73,403.19) --
	(192.03,403.13) --
	(192.33,403.06) --
	(192.63,403.00) --
	(192.93,402.94) --
	(193.22,402.88) --
	(193.52,402.82) --
	(193.82,402.76) --
	(194.12,402.70) --
	(194.42,402.64) --
	(194.72,402.58) --
	(195.02,402.52) --
	(195.32,402.46) --
	(195.62,402.40) --
	(195.92,402.34) --
	(196.22,402.28) --
	(196.52,402.22) --
	(196.82,402.17) --
	(197.12,402.11) --
	(197.42,402.05) --
	(197.72,401.99) --
	(198.02,401.94) --
	(198.32,401.88) --
	(198.61,401.82) --
	(198.91,401.77) --
	(199.21,401.71) --
	(199.51,401.65) --
	(199.81,401.60) --
	(200.11,401.54) --
	(200.41,401.49) --
	(200.71,401.43) --
	(201.01,401.38) --
	(201.31,401.32) --
	(201.61,401.27) --
	(201.91,401.22) --
	(202.21,401.16) --
	(202.51,401.11) --
	(202.81,401.06) --
	(203.11,401.00) --
	(203.41,400.95) --
	(203.71,400.90) --
	(204.00,400.84) --
	(204.30,400.79) --
	(204.60,400.74) --
	(204.90,400.69) --
	(205.20,400.64) --
	(205.50,400.58) --
	(205.80,400.53) --
	(206.10,400.48) --
	(206.40,400.43) --
	(206.70,400.38) --
	(207.00,400.33) --
	(207.30,400.28) --
	(207.60,400.23) --
	(207.90,400.18) --
	(208.20,400.13) --
	(208.50,400.08) --
	(208.80,400.03) --
	(209.10,399.98) --
	(209.39,399.94) --
	(209.69,399.89) --
	(209.99,399.84) --
	(210.29,399.79) --
	(210.59,399.74) --
	(210.89,399.69) --
	(211.19,399.65) --
	(211.49,399.60) --
	(211.79,399.55) --
	(212.09,399.50) --
	(212.39,399.46) --
	(212.69,399.41) --
	(212.99,399.36) --
	(213.29,399.32) --
	(213.59,399.27) --
	(213.89,399.23) --
	(214.19,399.18) --
	(214.49,399.13) --
	(214.79,399.09) --
	(215.08,399.04) --
	(215.38,399.00) --
	(215.68,398.95) --
	(215.98,398.91) --
	(216.28,398.86) --
	(216.58,398.82) --
	(216.88,398.78) --
	(217.18,398.73) --
	(217.48,398.69) --
	(217.78,398.64) --
	(218.08,398.60) --
	(218.38,398.56) --
	(218.68,398.51) --
	(218.98,398.47) --
	(219.28,398.43) --
	(219.58,398.38) --
	(219.88,398.34) --
	(220.18,398.30) --
	(220.47,398.26) --
	(220.77,398.21) --
	(221.07,398.17) --
	(221.37,398.13) --
	(221.67,398.09) --
	(221.97,398.05) --
	(222.27,398.01) --
	(222.57,397.96) --
	(222.87,397.92) --
	(223.17,397.88) --
	(223.47,397.84) --
	(223.77,397.80) --
	(224.07,397.76) --
	(224.37,397.72) --
	(224.67,397.68) --
	(224.97,397.64) --
	(225.27,397.60) --
	(225.57,397.56) --
	(225.86,397.52) --
	(226.16,397.48) --
	(226.46,397.44) --
	(226.76,397.40) --
	(227.06,397.36) --
	(227.36,397.32) --
	(227.66,397.28) --
	(227.96,397.25) --
	(228.26,397.21) --
	(228.56,397.17) --
	(228.86,397.13) --
	(229.16,397.09) --
	(229.46,397.05) --
	(229.76,397.02) --
	(230.06,396.98) --
	(230.36,396.94) --
	(230.66,396.90) --
	(230.96,396.87) --
	(231.25,396.83) --
	(231.55,396.79) --
	(231.85,396.76) --
	(232.15,396.72) --
	(232.45,396.68) --
	(232.75,396.64) --
	(233.05,396.61) --
	(233.35,396.57) --
	(233.65,396.54) --
	(233.95,396.50) --
	(234.25,396.46) --
	(234.55,396.43) --
	(234.85,396.39) --
	(235.15,396.36) --
	(235.45,396.32) --
	(235.75,396.29) --
	(236.05,396.25) --
	(236.35,396.22) --
	(236.65,396.18) --
	(236.94,396.15) --
	(237.24,396.11) --
	(237.54,396.08) --
	(237.84,396.04) --
	(238.14,396.01) --
	(238.44,395.97) --
	(238.74,395.94) --
	(239.04,395.90) --
	(239.34,395.87) --
	(239.64,395.84) --
	(239.94,395.80) --
	(240.24,395.77) --
	(240.54,395.74) --
	(240.84,395.70) --
	(241.14,395.67) --
	(241.44,395.64) --
	(241.74,395.60) --
	(242.04,395.57) --
	(242.33,395.54) --
	(242.63,395.50) --
	(242.93,395.47) --
	(243.23,395.44) --
	(243.53,395.41) --
	(243.83,395.37) --
	(244.13,395.34) --
	(244.43,395.31) --
	(244.73,395.28) --
	(245.03,395.25) --
	(245.33,395.21) --
	(245.63,395.18) --
	(245.93,395.15) --
	(246.23,395.12) --
	(246.53,395.09) --
	(246.83,395.06) --
	(247.13,395.02) --
	(247.43,394.99) --
	(247.72,394.96) --
	(248.02,394.93) --
	(248.32,394.90) --
	(248.62,394.87) --
	(248.92,394.84) --
	(249.22,394.81) --
	(249.52,394.78) --
	(249.82,394.75) --
	(250.12,394.72) --
	(250.42,394.69) --
	(250.72,394.66) --
	(251.02,394.63) --
	(251.32,394.60) --
	(251.62,394.57) --
	(251.92,394.54) --
	(252.22,394.51) --
	(252.52,394.48) --
	(252.82,394.45) --
	(253.11,394.42) --
	(253.41,394.39) --
	(253.71,394.36) --
	(254.01,394.33) --
	(254.31,394.30) --
	(254.61,394.27) --
	(254.91,394.25) --
	(255.21,394.22) --
	(255.51,394.19) --
	(255.81,394.16) --
	(256.11,394.13) --
	(256.41,394.10) --
	(256.71,394.07) --
	(257.01,394.05) --
	(257.31,394.02) --
	(257.61,393.99) --
	(257.91,393.96) --
	(258.21,393.93) --
	(258.51,393.91) --
	(258.80,393.88) --
	(259.10,393.85) --
	(259.40,393.82) --
	(259.70,393.80) --
	(260.00,393.77) --
	(260.30,393.74) --
	(260.60,393.71) --
	(260.90,393.69) --
	(261.20,393.66) --
	(261.50,393.63) --
	(261.80,393.61) --
	(262.10,393.58) --
	(262.40,393.55) --
	(262.70,393.53) --
	(263.00,393.50) --
	(263.30,393.47) --
	(263.60,393.45) --
	(263.90,393.42) --
	(264.19,393.39) --
	(264.49,393.37) --
	(264.79,393.34) --
	(265.09,393.32) --
	(265.39,393.29) --
	(265.69,393.26) --
	(265.99,393.24) --
	(266.29,393.21) --
	(266.59,393.19) --
	(266.89,393.16) --
	(267.19,393.14) --
	(267.49,393.11) --
	(267.79,393.08) --
	(268.09,393.06) --
	(268.39,393.03) --
	(268.69,393.01) --
	(268.99,392.98) --
	(269.29,392.96) --
	(269.58,392.93) --
	(269.88,392.91) --
	(270.18,392.88) --
	(270.48,392.86) --
	(270.78,392.83) --
	(271.08,392.81) --
	(271.38,392.79) --
	(271.68,392.76) --
	(271.98,392.74) --
	(272.28,392.71) --
	(272.58,392.69) --
	(272.88,392.66) --
	(273.18,392.64) --
	(273.48,392.62) --
	(273.78,392.59) --
	(274.08,392.57) --
	(274.38,392.54) --
	(274.68,392.52) --
	(274.98,392.50) --
	(275.27,392.47) --
	(275.57,392.45) --
	(275.87,392.43) --
	(276.17,392.40) --
	(276.47,392.38) --
	(276.77,392.36) --
	(277.07,392.33) --
	(277.37,392.31) --
	(277.67,392.29) --
	(277.97,392.26) --
	(278.27,392.24) --
	(278.57,392.22) --
	(278.87,392.19) --
	(279.17,392.17) --
	(279.47,392.15) --
	(279.77,392.13) --
	(280.07,392.10) --
	(280.37,392.08) --
	(280.66,392.06) --
	(280.96,392.04) --
	(281.26,392.01) --
	(281.56,391.99) --
	(281.86,391.97) --
	(282.16,391.95) --
	(282.46,391.92) --
	(282.76,391.90) --
	(283.06,391.88) --
	(283.36,391.86) --
	(283.66,391.84) --
	(283.96,391.81) --
	(284.26,391.79) --
	(284.56,391.77) --
	(284.86,391.75) --
	(285.16,391.73) --
	(285.46,391.71) --
	(285.76,391.68) --
	(286.05,391.66) --
	(286.35,391.64) --
	(286.65,391.62) --
	(286.95,391.60) --
	(287.25,391.58) --
	(287.55,391.56) --
	(287.85,391.53) --
	(288.15,391.51) --
	(288.45,391.49) --
	(288.75,391.47) --
	(289.05,391.45) --
	(289.35,391.43) --
	(289.65,391.41) --
	(289.95,391.39) --
	(290.25,391.37) --
	(290.55,391.35) --
	(290.85,391.33) --
	(291.15,391.31) --
	(291.44,391.29) --
	(291.74,391.26) --
	(292.04,391.24) --
	(292.34,391.22) --
	(292.64,391.20) --
	(292.94,391.18) --
	(293.24,391.16) --
	(293.54,391.14) --
	(293.84,391.12) --
	(294.14,391.10) --
	(294.44,391.08) --
	(294.74,391.06) --
	(295.04,391.04) --
	(295.34,391.02) --
	(295.64,391.00) --
	(295.94,390.98) --
	(296.24,390.96) --
	(296.54,390.94) --
	(296.84,390.93) --
	(297.13,390.91) --
	(297.43,390.89) --
	(297.73,390.87) --
	(298.03,390.85) --
	(298.33,390.83) --
	(298.63,390.81) --
	(298.93,390.79) --
	(299.23,390.77) --
	(299.53,390.75) --
	(299.83,390.73) --
	(300.13,390.71) --
	(300.43,390.69) --
	(300.73,390.67) --
	(301.03,390.66) --
	(301.33,390.64) --
	(301.63,390.62) --
	(301.93,390.60) --
	(302.23,390.58) --
	(302.52,390.56) --
	(302.82,390.54) --
	(303.12,390.53) --
	(303.42,390.51) --
	(303.72,390.49) --
	(304.02,390.47) --
	(304.32,390.45) --
	(304.62,390.43) --
	(304.92,390.41) --
	(305.22,390.40) --
	(305.52,390.38) --
	(305.82,390.36) --
	(306.12,390.34) --
	(306.42,390.32) --
	(306.72,390.31) --
	(307.02,390.29) --
	(307.32,390.27) --
	(307.62,390.25) --
	(307.91,390.23) --
	(308.21,390.22) --
	(308.51,390.20) --
	(308.81,390.18) --
	(309.11,390.16) --
	(309.41,390.14) --
	(309.71,390.13) --
	(310.01,390.11) --
	(310.31,390.09) --
	(310.61,390.07) --
	(310.91,390.06) --
	(311.21,390.04) --
	(311.51,390.02) --
	(311.81,390.01) --
	(312.11,389.99) --
	(312.41,389.97) --
	(312.71,389.95) --
	(313.01,389.94) --
	(313.30,389.92) --
	(313.60,389.90) --
	(313.90,389.88) --
	(314.20,389.87) --
	(314.50,389.85) --
	(314.80,389.83) --
	(315.10,389.82) --
	(315.40,389.80) --
	(315.70,389.78) --
	(316.00,389.77) --
	(316.30,389.75) --
	(316.60,389.73) --
	(316.90,389.72) --
	(317.20,389.70) --
	(317.50,389.68) --
	(317.80,389.67) --
	(318.10,389.65) --
	(318.40,389.63) --
	(318.70,389.62) --
	(318.99,389.60) --
	(319.29,389.58) --
	(319.59,389.57) --
	(319.89,389.55) --
	(320.19,389.54) --
	(320.49,389.52) --
	(320.79,389.50) --
	(321.09,389.49) --
	(321.39,389.47) --
	(321.69,389.46) --
	(321.99,389.44) --
	(322.29,389.42) --
	(322.59,389.41) --
	(322.89,389.39) --
	(323.19,389.38) --
	(323.49,389.36) --
	(323.79,389.34) --
	(324.09,389.33) --
	(324.38,389.31) --
	(324.68,389.30) --
	(324.98,389.28) --
	(325.28,389.27) --
	(325.58,389.25) --
	(325.88,389.23) --
	(326.18,389.22) --
	(326.48,389.20) --
	(326.78,389.19) --
	(327.08,389.17) --
	(327.38,389.16) --
	(327.68,389.14) --
	(327.98,389.13) --
	(328.28,389.11) --
	(328.58,389.10) --
	(328.88,389.08) --
	(329.18,389.07) --
	(329.48,389.05) --
	(329.77,389.04) --
	(330.07,389.02) --
	(330.37,389.00) --
	(330.67,388.99) --
	(330.97,388.97) --
	(331.27,388.96) --
	(331.57,388.95) --
	(331.87,388.93) --
	(332.17,388.92) --
	(332.47,388.90) --
	(332.77,388.89) --
	(333.07,388.87) --
	(333.37,388.86) --
	(333.67,388.84) --
	(333.97,388.83) --
	(334.27,388.81) --
	(334.57,388.80) --
	(334.87,388.78) --
	(335.16,388.77) --
	(335.46,388.75) --
	(335.76,388.74) --
	(336.06,388.73) --
	(336.36,388.71) --
	(336.66,388.70) --
	(336.96,388.68) --
	(337.26,388.67) --
	(337.56,388.65) --
	(337.86,388.64) --
	(338.16,388.63) --
	(338.46,388.61) --
	(338.76,388.60) --
	(339.06,388.58) --
	(339.36,388.57) --
	(339.66,388.55) --
	(339.96,388.54) --
	(340.26,388.53) --
	(340.56,388.51) --
	(340.85,388.50) --
	(341.15,388.48) --
	(341.45,388.47) --
	(341.75,388.46) --
	(342.05,388.44) --
	(342.35,388.43) --
	(342.65,388.42) --
	(342.95,388.40) --
	(343.25,388.39) --
	(343.55,388.37) --
	(343.85,388.36) --
	(344.15,388.35) --
	(344.45,388.33) --
	(344.75,388.32) --
	(345.05,388.31) --
	(345.35,388.29) --
	(345.65,388.28) --
	(345.95,388.27) --
	(346.24,388.25) --
	(346.54,388.24) --
	(346.84,388.23) --
	(347.14,388.21) --
	(347.44,388.20) --
	(347.74,388.19) --
	(348.04,388.17) --
	(348.34,388.16) --
	(348.64,388.15) --
	(348.94,388.13) --
	(349.24,388.12) --
	(349.54,388.11) --
	(349.84,388.09) --
	(350.14,388.08) --
	(350.44,388.07) --
	(350.74,388.05) --
	(351.04,388.04) --
	(351.34,388.03) --
	(351.63,388.02) --
	(351.93,388.00) --
	(352.23,387.99) --
	(352.53,387.98) --
	(352.83,387.96) --
	(353.13,387.95) --
	(353.43,387.94) --
	(353.73,387.93) --
	(354.03,387.91) --
	(354.33,387.90) --
	(354.63,387.89) --
	(354.93,387.87) --
	(355.23,387.86) --
	(355.53,387.85) --
	(355.83,387.84) --
	(356.13,387.82) --
	(356.43,387.81) --
	(356.73,387.80) --
	(357.02,387.79) --
	(357.32,387.77) --
	(357.62,387.76) --
	(357.92,387.75) --
	(358.22,387.74) --
	(358.52,387.72) --
	(358.82,387.71) --
	(359.12,387.70) --
	(359.42,387.69) --
	(359.72,387.68) --
	(360.02,387.66) --
	(360.32,387.65);

\path[draw=drawColor,line width= 0.4pt,dash pattern=on 4pt off 4pt ,line join=round,line cap=round] ( 67.07,542.02) --
	( 67.16,540.48) --
	( 67.45,535.17) --
	( 67.75,530.18) --
	( 68.05,525.51) --
	( 68.35,521.10) --
	( 68.65,516.95) --
	( 68.95,513.02) --
	( 69.25,509.31) --
	( 69.55,505.79) --
	( 69.85,502.46) --
	( 70.15,499.29) --
	( 70.45,496.27) --
	( 70.75,493.40) --
	( 71.05,490.66) --
	( 71.35,488.04) --
	( 71.65,485.54) --
	( 71.95,483.15) --
	( 72.25,480.86) --
	( 72.55,478.67) --
	( 72.84,476.56) --
	( 73.14,474.54) --
	( 73.44,472.60) --
	( 73.74,470.73) --
	( 74.04,468.94) --
	( 74.34,467.21) --
	( 74.64,465.54) --
	( 74.94,463.93) --
	( 75.24,462.38) --
	( 75.54,460.88) --
	( 75.84,459.43) --
	( 76.14,458.03) --
	( 76.44,456.68) --
	( 76.74,455.37) --
	( 77.04,454.10) --
	( 77.34,452.87) --
	( 77.64,451.68) --
	( 77.94,450.53) --
	( 78.23,449.41) --
	( 78.53,448.32) --
	( 78.83,447.26) --
	( 79.13,446.24) --
	( 79.43,445.24) --
	( 79.73,444.27) --
	( 80.03,443.33) --
	( 80.33,442.41) --
	( 80.63,441.51) --
	( 80.93,440.64) --
	( 81.23,439.80) --
	( 81.53,438.97) --
	( 81.83,438.17) --
	( 82.13,437.38) --
	( 82.43,436.62) --
	( 82.73,435.87) --
	( 83.03,435.14) --
	( 83.33,434.43) --
	( 83.63,433.74) --
	( 83.92,433.06) --
	( 84.22,432.40) --
	( 84.52,431.75) --
	( 84.82,431.12) --
	( 85.12,430.50) --
	( 85.42,429.89) --
	( 85.72,429.30) --
	( 86.02,428.72) --
	( 86.32,428.15) --
	( 86.62,427.60) --
	( 86.92,427.05) --
	( 87.22,426.52) --
	( 87.52,426.00) --
	( 87.82,425.49) --
	( 88.12,424.99) --
	( 88.42,424.49) --
	( 88.72,424.01) --
	( 89.02,423.54) --
	( 89.31,423.08) --
	( 89.61,422.62) --
	( 89.91,422.18) --
	( 90.21,421.74) --
	( 90.51,421.31) --
	( 90.81,420.89) --
	( 91.11,420.48) --
	( 91.41,420.07) --
	( 91.71,419.67) --
	( 92.01,419.28) --
	( 92.31,418.89) --
	( 92.61,418.52) --
	( 92.91,418.14) --
	( 93.21,417.78) --
	( 93.51,417.42) --
	( 93.81,417.07) --
	( 94.11,416.72) --
	( 94.41,416.38) --
	( 94.70,416.04) --
	( 95.00,415.71) --
	( 95.30,415.38) --
	( 95.60,415.06) --
	( 95.90,414.75) --
	( 96.20,414.44) --
	( 96.50,414.13) --
	( 96.80,413.83) --
	( 97.10,413.54) --
	( 97.40,413.25) --
	( 97.70,412.96) --
	( 98.00,412.68) --
	( 98.30,412.40) --
	( 98.60,412.12) --
	( 98.90,411.85) --
	( 99.20,411.59) --
	( 99.50,411.32) --
	( 99.80,411.07) --
	(100.09,410.81) --
	(100.39,410.56) --
	(100.69,410.31) --
	(100.99,410.07) --
	(101.29,409.83) --
	(101.59,409.59) --
	(101.89,409.35) --
	(102.19,409.12) --
	(102.49,408.89) --
	(102.79,408.67) --
	(103.09,408.45) --
	(103.39,408.23) --
	(103.69,408.01) --
	(103.99,407.80) --
	(104.29,407.59) --
	(104.59,407.38) --
	(104.89,407.17) --
	(105.19,406.97) --
	(105.49,406.77) --
	(105.78,406.57) --
	(106.08,406.38) --
	(106.38,406.19) --
	(106.68,406.00) --
	(106.98,405.81) --
	(107.28,405.62) --
	(107.58,405.44) --
	(107.88,405.26) --
	(108.18,405.08) --
	(108.48,404.90) --
	(108.78,404.73) --
	(109.08,404.55) --
	(109.38,404.38) --
	(109.68,404.21) --
	(109.98,404.05) --
	(110.28,403.88) --
	(110.58,403.72) --
	(110.88,403.56) --
	(111.17,403.40) --
	(111.47,403.24) --
	(111.77,403.09) --
	(112.07,402.93) --
	(112.37,402.78) --
	(112.67,402.63) --
	(112.97,402.48) --
	(113.27,402.33) --
	(113.57,402.19) --
	(113.87,402.04) --
	(114.17,401.90) --
	(114.47,401.76) --
	(114.77,401.62) --
	(115.07,401.48) --
	(115.37,401.34) --
	(115.67,401.21) --
	(115.97,401.07) --
	(116.27,400.94) --
	(116.56,400.81) --
	(116.86,400.68) --
	(117.16,400.55) --
	(117.46,400.42) --
	(117.76,400.30) --
	(118.06,400.17) --
	(118.36,400.05) --
	(118.66,399.93) --
	(118.96,399.81) --
	(119.26,399.69) --
	(119.56,399.57) --
	(119.86,399.45) --
	(120.16,399.33) --
	(120.46,399.22) --
	(120.76,399.11) --
	(121.06,398.99) --
	(121.36,398.88) --
	(121.66,398.77) --
	(121.95,398.66) --
	(122.25,398.55) --
	(122.55,398.44) --
	(122.85,398.34) --
	(123.15,398.23) --
	(123.45,398.12) --
	(123.75,398.02) --
	(124.05,397.92) --
	(124.35,397.82) --
	(124.65,397.71) --
	(124.95,397.61) --
	(125.25,397.51) --
	(125.55,397.42) --
	(125.85,397.32) --
	(126.15,397.22) --
	(126.45,397.13) --
	(126.75,397.03) --
	(127.05,396.94) --
	(127.35,396.84) --
	(127.64,396.75) --
	(127.94,396.66) --
	(128.24,396.57) --
	(128.54,396.48) --
	(128.84,396.39) --
	(129.14,396.30) --
	(129.44,396.21) --
	(129.74,396.12) --
	(130.04,396.04) --
	(130.34,395.95) --
	(130.64,395.87) --
	(130.94,395.78) --
	(131.24,395.70) --
	(131.54,395.61) --
	(131.84,395.53) --
	(132.14,395.45) --
	(132.44,395.37) --
	(132.74,395.29) --
	(133.03,395.21) --
	(133.33,395.13) --
	(133.63,395.05) --
	(133.93,394.97) --
	(134.23,394.90) --
	(134.53,394.82) --
	(134.83,394.74) --
	(135.13,394.67) --
	(135.43,394.59) --
	(135.73,394.52) --
	(136.03,394.45) --
	(136.33,394.37) --
	(136.63,394.30) --
	(136.93,394.23) --
	(137.23,394.16) --
	(137.53,394.08) --
	(137.83,394.01) --
	(138.13,393.94) --
	(138.42,393.88) --
	(138.72,393.81) --
	(139.02,393.74) --
	(139.32,393.67) --
	(139.62,393.60) --
	(139.92,393.54) --
	(140.22,393.47) --
	(140.52,393.40) --
	(140.82,393.34) --
	(141.12,393.27) --
	(141.42,393.21) --
	(141.72,393.14) --
	(142.02,393.08) --
	(142.32,393.02) --
	(142.62,392.96) --
	(142.92,392.89) --
	(143.22,392.83) --
	(143.52,392.77) --
	(143.81,392.71) --
	(144.11,392.65) --
	(144.41,392.59) --
	(144.71,392.53) --
	(145.01,392.47) --
	(145.31,392.41) --
	(145.61,392.35) --
	(145.91,392.29) --
	(146.21,392.24) --
	(146.51,392.18) --
	(146.81,392.12) --
	(147.11,392.07) --
	(147.41,392.01) --
	(147.71,391.95) --
	(148.01,391.90) --
	(148.31,391.84) --
	(148.61,391.79) --
	(148.91,391.73) --
	(149.21,391.68) --
	(149.50,391.63) --
	(149.80,391.57) --
	(150.10,391.52) --
	(150.40,391.47) --
	(150.70,391.42) --
	(151.00,391.36) --
	(151.30,391.31) --
	(151.60,391.26) --
	(151.90,391.21) --
	(152.20,391.16) --
	(152.50,391.11) --
	(152.80,391.06) --
	(153.10,391.01) --
	(153.40,390.96) --
	(153.70,390.91) --
	(154.00,390.86) --
	(154.30,390.82) --
	(154.60,390.77) --
	(154.89,390.72) --
	(155.19,390.67) --
	(155.49,390.63) --
	(155.79,390.58) --
	(156.09,390.53) --
	(156.39,390.49) --
	(156.69,390.44) --
	(156.99,390.39) --
	(157.29,390.35) --
	(157.59,390.30) --
	(157.89,390.26) --
	(158.19,390.21) --
	(158.49,390.17) --
	(158.79,390.12) --
	(159.09,390.08) --
	(159.39,390.04) --
	(159.69,389.99) --
	(159.99,389.95) --
	(160.28,389.91) --
	(160.58,389.87) --
	(160.88,389.82) --
	(161.18,389.78) --
	(161.48,389.74) --
	(161.78,389.70) --
	(162.08,389.66) --
	(162.38,389.62) --
	(162.68,389.57) --
	(162.98,389.53) --
	(163.28,389.49) --
	(163.58,389.45) --
	(163.88,389.41) --
	(164.18,389.37) --
	(164.48,389.33) --
	(164.78,389.29) --
	(165.08,389.26) --
	(165.38,389.22) --
	(165.67,389.18) --
	(165.97,389.14) --
	(166.27,389.10) --
	(166.57,389.06) --
	(166.87,389.03) --
	(167.17,388.99) --
	(167.47,388.95) --
	(167.77,388.91) --
	(168.07,388.88) --
	(168.37,388.84) --
	(168.67,388.80) --
	(168.97,388.77) --
	(169.27,388.73) --
	(169.57,388.69) --
	(169.87,388.66) --
	(170.17,388.62) --
	(170.47,388.59) --
	(170.77,388.55) --
	(171.07,388.52) --
	(171.36,388.48) --
	(171.66,388.45) --
	(171.96,388.41) --
	(172.26,388.38) --
	(172.56,388.34) --
	(172.86,388.31) --
	(173.16,388.28) --
	(173.46,388.24) --
	(173.76,388.21) --
	(174.06,388.18) --
	(174.36,388.14) --
	(174.66,388.11) --
	(174.96,388.08) --
	(175.26,388.05) --
	(175.56,388.01) --
	(175.86,387.98) --
	(176.16,387.95) --
	(176.46,387.92) --
	(176.75,387.89) --
	(177.05,387.85) --
	(177.35,387.82) --
	(177.65,387.79) --
	(177.95,387.76) --
	(178.25,387.73) --
	(178.55,387.70) --
	(178.85,387.67) --
	(179.15,387.64) --
	(179.45,387.61) --
	(179.75,387.58) --
	(180.05,387.55) --
	(180.35,387.52) --
	(180.65,387.49) --
	(180.95,387.46) --
	(181.25,387.43) --
	(181.55,387.40) --
	(181.85,387.37) --
	(182.14,387.34) --
	(182.44,387.31) --
	(182.74,387.28) --
	(183.04,387.25) --
	(183.34,387.23) --
	(183.64,387.20) --
	(183.94,387.17) --
	(184.24,387.14) --
	(184.54,387.11) --
	(184.84,387.08) --
	(185.14,387.06) --
	(185.44,387.03) --
	(185.74,387.00) --
	(186.04,386.97) --
	(186.34,386.95) --
	(186.64,386.92) --
	(186.94,386.89) --
	(187.24,386.87) --
	(187.53,386.84) --
	(187.83,386.81) --
	(188.13,386.79) --
	(188.43,386.76) --
	(188.73,386.73) --
	(189.03,386.71) --
	(189.33,386.68) --
	(189.63,386.66) --
	(189.93,386.63) --
	(190.23,386.61) --
	(190.53,386.58) --
	(190.83,386.55) --
	(191.13,386.53) --
	(191.43,386.50) --
	(191.73,386.48) --
	(192.03,386.45) --
	(192.33,386.43) --
	(192.63,386.40) --
	(192.93,386.38) --
	(193.22,386.36) --
	(193.52,386.33) --
	(193.82,386.31) --
	(194.12,386.28) --
	(194.42,386.26) --
	(194.72,386.23) --
	(195.02,386.21) --
	(195.32,386.19) --
	(195.62,386.16) --
	(195.92,386.14) --
	(196.22,386.12) --
	(196.52,386.09) --
	(196.82,386.07) --
	(197.12,386.05) --
	(197.42,386.02) --
	(197.72,386.00) --
	(198.02,385.98) --
	(198.32,385.96) --
	(198.61,385.93) --
	(198.91,385.91) --
	(199.21,385.89) --
	(199.51,385.87) --
	(199.81,385.84) --
	(200.11,385.82) --
	(200.41,385.80) --
	(200.71,385.78) --
	(201.01,385.76) --
	(201.31,385.73) --
	(201.61,385.71) --
	(201.91,385.69) --
	(202.21,385.67) --
	(202.51,385.65) --
	(202.81,385.63) --
	(203.11,385.60) --
	(203.41,385.58) --
	(203.71,385.56) --
	(204.00,385.54) --
	(204.30,385.52) --
	(204.60,385.50) --
	(204.90,385.48) --
	(205.20,385.46) --
	(205.50,385.44) --
	(205.80,385.42) --
	(206.10,385.40) --
	(206.40,385.38) --
	(206.70,385.36) --
	(207.00,385.34) --
	(207.30,385.32) --
	(207.60,385.30) --
	(207.90,385.28) --
	(208.20,385.26) --
	(208.50,385.24) --
	(208.80,385.22) --
	(209.10,385.20) --
	(209.39,385.18) --
	(209.69,385.16) --
	(209.99,385.14) --
	(210.29,385.12) --
	(210.59,385.10) --
	(210.89,385.08) --
	(211.19,385.06) --
	(211.49,385.04) --
	(211.79,385.02) --
	(212.09,385.01) --
	(212.39,384.99) --
	(212.69,384.97) --
	(212.99,384.95) --
	(213.29,384.93) --
	(213.59,384.91) --
	(213.89,384.89) --
	(214.19,384.88) --
	(214.49,384.86) --
	(214.79,384.84) --
	(215.08,384.82) --
	(215.38,384.80) --
	(215.68,384.79) --
	(215.98,384.77) --
	(216.28,384.75) --
	(216.58,384.73) --
	(216.88,384.71) --
	(217.18,384.70) --
	(217.48,384.68) --
	(217.78,384.66) --
	(218.08,384.64) --
	(218.38,384.63) --
	(218.68,384.61) --
	(218.98,384.59) --
	(219.28,384.57) --
	(219.58,384.56) --
	(219.88,384.54) --
	(220.18,384.52) --
	(220.47,384.51) --
	(220.77,384.49) --
	(221.07,384.47) --
	(221.37,384.46) --
	(221.67,384.44) --
	(221.97,384.42) --
	(222.27,384.41) --
	(222.57,384.39) --
	(222.87,384.37) --
	(223.17,384.36) --
	(223.47,384.34) --
	(223.77,384.32) --
	(224.07,384.31) --
	(224.37,384.29) --
	(224.67,384.28) --
	(224.97,384.26) --
	(225.27,384.24) --
	(225.57,384.23) --
	(225.86,384.21) --
	(226.16,384.20) --
	(226.46,384.18) --
	(226.76,384.16) --
	(227.06,384.15) --
	(227.36,384.13) --
	(227.66,384.12) --
	(227.96,384.10) --
	(228.26,384.09) --
	(228.56,384.07) --
	(228.86,384.06) --
	(229.16,384.04) --
	(229.46,384.03) --
	(229.76,384.01) --
	(230.06,384.00) --
	(230.36,383.98) --
	(230.66,383.97) --
	(230.96,383.95) --
	(231.25,383.94) --
	(231.55,383.92) --
	(231.85,383.91) --
	(232.15,383.89) --
	(232.45,383.88) --
	(232.75,383.86) --
	(233.05,383.85) --
	(233.35,383.83) --
	(233.65,383.82) --
	(233.95,383.80) --
	(234.25,383.79) --
	(234.55,383.78) --
	(234.85,383.76) --
	(235.15,383.75) --
	(235.45,383.73) --
	(235.75,383.72) --
	(236.05,383.70) --
	(236.35,383.69) --
	(236.65,383.68) --
	(236.94,383.66) --
	(237.24,383.65) --
	(237.54,383.63) --
	(237.84,383.62) --
	(238.14,383.61) --
	(238.44,383.59) --
	(238.74,383.58) --
	(239.04,383.57) --
	(239.34,383.55) --
	(239.64,383.54) --
	(239.94,383.53) --
	(240.24,383.51) --
	(240.54,383.50) --
	(240.84,383.48) --
	(241.14,383.47) --
	(241.44,383.46) --
	(241.74,383.44) --
	(242.04,383.43) --
	(242.33,383.42) --
	(242.63,383.41) --
	(242.93,383.39) --
	(243.23,383.38) --
	(243.53,383.37) --
	(243.83,383.35) --
	(244.13,383.34) --
	(244.43,383.33) --
	(244.73,383.31) --
	(245.03,383.30) --
	(245.33,383.29) --
	(245.63,383.28) --
	(245.93,383.26) --
	(246.23,383.25) --
	(246.53,383.24) --
	(246.83,383.23) --
	(247.13,383.21) --
	(247.43,383.20) --
	(247.72,383.19) --
	(248.02,383.18) --
	(248.32,383.16) --
	(248.62,383.15) --
	(248.92,383.14) --
	(249.22,383.13) --
	(249.52,383.12) --
	(249.82,383.10) --
	(250.12,383.09) --
	(250.42,383.08) --
	(250.72,383.07) --
	(251.02,383.05) --
	(251.32,383.04) --
	(251.62,383.03) --
	(251.92,383.02) --
	(252.22,383.01) --
	(252.52,383.00) --
	(252.82,382.98) --
	(253.11,382.97) --
	(253.41,382.96) --
	(253.71,382.95) --
	(254.01,382.94) --
	(254.31,382.93) --
	(254.61,382.91) --
	(254.91,382.90) --
	(255.21,382.89) --
	(255.51,382.88) --
	(255.81,382.87) --
	(256.11,382.86) --
	(256.41,382.85) --
	(256.71,382.83) --
	(257.01,382.82) --
	(257.31,382.81) --
	(257.61,382.80) --
	(257.91,382.79) --
	(258.21,382.78) --
	(258.51,382.77) --
	(258.80,382.76) --
	(259.10,382.74) --
	(259.40,382.73) --
	(259.70,382.72) --
	(260.00,382.71) --
	(260.30,382.70) --
	(260.60,382.69) --
	(260.90,382.68) --
	(261.20,382.67) --
	(261.50,382.66) --
	(261.80,382.65) --
	(262.10,382.64) --
	(262.40,382.63) --
	(262.70,382.61) --
	(263.00,382.60) --
	(263.30,382.59) --
	(263.60,382.58) --
	(263.90,382.57) --
	(264.19,382.56) --
	(264.49,382.55) --
	(264.79,382.54) --
	(265.09,382.53) --
	(265.39,382.52) --
	(265.69,382.51) --
	(265.99,382.50) --
	(266.29,382.49) --
	(266.59,382.48) --
	(266.89,382.47) --
	(267.19,382.46) --
	(267.49,382.45) --
	(267.79,382.44) --
	(268.09,382.43) --
	(268.39,382.42) --
	(268.69,382.41) --
	(268.99,382.40) --
	(269.29,382.39) --
	(269.58,382.38) --
	(269.88,382.37) --
	(270.18,382.36) --
	(270.48,382.35) --
	(270.78,382.34) --
	(271.08,382.33) --
	(271.38,382.32) --
	(271.68,382.31) --
	(271.98,382.30) --
	(272.28,382.29) --
	(272.58,382.28) --
	(272.88,382.27) --
	(273.18,382.26) --
	(273.48,382.25) --
	(273.78,382.24) --
	(274.08,382.23) --
	(274.38,382.22) --
	(274.68,382.21) --
	(274.98,382.20) --
	(275.27,382.19) --
	(275.57,382.18) --
	(275.87,382.17) --
	(276.17,382.16) --
	(276.47,382.16) --
	(276.77,382.15) --
	(277.07,382.14) --
	(277.37,382.13) --
	(277.67,382.12) --
	(277.97,382.11) --
	(278.27,382.10) --
	(278.57,382.09) --
	(278.87,382.08) --
	(279.17,382.07) --
	(279.47,382.06) --
	(279.77,382.05) --
	(280.07,382.05) --
	(280.37,382.04) --
	(280.66,382.03) --
	(280.96,382.02) --
	(281.26,382.01) --
	(281.56,382.00) --
	(281.86,381.99) --
	(282.16,381.98) --
	(282.46,381.97) --
	(282.76,381.96) --
	(283.06,381.96) --
	(283.36,381.95) --
	(283.66,381.94) --
	(283.96,381.93) --
	(284.26,381.92) --
	(284.56,381.91) --
	(284.86,381.90) --
	(285.16,381.89) --
	(285.46,381.89) --
	(285.76,381.88) --
	(286.05,381.87) --
	(286.35,381.86) --
	(286.65,381.85) --
	(286.95,381.84) --
	(287.25,381.83) --
	(287.55,381.83) --
	(287.85,381.82) --
	(288.15,381.81) --
	(288.45,381.80) --
	(288.75,381.79) --
	(289.05,381.78) --
	(289.35,381.78) --
	(289.65,381.77) --
	(289.95,381.76) --
	(290.25,381.75) --
	(290.55,381.74) --
	(290.85,381.73) --
	(291.15,381.73) --
	(291.44,381.72) --
	(291.74,381.71) --
	(292.04,381.70) --
	(292.34,381.69) --
	(292.64,381.69) --
	(292.94,381.68) --
	(293.24,381.67) --
	(293.54,381.66) --
	(293.84,381.65) --
	(294.14,381.65) --
	(294.44,381.64) --
	(294.74,381.63) --
	(295.04,381.62) --
	(295.34,381.61) --
	(295.64,381.61) --
	(295.94,381.60) --
	(296.24,381.59) --
	(296.54,381.58) --
	(296.84,381.57) --
	(297.13,381.57) --
	(297.43,381.56) --
	(297.73,381.55) --
	(298.03,381.54) --
	(298.33,381.54) --
	(298.63,381.53) --
	(298.93,381.52) --
	(299.23,381.51) --
	(299.53,381.50) --
	(299.83,381.50) --
	(300.13,381.49) --
	(300.43,381.48) --
	(300.73,381.47) --
	(301.03,381.47) --
	(301.33,381.46) --
	(301.63,381.45) --
	(301.93,381.44) --
	(302.23,381.44) --
	(302.52,381.43) --
	(302.82,381.42) --
	(303.12,381.41) --
	(303.42,381.41) --
	(303.72,381.40) --
	(304.02,381.39) --
	(304.32,381.38) --
	(304.62,381.38) --
	(304.92,381.37) --
	(305.22,381.36) --
	(305.52,381.36) --
	(305.82,381.35) --
	(306.12,381.34) --
	(306.42,381.33) --
	(306.72,381.33) --
	(307.02,381.32) --
	(307.32,381.31) --
	(307.62,381.30) --
	(307.91,381.30) --
	(308.21,381.29) --
	(308.51,381.28) --
	(308.81,381.28) --
	(309.11,381.27) --
	(309.41,381.26) --
	(309.71,381.25) --
	(310.01,381.25) --
	(310.31,381.24) --
	(310.61,381.23) --
	(310.91,381.23) --
	(311.21,381.22) --
	(311.51,381.21) --
	(311.81,381.21) --
	(312.11,381.20) --
	(312.41,381.19) --
	(312.71,381.19) --
	(313.01,381.18) --
	(313.30,381.17) --
	(313.60,381.16) --
	(313.90,381.16) --
	(314.20,381.15) --
	(314.50,381.14) --
	(314.80,381.14) --
	(315.10,381.13) --
	(315.40,381.12) --
	(315.70,381.12) --
	(316.00,381.11) --
	(316.30,381.10) --
	(316.60,381.10) --
	(316.90,381.09) --
	(317.20,381.08) --
	(317.50,381.08) --
	(317.80,381.07) --
	(318.10,381.06) --
	(318.40,381.06) --
	(318.70,381.05) --
	(318.99,381.04) --
	(319.29,381.04) --
	(319.59,381.03) --
	(319.89,381.02) --
	(320.19,381.02) --
	(320.49,381.01) --
	(320.79,381.01) --
	(321.09,381.00) --
	(321.39,380.99) --
	(321.69,380.99) --
	(321.99,380.98) --
	(322.29,380.97) --
	(322.59,380.97) --
	(322.89,380.96) --
	(323.19,380.95) --
	(323.49,380.95) --
	(323.79,380.94) --
	(324.09,380.94) --
	(324.38,380.93) --
	(324.68,380.92) --
	(324.98,380.92) --
	(325.28,380.91) --
	(325.58,380.90) --
	(325.88,380.90) --
	(326.18,380.89) --
	(326.48,380.89) --
	(326.78,380.88) --
	(327.08,380.87) --
	(327.38,380.87) --
	(327.68,380.86) --
	(327.98,380.85) --
	(328.28,380.85) --
	(328.58,380.84) --
	(328.88,380.84) --
	(329.18,380.83) --
	(329.48,380.82) --
	(329.77,380.82) --
	(330.07,380.81) --
	(330.37,380.81) --
	(330.67,380.80) --
	(330.97,380.79) --
	(331.27,380.79) --
	(331.57,380.78) --
	(331.87,380.78) --
	(332.17,380.77) --
	(332.47,380.76) --
	(332.77,380.76) --
	(333.07,380.75) --
	(333.37,380.75) --
	(333.67,380.74) --
	(333.97,380.73) --
	(334.27,380.73) --
	(334.57,380.72) --
	(334.87,380.72) --
	(335.16,380.71) --
	(335.46,380.71) --
	(335.76,380.70) --
	(336.06,380.69) --
	(336.36,380.69) --
	(336.66,380.68) --
	(336.96,380.68) --
	(337.26,380.67) --
	(337.56,380.67) --
	(337.86,380.66) --
	(338.16,380.65) --
	(338.46,380.65) --
	(338.76,380.64) --
	(339.06,380.64) --
	(339.36,380.63) --
	(339.66,380.63) --
	(339.96,380.62) --
	(340.26,380.61) --
	(340.56,380.61) --
	(340.85,380.60) --
	(341.15,380.60) --
	(341.45,380.59) --
	(341.75,380.59) --
	(342.05,380.58) --
	(342.35,380.58) --
	(342.65,380.57) --
	(342.95,380.56) --
	(343.25,380.56) --
	(343.55,380.55) --
	(343.85,380.55) --
	(344.15,380.54) --
	(344.45,380.54) --
	(344.75,380.53) --
	(345.05,380.53) --
	(345.35,380.52) --
	(345.65,380.52) --
	(345.95,380.51) --
	(346.24,380.50) --
	(346.54,380.50) --
	(346.84,380.49) --
	(347.14,380.49) --
	(347.44,380.48) --
	(347.74,380.48) --
	(348.04,380.47) --
	(348.34,380.47) --
	(348.64,380.46) --
	(348.94,380.46) --
	(349.24,380.45) --
	(349.54,380.45) --
	(349.84,380.44) --
	(350.14,380.44) --
	(350.44,380.43) --
	(350.74,380.43) --
	(351.04,380.42) --
	(351.34,380.42) --
	(351.63,380.41) --
	(351.93,380.40) --
	(352.23,380.40) --
	(352.53,380.39) --
	(352.83,380.39) --
	(353.13,380.38) --
	(353.43,380.38) --
	(353.73,380.37) --
	(354.03,380.37) --
	(354.33,380.36) --
	(354.63,380.36) --
	(354.93,380.35) --
	(355.23,380.35) --
	(355.53,380.34) --
	(355.83,380.34) --
	(356.13,380.33) --
	(356.43,380.33) --
	(356.73,380.32) --
	(357.02,380.32) --
	(357.32,380.31) --
	(357.62,380.31) --
	(357.92,380.30) --
	(358.22,380.30) --
	(358.52,380.29) --
	(358.82,380.29) --
	(359.12,380.28) --
	(359.42,380.28) --
	(359.72,380.27) --
	(360.02,380.27) --
	(360.32,380.26);
\definecolor{drawColor}{RGB}{190,190,190}

\path[draw=drawColor,line width= 0.4pt,line join=round,line cap=round] ( 49.20,375.34) -- (372.28,375.34);
\definecolor{drawColor}{RGB}{0,0,0}

\path[draw=drawColor,line width= 0.4pt,line join=round,line cap=round] (275.67,477.59) -- (286.47,477.59);

\path[draw=drawColor,line width= 0.4pt,dash pattern=on 4pt off 4pt ,line join=round,line cap=round] (275.67,463.19) -- (286.47,463.19);
\definecolor{drawColor}{RGB}{190,190,190}

\path[draw=drawColor,line width= 0.4pt,line join=round,line cap=round] (275.67,448.79) -- (286.47,448.79);
\definecolor{drawColor}{RGB}{0,0,0}

\node[text=drawColor,anchor=base west,inner sep=0pt, outer sep=0pt, scale=  0.60] at (291.87,475.53) {MIPS};

\node[text=drawColor,anchor=base west,inner sep=0pt, outer sep=0pt, scale=  0.60] at (291.87,461.13) {$\text{MIPS}^\text{max}\text{ und }\text{MIPS}^\text{min}$};

\node[text=drawColor,anchor=base west,inner sep=0pt, outer sep=0pt, scale=  0.60] at (291.87,446.73) {$\lim\limits_{t_\text{max}\rightarrow\infty} \text{MIPS}$};
\end{scope}
\begin{scope}
\path[clip] ( 49.20, 61.20) rectangle (372.28,221.81);
\definecolor{drawColor}{RGB}{190,190,190}

\path[draw=drawColor,line width= 0.4pt,line join=round,line cap=round] ( 49.20,104.33) -- (372.28,104.33);
\definecolor{drawColor}{RGB}{0,0,0}

\path[draw=drawColor,line width= 0.4pt,line join=round,line cap=round] ( 61.52,542.02) --
	( 61.77,509.26) --
	( 62.06,475.50) --
	( 62.36,446.53) --
	( 62.66,421.39) --
	( 62.96,399.38) --
	( 63.26,379.95) --
	( 63.56,362.67) --
	( 63.86,347.19) --
	( 64.16,333.26) --
	( 64.46,320.65) --
	( 64.76,309.18) --
	( 65.06,298.70) --
	( 65.36,289.09) --
	( 65.66,280.25) --
	( 65.96,272.09) --
	( 66.26,264.53) --
	( 66.56,257.50) --
	( 66.86,250.96) --
	( 67.16,244.86) --
	( 67.45,239.15) --
	( 67.75,233.79) --
	( 68.05,228.76) --
	( 68.35,224.02) --
	( 68.65,219.55) --
	( 68.95,215.33) --
	( 69.25,211.34) --
	( 69.55,207.56) --
	( 69.85,203.97) --
	( 70.15,200.56) --
	( 70.45,197.31) --
	( 70.75,194.22) --
	( 71.05,191.28) --
	( 71.35,188.46) --
	( 71.65,185.78) --
	( 71.95,183.21) --
	( 72.25,180.74) --
	( 72.55,178.38) --
	( 72.84,176.12) --
	( 73.14,173.95) --
	( 73.44,171.86) --
	( 73.74,169.85) --
	( 74.04,167.92) --
	( 74.34,166.06) --
	( 74.64,164.26) --
	( 74.94,162.53) --
	( 75.24,160.86) --
	( 75.54,159.25) --
	( 75.84,157.70) --
	( 76.14,156.19) --
	( 76.44,154.74) --
	( 76.74,153.33) --
	( 77.04,151.96) --
	( 77.34,150.64) --
	( 77.64,149.36) --
	( 77.94,148.12) --
	( 78.23,146.91) --
	( 78.53,145.74) --
	( 78.83,144.61) --
	( 79.13,143.50) --
	( 79.43,142.43) --
	( 79.73,141.39) --
	( 80.03,140.37) --
	( 80.33,139.39) --
	( 80.63,138.43) --
	( 80.93,137.49) --
	( 81.23,136.58) --
	( 81.53,135.69) --
	( 81.83,134.83) --
	( 82.13,133.98) --
	( 82.43,133.16) --
	( 82.73,132.36) --
	( 83.03,131.57) --
	( 83.33,130.81) --
	( 83.63,130.06) --
	( 83.92,129.33) --
	( 84.22,128.62) --
	( 84.52,127.92) --
	( 84.82,127.24) --
	( 85.12,126.58) --
	( 85.42,125.92) --
	( 85.72,125.29) --
	( 86.02,124.66) --
	( 86.32,124.05) --
	( 86.62,123.46) --
	( 86.92,122.87) --
	( 87.22,122.30) --
	( 87.52,121.74) --
	( 87.82,121.19) --
	( 88.12,120.65) --
	( 88.42,120.12) --
	( 88.72,119.60) --
	( 89.02,119.09) --
	( 89.31,118.60) --
	( 89.61,118.11) --
	( 89.91,117.63) --
	( 90.21,117.16) --
	( 90.51,116.70) --
	( 90.81,116.24) --
	( 91.11,115.80) --
	( 91.41,115.36) --
	( 91.71,114.93) --
	( 92.01,114.51) --
	( 92.31,114.10) --
	( 92.61,113.69) --
	( 92.91,113.29) --
	( 93.21,112.90) --
	( 93.51,112.51) --
	( 93.81,112.13) --
	( 94.11,111.76) --
	( 94.41,111.39) --
	( 94.70,111.03) --
	( 95.00,110.67) --
	( 95.30,110.32) --
	( 95.60,109.98) --
	( 95.90,109.64) --
	( 96.20,109.31) --
	( 96.50,108.98) --
	( 96.80,108.65) --
	( 97.10,108.34) --
	( 97.40,108.02) --
	( 97.70,107.71) --
	( 98.00,107.41) --
	( 98.30,107.11) --
	( 98.60,106.82) --
	( 98.90,106.52) --
	( 99.20,106.24) --
	( 99.50,105.96) --
	( 99.80,105.68) --
	(100.09,105.40) --
	(100.39,105.13) --
	(100.69,104.87) --
	(100.99,104.60) --
	(101.29,104.34) --
	(101.59,104.09) --
	(101.89,103.84) --
	(102.19,103.59) --
	(102.49,103.34) --
	(102.79,103.10) --
	(103.09,102.86) --
	(103.39,102.63) --
	(103.69,102.39) --
	(103.99,102.16) --
	(104.29,101.94) --
	(104.59,101.71) --
	(104.89,101.49) --
	(105.19,101.27) --
	(105.49,101.06) --
	(105.78,100.85) --
	(106.08,100.64) --
	(106.38,100.43) --
	(106.68,100.23) --
	(106.98,100.02) --
	(107.28, 99.82) --
	(107.58, 99.63) --
	(107.88, 99.43) --
	(108.18, 99.24) --
	(108.48, 99.05) --
	(108.78, 98.86) --
	(109.08, 98.67) --
	(109.38, 98.49) --
	(109.68, 98.31) --
	(109.98, 98.13) --
	(110.28, 97.95) --
	(110.58, 97.78) --
	(110.88, 97.60) --
	(111.17, 97.43) --
	(111.47, 97.26) --
	(111.77, 97.10) --
	(112.07, 96.93) --
	(112.37, 96.77) --
	(112.67, 96.60) --
	(112.97, 96.44) --
	(113.27, 96.29) --
	(113.57, 96.13) --
	(113.87, 95.97) --
	(114.17, 95.82) --
	(114.47, 95.67) --
	(114.77, 95.52) --
	(115.07, 95.37) --
	(115.37, 95.22) --
	(115.67, 95.08) --
	(115.97, 94.93) --
	(116.27, 94.79) --
	(116.56, 94.65) --
	(116.86, 94.51) --
	(117.16, 94.37) --
	(117.46, 94.23) --
	(117.76, 94.10) --
	(118.06, 93.96) --
	(118.36, 93.83) --
	(118.66,130.72) --
	(118.96,130.41) --
	(119.26,130.10) --
	(119.56,129.79) --
	(119.86,129.49) --
	(120.16,129.19) --
	(120.46,128.89) --
	(120.76,128.60) --
	(121.06,128.31) --
	(121.36,128.02) --
	(121.66,127.73) --
	(121.95,127.45) --
	(122.25,127.16) --
	(122.55,126.89) --
	(122.85,126.61) --
	(123.15,126.34) --
	(123.45,126.07) --
	(123.75,125.80) --
	(124.05,125.53) --
	(124.35,125.27) --
	(124.65,125.01) --
	(124.95,124.75) --
	(125.25,124.49) --
	(125.55,124.24) --
	(125.85,123.99) --
	(126.15,123.74) --
	(126.45,123.49) --
	(126.75,123.24) --
	(127.05,123.00) --
	(127.35,122.76) --
	(127.64,122.52) --
	(127.94,122.28) --
	(128.24,122.05) --
	(128.54,121.81) --
	(128.84,121.58) --
	(129.14,121.35) --
	(129.44,121.13) --
	(129.74,120.90) --
	(130.04,120.68) --
	(130.34,120.46) --
	(130.64,120.24) --
	(130.94,120.02) --
	(131.24,119.80) --
	(131.54,119.59) --
	(131.84,119.37) --
	(132.14,119.16) --
	(132.44,118.95) --
	(132.74,118.75) --
	(133.03,118.54) --
	(133.33,118.34) --
	(133.63,118.13) --
	(133.93,117.93) --
	(134.23,117.73) --
	(134.53,117.54) --
	(134.83,117.34) --
	(135.13,117.14) --
	(135.43,116.95) --
	(135.73,116.76) --
	(136.03,116.57) --
	(136.33,116.38) --
	(136.63,116.19) --
	(136.93,116.01) --
	(137.23,115.82) --
	(137.53,115.64) --
	(137.83,115.46) --
	(138.13,115.28) --
	(138.42,115.10) --
	(138.72,114.92) --
	(139.02,114.74) --
	(139.32,114.57) --
	(139.62,114.39) --
	(139.92,114.22) --
	(140.22,114.05) --
	(140.52,113.88) --
	(140.82,113.71) --
	(141.12,113.54) --
	(141.42,113.38) --
	(141.72,113.21) --
	(142.02,113.05) --
	(142.32,112.88) --
	(142.62,112.72) --
	(142.92,112.56) --
	(143.22,112.40) --
	(143.52,112.24) --
	(143.81,112.09) --
	(144.11,111.93) --
	(144.41,111.78) --
	(144.71,111.62) --
	(145.01,111.47) --
	(145.31,111.32) --
	(145.61,111.17) --
	(145.91,111.02) --
	(146.21,110.87) --
	(146.51,110.72) --
	(146.81,110.57) --
	(147.11,110.43) --
	(147.41,110.28) --
	(147.71,110.14) --
	(148.01,110.00) --
	(148.31,109.85) --
	(148.61,109.71) --
	(148.91,109.57) --
	(149.21,109.43) --
	(149.50,109.30) --
	(149.80,109.16) --
	(150.10,109.02) --
	(150.40,108.89) --
	(150.70,108.75) --
	(151.00,108.62) --
	(151.30,108.48) --
	(151.60,108.35) --
	(151.90,108.22) --
	(152.20,108.09) --
	(152.50,107.96) --
	(152.80,107.83) --
	(153.10,107.70) --
	(153.40,107.58) --
	(153.70,107.45) --
	(154.00,107.33) --
	(154.30,107.20) --
	(154.60,107.08) --
	(154.89,106.95) --
	(155.19,106.83) --
	(155.49,106.71) --
	(155.79,106.59) --
	(156.09,106.47) --
	(156.39,106.35) --
	(156.69,106.23) --
	(156.99,106.11) --
	(157.29,105.99) --
	(157.59,105.88) --
	(157.89,105.76) --
	(158.19,105.65) --
	(158.49,105.53) --
	(158.79,105.42) --
	(159.09,105.30) --
	(159.39,105.19) --
	(159.69,105.08) --
	(159.99,104.97) --
	(160.28,104.86) --
	(160.58,104.75) --
	(160.88,104.64) --
	(161.18,104.53) --
	(161.48,104.42) --
	(161.78,104.31) --
	(162.08,104.21) --
	(162.38,104.10) --
	(162.68,104.00) --
	(162.98,103.89) --
	(163.28,103.79) --
	(163.58,103.68) --
	(163.88,103.58) --
	(164.18,103.48) --
	(164.48,103.38) --
	(164.78,103.27) --
	(165.08,103.17) --
	(165.38,103.07) --
	(165.67,102.97) --
	(165.97,102.87) --
	(166.27,102.78) --
	(166.57,102.68) --
	(166.87,102.58) --
	(167.17,102.48) --
	(167.47,102.39) --
	(167.77,102.29) --
	(168.07,102.19) --
	(168.37,102.10) --
	(168.67,102.01) --
	(168.97,101.91) --
	(169.27,101.82) --
	(169.57,101.73) --
	(169.87,101.63) --
	(170.17,101.54) --
	(170.47,101.45) --
	(170.77,101.36) --
	(171.07,101.27) --
	(171.36,101.18) --
	(171.66,101.09) --
	(171.96,101.00) --
	(172.26,100.91) --
	(172.56,100.82) --
	(172.86,100.74) --
	(173.16,100.65) --
	(173.46,100.56) --
	(173.76,100.48) --
	(174.06,100.39) --
	(174.36,100.30) --
	(174.66,100.22) --
	(174.96,100.13) --
	(175.26,100.05) --
	(175.56, 99.97) --
	(175.86, 99.88) --
	(176.16, 99.80) --
	(176.46, 99.72) --
	(176.75, 99.64) --
	(177.05, 99.56) --
	(177.35, 99.47) --
	(177.65, 99.39) --
	(177.95, 99.31) --
	(178.25, 99.23) --
	(178.55, 99.15) --
	(178.85, 99.07) --
	(179.15,117.51) --
	(179.45,117.39) --
	(179.75,117.26) --
	(180.05,117.14) --
	(180.35,117.02) --
	(180.65,116.90) --
	(180.95,116.78) --
	(181.25,116.66) --
	(181.55,116.54) --
	(181.85,116.42) --
	(182.14,116.30) --
	(182.44,116.18) --
	(182.74,116.06) --
	(183.04,115.94) --
	(183.34,115.83) --
	(183.64,115.71) --
	(183.94,115.60) --
	(184.24,115.48) --
	(184.54,115.37) --
	(184.84,115.25) --
	(185.14,115.14) --
	(185.44,115.03) --
	(185.74,114.92) --
	(186.04,114.80) --
	(186.34,114.69) --
	(186.64,114.58) --
	(186.94,114.47) --
	(187.24,114.36) --
	(187.53,114.25) --
	(187.83,114.15) --
	(188.13,114.04) --
	(188.43,113.93) --
	(188.73,113.82) --
	(189.03,113.72) --
	(189.33,113.61) --
	(189.63,113.51) --
	(189.93,113.40) --
	(190.23,113.30) --
	(190.53,113.19) --
	(190.83,113.09) --
	(191.13,112.98) --
	(191.43,112.88) --
	(191.73,112.78) --
	(192.03,112.68) --
	(192.33,112.58) --
	(192.63,112.48) --
	(192.93,112.37) --
	(193.22,112.27) --
	(193.52,112.18) --
	(193.82,112.08) --
	(194.12,111.98) --
	(194.42,111.88) --
	(194.72,111.78) --
	(195.02,111.68) --
	(195.32,111.59) --
	(195.62,111.49) --
	(195.92,111.39) --
	(196.22,111.30) --
	(196.52,111.20) --
	(196.82,111.11) --
	(197.12,111.01) --
	(197.42,110.92) --
	(197.72,110.83) --
	(198.02,110.73) --
	(198.32,110.64) --
	(198.61,110.55) --
	(198.91,110.46) --
	(199.21,110.36) --
	(199.51,110.27) --
	(199.81,110.18) --
	(200.11,110.09) --
	(200.41,110.00) --
	(200.71,109.91) --
	(201.01,109.82) --
	(201.31,109.73) --
	(201.61,109.64) --
	(201.91,109.56) --
	(202.21,109.47) --
	(202.51,109.38) --
	(202.81,109.29) --
	(203.11,109.21) --
	(203.41,109.12) --
	(203.71,109.03) --
	(204.00,108.95) --
	(204.30,108.86) --
	(204.60,108.78) --
	(204.90,108.69) --
	(205.20,108.61) --
	(205.50,108.52) --
	(205.80,108.44) --
	(206.10,108.36) --
	(206.40,108.27) --
	(206.70,108.19) --
	(207.00,108.11) --
	(207.30,108.03) --
	(207.60,107.95) --
	(207.90,107.86) --
	(208.20,107.78) --
	(208.50,107.70) --
	(208.80,107.62) --
	(209.10,107.54) --
	(209.39,107.46) --
	(209.69,107.38) --
	(209.99,107.30) --
	(210.29,107.22) --
	(210.59,107.15) --
	(210.89,107.07) --
	(211.19,106.99) --
	(211.49,106.91) --
	(211.79,106.83) --
	(212.09,106.76) --
	(212.39,106.68) --
	(212.69,106.60) --
	(212.99,106.53) --
	(213.29,106.45) --
	(213.59,106.38) --
	(213.89,106.30) --
	(214.19,106.23) --
	(214.49,106.15) --
	(214.79,106.08) --
	(215.08,106.00) --
	(215.38,105.93) --
	(215.68,105.86) --
	(215.98,105.78) --
	(216.28,105.71) --
	(216.58,105.64) --
	(216.88,105.57) --
	(217.18,105.49) --
	(217.48,105.42) --
	(217.78,105.35) --
	(218.08,105.28) --
	(218.38,105.21) --
	(218.68,105.14) --
	(218.98,105.07) --
	(219.28,105.00) --
	(219.58,104.93) --
	(219.88,104.86) --
	(220.18,104.79) --
	(220.47,104.72) --
	(220.77,104.65) --
	(221.07,104.58) --
	(221.37,104.51) --
	(221.67,104.44) --
	(221.97,104.37) --
	(222.27,104.31) --
	(222.57,104.24) --
	(222.87,104.17) --
	(223.17,104.11) --
	(223.47,104.04) --
	(223.77,103.97) --
	(224.07,103.91) --
	(224.37,103.84) --
	(224.67,103.77) --
	(224.97,103.71) --
	(225.27,103.64) --
	(225.57,103.58) --
	(225.86,103.51) --
	(226.16,103.45) --
	(226.46,103.38) --
	(226.76,103.32) --
	(227.06,103.26) --
	(227.36,103.19) --
	(227.66,103.13) --
	(227.96,103.07) --
	(228.26,103.00) --
	(228.56,102.94) --
	(228.86,102.88) --
	(229.16,102.81) --
	(229.46,102.75) --
	(229.76,102.69) --
	(230.06,102.63) --
	(230.36,102.57) --
	(230.66,102.51) --
	(230.96,102.45) --
	(231.25,102.38) --
	(231.55,102.32) --
	(231.85,102.26) --
	(232.15,102.20) --
	(232.45,102.14) --
	(232.75,102.08) --
	(233.05,102.02) --
	(233.35,101.96) --
	(233.65,101.90) --
	(233.95,101.85) --
	(234.25,101.79) --
	(234.55,101.73) --
	(234.85,101.67) --
	(235.15,101.61) --
	(235.45,101.55) --
	(235.75,101.50) --
	(236.05,101.44) --
	(236.35,101.38) --
	(236.65,101.32) --
	(236.94,101.27) --
	(237.24,101.21) --
	(237.54,101.15) --
	(237.84,101.10) --
	(238.14,101.04) --
	(238.44,100.98) --
	(238.74,100.93) --
	(239.04,100.87) --
	(239.34,100.82) --
	(239.64,113.11) --
	(239.94,113.03) --
	(240.24,112.96) --
	(240.54,112.88) --
	(240.84,112.81) --
	(241.14,112.73) --
	(241.44,112.66) --
	(241.74,112.58) --
	(242.04,112.51) --
	(242.33,112.44) --
	(242.63,112.36) --
	(242.93,112.29) --
	(243.23,112.22) --
	(243.53,112.14) --
	(243.83,112.07) --
	(244.13,112.00) --
	(244.43,111.93) --
	(244.73,111.86) --
	(245.03,111.78) --
	(245.33,111.71) --
	(245.63,111.64) --
	(245.93,111.57) --
	(246.23,111.50) --
	(246.53,111.43) --
	(246.83,111.36) --
	(247.13,111.29) --
	(247.43,111.22) --
	(247.72,111.15) --
	(248.02,111.08) --
	(248.32,111.01) --
	(248.62,110.94) --
	(248.92,110.88) --
	(249.22,110.81) --
	(249.52,110.74) --
	(249.82,110.67) --
	(250.12,110.60) --
	(250.42,110.54) --
	(250.72,110.47) --
	(251.02,110.40) --
	(251.32,110.33) --
	(251.62,110.27) --
	(251.92,110.20) --
	(252.22,110.13) --
	(252.52,110.07) --
	(252.82,110.00) --
	(253.11,109.94) --
	(253.41,109.87) --
	(253.71,109.81) --
	(254.01,109.74) --
	(254.31,109.68) --
	(254.61,109.61) --
	(254.91,109.55) --
	(255.21,109.48) --
	(255.51,109.42) --
	(255.81,109.36) --
	(256.11,109.29) --
	(256.41,109.23) --
	(256.71,109.16) --
	(257.01,109.10) --
	(257.31,109.04) --
	(257.61,108.98) --
	(257.91,108.91) --
	(258.21,108.85) --
	(258.51,108.79) --
	(258.80,108.73) --
	(259.10,108.67) --
	(259.40,108.60) --
	(259.70,108.54) --
	(260.00,108.48) --
	(260.30,108.42) --
	(260.60,108.36) --
	(260.90,108.30) --
	(261.20,108.24) --
	(261.50,108.18) --
	(261.80,108.12) --
	(262.10,108.06) --
	(262.40,108.00) --
	(262.70,107.94) --
	(263.00,107.88) --
	(263.30,107.82) --
	(263.60,107.76) --
	(263.90,107.70) --
	(264.19,107.64) --
	(264.49,107.58) --
	(264.79,107.52) --
	(265.09,107.47) --
	(265.39,107.41) --
	(265.69,107.35) --
	(265.99,107.29) --
	(266.29,107.24) --
	(266.59,107.18) --
	(266.89,107.12) --
	(267.19,107.06) --
	(267.49,107.01) --
	(267.79,106.95) --
	(268.09,106.89) --
	(268.39,106.84) --
	(268.69,106.78) --
	(268.99,106.72) --
	(269.29,106.67) --
	(269.58,106.61) --
	(269.88,106.56) --
	(270.18,106.50) --
	(270.48,106.45) --
	(270.78,106.39) --
	(271.08,106.34) --
	(271.38,106.28) --
	(271.68,106.23) --
	(271.98,106.17) --
	(272.28,106.12) --
	(272.58,106.06) --
	(272.88,106.01) --
	(273.18,105.95) --
	(273.48,105.90) --
	(273.78,105.85) --
	(274.08,105.79) --
	(274.38,105.74) --
	(274.68,105.69) --
	(274.98,105.63) --
	(275.27,105.58) --
	(275.57,105.53) --
	(275.87,105.48) --
	(276.17,105.42) --
	(276.47,105.37) --
	(276.77,105.32) --
	(277.07,105.27) --
	(277.37,105.21) --
	(277.67,105.16) --
	(277.97,105.11) --
	(278.27,105.06) --
	(278.57,105.01) --
	(278.87,104.96) --
	(279.17,104.91) --
	(279.47,104.86) --
	(279.77,104.80) --
	(280.07,104.75) --
	(280.37,104.70) --
	(280.66,104.65) --
	(280.96,104.60) --
	(281.26,104.55) --
	(281.56,104.50) --
	(281.86,104.45) --
	(282.16,104.40) --
	(282.46,104.35) --
	(282.76,104.30) --
	(283.06,104.25) --
	(283.36,104.21) --
	(283.66,104.16) --
	(283.96,104.11) --
	(284.26,104.06) --
	(284.56,104.01) --
	(284.86,103.96) --
	(285.16,103.91) --
	(285.46,103.86) --
	(285.76,103.82) --
	(286.05,103.77) --
	(286.35,103.72) --
	(286.65,103.67) --
	(286.95,103.62) --
	(287.25,103.58) --
	(287.55,103.53) --
	(287.85,103.48) --
	(288.15,103.44) --
	(288.45,103.39) --
	(288.75,103.34) --
	(289.05,103.29) --
	(289.35,103.25) --
	(289.65,103.20) --
	(289.95,103.16) --
	(290.25,103.11) --
	(290.55,103.06) --
	(290.85,103.02) --
	(291.15,102.97) --
	(291.44,102.92) --
	(291.74,102.88) --
	(292.04,102.83) --
	(292.34,102.79) --
	(292.64,102.74) --
	(292.94,102.70) --
	(293.24,102.65) --
	(293.54,102.61) --
	(293.84,102.56) --
	(294.14,102.52) --
	(294.44,102.47) --
	(294.74,102.43) --
	(295.04,102.38) --
	(295.34,102.34) --
	(295.64,102.29) --
	(295.94,102.25) --
	(296.24,102.21) --
	(296.54,102.16) --
	(296.84,102.12) --
	(297.13,102.08) --
	(297.43,102.03) --
	(297.73,101.99) --
	(298.03,101.95) --
	(298.33,101.90) --
	(298.63,101.86) --
	(298.93,101.82) --
	(299.23,101.77) --
	(299.53,101.73) --
	(299.83,101.69) --
	(300.13,110.90) --
	(300.43,110.85) --
	(300.73,110.80) --
	(301.03,110.74) --
	(301.33,110.69) --
	(301.63,110.63) --
	(301.93,110.58) --
	(302.23,110.53) --
	(302.52,110.48) --
	(302.82,110.42) --
	(303.12,110.37) --
	(303.42,110.32) --
	(303.72,110.26) --
	(304.02,110.21) --
	(304.32,110.16) --
	(304.62,110.11) --
	(304.92,110.06) --
	(305.22,110.00) --
	(305.52,109.95) --
	(305.82,109.90) --
	(306.12,109.85) --
	(306.42,109.80) --
	(306.72,109.75) --
	(307.02,109.70) --
	(307.32,109.64) --
	(307.62,109.59) --
	(307.91,109.54) --
	(308.21,109.49) --
	(308.51,109.44) --
	(308.81,109.39) --
	(309.11,109.34) --
	(309.41,109.29) --
	(309.71,109.24) --
	(310.01,109.19) --
	(310.31,109.14) --
	(310.61,109.09) --
	(310.91,109.04) --
	(311.21,108.99) --
	(311.51,108.94) --
	(311.81,108.89) --
	(312.11,108.84) --
	(312.41,108.80) --
	(312.71,108.75) --
	(313.01,108.70) --
	(313.30,108.65) --
	(313.60,108.60) --
	(313.90,108.55) --
	(314.20,108.50) --
	(314.50,108.46) --
	(314.80,108.41) --
	(315.10,108.36) --
	(315.40,108.31) --
	(315.70,108.26) --
	(316.00,108.22) --
	(316.30,108.17) --
	(316.60,108.12) --
	(316.90,108.07) --
	(317.20,108.03) --
	(317.50,107.98) --
	(317.80,107.93) --
	(318.10,107.89) --
	(318.40,107.84) --
	(318.70,107.79) --
	(318.99,107.75) --
	(319.29,107.70) --
	(319.59,107.65) --
	(319.89,107.61) --
	(320.19,107.56) --
	(320.49,107.52) --
	(320.79,107.47) --
	(321.09,107.42) --
	(321.39,107.38) --
	(321.69,107.33) --
	(321.99,107.29) --
	(322.29,107.24) --
	(322.59,107.20) --
	(322.89,107.15) --
	(323.19,107.11) --
	(323.49,107.06) --
	(323.79,107.02) --
	(324.09,106.97) --
	(324.38,106.93) --
	(324.68,106.88) --
	(324.98,106.84) --
	(325.28,106.79) --
	(325.58,106.75) --
	(325.88,106.70) --
	(326.18,106.66) --
	(326.48,106.62) --
	(326.78,106.57) --
	(327.08,106.53) --
	(327.38,106.49) --
	(327.68,106.44) --
	(327.98,106.40) --
	(328.28,106.36) --
	(328.58,106.31) --
	(328.88,106.27) --
	(329.18,106.23) --
	(329.48,106.18) --
	(329.77,106.14) --
	(330.07,106.10) --
	(330.37,106.05) --
	(330.67,106.01) --
	(330.97,105.97) --
	(331.27,105.93) --
	(331.57,105.88) --
	(331.87,105.84) --
	(332.17,105.80) --
	(332.47,105.76) --
	(332.77,105.72) --
	(333.07,105.67) --
	(333.37,105.63) --
	(333.67,105.59) --
	(333.97,105.55) --
	(334.27,105.51) --
	(334.57,105.47) --
	(334.87,105.42) --
	(335.16,105.38) --
	(335.46,105.34) --
	(335.76,105.30) --
	(336.06,105.26) --
	(336.36,105.22) --
	(336.66,105.18) --
	(336.96,105.14) --
	(337.26,105.10) --
	(337.56,105.06) --
	(337.86,105.02) --
	(338.16,104.98) --
	(338.46,104.93) --
	(338.76,104.89) --
	(339.06,104.85) --
	(339.36,104.81) --
	(339.66,104.77) --
	(339.96,104.73) --
	(340.26,104.69) --
	(340.56,104.66) --
	(340.85,104.62) --
	(341.15,104.58) --
	(341.45,104.54) --
	(341.75,104.50) --
	(342.05,104.46) --
	(342.35,104.42) --
	(342.65,104.38) --
	(342.95,104.34) --
	(343.25,104.30) --
	(343.55,104.26) --
	(343.85,104.22) --
	(344.15,104.19) --
	(344.45,104.15) --
	(344.75,104.11) --
	(345.05,104.07) --
	(345.35,104.03) --
	(345.65,103.99) --
	(345.95,103.95) --
	(346.24,103.92) --
	(346.54,103.88) --
	(346.84,103.84) --
	(347.14,103.80) --
	(347.44,103.76) --
	(347.74,103.73) --
	(348.04,103.69) --
	(348.34,103.65) --
	(348.64,103.61) --
	(348.94,103.58) --
	(349.24,103.54) --
	(349.54,103.50) --
	(349.84,103.46) --
	(350.14,103.43) --
	(350.44,103.39) --
	(350.74,103.35) --
	(351.04,103.32) --
	(351.34,103.28) --
	(351.63,103.24) --
	(351.93,103.21) --
	(352.23,103.17) --
	(352.53,103.13) --
	(352.83,103.10) --
	(353.13,103.06) --
	(353.43,103.02) --
	(353.73,102.99) --
	(354.03,102.95) --
	(354.33,102.92) --
	(354.63,102.88) --
	(354.93,102.84) --
	(355.23,102.81) --
	(355.53,102.77) --
	(355.83,102.74) --
	(356.13,102.70) --
	(356.43,102.67) --
	(356.73,102.63) --
	(357.02,102.59) --
	(357.32,102.56) --
	(357.62,102.52) --
	(357.92,102.49) --
	(358.22,102.45) --
	(358.52,102.42) --
	(358.82,102.38) --
	(359.12,102.35) --
	(359.42,102.31) --
	(359.72,102.28) --
	(360.02,102.24) --
	(360.32,102.21);
\end{scope}
\begin{scope}
\path[clip] (  0.00,  0.00) rectangle (397.48,542.02);
\definecolor{drawColor}{RGB}{0,0,0}

\path[draw=drawColor,line width= 0.4pt,line join=round,line cap=round] ( 58.14, 61.20) -- (360.32, 61.20);

\path[draw=drawColor,line width= 0.4pt,line join=round,line cap=round] ( 58.14, 61.20) -- ( 58.14, 55.20);

\path[draw=drawColor,line width= 0.4pt,line join=round,line cap=round] (118.58, 61.20) -- (118.58, 55.20);

\path[draw=drawColor,line width= 0.4pt,line join=round,line cap=round] (179.01, 61.20) -- (179.01, 55.20);

\path[draw=drawColor,line width= 0.4pt,line join=round,line cap=round] (239.45, 61.20) -- (239.45, 55.20);

\path[draw=drawColor,line width= 0.4pt,line join=round,line cap=round] (299.88, 61.20) -- (299.88, 55.20);

\path[draw=drawColor,line width= 0.4pt,line join=round,line cap=round] (360.32, 61.20) -- (360.32, 55.20);

\node[text=drawColor,anchor=base,inner sep=0pt, outer sep=0pt, scale=  1.00] at ( 58.14, 39.60) {0};

\node[text=drawColor,anchor=base,inner sep=0pt, outer sep=0pt, scale=  1.00] at (118.58, 39.60) {2};

\node[text=drawColor,anchor=base,inner sep=0pt, outer sep=0pt, scale=  1.00] at (179.01, 39.60) {4};

\node[text=drawColor,anchor=base,inner sep=0pt, outer sep=0pt, scale=  1.00] at (239.45, 39.60) {6};

\node[text=drawColor,anchor=base,inner sep=0pt, outer sep=0pt, scale=  1.00] at (299.88, 39.60) {8};

\node[text=drawColor,anchor=base,inner sep=0pt, outer sep=0pt, scale=  1.00] at (360.32, 39.60) {10};

\path[draw=drawColor,line width= 0.4pt,line join=round,line cap=round] ( 49.20, 67.15) -- ( 49.20,215.44);

\path[draw=drawColor,line width= 0.4pt,line join=round,line cap=round] ( 49.20, 67.15) -- ( 43.20, 67.15);

\path[draw=drawColor,line width= 0.4pt,line join=round,line cap=round] ( 49.20, 88.33) -- ( 43.20, 88.33);

\path[draw=drawColor,line width= 0.4pt,line join=round,line cap=round] ( 49.20,109.52) -- ( 43.20,109.52);

\path[draw=drawColor,line width= 0.4pt,line join=round,line cap=round] ( 49.20,130.70) -- ( 43.20,130.70);

\path[draw=drawColor,line width= 0.4pt,line join=round,line cap=round] ( 49.20,151.89) -- ( 43.20,151.89);

\path[draw=drawColor,line width= 0.4pt,line join=round,line cap=round] ( 49.20,173.07) -- ( 43.20,173.07);

\path[draw=drawColor,line width= 0.4pt,line join=round,line cap=round] ( 49.20,194.26) -- ( 43.20,194.26);

\path[draw=drawColor,line width= 0.4pt,line join=round,line cap=round] ( 49.20,215.44) -- ( 43.20,215.44);

\node[text=drawColor,rotate= 90.00,anchor=base,inner sep=0pt, outer sep=0pt, scale=  1.00] at ( 34.80, 67.15) {0};

\node[text=drawColor,rotate= 90.00,anchor=base,inner sep=0pt, outer sep=0pt, scale=  1.00] at ( 34.80, 88.33) {2};

\node[text=drawColor,rotate= 90.00,anchor=base,inner sep=0pt, outer sep=0pt, scale=  1.00] at ( 34.80,109.52) {4};

\node[text=drawColor,rotate= 90.00,anchor=base,inner sep=0pt, outer sep=0pt, scale=  1.00] at ( 34.80,130.70) {6};

\node[text=drawColor,rotate= 90.00,anchor=base,inner sep=0pt, outer sep=0pt, scale=  1.00] at ( 34.80,151.89) {8};

\node[text=drawColor,rotate= 90.00,anchor=base,inner sep=0pt, outer sep=0pt, scale=  1.00] at ( 34.80,173.07) {10};

\node[text=drawColor,rotate= 90.00,anchor=base,inner sep=0pt, outer sep=0pt, scale=  1.00] at ( 34.80,194.26) {12};

\node[text=drawColor,rotate= 90.00,anchor=base,inner sep=0pt, outer sep=0pt, scale=  1.00] at ( 34.80,215.44) {14};

\path[draw=drawColor,line width= 0.4pt,line join=round,line cap=round] ( 49.20, 61.20) --
	(372.28, 61.20) --
	(372.28,221.81) --
	( 49.20,221.81) --
	( 49.20, 61.20);
\end{scope}
\begin{scope}
\path[clip] (  0.00,  0.00) rectangle (397.48,271.01);
\definecolor{drawColor}{RGB}{0,0,0}

\node[text=drawColor,anchor=base,inner sep=0pt, outer sep=0pt, scale=  1.20] at (210.74,242.27) {\bfseries $\tilde{i}_P < i_R$};

\node[text=drawColor,anchor=base,inner sep=0pt, outer sep=0pt, scale=  1.00] at (210.74, 27.60) {$t_\text{max}$};

\node[text=drawColor,rotate= 90.00,anchor=base,inner sep=0pt, outer sep=0pt, scale=  1.00] at ( 22.80,141.51) {MIPS};
\end{scope}
\begin{scope}
\path[clip] ( 49.20, 61.20) rectangle (372.28,221.81);
\definecolor{drawColor}{RGB}{0,0,0}

\path[draw=drawColor,line width= 0.4pt,dash pattern=on 4pt off 4pt ,line join=round,line cap=round] ( 61.80,542.02) --
	( 62.06,512.57) --
	( 62.36,483.60) --
	( 62.66,458.47) --
	( 62.96,436.46) --
	( 63.26,417.02) --
	( 63.56,399.74) --
	( 63.86,384.27) --
	( 64.16,370.33) --
	( 64.46,357.72) --
	( 64.76,346.25) --
	( 65.06,335.77) --
	( 65.36,326.17) --
	( 65.66,317.32) --
	( 65.96,309.16) --
	( 66.26,301.60) --
	( 66.56,294.58) --
	( 66.86,288.04) --
	( 67.16,281.93) --
	( 67.45,276.22) --
	( 67.75,270.86) --
	( 68.05,265.83) --
	( 68.35,261.09) --
	( 68.65,256.63) --
	( 68.95,252.41) --
	( 69.25,248.41) --
	( 69.55,244.63) --
	( 69.85,241.04) --
	( 70.15,237.63) --
	( 70.45,234.39) --
	( 70.75,231.30) --
	( 71.05,228.35) --
	( 71.35,225.54) --
	( 71.65,222.85) --
	( 71.95,220.28) --
	( 72.25,217.82) --
	( 72.55,215.46) --
	( 72.84,213.19) --
	( 73.14,211.02) --
	( 73.44,208.93) --
	( 73.74,206.92) --
	( 74.04,204.99) --
	( 74.34,203.13) --
	( 74.64,201.34) --
	( 74.94,199.61) --
	( 75.24,197.94) --
	( 75.54,196.33) --
	( 75.84,194.77) --
	( 76.14,193.26) --
	( 76.44,191.81) --
	( 76.74,190.40) --
	( 77.04,189.03) --
	( 77.34,187.71) --
	( 77.64,186.43) --
	( 77.94,185.19) --
	( 78.23,183.98) --
	( 78.53,182.81) --
	( 78.83,181.68) --
	( 79.13,180.58) --
	( 79.43,179.50) --
	( 79.73,178.46) --
	( 80.03,177.45) --
	( 80.33,176.46) --
	( 80.63,175.50) --
	( 80.93,174.56) --
	( 81.23,173.65) --
	( 81.53,172.76) --
	( 81.83,171.90) --
	( 82.13,171.06) --
	( 82.43,170.23) --
	( 82.73,169.43) --
	( 83.03,168.65) --
	( 83.33,167.88) --
	( 83.63,167.13) --
	( 83.92,166.40) --
	( 84.22,165.69) --
	( 84.52,165.00) --
	( 84.82,164.31) --
	( 85.12,163.65) --
	( 85.42,163.00) --
	( 85.72,162.36) --
	( 86.02,161.74) --
	( 86.32,161.13) --
	( 86.62,160.53) --
	( 86.92,159.94) --
	( 87.22,159.37) --
	( 87.52,158.81) --
	( 87.82,158.26) --
	( 88.12,157.72) --
	( 88.42,157.19) --
	( 88.72,156.68) --
	( 89.02,156.17) --
	( 89.31,155.67) --
	( 89.61,155.18) --
	( 89.91,154.70) --
	( 90.21,154.23) --
	( 90.51,153.77) --
	( 90.81,153.32) --
	( 91.11,152.87) --
	( 91.41,152.43) --
	( 91.71,152.01) --
	( 92.01,151.58) --
	( 92.31,151.17) --
	( 92.61,150.76) --
	( 92.91,150.36) --
	( 93.21,149.97) --
	( 93.51,149.58) --
	( 93.81,149.20) --
	( 94.11,148.83) --
	( 94.41,148.46) --
	( 94.70,148.10) --
	( 95.00,147.74) --
	( 95.30,147.39) --
	( 95.60,147.05) --
	( 95.90,146.71) --
	( 96.20,146.38) --
	( 96.50,146.05) --
	( 96.80,145.73) --
	( 97.10,145.41) --
	( 97.40,145.10) --
	( 97.70,144.79) --
	( 98.00,144.48) --
	( 98.30,144.18) --
	( 98.60,143.89) --
	( 98.90,143.60) --
	( 99.20,143.31) --
	( 99.50,143.03) --
	( 99.80,142.75) --
	(100.09,142.48) --
	(100.39,142.21) --
	(100.69,141.94) --
	(100.99,141.68) --
	(101.29,141.42) --
	(101.59,141.16) --
	(101.89,140.91) --
	(102.19,140.66) --
	(102.49,140.42) --
	(102.79,140.17) --
	(103.09,139.93) --
	(103.39,139.70) --
	(103.69,139.47) --
	(103.99,139.24) --
	(104.29,139.01) --
	(104.59,138.79) --
	(104.89,138.57) --
	(105.19,138.35) --
	(105.49,138.13) --
	(105.78,137.92) --
	(106.08,137.71) --
	(106.38,137.50) --
	(106.68,137.30) --
	(106.98,137.10) --
	(107.28,136.90) --
	(107.58,136.70) --
	(107.88,136.50) --
	(108.18,136.31) --
	(108.48,136.12) --
	(108.78,135.93) --
	(109.08,135.75) --
	(109.38,135.56) --
	(109.68,135.38) --
	(109.98,135.20) --
	(110.28,135.03) --
	(110.58,134.85) --
	(110.88,134.68) --
	(111.17,134.51) --
	(111.47,134.34) --
	(111.77,134.17) --
	(112.07,134.00) --
	(112.37,133.84) --
	(112.67,133.68) --
	(112.97,133.52) --
	(113.27,133.36) --
	(113.57,133.20) --
	(113.87,133.05) --
	(114.17,132.89) --
	(114.47,132.74) --
	(114.77,132.59) --
	(115.07,132.44) --
	(115.37,132.29) --
	(115.67,132.15) --
	(115.97,132.00) --
	(116.27,131.86) --
	(116.56,131.72) --
	(116.86,131.58) --
	(117.16,131.44) --
	(117.46,131.31) --
	(117.76,131.17) --
	(118.06,131.04) --
	(118.36,130.90) --
	(118.66,130.77) --
	(118.96,130.64) --
	(119.26,130.51) --
	(119.56,130.39) --
	(119.86,130.26) --
	(120.16,130.13) --
	(120.46,130.01) --
	(120.76,129.89) --
	(121.06,129.77) --
	(121.36,129.64) --
	(121.66,129.53) --
	(121.95,129.41) --
	(122.25,129.29) --
	(122.55,129.17) --
	(122.85,129.06) --
	(123.15,128.95) --
	(123.45,128.83) --
	(123.75,128.72) --
	(124.05,128.61) --
	(124.35,128.50) --
	(124.65,128.39) --
	(124.95,128.28) --
	(125.25,128.18) --
	(125.55,128.07) --
	(125.85,127.97) --
	(126.15,127.86) --
	(126.45,127.76) --
	(126.75,127.66) --
	(127.05,127.55) --
	(127.35,127.45) --
	(127.64,127.35) --
	(127.94,127.26) --
	(128.24,127.16) --
	(128.54,127.06) --
	(128.84,126.96) --
	(129.14,126.87) --
	(129.44,126.77) --
	(129.74,126.68) --
	(130.04,126.59) --
	(130.34,126.49) --
	(130.64,126.40) --
	(130.94,126.31) --
	(131.24,126.22) --
	(131.54,126.13) --
	(131.84,126.04) --
	(132.14,125.96) --
	(132.44,125.87) --
	(132.74,125.78) --
	(133.03,125.70) --
	(133.33,125.61) --
	(133.63,125.53) --
	(133.93,125.44) --
	(134.23,125.36) --
	(134.53,125.28) --
	(134.83,125.20) --
	(135.13,125.11) --
	(135.43,125.03) --
	(135.73,124.95) --
	(136.03,124.88) --
	(136.33,124.80) --
	(136.63,124.72) --
	(136.93,124.64) --
	(137.23,124.56) --
	(137.53,124.49) --
	(137.83,124.41) --
	(138.13,124.34) --
	(138.42,124.26) --
	(138.72,124.19) --
	(139.02,124.11) --
	(139.32,124.04) --
	(139.62,123.97) --
	(139.92,123.90) --
	(140.22,123.83) --
	(140.52,123.75) --
	(140.82,123.68) --
	(141.12,123.61) --
	(141.42,123.55) --
	(141.72,123.48) --
	(142.02,123.41) --
	(142.32,123.34) --
	(142.62,123.27) --
	(142.92,123.21) --
	(143.22,123.14) --
	(143.52,123.07) --
	(143.81,123.01) --
	(144.11,122.94) --
	(144.41,122.88) --
	(144.71,122.81) --
	(145.01,122.75) --
	(145.31,122.69) --
	(145.61,122.62) --
	(145.91,122.56) --
	(146.21,122.50) --
	(146.51,122.44) --
	(146.81,122.38) --
	(147.11,122.32) --
	(147.41,122.26) --
	(147.71,122.20) --
	(148.01,122.14) --
	(148.31,122.08) --
	(148.61,122.02) --
	(148.91,121.96) --
	(149.21,121.90) --
	(149.50,121.84) --
	(149.80,121.79) --
	(150.10,121.73) --
	(150.40,121.67) --
	(150.70,121.62) --
	(151.00,121.56) --
	(151.30,121.51) --
	(151.60,121.45) --
	(151.90,121.40) --
	(152.20,121.34) --
	(152.50,121.29) --
	(152.80,121.23) --
	(153.10,121.18) --
	(153.40,121.13) --
	(153.70,121.08) --
	(154.00,121.02) --
	(154.30,120.97) --
	(154.60,120.92) --
	(154.89,120.87) --
	(155.19,120.82) --
	(155.49,120.77) --
	(155.79,120.72) --
	(156.09,120.67) --
	(156.39,120.62) --
	(156.69,120.57) --
	(156.99,120.52) --
	(157.29,120.47) --
	(157.59,120.42) --
	(157.89,120.37) --
	(158.19,120.32) --
	(158.49,120.28) --
	(158.79,120.23) --
	(159.09,120.18) --
	(159.39,120.13) --
	(159.69,120.09) --
	(159.99,120.04) --
	(160.28,120.00) --
	(160.58,119.95) --
	(160.88,119.90) --
	(161.18,119.86) --
	(161.48,119.81) --
	(161.78,119.77) --
	(162.08,119.72) --
	(162.38,119.68) --
	(162.68,119.64) --
	(162.98,119.59) --
	(163.28,119.55) --
	(163.58,119.51) --
	(163.88,119.46) --
	(164.18,119.42) --
	(164.48,119.38) --
	(164.78,119.34) --
	(165.08,119.29) --
	(165.38,119.25) --
	(165.67,119.21) --
	(165.97,119.17) --
	(166.27,119.13) --
	(166.57,119.09) --
	(166.87,119.05) --
	(167.17,119.01) --
	(167.47,118.97) --
	(167.77,118.93) --
	(168.07,118.89) --
	(168.37,118.85) --
	(168.67,118.81) --
	(168.97,118.77) --
	(169.27,118.73) --
	(169.57,118.69) --
	(169.87,118.65) --
	(170.17,118.61) --
	(170.47,118.58) --
	(170.77,118.54) --
	(171.07,118.50) --
	(171.36,118.46) --
	(171.66,118.43) --
	(171.96,118.39) --
	(172.26,118.35) --
	(172.56,118.31) --
	(172.86,118.28) --
	(173.16,118.24) --
	(173.46,118.21) --
	(173.76,118.17) --
	(174.06,118.13) --
	(174.36,118.10) --
	(174.66,118.06) --
	(174.96,118.03) --
	(175.26,117.99) --
	(175.56,117.96) --
	(175.86,117.92) --
	(176.16,117.89) --
	(176.46,117.85) --
	(176.75,117.82) --
	(177.05,117.79) --
	(177.35,117.75) --
	(177.65,117.72) --
	(177.95,117.69) --
	(178.25,117.65) --
	(178.55,117.62) --
	(178.85,117.59) --
	(179.15,117.55) --
	(179.45,117.52) --
	(179.75,117.49) --
	(180.05,117.46) --
	(180.35,117.42) --
	(180.65,117.39) --
	(180.95,117.36) --
	(181.25,117.33) --
	(181.55,117.30) --
	(181.85,117.26) --
	(182.14,117.23) --
	(182.44,117.20) --
	(182.74,117.17) --
	(183.04,117.14) --
	(183.34,117.11) --
	(183.64,117.08) --
	(183.94,117.05) --
	(184.24,117.02) --
	(184.54,116.99) --
	(184.84,116.96) --
	(185.14,116.93) --
	(185.44,116.90) --
	(185.74,116.87) --
	(186.04,116.84) --
	(186.34,116.81) --
	(186.64,116.78) --
	(186.94,116.75) --
	(187.24,116.72) --
	(187.53,116.70) --
	(187.83,116.67) --
	(188.13,116.64) --
	(188.43,116.61) --
	(188.73,116.58) --
	(189.03,116.55) --
	(189.33,116.53) --
	(189.63,116.50) --
	(189.93,116.47) --
	(190.23,116.44) --
	(190.53,116.42) --
	(190.83,116.39) --
	(191.13,116.36) --
	(191.43,116.33) --
	(191.73,116.31) --
	(192.03,116.28) --
	(192.33,116.25) --
	(192.63,116.23) --
	(192.93,116.20) --
	(193.22,116.17) --
	(193.52,116.15) --
	(193.82,116.12) --
	(194.12,116.10) --
	(194.42,116.07) --
	(194.72,116.04) --
	(195.02,116.02) --
	(195.32,115.99) --
	(195.62,115.97) --
	(195.92,115.94) --
	(196.22,115.92) --
	(196.52,115.89) --
	(196.82,115.87) --
	(197.12,115.84) --
	(197.42,115.82) --
	(197.72,115.79) --
	(198.02,115.77) --
	(198.32,115.74) --
	(198.61,115.72) --
	(198.91,115.70) --
	(199.21,115.67) --
	(199.51,115.65) --
	(199.81,115.62) --
	(200.11,115.60) --
	(200.41,115.58) --
	(200.71,115.55) --
	(201.01,115.53) --
	(201.31,115.51) --
	(201.61,115.48) --
	(201.91,115.46) --
	(202.21,115.44) --
	(202.51,115.41) --
	(202.81,115.39) --
	(203.11,115.37) --
	(203.41,115.34) --
	(203.71,115.32) --
	(204.00,115.30) --
	(204.30,115.28) --
	(204.60,115.25) --
	(204.90,115.23) --
	(205.20,115.21) --
	(205.50,115.19) --
	(205.80,115.17) --
	(206.10,115.14) --
	(206.40,115.12) --
	(206.70,115.10) --
	(207.00,115.08) --
	(207.30,115.06) --
	(207.60,115.04) --
	(207.90,115.01) --
	(208.20,114.99) --
	(208.50,114.97) --
	(208.80,114.95) --
	(209.10,114.93) --
	(209.39,114.91) --
	(209.69,114.89) --
	(209.99,114.87) --
	(210.29,114.85) --
	(210.59,114.83) --
	(210.89,114.80) --
	(211.19,114.78) --
	(211.49,114.76) --
	(211.79,114.74) --
	(212.09,114.72) --
	(212.39,114.70) --
	(212.69,114.68) --
	(212.99,114.66) --
	(213.29,114.64) --
	(213.59,114.62) --
	(213.89,114.60) --
	(214.19,114.58) --
	(214.49,114.56) --
	(214.79,114.54) --
	(215.08,114.52) --
	(215.38,114.51) --
	(215.68,114.49) --
	(215.98,114.47) --
	(216.28,114.45) --
	(216.58,114.43) --
	(216.88,114.41) --
	(217.18,114.39) --
	(217.48,114.37) --
	(217.78,114.35) --
	(218.08,114.33) --
	(218.38,114.32) --
	(218.68,114.30) --
	(218.98,114.28) --
	(219.28,114.26) --
	(219.58,114.24) --
	(219.88,114.22) --
	(220.18,114.20) --
	(220.47,114.19) --
	(220.77,114.17) --
	(221.07,114.15) --
	(221.37,114.13) --
	(221.67,114.11) --
	(221.97,114.10) --
	(222.27,114.08) --
	(222.57,114.06) --
	(222.87,114.04) --
	(223.17,114.03) --
	(223.47,114.01) --
	(223.77,113.99) --
	(224.07,113.97) --
	(224.37,113.96) --
	(224.67,113.94) --
	(224.97,113.92) --
	(225.27,113.90) --
	(225.57,113.89) --
	(225.86,113.87) --
	(226.16,113.85) --
	(226.46,113.84) --
	(226.76,113.82) --
	(227.06,113.80) --
	(227.36,113.78) --
	(227.66,113.77) --
	(227.96,113.75) --
	(228.26,113.73) --
	(228.56,113.72) --
	(228.86,113.70) --
	(229.16,113.69) --
	(229.46,113.67) --
	(229.76,113.65) --
	(230.06,113.64) --
	(230.36,113.62) --
	(230.66,113.60) --
	(230.96,113.59) --
	(231.25,113.57) --
	(231.55,113.56) --
	(231.85,113.54) --
	(232.15,113.52) --
	(232.45,113.51) --
	(232.75,113.49) --
	(233.05,113.48) --
	(233.35,113.46) --
	(233.65,113.45) --
	(233.95,113.43) --
	(234.25,113.41) --
	(234.55,113.40) --
	(234.85,113.38) --
	(235.15,113.37) --
	(235.45,113.35) --
	(235.75,113.34) --
	(236.05,113.32) --
	(236.35,113.31) --
	(236.65,113.29) --
	(236.94,113.28) --
	(237.24,113.26) --
	(237.54,113.25) --
	(237.84,113.23) --
	(238.14,113.22) --
	(238.44,113.20) --
	(238.74,113.19) --
	(239.04,113.17) --
	(239.34,113.16) --
	(239.64,113.15) --
	(239.94,113.13) --
	(240.24,113.12) --
	(240.54,113.10) --
	(240.84,113.09) --
	(241.14,113.07) --
	(241.44,113.06) --
	(241.74,113.04) --
	(242.04,113.03) --
	(242.33,113.02) --
	(242.63,113.00) --
	(242.93,112.99) --
	(243.23,112.97) --
	(243.53,112.96) --
	(243.83,112.95) --
	(244.13,112.93) --
	(244.43,112.92) --
	(244.73,112.90) --
	(245.03,112.89) --
	(245.33,112.88) --
	(245.63,112.86) --
	(245.93,112.85) --
	(246.23,112.84) --
	(246.53,112.82) --
	(246.83,112.81) --
	(247.13,112.80) --
	(247.43,112.78) --
	(247.72,112.77) --
	(248.02,112.76) --
	(248.32,112.74) --
	(248.62,112.73) --
	(248.92,112.72) --
	(249.22,112.70) --
	(249.52,112.69) --
	(249.82,112.68) --
	(250.12,112.66) --
	(250.42,112.65) --
	(250.72,112.64) --
	(251.02,112.62) --
	(251.32,112.61) --
	(251.62,112.60) --
	(251.92,112.59) --
	(252.22,112.57) --
	(252.52,112.56) --
	(252.82,112.55) --
	(253.11,112.54) --
	(253.41,112.52) --
	(253.71,112.51) --
	(254.01,112.50) --
	(254.31,112.49) --
	(254.61,112.47) --
	(254.91,112.46) --
	(255.21,112.45) --
	(255.51,112.44) --
	(255.81,112.42) --
	(256.11,112.41) --
	(256.41,112.40) --
	(256.71,112.39) --
	(257.01,112.37) --
	(257.31,112.36) --
	(257.61,112.35) --
	(257.91,112.34) --
	(258.21,112.33) --
	(258.51,112.31) --
	(258.80,112.30) --
	(259.10,112.29) --
	(259.40,112.28) --
	(259.70,112.27) --
	(260.00,112.26) --
	(260.30,112.24) --
	(260.60,112.23) --
	(260.90,112.22) --
	(261.20,112.21) --
	(261.50,112.20) --
	(261.80,112.19) --
	(262.10,112.17) --
	(262.40,112.16) --
	(262.70,112.15) --
	(263.00,112.14) --
	(263.30,112.13) --
	(263.60,112.12) --
	(263.90,112.11) --
	(264.19,112.09) --
	(264.49,112.08) --
	(264.79,112.07) --
	(265.09,112.06) --
	(265.39,112.05) --
	(265.69,112.04) --
	(265.99,112.03) --
	(266.29,112.02) --
	(266.59,112.00) --
	(266.89,111.99) --
	(267.19,111.98) --
	(267.49,111.97) --
	(267.79,111.96) --
	(268.09,111.95) --
	(268.39,111.94) --
	(268.69,111.93) --
	(268.99,111.92) --
	(269.29,111.91) --
	(269.58,111.90) --
	(269.88,111.89) --
	(270.18,111.87) --
	(270.48,111.86) --
	(270.78,111.85) --
	(271.08,111.84) --
	(271.38,111.83) --
	(271.68,111.82) --
	(271.98,111.81) --
	(272.28,111.80) --
	(272.58,111.79) --
	(272.88,111.78) --
	(273.18,111.77) --
	(273.48,111.76) --
	(273.78,111.75) --
	(274.08,111.74) --
	(274.38,111.73) --
	(274.68,111.72) --
	(274.98,111.71) --
	(275.27,111.70) --
	(275.57,111.69) --
	(275.87,111.68) --
	(276.17,111.67) --
	(276.47,111.66) --
	(276.77,111.65) --
	(277.07,111.64) --
	(277.37,111.63) --
	(277.67,111.62) --
	(277.97,111.61) --
	(278.27,111.60) --
	(278.57,111.59) --
	(278.87,111.58) --
	(279.17,111.57) --
	(279.47,111.56) --
	(279.77,111.55) --
	(280.07,111.54) --
	(280.37,111.53) --
	(280.66,111.52) --
	(280.96,111.51) --
	(281.26,111.50) --
	(281.56,111.49) --
	(281.86,111.48) --
	(282.16,111.47) --
	(282.46,111.46) --
	(282.76,111.45) --
	(283.06,111.44) --
	(283.36,111.43) --
	(283.66,111.42) --
	(283.96,111.41) --
	(284.26,111.41) --
	(284.56,111.40) --
	(284.86,111.39) --
	(285.16,111.38) --
	(285.46,111.37) --
	(285.76,111.36) --
	(286.05,111.35) --
	(286.35,111.34) --
	(286.65,111.33) --
	(286.95,111.32) --
	(287.25,111.31) --
	(287.55,111.30) --
	(287.85,111.29) --
	(288.15,111.29) --
	(288.45,111.28) --
	(288.75,111.27) --
	(289.05,111.26) --
	(289.35,111.25) --
	(289.65,111.24) --
	(289.95,111.23) --
	(290.25,111.22) --
	(290.55,111.21) --
	(290.85,111.20) --
	(291.15,111.20) --
	(291.44,111.19) --
	(291.74,111.18) --
	(292.04,111.17) --
	(292.34,111.16) --
	(292.64,111.15) --
	(292.94,111.14) --
	(293.24,111.13) --
	(293.54,111.13) --
	(293.84,111.12) --
	(294.14,111.11) --
	(294.44,111.10) --
	(294.74,111.09) --
	(295.04,111.08) --
	(295.34,111.07) --
	(295.64,111.07) --
	(295.94,111.06) --
	(296.24,111.05) --
	(296.54,111.04) --
	(296.84,111.03) --
	(297.13,111.02) --
	(297.43,111.02) --
	(297.73,111.01) --
	(298.03,111.00) --
	(298.33,110.99) --
	(298.63,110.98) --
	(298.93,110.97) --
	(299.23,110.97) --
	(299.53,110.96) --
	(299.83,110.95) --
	(300.13,110.94) --
	(300.43,110.93) --
	(300.73,110.92) --
	(301.03,110.92) --
	(301.33,110.91) --
	(301.63,110.90) --
	(301.93,110.89) --
	(302.23,110.88) --
	(302.52,110.88) --
	(302.82,110.87) --
	(303.12,110.86) --
	(303.42,110.85) --
	(303.72,110.84) --
	(304.02,110.84) --
	(304.32,110.83) --
	(304.62,110.82) --
	(304.92,110.81) --
	(305.22,110.80) --
	(305.52,110.80) --
	(305.82,110.79) --
	(306.12,110.78) --
	(306.42,110.77) --
	(306.72,110.77) --
	(307.02,110.76) --
	(307.32,110.75) --
	(307.62,110.74) --
	(307.91,110.73) --
	(308.21,110.73) --
	(308.51,110.72) --
	(308.81,110.71) --
	(309.11,110.70) --
	(309.41,110.70) --
	(309.71,110.69) --
	(310.01,110.68) --
	(310.31,110.67) --
	(310.61,110.67) --
	(310.91,110.66) --
	(311.21,110.65) --
	(311.51,110.64) --
	(311.81,110.64) --
	(312.11,110.63) --
	(312.41,110.62) --
	(312.71,110.61) --
	(313.01,110.61) --
	(313.30,110.60) --
	(313.60,110.59) --
	(313.90,110.58) --
	(314.20,110.58) --
	(314.50,110.57) --
	(314.80,110.56) --
	(315.10,110.56) --
	(315.40,110.55) --
	(315.70,110.54) --
	(316.00,110.53) --
	(316.30,110.53) --
	(316.60,110.52) --
	(316.90,110.51) --
	(317.20,110.51) --
	(317.50,110.50) --
	(317.80,110.49) --
	(318.10,110.48) --
	(318.40,110.48) --
	(318.70,110.47) --
	(318.99,110.46) --
	(319.29,110.46) --
	(319.59,110.45) --
	(319.89,110.44) --
	(320.19,110.43) --
	(320.49,110.43) --
	(320.79,110.42) --
	(321.09,110.41) --
	(321.39,110.41) --
	(321.69,110.40) --
	(321.99,110.39) --
	(322.29,110.39) --
	(322.59,110.38) --
	(322.89,110.37) --
	(323.19,110.37) --
	(323.49,110.36) --
	(323.79,110.35) --
	(324.09,110.35) --
	(324.38,110.34) --
	(324.68,110.33) --
	(324.98,110.32) --
	(325.28,110.32) --
	(325.58,110.31) --
	(325.88,110.30) --
	(326.18,110.30) --
	(326.48,110.29) --
	(326.78,110.28) --
	(327.08,110.28) --
	(327.38,110.27) --
	(327.68,110.26) --
	(327.98,110.26) --
	(328.28,110.25) --
	(328.58,110.25) --
	(328.88,110.24) --
	(329.18,110.23) --
	(329.48,110.23) --
	(329.77,110.22) --
	(330.07,110.21) --
	(330.37,110.21) --
	(330.67,110.20) --
	(330.97,110.19) --
	(331.27,110.19) --
	(331.57,110.18) --
	(331.87,110.17) --
	(332.17,110.17) --
	(332.47,110.16) --
	(332.77,110.15) --
	(333.07,110.15) --
	(333.37,110.14) --
	(333.67,110.14) --
	(333.97,110.13) --
	(334.27,110.12) --
	(334.57,110.12) --
	(334.87,110.11) --
	(335.16,110.10) --
	(335.46,110.10) --
	(335.76,110.09) --
	(336.06,110.09) --
	(336.36,110.08) --
	(336.66,110.07) --
	(336.96,110.07) --
	(337.26,110.06) --
	(337.56,110.05) --
	(337.86,110.05) --
	(338.16,110.04) --
	(338.46,110.04) --
	(338.76,110.03) --
	(339.06,110.02) --
	(339.36,110.02) --
	(339.66,110.01) --
	(339.96,110.01) --
	(340.26,110.00) --
	(340.56,109.99) --
	(340.85,109.99) --
	(341.15,109.98) --
	(341.45,109.98) --
	(341.75,109.97) --
	(342.05,109.96) --
	(342.35,109.96) --
	(342.65,109.95) --
	(342.95,109.95) --
	(343.25,109.94) --
	(343.55,109.93) --
	(343.85,109.93) --
	(344.15,109.92) --
	(344.45,109.92) --
	(344.75,109.91) --
	(345.05,109.91) --
	(345.35,109.90) --
	(345.65,109.89) --
	(345.95,109.89) --
	(346.24,109.88) --
	(346.54,109.88) --
	(346.84,109.87) --
	(347.14,109.87) --
	(347.44,109.86) --
	(347.74,109.85) --
	(348.04,109.85) --
	(348.34,109.84) --
	(348.64,109.84) --
	(348.94,109.83) --
	(349.24,109.83) --
	(349.54,109.82) --
	(349.84,109.81) --
	(350.14,109.81) --
	(350.44,109.80) --
	(350.74,109.80) --
	(351.04,109.79) --
	(351.34,109.79) --
	(351.63,109.78) --
	(351.93,109.77) --
	(352.23,109.77) --
	(352.53,109.76) --
	(352.83,109.76) --
	(353.13,109.75) --
	(353.43,109.75) --
	(353.73,109.74) --
	(354.03,109.74) --
	(354.33,109.73) --
	(354.63,109.73) --
	(354.93,109.72) --
	(355.23,109.71) --
	(355.53,109.71) --
	(355.83,109.70) --
	(356.13,109.70) --
	(356.43,109.69) --
	(356.73,109.69) --
	(357.02,109.68) --
	(357.32,109.68) --
	(357.62,109.67) --
	(357.92,109.67) --
	(358.22,109.66) --
	(358.52,109.66) --
	(358.82,109.65) --
	(359.12,109.64) --
	(359.42,109.64) --
	(359.72,109.63) --
	(360.02,109.63) --
	(360.32,109.62);

\path[draw=drawColor,line width= 0.4pt,dash pattern=on 4pt off 4pt ,line join=round,line cap=round] ( 64.28,  0.00) --
	( 64.46,  2.97) --
	( 64.76,  7.56) --
	( 65.06, 11.75) --
	( 65.36, 15.59) --
	( 65.66, 19.13) --
	( 65.96, 22.39) --
	( 66.26, 25.42) --
	( 66.56, 28.23) --
	( 66.86, 30.84) --
	( 67.16, 33.29) --
	( 67.45, 35.57) --
	( 67.75, 37.71) --
	( 68.05, 39.73) --
	( 68.35, 41.62) --
	( 68.65, 43.41) --
	( 68.95, 45.10) --
	( 69.25, 46.69) --
	( 69.55, 48.21) --
	( 69.85, 49.64) --
	( 70.15, 51.01) --
	( 70.45, 52.30) --
	( 70.75, 53.54) --
	( 71.05, 54.72) --
	( 71.35, 55.84) --
	( 71.65, 56.92) --
	( 71.95, 57.95) --
	( 72.25, 58.93) --
	( 72.55, 59.88) --
	( 72.84, 60.78) --
	( 73.14, 61.65) --
	( 73.44, 62.49) --
	( 73.74, 63.29) --
	( 74.04, 64.06) --
	( 74.34, 64.81) --
	( 74.64, 65.52) --
	( 74.94, 66.22) --
	( 75.24, 66.88) --
	( 75.54, 67.53) --
	( 75.84, 68.15) --
	( 76.14, 68.75) --
	( 76.44, 69.34) --
	( 76.74, 69.90) --
	( 77.04, 70.44) --
	( 77.34, 70.97) --
	( 77.64, 71.49) --
	( 77.94, 71.98) --
	( 78.23, 72.46) --
	( 78.53, 72.93) --
	( 78.83, 73.39) --
	( 79.13, 73.83) --
	( 79.43, 74.26) --
	( 79.73, 74.67) --
	( 80.03, 75.08) --
	( 80.33, 75.47) --
	( 80.63, 75.86) --
	( 80.93, 76.23) --
	( 81.23, 76.60) --
	( 81.53, 76.95) --
	( 81.83, 77.30) --
	( 82.13, 77.64) --
	( 82.43, 77.97) --
	( 82.73, 78.29) --
	( 83.03, 78.60) --
	( 83.33, 78.91) --
	( 83.63, 79.20) --
	( 83.92, 79.50) --
	( 84.22, 79.78) --
	( 84.52, 80.06) --
	( 84.82, 80.33) --
	( 85.12, 80.60) --
	( 85.42, 80.86) --
	( 85.72, 81.11) --
	( 86.02, 81.36) --
	( 86.32, 81.61) --
	( 86.62, 81.85) --
	( 86.92, 82.08) --
	( 87.22, 82.31) --
	( 87.52, 82.53) --
	( 87.82, 82.75) --
	( 88.12, 82.97) --
	( 88.42, 83.18) --
	( 88.72, 83.39) --
	( 89.02, 83.59) --
	( 89.31, 83.79) --
	( 89.61, 83.99) --
	( 89.91, 84.18) --
	( 90.21, 84.37) --
	( 90.51, 84.55) --
	( 90.81, 84.73) --
	( 91.11, 84.91) --
	( 91.41, 85.08) --
	( 91.71, 85.26) --
	( 92.01, 85.42) --
	( 92.31, 85.59) --
	( 92.61, 85.75) --
	( 92.91, 85.91) --
	( 93.21, 86.07) --
	( 93.51, 86.23) --
	( 93.81, 86.38) --
	( 94.11, 86.53) --
	( 94.41, 86.67) --
	( 94.70, 86.82) --
	( 95.00, 86.96) --
	( 95.30, 87.10) --
	( 95.60, 87.24) --
	( 95.90, 87.37) --
	( 96.20, 87.51) --
	( 96.50, 87.64) --
	( 96.80, 87.77) --
	( 97.10, 87.89) --
	( 97.40, 88.02) --
	( 97.70, 88.14) --
	( 98.00, 88.27) --
	( 98.30, 88.39) --
	( 98.60, 88.50) --
	( 98.90, 88.62) --
	( 99.20, 88.73) --
	( 99.50, 88.85) --
	( 99.80, 88.96) --
	(100.09, 89.07) --
	(100.39, 89.18) --
	(100.69, 89.28) --
	(100.99, 89.39) --
	(101.29, 89.49) --
	(101.59, 89.59) --
	(101.89, 89.69) --
	(102.19, 89.79) --
	(102.49, 89.89) --
	(102.79, 89.99) --
	(103.09, 90.08) --
	(103.39, 90.18) --
	(103.69, 90.27) --
	(103.99, 90.36) --
	(104.29, 90.45) --
	(104.59, 90.54) --
	(104.89, 90.63) --
	(105.19, 90.72) --
	(105.49, 90.81) --
	(105.78, 90.89) --
	(106.08, 90.97) --
	(106.38, 91.06) --
	(106.68, 91.14) --
	(106.98, 91.22) --
	(107.28, 91.30) --
	(107.58, 91.38) --
	(107.88, 91.46) --
	(108.18, 91.53) --
	(108.48, 91.61) --
	(108.78, 91.69) --
	(109.08, 91.76) --
	(109.38, 91.83) --
	(109.68, 91.91) --
	(109.98, 91.98) --
	(110.28, 92.05) --
	(110.58, 92.12) --
	(110.88, 92.19) --
	(111.17, 92.26) --
	(111.47, 92.32) --
	(111.77, 92.39) --
	(112.07, 92.46) --
	(112.37, 92.52) --
	(112.67, 92.59) --
	(112.97, 92.65) --
	(113.27, 92.72) --
	(113.57, 92.78) --
	(113.87, 92.84) --
	(114.17, 92.90) --
	(114.47, 92.96) --
	(114.77, 93.02) --
	(115.07, 93.08) --
	(115.37, 93.14) --
	(115.67, 93.20) --
	(115.97, 93.26) --
	(116.27, 93.31) --
	(116.56, 93.37) --
	(116.86, 93.43) --
	(117.16, 93.48) --
	(117.46, 93.54) --
	(117.76, 93.59) --
	(118.06, 93.64) --
	(118.36, 93.70) --
	(118.66, 93.75) --
	(118.96, 93.80) --
	(119.26, 93.85) --
	(119.56, 93.90) --
	(119.86, 93.95) --
	(120.16, 94.00) --
	(120.46, 94.05) --
	(120.76, 94.10) --
	(121.06, 94.15) --
	(121.36, 94.20) --
	(121.66, 94.25) --
	(121.95, 94.30) --
	(122.25, 94.34) --
	(122.55, 94.39) --
	(122.85, 94.43) --
	(123.15, 94.48) --
	(123.45, 94.53) --
	(123.75, 94.57) --
	(124.05, 94.61) --
	(124.35, 94.66) --
	(124.65, 94.70) --
	(124.95, 94.75) --
	(125.25, 94.79) --
	(125.55, 94.83) --
	(125.85, 94.87) --
	(126.15, 94.91) --
	(126.45, 94.96) --
	(126.75, 95.00) --
	(127.05, 95.04) --
	(127.35, 95.08) --
	(127.64, 95.12) --
	(127.94, 95.16) --
	(128.24, 95.20) --
	(128.54, 95.23) --
	(128.84, 95.27) --
	(129.14, 95.31) --
	(129.44, 95.35) --
	(129.74, 95.39) --
	(130.04, 95.42) --
	(130.34, 95.46) --
	(130.64, 95.50) --
	(130.94, 95.53) --
	(131.24, 95.57) --
	(131.54, 95.61) --
	(131.84, 95.64) --
	(132.14, 95.68) --
	(132.44, 95.71) --
	(132.74, 95.75) --
	(133.03, 95.78) --
	(133.33, 95.81) --
	(133.63, 95.85) --
	(133.93, 95.88) --
	(134.23, 95.91) --
	(134.53, 95.95) --
	(134.83, 95.98) --
	(135.13, 96.01) --
	(135.43, 96.04) --
	(135.73, 96.08) --
	(136.03, 96.11) --
	(136.33, 96.14) --
	(136.63, 96.17) --
	(136.93, 96.20) --
	(137.23, 96.23) --
	(137.53, 96.26) --
	(137.83, 96.29) --
	(138.13, 96.32) --
	(138.42, 96.35) --
	(138.72, 96.38) --
	(139.02, 96.41) --
	(139.32, 96.44) --
	(139.62, 96.47) --
	(139.92, 96.50) --
	(140.22, 96.53) --
	(140.52, 96.56) --
	(140.82, 96.58) --
	(141.12, 96.61) --
	(141.42, 96.64) --
	(141.72, 96.67) --
	(142.02, 96.70) --
	(142.32, 96.72) --
	(142.62, 96.75) --
	(142.92, 96.78) --
	(143.22, 96.80) --
	(143.52, 96.83) --
	(143.81, 96.86) --
	(144.11, 96.88) --
	(144.41, 96.91) --
	(144.71, 96.93) --
	(145.01, 96.96) --
	(145.31, 96.98) --
	(145.61, 97.01) --
	(145.91, 97.03) --
	(146.21, 97.06) --
	(146.51, 97.08) --
	(146.81, 97.11) --
	(147.11, 97.13) --
	(147.41, 97.16) --
	(147.71, 97.18) --
	(148.01, 97.20) --
	(148.31, 97.23) --
	(148.61, 97.25) --
	(148.91, 97.27) --
	(149.21, 97.30) --
	(149.50, 97.32) --
	(149.80, 97.34) --
	(150.10, 97.37) --
	(150.40, 97.39) --
	(150.70, 97.41) --
	(151.00, 97.43) --
	(151.30, 97.46) --
	(151.60, 97.48) --
	(151.90, 97.50) --
	(152.20, 97.52) --
	(152.50, 97.54) --
	(152.80, 97.56) --
	(153.10, 97.59) --
	(153.40, 97.61) --
	(153.70, 97.63) --
	(154.00, 97.65) --
	(154.30, 97.67) --
	(154.60, 97.69) --
	(154.89, 97.71) --
	(155.19, 97.73) --
	(155.49, 97.75) --
	(155.79, 97.77) --
	(156.09, 97.79) --
	(156.39, 97.81) --
	(156.69, 97.83) --
	(156.99, 97.85) --
	(157.29, 97.87) --
	(157.59, 97.89) --
	(157.89, 97.91) --
	(158.19, 97.93) --
	(158.49, 97.95) --
	(158.79, 97.97) --
	(159.09, 97.99) --
	(159.39, 98.00) --
	(159.69, 98.02) --
	(159.99, 98.04) --
	(160.28, 98.06) --
	(160.58, 98.08) --
	(160.88, 98.10) --
	(161.18, 98.11) --
	(161.48, 98.13) --
	(161.78, 98.15) --
	(162.08, 98.17) --
	(162.38, 98.19) --
	(162.68, 98.20) --
	(162.98, 98.22) --
	(163.28, 98.24) --
	(163.58, 98.26) --
	(163.88, 98.27) --
	(164.18, 98.29) --
	(164.48, 98.31) --
	(164.78, 98.32) --
	(165.08, 98.34) --
	(165.38, 98.36) --
	(165.67, 98.37) --
	(165.97, 98.39) --
	(166.27, 98.41) --
	(166.57, 98.42) --
	(166.87, 98.44) --
	(167.17, 98.46) --
	(167.47, 98.47) --
	(167.77, 98.49) --
	(168.07, 98.50) --
	(168.37, 98.52) --
	(168.67, 98.54) --
	(168.97, 98.55) --
	(169.27, 98.57) --
	(169.57, 98.58) --
	(169.87, 98.60) --
	(170.17, 98.61) --
	(170.47, 98.63) --
	(170.77, 98.64) --
	(171.07, 98.66) --
	(171.36, 98.67) --
	(171.66, 98.69) --
	(171.96, 98.70) --
	(172.26, 98.72) --
	(172.56, 98.73) --
	(172.86, 98.75) --
	(173.16, 98.76) --
	(173.46, 98.78) --
	(173.76, 98.79) --
	(174.06, 98.80) --
	(174.36, 98.82) --
	(174.66, 98.83) --
	(174.96, 98.85) --
	(175.26, 98.86) --
	(175.56, 98.88) --
	(175.86, 98.89) --
	(176.16, 98.90) --
	(176.46, 98.92) --
	(176.75, 98.93) --
	(177.05, 98.94) --
	(177.35, 98.96) --
	(177.65, 98.97) --
	(177.95, 98.98) --
	(178.25, 99.00) --
	(178.55, 99.01) --
	(178.85, 99.02) --
	(179.15, 99.04) --
	(179.45, 99.05) --
	(179.75, 99.06) --
	(180.05, 99.08) --
	(180.35, 99.09) --
	(180.65, 99.10) --
	(180.95, 99.11) --
	(181.25, 99.13) --
	(181.55, 99.14) --
	(181.85, 99.15) --
	(182.14, 99.17) --
	(182.44, 99.18) --
	(182.74, 99.19) --
	(183.04, 99.20) --
	(183.34, 99.21) --
	(183.64, 99.23) --
	(183.94, 99.24) --
	(184.24, 99.25) --
	(184.54, 99.26) --
	(184.84, 99.27) --
	(185.14, 99.29) --
	(185.44, 99.30) --
	(185.74, 99.31) --
	(186.04, 99.32) --
	(186.34, 99.33) --
	(186.64, 99.35) --
	(186.94, 99.36) --
	(187.24, 99.37) --
	(187.53, 99.38) --
	(187.83, 99.39) --
	(188.13, 99.40) --
	(188.43, 99.41) --
	(188.73, 99.43) --
	(189.03, 99.44) --
	(189.33, 99.45) --
	(189.63, 99.46) --
	(189.93, 99.47) --
	(190.23, 99.48) --
	(190.53, 99.49) --
	(190.83, 99.50) --
	(191.13, 99.51) --
	(191.43, 99.52) --
	(191.73, 99.54) --
	(192.03, 99.55) --
	(192.33, 99.56) --
	(192.63, 99.57) --
	(192.93, 99.58) --
	(193.22, 99.59) --
	(193.52, 99.60) --
	(193.82, 99.61) --
	(194.12, 99.62) --
	(194.42, 99.63) --
	(194.72, 99.64) --
	(195.02, 99.65) --
	(195.32, 99.66) --
	(195.62, 99.67) --
	(195.92, 99.68) --
	(196.22, 99.69) --
	(196.52, 99.70) --
	(196.82, 99.71) --
	(197.12, 99.72) --
	(197.42, 99.73) --
	(197.72, 99.74) --
	(198.02, 99.75) --
	(198.32, 99.76) --
	(198.61, 99.77) --
	(198.91, 99.78) --
	(199.21, 99.79) --
	(199.51, 99.80) --
	(199.81, 99.81) --
	(200.11, 99.82) --
	(200.41, 99.83) --
	(200.71, 99.84) --
	(201.01, 99.85) --
	(201.31, 99.86) --
	(201.61, 99.87) --
	(201.91, 99.87) --
	(202.21, 99.88) --
	(202.51, 99.89) --
	(202.81, 99.90) --
	(203.11, 99.91) --
	(203.41, 99.92) --
	(203.71, 99.93) --
	(204.00, 99.94) --
	(204.30, 99.95) --
	(204.60, 99.96) --
	(204.90, 99.97) --
	(205.20, 99.97) --
	(205.50, 99.98) --
	(205.80, 99.99) --
	(206.10,100.00) --
	(206.40,100.01) --
	(206.70,100.02) --
	(207.00,100.03) --
	(207.30,100.04) --
	(207.60,100.04) --
	(207.90,100.05) --
	(208.20,100.06) --
	(208.50,100.07) --
	(208.80,100.08) --
	(209.10,100.09) --
	(209.39,100.10) --
	(209.69,100.10) --
	(209.99,100.11) --
	(210.29,100.12) --
	(210.59,100.13) --
	(210.89,100.14) --
	(211.19,100.14) --
	(211.49,100.15) --
	(211.79,100.16) --
	(212.09,100.17) --
	(212.39,100.18) --
	(212.69,100.19) --
	(212.99,100.19) --
	(213.29,100.20) --
	(213.59,100.21) --
	(213.89,100.22) --
	(214.19,100.23) --
	(214.49,100.23) --
	(214.79,100.24) --
	(215.08,100.25) --
	(215.38,100.26) --
	(215.68,100.26) --
	(215.98,100.27) --
	(216.28,100.28) --
	(216.58,100.29) --
	(216.88,100.29) --
	(217.18,100.30) --
	(217.48,100.31) --
	(217.78,100.32) --
	(218.08,100.32) --
	(218.38,100.33) --
	(218.68,100.34) --
	(218.98,100.35) --
	(219.28,100.35) --
	(219.58,100.36) --
	(219.88,100.37) --
	(220.18,100.38) --
	(220.47,100.38) --
	(220.77,100.39) --
	(221.07,100.40) --
	(221.37,100.41) --
	(221.67,100.41) --
	(221.97,100.42) --
	(222.27,100.43) --
	(222.57,100.43) --
	(222.87,100.44) --
	(223.17,100.45) --
	(223.47,100.46) --
	(223.77,100.46) --
	(224.07,100.47) --
	(224.37,100.48) --
	(224.67,100.48) --
	(224.97,100.49) --
	(225.27,100.50) --
	(225.57,100.50) --
	(225.86,100.51) --
	(226.16,100.52) --
	(226.46,100.52) --
	(226.76,100.53) --
	(227.06,100.54) --
	(227.36,100.54) --
	(227.66,100.55) --
	(227.96,100.56) --
	(228.26,100.56) --
	(228.56,100.57) --
	(228.86,100.58) --
	(229.16,100.58) --
	(229.46,100.59) --
	(229.76,100.60) --
	(230.06,100.60) --
	(230.36,100.61) --
	(230.66,100.62) --
	(230.96,100.62) --
	(231.25,100.63) --
	(231.55,100.64) --
	(231.85,100.64) --
	(232.15,100.65) --
	(232.45,100.65) --
	(232.75,100.66) --
	(233.05,100.67) --
	(233.35,100.67) --
	(233.65,100.68) --
	(233.95,100.69) --
	(234.25,100.69) --
	(234.55,100.70) --
	(234.85,100.70) --
	(235.15,100.71) --
	(235.45,100.72) --
	(235.75,100.72) --
	(236.05,100.73) --
	(236.35,100.74) --
	(236.65,100.74) --
	(236.94,100.75) --
	(237.24,100.75) --
	(237.54,100.76) --
	(237.84,100.77) --
	(238.14,100.77) --
	(238.44,100.78) --
	(238.74,100.78) --
	(239.04,100.79) --
	(239.34,100.79) --
	(239.64,100.80) --
	(239.94,100.81) --
	(240.24,100.81) --
	(240.54,100.82) --
	(240.84,100.82) --
	(241.14,100.83) --
	(241.44,100.83) --
	(241.74,100.84) --
	(242.04,100.85) --
	(242.33,100.85) --
	(242.63,100.86) --
	(242.93,100.86) --
	(243.23,100.87) --
	(243.53,100.87) --
	(243.83,100.88) --
	(244.13,100.89) --
	(244.43,100.89) --
	(244.73,100.90) --
	(245.03,100.90) --
	(245.33,100.91) --
	(245.63,100.91) --
	(245.93,100.92) --
	(246.23,100.92) --
	(246.53,100.93) --
	(246.83,100.93) --
	(247.13,100.94) --
	(247.43,100.95) --
	(247.72,100.95) --
	(248.02,100.96) --
	(248.32,100.96) --
	(248.62,100.97) --
	(248.92,100.97) --
	(249.22,100.98) --
	(249.52,100.98) --
	(249.82,100.99) --
	(250.12,100.99) --
	(250.42,101.00) --
	(250.72,101.00) --
	(251.02,101.01) --
	(251.32,101.01) --
	(251.62,101.02) --
	(251.92,101.02) --
	(252.22,101.03) --
	(252.52,101.03) --
	(252.82,101.04) --
	(253.11,101.04) --
	(253.41,101.05) --
	(253.71,101.05) --
	(254.01,101.06) --
	(254.31,101.06) --
	(254.61,101.07) --
	(254.91,101.07) --
	(255.21,101.08) --
	(255.51,101.08) --
	(255.81,101.09) --
	(256.11,101.09) --
	(256.41,101.10) --
	(256.71,101.10) --
	(257.01,101.11) --
	(257.31,101.11) --
	(257.61,101.12) --
	(257.91,101.12) --
	(258.21,101.13) --
	(258.51,101.13) --
	(258.80,101.14) --
	(259.10,101.14) --
	(259.40,101.15) --
	(259.70,101.15) --
	(260.00,101.16) --
	(260.30,101.16) --
	(260.60,101.17) --
	(260.90,101.17) --
	(261.20,101.17) --
	(261.50,101.18) --
	(261.80,101.18) --
	(262.10,101.19) --
	(262.40,101.19) --
	(262.70,101.20) --
	(263.00,101.20) --
	(263.30,101.21) --
	(263.60,101.21) --
	(263.90,101.22) --
	(264.19,101.22) --
	(264.49,101.23) --
	(264.79,101.23) --
	(265.09,101.23) --
	(265.39,101.24) --
	(265.69,101.24) --
	(265.99,101.25) --
	(266.29,101.25) --
	(266.59,101.26) --
	(266.89,101.26) --
	(267.19,101.27) --
	(267.49,101.27) --
	(267.79,101.27) --
	(268.09,101.28) --
	(268.39,101.28) --
	(268.69,101.29) --
	(268.99,101.29) --
	(269.29,101.30) --
	(269.58,101.30) --
	(269.88,101.30) --
	(270.18,101.31) --
	(270.48,101.31) --
	(270.78,101.32) --
	(271.08,101.32) --
	(271.38,101.33) --
	(271.68,101.33) --
	(271.98,101.33) --
	(272.28,101.34) --
	(272.58,101.34) --
	(272.88,101.35) --
	(273.18,101.35) --
	(273.48,101.35) --
	(273.78,101.36) --
	(274.08,101.36) --
	(274.38,101.37) --
	(274.68,101.37) --
	(274.98,101.38) --
	(275.27,101.38) --
	(275.57,101.38) --
	(275.87,101.39) --
	(276.17,101.39) --
	(276.47,101.40) --
	(276.77,101.40) --
	(277.07,101.40) --
	(277.37,101.41) --
	(277.67,101.41) --
	(277.97,101.42) --
	(278.27,101.42) --
	(278.57,101.42) --
	(278.87,101.43) --
	(279.17,101.43) --
	(279.47,101.44) --
	(279.77,101.44) --
	(280.07,101.44) --
	(280.37,101.45) --
	(280.66,101.45) --
	(280.96,101.45) --
	(281.26,101.46) --
	(281.56,101.46) --
	(281.86,101.47) --
	(282.16,101.47) --
	(282.46,101.47) --
	(282.76,101.48) --
	(283.06,101.48) --
	(283.36,101.49) --
	(283.66,101.49) --
	(283.96,101.49) --
	(284.26,101.50) --
	(284.56,101.50) --
	(284.86,101.50) --
	(285.16,101.51) --
	(285.46,101.51) --
	(285.76,101.51) --
	(286.05,101.52) --
	(286.35,101.52) --
	(286.65,101.53) --
	(286.95,101.53) --
	(287.25,101.53) --
	(287.55,101.54) --
	(287.85,101.54) --
	(288.15,101.54) --
	(288.45,101.55) --
	(288.75,101.55) --
	(289.05,101.56) --
	(289.35,101.56) --
	(289.65,101.56) --
	(289.95,101.57) --
	(290.25,101.57) --
	(290.55,101.57) --
	(290.85,101.58) --
	(291.15,101.58) --
	(291.44,101.58) --
	(291.74,101.59) --
	(292.04,101.59) --
	(292.34,101.59) --
	(292.64,101.60) --
	(292.94,101.60) --
	(293.24,101.60) --
	(293.54,101.61) --
	(293.84,101.61) --
	(294.14,101.61) --
	(294.44,101.62) --
	(294.74,101.62) --
	(295.04,101.63) --
	(295.34,101.63) --
	(295.64,101.63) --
	(295.94,101.64) --
	(296.24,101.64) --
	(296.54,101.64) --
	(296.84,101.65) --
	(297.13,101.65) --
	(297.43,101.65) --
	(297.73,101.66) --
	(298.03,101.66) --
	(298.33,101.66) --
	(298.63,101.67) --
	(298.93,101.67) --
	(299.23,101.67) --
	(299.53,101.68) --
	(299.83,101.68) --
	(300.13,101.68) --
	(300.43,101.69) --
	(300.73,101.69) --
	(301.03,101.69) --
	(301.33,101.70) --
	(301.63,101.70) --
	(301.93,101.70) --
	(302.23,101.70) --
	(302.52,101.71) --
	(302.82,101.71) --
	(303.12,101.71) --
	(303.42,101.72) --
	(303.72,101.72) --
	(304.02,101.72) --
	(304.32,101.73) --
	(304.62,101.73) --
	(304.92,101.73) --
	(305.22,101.74) --
	(305.52,101.74) --
	(305.82,101.74) --
	(306.12,101.75) --
	(306.42,101.75) --
	(306.72,101.75) --
	(307.02,101.76) --
	(307.32,101.76) --
	(307.62,101.76) --
	(307.91,101.76) --
	(308.21,101.77) --
	(308.51,101.77) --
	(308.81,101.77) --
	(309.11,101.78) --
	(309.41,101.78) --
	(309.71,101.78) --
	(310.01,101.79) --
	(310.31,101.79) --
	(310.61,101.79) --
	(310.91,101.79) --
	(311.21,101.80) --
	(311.51,101.80) --
	(311.81,101.80) --
	(312.11,101.81) --
	(312.41,101.81) --
	(312.71,101.81) --
	(313.01,101.82) --
	(313.30,101.82) --
	(313.60,101.82) --
	(313.90,101.82) --
	(314.20,101.83) --
	(314.50,101.83) --
	(314.80,101.83) --
	(315.10,101.84) --
	(315.40,101.84) --
	(315.70,101.84) --
	(316.00,101.84) --
	(316.30,101.85) --
	(316.60,101.85) --
	(316.90,101.85) --
	(317.20,101.86) --
	(317.50,101.86) --
	(317.80,101.86) --
	(318.10,101.86) --
	(318.40,101.87) --
	(318.70,101.87) --
	(318.99,101.87) --
	(319.29,101.88) --
	(319.59,101.88) --
	(319.89,101.88) --
	(320.19,101.88) --
	(320.49,101.89) --
	(320.79,101.89) --
	(321.09,101.89) --
	(321.39,101.90) --
	(321.69,101.90) --
	(321.99,101.90) --
	(322.29,101.90) --
	(322.59,101.91) --
	(322.89,101.91) --
	(323.19,101.91) --
	(323.49,101.91) --
	(323.79,101.92) --
	(324.09,101.92) --
	(324.38,101.92) --
	(324.68,101.93) --
	(324.98,101.93) --
	(325.28,101.93) --
	(325.58,101.93) --
	(325.88,101.94) --
	(326.18,101.94) --
	(326.48,101.94) --
	(326.78,101.94) --
	(327.08,101.95) --
	(327.38,101.95) --
	(327.68,101.95) --
	(327.98,101.96) --
	(328.28,101.96) --
	(328.58,101.96) --
	(328.88,101.96) --
	(329.18,101.97) --
	(329.48,101.97) --
	(329.77,101.97) --
	(330.07,101.97) --
	(330.37,101.98) --
	(330.67,101.98) --
	(330.97,101.98) --
	(331.27,101.98) --
	(331.57,101.99) --
	(331.87,101.99) --
	(332.17,101.99) --
	(332.47,101.99) --
	(332.77,102.00) --
	(333.07,102.00) --
	(333.37,102.00) --
	(333.67,102.00) --
	(333.97,102.01) --
	(334.27,102.01) --
	(334.57,102.01) --
	(334.87,102.01) --
	(335.16,102.02) --
	(335.46,102.02) --
	(335.76,102.02) --
	(336.06,102.02) --
	(336.36,102.03) --
	(336.66,102.03) --
	(336.96,102.03) --
	(337.26,102.03) --
	(337.56,102.04) --
	(337.86,102.04) --
	(338.16,102.04) --
	(338.46,102.04) --
	(338.76,102.05) --
	(339.06,102.05) --
	(339.36,102.05) --
	(339.66,102.05) --
	(339.96,102.06) --
	(340.26,102.06) --
	(340.56,102.06) --
	(340.85,102.06) --
	(341.15,102.07) --
	(341.45,102.07) --
	(341.75,102.07) --
	(342.05,102.07) --
	(342.35,102.08) --
	(342.65,102.08) --
	(342.95,102.08) --
	(343.25,102.08) --
	(343.55,102.08) --
	(343.85,102.09) --
	(344.15,102.09) --
	(344.45,102.09) --
	(344.75,102.09) --
	(345.05,102.10) --
	(345.35,102.10) --
	(345.65,102.10) --
	(345.95,102.10) --
	(346.24,102.11) --
	(346.54,102.11) --
	(346.84,102.11) --
	(347.14,102.11) --
	(347.44,102.11) --
	(347.74,102.12) --
	(348.04,102.12) --
	(348.34,102.12) --
	(348.64,102.12) --
	(348.94,102.13) --
	(349.24,102.13) --
	(349.54,102.13) --
	(349.84,102.13) --
	(350.14,102.14) --
	(350.44,102.14) --
	(350.74,102.14) --
	(351.04,102.14) --
	(351.34,102.14) --
	(351.63,102.15) --
	(351.93,102.15) --
	(352.23,102.15) --
	(352.53,102.15) --
	(352.83,102.16) --
	(353.13,102.16) --
	(353.43,102.16) --
	(353.73,102.16) --
	(354.03,102.16) --
	(354.33,102.17) --
	(354.63,102.17) --
	(354.93,102.17) --
	(355.23,102.17) --
	(355.53,102.17) --
	(355.83,102.18) --
	(356.13,102.18) --
	(356.43,102.18) --
	(356.73,102.18) --
	(357.02,102.19) --
	(357.32,102.19) --
	(357.62,102.19) --
	(357.92,102.19) --
	(358.22,102.19) --
	(358.52,102.20) --
	(358.82,102.20) --
	(359.12,102.20) --
	(359.42,102.20) --
	(359.72,102.20) --
	(360.02,102.21) --
	(360.32,102.21);
\definecolor{drawColor}{RGB}{190,190,190}

\path[draw=drawColor,line width= 0.4pt,line join=round,line cap=round] ( 49.20,104.33) -- (372.28,104.33);
\definecolor{drawColor}{RGB}{0,0,0}

\path[draw=drawColor,line width= 0.4pt,line join=round,line cap=round] (275.67,206.58) -- (286.47,206.58);

\path[draw=drawColor,line width= 0.4pt,dash pattern=on 4pt off 4pt ,line join=round,line cap=round] (275.67,192.18) -- (286.47,192.18);
\definecolor{drawColor}{RGB}{190,190,190}

\path[draw=drawColor,line width= 0.4pt,line join=round,line cap=round] (275.67,177.78) -- (286.47,177.78);
\definecolor{drawColor}{RGB}{0,0,0}

\node[text=drawColor,anchor=base west,inner sep=0pt, outer sep=0pt, scale=  0.60] at (291.87,204.52) {MIPS};

\node[text=drawColor,anchor=base west,inner sep=0pt, outer sep=0pt, scale=  0.60] at (291.87,190.12) {$\text{MIPS}^\text{max}\text{ und }\text{MIPS}^\text{min}$};

\node[text=drawColor,anchor=base west,inner sep=0pt, outer sep=0pt, scale=  0.60] at (291.87,175.72) {$\lim\limits_{t_\text{max}\rightarrow\infty} \text{MIPS}$};
\end{scope}
\end{tikzpicture}

	\end{frame}

\section{Fallstudie}
	\begin{frame}{Fallstudie - Datenverfügbarkeit?}
            Szenarien: 
            \begin{itemize}
                \item Individuelle Waschmaschinennutzung (Ind.)
                \item Gemeinschaftliche Nutzung eines Waschkellers (Gem.)
                \item Nutzung eines öffentlichen Waschsalons (Sal.)
            \end{itemize}
		\begin{center}
\pause
\definecolor{lightorange}{RGB}{253,208,81}
% \definecolor{lightorange}{RGB}{255,238,193}
\definecolor{darkorange}{RGB}{252,187,6}
\rowcolors{1}{lightorange}{}
            \begin{tabular}[h]{lccc}
                % \toprule
                \hline
            \rowcolor{darkorange}
                Parameter & Ind. & Gem. & Sal.\\
                \hline
                % \midrule
                Servicebedarf $S_D$ &&& \pause\\
                Nutzungshäufigkeit $h$ &&& \pause\\
                absolute Produktauslastung $A$ &&&\pause\\
                Einzelnutzungsvorrat $\n{max}$ &&&\pause\\
                Inputs je Produkt $i_P$ &&&\pause\\
                gewünschte Nutzungsdauer $\t{max}$ &&&\pause\\
                technische Lebensdauer $\t{tech}$ &&&\pause\\
                reparaturbedingte Inputs $I_R$ &&&\pause\\
                Inputs je Reparatur $i_R$ &&&\pause\\
                fixe Inputs $\I{fix}$ &&&\\
                \hline
                % \bottomrule
            \end{tabular}
            % \onslide<2->
            % \begin{tabular}[h]{|l|c|c|c|}
            %     \hline
            %     \bfseries{Parameter }& \bfseries{Ind.} & \bfseries{Gem.} &
            %     \bfseries{Sal.}\\
            %     \hline\hline
            %     Servicebedarf $S_D$ &&&\\
            %     \hline
            %     Nutzungshäufigkeit $h$ &&&\\
            %     \hline
            %     absolute Produktauslastung $A$ &&&\\
            %     \hline
            %     Einzelnutzungsvorrat $\n{max}$ &&&\\
            %     \hline
            %     Inputs je Produkt $i_P$ &&&\\
            %     \hline
            %     gewünschte Nutzungsdauer $\t{max}$ &&&\\
            %     \hline
            %     technische Lebensdauer $\t{tech}$ &&&\\
            %     \hline
            %     reparaturbedingte Inputs $I_R$ &&&\\
            %     \hline
            %     Inputs je Reparatur $i_R$ &&&\\
            %     \hline
            %     fixe Inputs $\I{fix}$ &&&\\
            %     \hline
            % \end{tabular}
		\end{center}
	\end{frame}

\end{document}
