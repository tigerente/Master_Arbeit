\documentclass[beamer, xcolor=table]{beamer}
%\usepackage{mathpazo}
\usepackage[T1]{fontenc}
\usepackage[utf8]{inputenc}
\usepackage[ngerman]{babel}
\usepackage[babel,german=quotes]{csquotes}
\usepackage{lmodern} % for removing warning about font shape


%\usetheme{Madrid}
\usetheme{Frankfurt}
%\usetheme{Singapore}
\usecolortheme{crane}

\setbeamertemplate{footline}[frame number] % für Foliennummerierung
\beamertemplatenavigationsymbolsempty % Navigationsleiste 

\AtBeginSection{\frame{\sectionpage}}
\newtranslation[to=ngerman]{Section}{Abschnitt}
\newtranslation[to=ngerman]{Subsection}{Beispielmodell}

\usepackage{etoolbox}
\makeatletter
\patchcmd{\slideentry}{\ifnum#2>0}{\ifnum2>0}{}{\@error{unable to patch}}% replace the subsection number test with a test that always returns true
\makeatother

%\setbeameroption{show notes}

\usepackage{amsmath}
\usepackage{calc}

\usepackage{graphicx} %Zum Einbinden von Grafikdateien
\usepackage[font={footnotesize, sf}]{caption}
\usepackage{subfig}

\usepackage{booktabs} %Für schönere Tabellen
\usepackage{xspace}

\usepackage{tikz} %Für Skizzen

\usepackage[bibstyle=authortitle, citestyle=authoryear, isbn=false, doi=false,
dashed=false]{biblatex}
\addbibresource{Masterarbeit.bib}

%\usepackage{vmargin} %Fürs Seitenlayout
%\setmarginsrb{2.5cm}{1.5cm}{2cm}{1.5cm}{0mm}{0.5cm}{0mm}{1cm}

%%% Eigene Befehle %%%
\newcommand{\was}[1]{\small\textit{#1}}
\newcommand{\noteS}[1]{\todo[color=green!40]{\textbf{Sascha: }#1}}
\newcommand{\noteJ}[1]{\todo[color=blue!40]{\textbf{János: }#1}}
\newcommand{\textfrac}[2]{\hspace{2pt} \frac{\text{#1}}{\text{#2}}}
\newcommand{\eqnref}[1]{\overset{(\ref{#1})}{=}} % Gleichheitszeichen mit Referenz auf die verwendete Gleichung
\newcommand{\defeq}{\vcentcolon=} %Definitions-Gleichheitszeichen
\newcommand{\eqdef}{=\vcentcolon}
\newcommand{\pfrac}[2]{\frac{\partial #1}{\partial #2}}

% Formeln:
\newcommand{\MIPS}[1][]{
  \ifthenelse {\equal {#1} {}}
  {\text{MIPS}} % if argument is blank
  {\text{MIPS}({#1})} % if an optional argument is given
}
\renewcommand{\P}[1]{P_\text{#1}}
\newcommand{\I}[1]{I_\text{#1}}
\newcommand{\itext}[1]{i_\text{#1}}
\newcommand{\T}[1]{T_\text{#1}}
\newcommand{\n}[1]{n_\text{#1}}
\newcommand{\N}[1]{N_\text{#1}}
\renewcommand{\t}[1]{t_\text{#1}}



\title{Ökologische Nachhaltigkeit durch \\ \enquote{Nutzen statt Besitzen}}
\subtitle{Entwicklung eines Modells zur Ableitung von Kriterien für die Senkung des Umweltverbrauchs durch gemeinschaftliche Produktnutzung}
\author{Sascha \and János}
\date{29.04.2015}

\begin{document}

\frame{\titlepage}

\frame<beamer>{\tableofcontents[hidesubsections]}

\section{Thema, Fragestellung, Methodik}

	\begin{frame}{Thema und Fragestellung}
        Thema: 
        \begin{itemize}
            \item Gemeinschaftliche Nutzung von Produkten (Bspw. Waschmaschinen,
                Autos, Kleidung,\dots)
            \item Ökologische Vorteile solcher Nutzungsformen gegenüber
                individueller Nutzung
        \end{itemize}

        \begin{block}{Fragestellung:}
            \begin{itemize}
                \item Unter welchen Umständen kann der Umweltverbrauch eines Produktes durch gemeinschaftliche Nutzung gegenüber der individuellen Nutzung gesenkt werden?
                % \item Was sind die Mechanismen, die den Umweltverbrauch bei gemeinschaftlicher Nutzung bestimmen und wie wirken diese? 
                % \item Welche Eigenschaften des Produkts und der Nutzungsform beeinflussen die Wirkung dieser Mechanismen und wie fließen sie ein?
            \end{itemize}
        \end{block}
	\end{frame}
	
	\begin{frame}{Untersuchte Effekte}
        \only<1>{
            Was ist eigentlich ein Effekt/Mechanismus?
        }
		\begin{center}
            \scriptsize
            \only<2>{
			\begin{tabular}{p{5cm}p{5cm}}
				 \toprule
				 \multicolumn{2}{l}{\textbf{Umweltauswirkungen durch die Nutzung}}  \\
				 \textbf{positiv} & \textbf{negativ} \\
				 \midrule
				 Nutzungsintensivierung  &  Zusätzliche Transaktionen  \\
				 Einsatz langlebiger Produkte  &  zusätzlicher Ressourcenverbrauch für Langlebigkeit  \\
				 Verwendung verbrauchsarmer bzw. leistungsstarker Geräte  &  Beschleunigte Ausmusterung  \\
				 Maximierung der Geräteauslastung  & Tauschbedingter Verschleiß  \\
				 Berücksichtigung des technischen Fortschritts  &  \\
				 Wartung / Reparaturen & \\
				 \midrule
				 \multicolumn{2}{l}{\textbf{Umweltauswirkungen durch Nachfrageänderung}}  \\
				 \textbf{positiv} & \textbf{negativ} \\
				 \midrule
				 Nachfrageverringerung durch höhere Kostentransparenz  & Erleichterter Produktzugang  \\
				 Vermeidung von Fehlkäufen  & Wunsch nach Eigentum  \\
				  & Rebound-Effekt  \\
				 \bottomrule
			\end{tabular}
            \vspace{3pt}

			Mechanismen für die Umweltauswirkungen durch
            gemeinschaftliche Nutzung. Quelle: verändert nach
            \cite{scholl_marketing_2009}.}
            \only<3>{
            \normalsize
			\begin{tabular}{p{5cm}p{5cm}}
				 \toprule
				 \multicolumn{2}{l}{\textbf{Umweltauswirkungen durch die
                 Nutzung}}  \\[5pt]
				 \textbf{positiv} & \textbf{negativ} \\
				 \midrule
				 Nutzungsintensivierung  &  Zusätzliche Transaktionen
                 \\[3pt]
                 Maximierung der Geräteauslastung  & \\[3pt]
                 Wartung / Reparaturen & \\[3pt]
				 \bottomrule
			\end{tabular}
            \vspace{3pt}

			Betrachtete Mechanismen für die Umweltauswirkungen durch gemeinschaftliche Nutzung. Quelle: verändert nach \cite{scholl_marketing_2009}.}
        \end{center}
	\end{frame}
	
	\begin{frame}{Methodik}
		\begin{center}
            \begin{itemize}
                \item Modellierung
                    \begin{itemize}
                        \item Nutzungssystem
                            \begin{itemize}
                                \item Personen (Service-Nachfrage)
                                \item Produkte
                                \item Organisation der Nutzung
                            \end{itemize}
                        \item Vergleich verschiedener Nutzungssysteme
                            \begin{itemize}
                                \item Wie wirken die einzelnen Faktoren auf die MIPS?
                                \item MIPS gewährleistet Vergleichbarkeit
                            \end{itemize}
                    \end{itemize}

                \item Analyse
                    \begin{itemize}
                        \item Unter welchen Umständen verringert sich die MIPS
                            durch gemeinschaftliche Nutzung?
                        \item Grund-Annahme: Parameter ändern sich durch
                            gemeinschaftliche Nutzung in eine bestimmte Richtung
                        \item Partialanalyse: Nutzungssysteme, die sich in nur
                            einem Parameter unterscheiden
                        \item Modellkopplung: mehrere Parameter ändern sich	
                    \end{itemize}
            \end{itemize}
		\end{center}
	\end{frame}

\section{Modelle}
	\begin{frame}{Modell-Grundlagen}
		\begin{center}
			$S = S_D$ (in Bezug auf konstantes T)
			Abbildung 2.1
			T, p, q, P
			$t_max, n_max$
			
			Fragen / Anmerkungen?
		\end{center}
	\end{frame}

\subsection{Nutzungsintensivierung}
	\frame{\subsectionpage}
	\begin{frame}{Modellbeschreibung}
		\begin{center}
			Bla
		\end{center}
	\end{frame}
	
	\begin{frame}{Analyse}
		\begin{center}
			Bla
			Fragen / Diskussion
		\end{center}
	\end{frame}
	
\subsection{Reparatur}
\frame{\subsectionpage}
	\begin{frame}{Modellbeschreibung}
		\begin{center}
			Bla
		\end{center}
	\end{frame}
	
	\begin{frame}{Analyse}
		\begin{center}
			Bla
			Fragen / Diskussion
		\end{center}
	\end{frame}

\section{Fallstudie}
	\begin{frame}{Fallstudie - Datenverfügbarkeit?}
            Szenarien: 
            \begin{itemize}
                \item Individuelle Waschmaschinennutzung (Ind.)
                \item Gemeinschaftliche Nutzung eines Waschkellers (Gem.)
                \item Nutzung eines öffentlichen Waschsalons (Sal.)
            \end{itemize}
		\begin{center}
\pause
\rowcolors{1}{yellow!40}{white}
            \begin{tabular}[h]{lccc}
                % \toprule
                \hline
            \rowcolor{yellow!70!red}
                Parameter & Ind. & Gem. & Sal.\\
                \hline
                % \midrule
                Servicebedarf $S_D$ &&& \pause\\
                Nutzungshäufigkeit $h$ &&& \pause\\
                absolute Produktauslastung $A$ &&&\pause\\
                Einzelnutzungsvorrat $\n{max}$ &&&\pause\\
                Inputs je Produkt $i_P$ &&&\pause\\
                gewünschte Nutzungsdauer $\t{max}$ &&&\pause\\
                technische Lebensdauer $\t{tech}$ &&&\pause\\
                reparaturbedingte Inputs $I_R$ &&&\pause\\
                Inputs je Reparatur $i_R$ &&&\pause\\
                fixe Inputs $\I{fix}$ &&&\\
                \hline
                % \bottomrule
            \end{tabular}
            % \onslide<2->
            % \begin{tabular}[h]{|l|c|c|c|}
            %     \hline
            %     \bfseries{Parameter }& \bfseries{Ind.} & \bfseries{Gem.} &
            %     \bfseries{Sal.}\\
            %     \hline\hline
            %     Servicebedarf $S_D$ &&&\\
            %     \hline
            %     Nutzungshäufigkeit $h$ &&&\\
            %     \hline
            %     absolute Produktauslastung $A$ &&&\\
            %     \hline
            %     Einzelnutzungsvorrat $\n{max}$ &&&\\
            %     \hline
            %     Inputs je Produkt $i_P$ &&&\\
            %     \hline
            %     gewünschte Nutzungsdauer $\t{max}$ &&&\\
            %     \hline
            %     technische Lebensdauer $\t{tech}$ &&&\\
            %     \hline
            %     reparaturbedingte Inputs $I_R$ &&&\\
            %     \hline
            %     Inputs je Reparatur $i_R$ &&&\\
            %     \hline
            %     fixe Inputs $\I{fix}$ &&&\\
            %     \hline
            % \end{tabular}
		\end{center}
	\end{frame}

\end{document}
