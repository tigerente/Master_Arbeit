%        File: Masterarbeit.tex
%     Created: Di Mrz 04 02:00  2014 Mitteleuropäische Z
% Last Change: Di Mrz 04 02:00  2014 Mitteleuropäische Z
%
% !TEX TS-program = pdflatex
% !TEX encoding = UTF-8 Unicode

\documentclass[11pt, titlepage=true]{scrartcl} % use larger type; default would be 10pt

\usepackage[utf8]{inputenc} % set input encoding (not needed with XeLaTeX)
\usepackage[ngerman]{babel} 
\usepackage[babel,german=quotes]{csquotes}
\usepackage[T1]{fontenc} % for enabling non-standard characters in
                         % hyphenation-list and smallcaps-boldfont

\usepackage[bibstyle=authortitle, citestyle=authoryear, isbn=false, doi=false,
dashed=false]{biblatex}
\addbibresource{Masterarbeit.bib}

\def^^cc{XXXXX} % Zum Aufspüren von fehlerhaften Zeichen, falls komische Codierungsfehler mit der Bibliographie auftauchen.

% \DeclareUnicodeCharacter{00A0}{~} % for avoiding problems with no-break spaces

%%% PAGE DIMENSIONS
\usepackage{geometry} % to change the page dimensions
\geometry{a4paper} % or letterpaper (US) or a5paper or....
% \geometry{margin=2in} % for example, change the margins to 2 inches all round
% \geometry{landscape} % set up the page for landscape
%   read geometry.pdf for detailed page layout information

\usepackage{graphicx} % support the \includegraphics command and options
\usepackage{tikz} % for graphics produced by R

\graphicspath{{../Abbildungen/}{./Abbildungen/}}

\usepackage[parfill]{parskip} % Activate to begin paragraphs with an empty line rather than an indent

%%% PACKAGES
\usepackage{booktabs} % for much better looking tables
\usepackage[]{longtable}
\usepackage{array} % for better arrays (eg matrices) in maths
%\usepackage{paralist} % very flexible & customisable lists (eg. enumerate/itemize, etc.)
\usepackage{verbatim} % adds environment for commenting out blocks of text & for better verbatim
\usepackage{subfig} % make it possible to include more than one captioned figure/table in a single float
% These packages are all incorporated in the memoir class to one degree or another...
\usepackage{amsmath}
\usepackage{mdframed}
\usepackage{booktabs}

%%% HEADERS & FOOTERS
\usepackage{fancyhdr} % This should be set AFTER setting up the page geometry
\pagestyle{fancy} % options: empty , plain , fancy
\renewcommand{\headrulewidth}{0pt} % customise the layout...
\lhead{}\chead{}\rhead{}
\lfoot{}\cfoot{\thepage}\rfoot{}

%%% SECTION TITLE APPEARANCE
\usepackage{sectsty}
\allsectionsfont{\sffamily\mdseries\upshape} % (See the fntguide.pdf for font help)
% (This matches ConTeXt defaults)

%%% ToC (table of contents) APPEARANCE
\usepackage[nottoc,notlof,notlot]{tocbibind} % Put the bibliography in the ToC
\usepackage[titles,subfigure]{tocloft} % Alter the style of the Table of Contents
\renewcommand{\cftsecfont}{\rmfamily\mdseries\upshape}
\renewcommand{\cftsecpagefont}{\rmfamily\mdseries\upshape} % No bold!

\usepackage[hidelinks]{hyperref}
\usepackage{ragged2e} %für links und rechtsbündige Umgebungen in Tabellen
\usepackage[textsize=tiny]{todonotes}
\usepackage[normalem]{ulem} % ermöglicht es, Text mittels \sout und \xout
                            % durchzustreichen

%%% Eigene Befehle %%%
\newcommand{\was}[1]{\small\textit{#1}}
\newcommand{\noteS}[1]{\todo[color=green!40]{\textbf{Sascha: }#1}}
\newcommand{\noteJ}[1]{\todo[color=blue!40]{\textbf{János: }#1}}

% Formeln:
\newcommand{\MIPS}[1][]{\text{MIPS}_{#1}}
\renewcommand{\P}[1]{P_\text{#1}}
\newcommand{\I}[1]{I_\text{#1}}
\newcommand{\n}[1]{n_\text{#1}}
\newcommand{\N}[1]{N_\text{#1}}
\renewcommand{\t}[1]{t_\text{#1}}

%%% Allgemeine Daten %%%
\newcommand{\autorOne}{Alexander Müller}
\newcommand{\matrikelnrOne}{940597}
\newcommand{\adresseOne}{Gartlager Weg 37, \\49086 Osnabrück}
\newcommand{\telOne}{01578 1901578}
\newcommand{\emailOne}{alemuell@uos.de}

\newcommand{\autorTwo}{János Sebestyén}
\newcommand{\matrikelnrTwo}{939525}
\newcommand{\adresseTwo}{Gartlager Weg 37, \\49086 Osnabrück}
\newcommand{\telTwo}{}
\newcommand{\emailTwo}{jsebesty@uos.de}

%%% Titlepage Expose %%%
\newcommand{\titel}{Ökologische Nachhaltigkeit durch \enquote{Nutzen statt Besitzen}?}
\newcommand{\stitle}{Entwicklung eines Modells zur Ableitung von Kriterien für die Senkung des Umweltverbrauchs durch gemeinschaftliche Produktnutzung}
\newcommand{\art}{Master-Arbeit}
\newcommand{\fachgebiet}{Angewandte Systemwissenschaft}
\newcommand{\betreuungOne}{Katrin Bienge}
\newcommand{\betreuungTwo}{Prof. Dr. Claudia Pahl-Wostl}
\newcommand{\institutOne}{Wuppertal Institut für Klima, Umwelt, Energie}
\newcommand{\institutTwo}{Institut für Umweltsystemforschung, Osnabrück}
\newcommand{\ort}{Osnabrück}

%%% END Article customizations

\begin{document}
\begin{titlepage}
  \begin{center}
    \includegraphics[scale=0.5]{Logo-Uni-Osnabrueck.jpg}\\[2ex]

    \vfill
    \LARGE{\textbf{\art}}\\[1.5ex]
    \huge{\textbf{\textsc{\titel}}}\\[1.0ex]
    \LARGE{\textbf{\stitle}}\\[2ex]
    \vspace{1cm}

    \normalsize
    vorgelegt von:\\[12pt]

      \begin{minipage}[]{0.30\textwidth}
        \textbf{\autorOne}\\
        Matrikelnummer: \matrikelnrOne\\
        \emailOne\\
        \adresseOne\\
      \end{minipage}
      \hspace{0.15\textwidth}
      \begin{minipage}[]{0.3\textwidth}
        \textbf{\autorTwo}\\
        Matrikelnummer: \matrikelnrTwo\\
        \emailTwo\\
        \adresseTwo\\
      \end{minipage}

      \vspace{2cm}
      \begin{minipage}[t]{0.45\textwidth}
        \begin{flushright}
          Datum:              \\[1.2ex]
          Ort:                \\[1.2ex]
          Betreuende Personen:\\
        \end{flushright}
      \end{minipage}
      \hspace{0.01\textwidth}
      \begin{minipage}[t]{0.45\textwidth}
        \begin{flushleft}
          \today\\[1.2ex]
          \ort\\[1.2ex]
          \betreuungOne\\
          (\institutOne) \\[1.2ex]
          \betreuungTwo\\
          (\institutTwo) \\
        \end{flushleft}
      \end{minipage}
  \end{center}
\end{titlepage}
\tableofcontents



\section{Einleitung}
\subsection{Thema}
Die heutigen industrialisierten Kulturen sind durch einen umfangreichen
individuellen Bedarf an Produkten und Dienstleistungen gekennzeichnet. Jeder
Haushalt, jede Person verfügt über zahlreiche, in der Summe höchst
ressourcenintensive Güter, was gesamtgesellschaftlich zu einer hohen, nicht mehr
tragfähigen Umweltbelastung führt.

Ein Problem heutiger Konsumkulturen, an dem Nachhaltigkeitsstrategien ansetzen
können, ist die ineffiziente Nutzung der produzierten Gütermenge: So wurde eine
durchschnittliche Bohrmaschine bei ihrer Entsorgung lediglich 45 Stunden
genutzt, obwohl sie 300
Stunden hätte genutzt werden können \parencite[S. 31]{behrendt_oko-rent_2000}. Autos
sind im Durchschnitt mit nur 1,6 Personen besetzt, obwohl sie in der Regel
Platz für vier Menschen bieten \parencites[zitiert nach: ]{baum_untersuchungen_1994}[S. 43]{behrendt_car-sharing_2000}. Diese Beispiele
verdeutlichen, dass bei der individuellen Nutzung die maximale Nutzungsdauer und
Auslastung von Produkten häufig nicht erreicht werden. Formen gemeinschaftlicher
Nutzung, um die es in unserer Arbeit geht, können durch
Nutzungsdauerverlängerung und Nutzungsintensivierung die daraus resultierenden
Effizienzpotentiale heben und dadurch zu einer Reduktion der Umweltbelastung
führen.

Zahlreiche dieser Nutzungsformen, die sich über alle Produktgruppen erstrecken,
sind bereits seit geraumer Zeit etabliert. Zum einen finden sich hier alle Formen des Gebrauchtwarenmarkts von Second-Hand-Läden für Kleidung über Geschäfte des An- und Verkaufs bis hin zu Flohmärkten. Zum anderen existiert eine breite Palette an Miet- und Verleihsystemen, wie zum
Beispiel Bibliotheken, Videotheken oder Autovermietungen. Neben diesen kommerziellen Angeboten gibt es auch vielfältige Formen der privaten gemeinschaftlichen Nutzung, wie
beispielsweise die gemeinsame Nutzung von Waschmaschinen, Rasenmähern oder Kühlschränken in
Wohnanlagen und Wohngemeinschaften, oder die Bildung von Fahrgemeinschaften.

Mit dem Aufkommen der heutigen Informations- und Kommunikationstechnologien
erhalten viele dieser Formen eine neue, ungeahnte Dimension: Das Internet
vergrößert die räumlichen Entfernungen und die Anzahl beteiligter
Teilnehmer*innen der herkömmlichen Gebraucht-, Miet- und Verleihmärkte um ein
Vielfaches. Durch das große Angebot und die einfache Möglichkeit der gezielten
Suche verringert sich der Aufwand zum Zustandekommen des Warenaustauschs.
Ermöglicht durch den einfachen, ortsungebundenen Zugang zum Internet und die
Vernetzung von Menschen mit ähnlichen Bedürfnissen entstanden ganz neue Formen
der gemeinschaftlichen Nutzung: So ermöglichen es entsprechende
Internetplattformen beispielsweise, anstatt in einem Hotel zu übernachten, auf
einer Reise ganz bequem bei Privatpersonen unterzukommen oder mit Hilfe eines
Smartphones ganz einfach das nächste Free Floating Car zu finden.

Hinter diesen neuen Formen werden unter Verwendung der Begriffe
\enquote{collaborative consumption} oder \enquote{Kokonsum} soziale Innovationen
gesehen, die zu mehr Nachhaltigkeit bei gleichbleibendem oder sogar gesteigertem
Wohlstand führen können. Und aus Unternehmensperspektive stellen sich diese
neuen Nutzungsformen als wichtiger neuer ökonomischer Wachstumsmarkt -- die
\enquote{Sharing Economy} -- dar. In der Forschung wird das Thema mit jeweils
leicht unterschiedlichen Perspektiven unter den Begriffen \enquote{Nutzen statt
Besitzen}, \enquote{eigentumsersetzender Konsum}, \enquote{eco services} und
\enquote{product-service systems} diskutiert. 

Im Vergleich zur Euphorie, die von der Sharing Economy ausgeht, zeigt sich aus wissenschaftlicher Perspektive hinsichtlich der Umweltwirkungen gemeinschaftlicher Nutzungssysteme ein differenzierteres Bild: Zum einen werden die positiven Primäreffekte der
Nutzungsdauerverlängerung und -intensivierung um weitere sekundäre
Ressourceneinsparmöglichkeiten ergänzt. So kann es beispielsweise bei der
gemeinschaftlichen Nutzung einer Waschmaschine sinnvoll sein, eine
halbgewerbliche Maschine mit entsprechend längerer Lebensdauer und höherer
Effizienz zu verwenden. Zum anderen werden die durch gemeinschaftliche Nutzung
ermöglichten Ressourceneinsparungen jedoch auch durch negative Umweltwirkungen
begleitet: Beispielsweise müssen für die Koordination der gemeinschaftlichen
Nutzung und den Transport der genutzten Produkte zwischen den Nutzer*innen zusätzliche
Ressourcen aufgewendet werden. Und auch positive sekundäre Effekte können mit negativen Nebenwirkungen einhergehen, wie beispielsweise für die Produktion einer halbgewerblichen Waschmaschine im Vergleich zu einer handelsüblichen
mehr Ressourcen benötigt werden.

Nicht zuletzt wirken sich die aus der Perspektive der Sharing Economy positiven
finanziellen Einsparungen bei der gemeinschaftlichen Nutzung auf die Nachfrageseite der
Konsumenten aus: Das eingesparte Geld kann in weiteren Konsum und damit
zusätzlichen Ressourcenverbrauch investiert werden (Rebound-Effekt). Außerdem
bleiben in einer solchermaßen veränderten Ökonomie, welche den Zugang und nicht
den Besitz in den Vordergrund stellt, die Bedürfnisstrukturen nicht unberührt:
Die Möglichkeit, auf viele (Luxus-)Güter zuzugreifen, könnte mit einem
verstärkten Wunsch einhergehen, dies auch tatsächlich zu tun.

Welche Netto-Umweltwirkungen schließlich durch die gemeinschaftliche Nutzung zu
erwarten sind, hängt in hohem Maße von dem betrachteten Produkt und dem konkreten
Nutzungssystem ab. Entscheidende Produktmerkmale sind zum Beispiel, ob während der Nutzung des Produkts Ressourcen verbraucht werden (wie bei Elektrogeräten), oder wie groß das Verhältnis zwischen typischer individueller und maximal möglicher Nutzungsdauer ist (im oben genannten Beispiel der Bohrmaschine 1:6). Zu den wichtigen Eigenschaften des Nutzungssystems zählen eventuell anfallende zusätzliche Transporte (wie beim Second-Hand-Handel) und die Auslastung während der Nutzung (zum Beispiel 3 Personen pro Auto). Die Zusammenhänge zwischen Merkmalen der Produkte und Nutzungssysteme und den resultierenden Umweltwirkungen zu analysieren ist der Gegenstand unserer Arbeit.

\subsection{Forschungsstand}\label{sec:forschung}
Unter die neuere Literatur, die sich mit Umwelteffekten gemeinschaftlicher Produktnutzung beschäftigt, fällt zunächst eine Reihe von Publikationen im Zusammenhang mit zwei größeren Forschungsprojekten um die Jahrtausendwende. Gegen Ende der Nuller Jahre folgen einige weitere Publikationen aus verschiedenen Forschungszusammenhängen. Schließlich wird das Thema auch in der ökonomischen Fachliteratur behandelt. Die einzelnen Bereiche werden im Folgenden grob umrissen.

In den Jahren 1997-2000 bearbeitete das Institut für ökologische
Wirtschaftsforschung (IÖW) das vom Bundesministerium für Bildung und Forschung geförderte
Forschungsvorhaben \enquote{Ökologische Entlastungspotentiale,
  Umsetzungsprobleme und Entwicklungsperspektiven von Strategien zur
Nutzungsdauerverlängerung (NV) und Nutzungsintensivierung (NI)}.
\textcite{scholl_produkte_1998} steckt zunächst den
begrifflichen und theoretischen Rahmen für die Beschreibung gemeinschaftlicher
Nutzungsformen und deren ökologischen Bewertung ab. Hieran
schließen sich eine Fallstudie zur privaten Textilwäsche
\parencite{hirschl_produkte_2000} und eine zum Wintersport
\parencite{konrad_produkte_2000} an, wobei hier insbesondere der
Frage nachgegangen wird, wie das ökologische Entlastungspotenzial
von Nutzungskonzepten im Vergleich zur herkömmlichen individuellen
Nutzung einzuschätzen ist. Eine zusammenfassende Darstellung der
Projektergebnisse gibt schließlich
\textcite{hirschl_nachhaltige_2001}. Aus einem Vergleich der beiden
Fallstudien werden dort allgemeinere Erkenntnisse für die
Nachhaltigkeitsstrategien Nutzungsdauerverlängerung und
-Intensivierung abgeleitet.

In einem ähnlichen Zeitraum wie die Studie des IÖW wurde
von verschiedenen anderen
Forschungseinrichtung ein EU finanziertes Forschungsprojekt
unter dem Titel \enquote{Eco-Services for Sustainable Development in
the European Union} durchgeführt. Diese Studie hat einen sehr ähnlichen Aufbau, betrachtet
allerdings zum Teil andere Fallbeispiele und lenkt durch
den internationalen Kontext den Blick auch auf länderspezifische
Unterschiede. Die vertiefenden Fallstudien, welche ökologische Betrachtungen
einschließen, wurden in den  Bereichen Heimwerken, Baueigenleistung
und Gartenpflege \parencite{behrendt_oko-rent_2000}, nachhaltiges
Waschen \parencite{behrendt_nachhaltig_2000} und
Car-Sharing \parencite{behrendt_car-sharing_2000} durchgeführt. Eine
Systematisierung der betrachteten Produktdienstleistung und eine
vergleichende Darstellung der Fallstudien, Zusammenfassung des
Projekts liefert schließlich \textcite{behrendt_eco-service_2003}.

In der jüngeren Literatur finden sich weitere Publikationen,
die sich zum einen Teil theoretisch dem Thema gemeinschaftlicher
Nutzungsformen zuwenden
\parencites{scholl_marketing_2009}{scholl_nutzen_2012} und zum
anderen Teil empirisch-analytisch die Umwelteffekte ausgewählter
Produkt-Nutzungs-Systeme untersuchen
\parencites{rabelt_nachhaltiger_2007}{erdmann_quantifizierung_2011}.
Erwähnenswert ist, dass \textcite{scholl_marketing_2009} eine
umfassende Liste der bei gemeinschaftlichen Nutzungsformen 
auftretenden positiven und negativen Umwelteffekte aufstellt. Er
gliedert diese in solche, die direkt durch die
Nutzungsform auftreten und solche, die indirekt aus veränderten
Nachfragemustern resultieren.
Während die meisten der bisher zitierten Studien die letztgenannte Gruppe der Nachfrage-Effekte
nicht näher quantifizieren, unternimmt
\textcite{erdmann_quantifizierung_2011}  genau das anhand des
online-basierten Gebrauchtwarenhandels.

Auch in der ökonomischen Literatur lassen sich Autor*innen finden, die sich mit gemeinschaftlichen Nutzungsformen -- unter Verwendung einer deutlich formalisierteren Methodik --
auseinandersetzen. \textcite{yokoo_economic_2010} entwickelt ein
mikroökonomisches Modell der Wiederverwendung um Effekte von Second-Hand-Märkten auf die Wohlfahrt der Konsument*innen und die Abfallmenge einer Wirtschaft zu untersuchen.
\textcite{thomas_environmental_2011} stellt ein ähnliches
Modell vor, lenkt jedoch den Blick stärker auf die ökologischen Effekte auf Second-Hand-Märkten, in dem sie das Marktmodell mit einem einfachen Modell für die Lebenszyklusbilanz verknüpft. Als Anwendungsbeispiel wird hier der Gebrauchtwarenhandel mit Büchern betrachtet.

Während diese Zusammenstellung der Literatur sich auf die für unsere Arbeit relevanten Inhalte konzentriert, sind die angegebenen Publikationen in der Regel wesentlich umfassender angelegt. So werden neben den Umweltwirkungen je nach Studie zahlreiche weitere Aspekte behandelt -- wie beispielsweise Handlungsempfehlungen, Strategien zur Förderung, mögliche Hemmnisse, Marktanalysen, Anschlussfähigkeit, Typisierung von Konsument*innen und die
Motivation für gemeinschaftliche Nutzungsformen.

Generell stellen wir fest, dass zu dem Thema \enquote{gemeinschaftliche Nutzungsformen}
sowohl auf empirischer als auch auf theoretisch-konzeptioneller Ebene bereits einige Forschung
betrieben wurde, auf die wir mit unserer Arbeit aufbauen können. In Hinblick auf einige Aspekte sehen wir jedoch weiteren Forschungsbedarf. Zum einen beschränkt sich die Quantifizierung der Umwelteffekte in den meisten angegebenen Quellen auf spezielle Produkte und Nutzungssysteme. Eine Ausnahme bilden die formalen Modelle der ökonomischen Fachliteratur, die allerdings wiederum nur spezielle Umwelteffekte abbilden. Der Versuch einer allgemeinen Betrachtung wurde unseres Wissens nach noch nicht unternommen. Damit einhergehend weisen die Einzeluntersuchungen kein einheitliches methodisches Vorgehen auf, sondern variieren in den verwendeten Zielgrößen, Betrachtungsrahmen und Berechnungsmethoden. Es werden in der Regel Einzelfallberechnungen durchgeführt, ohne den Einfluss der (implizit) verwendeten Parameter zu untersuchen. Was die Zielgrößen angeht, so konzentriert sich ein Großteil der Literatur auf Emissionen, während die Ressourcen-Seite eher vernachlässigt wird. Diese Forschungslücke möchten wir mit unserer Arbeit schließen.

\subsection{Forschungsziel und Fragestellung}
Das Hauptziel unserer Arbeit ist die Entwicklung von Kriterien für die Senkung des Umweltverbrauchs durch gemeinschaftliche Produktnutzung. Darunter fällt eine systematische und möglichst allgemeine Betrachtung der verschiedenen Umweltwirkungen von gemeinschaftlichen Nutzungsformen. Ein methodologisches Ziel der Arbeit ist es, einen einheitlichen Methodenrahmen für die Untersuchung verschiedener Produkte und Nutzungssysteme zu entwickeln.

Eine mit diesem Forschungsziel korrespondierende Fragestellung lautet:
\begin{itemize}
  \item Unter welchen Umständen kann der Umweltverbrauch eines Produktes durch gemeinschaftliche Nutzung gegenüber der individuellen Nutzung gesenkt werden?
\end{itemize}
Etwas feinteiliger betrachtet, lässt sich diese Fragestellung in die folgenden zwei Teilfragen unterteilen:
\begin{itemize}
  \item Was sind die Mechanismen, die den Umweltverbrauch bei gemeinschaftlicher Nutzung bestimmen und wie wirken diese? \todo{"wie wirken diese?": spezifizieren / besser beschreiben, was wir meinen}
  \item Welche Eigenschaften des Produkts und der Nutzungsform beeinflussen die Wirkung dieser Mechanismen und wie fließen sie ein?
\end{itemize}
\section{Methodisches Vorgehen}
		\subsection{Modellansatz}
		Eine Grundannahme, unter der wir an die Bearbeitung der Fragestellung herangehen, ist, dass der Umweltverbrauch eines Produktnutzungssystems durch Merkmale des Produktes (einschließlich seines Produktions- und Entsorgungssystems) und Merkmale des Nutzungssystems bestimmt wird. Ziel der Untersuchung ist es also, den Zusammenhang zwischen diesen Merkmalen und dem Umweltverbrauch zu bestimmen.
		
		Die Zielgröße \enquote{Umweltverbrauch} operationalisieren wir in unserer Untersuchung durch die Menge der eingesetzten Ressourcen und der ausgestoßenen Emissionen. Für die Ressourcenseite werden wir auf das MIPS-Konzept (\cite{schmidt-bleek_maia:_1998}) zurückgreifen. Bei der Betrachtung der Emissionen beschränken wir uns auf Treibhausgase (CO$_{2}$-Äquivalente). Je nach Wirkungsmechanismus wird es erforderlich sein, unterschiedliche Betrachtungsrahmen für die Bilanzierung heranzuziehen. So können die Umwelteffekte beispielsweise auf eine einzelne Produktnutzung, auf den gesamten Konsum einer Person oder gar auf den Konsum einer ganzen Volkswirtschaft bezogen werden. Wir streben eine möglichst einheitliche Betrachtung an, die allerdings an ihre Grenzen stoßen wird, da beispielsweise Effekte einer gemeinschaftlichen Produktnutzung auf andere Konsumbereiche mit einer Produkt-zentrierten Betrachtung nicht erfasst werden können.
		
		Um die mögliche Senkung des Umweltverbrauchs durch gemeinschaftliche Nutzung zu berechnen, wird als Referenzszenario jeweils die individuelle Nutzung herangezogen. Ergänzend dazu kann das Resultat auch mit den Effekten durch andere Maßnahmen zur Senkung des Umweltverbrauchs ins Verhältnis gesetzt werden, sofern dies sinnvoll erscheint. Beispielsweise könnten technische Effizienzsteigerungen oder Konsumreduzierungen betrachtet werden.
		
		Die in Tabelle \ref{tab:Mechanismen} aufgeführten Wirkungsmechanismen möchten wir prinzipiell in unsere Betrachtung miteinbeziehen. Die Liste basiert auf \cite{scholl_marketing_2009}, wobei wir einige der dort aufgeführten Mechanismen aus Plausibilitätsüberlegungen nicht aufgenommen und andere selbst hinzugefügt haben. \todo{Veränderungen genau benennen und begründen.} %  z.B. Umbenennungen von Begriffen, Weglassen und Hinzunehmen von Effekten. Idee: Textlich die Effekte anhand von Scholl 2009 durchgehen und grob beschreiben, dort auch direkt Veränderungen beschreiben und begründen. Zusammenfassend eine Tabelle mit den von uns betrachteten Effekten.
		
		\begin{table}[h]
			\begin{tabular}{p{7cm}p{7cm}}
				 \toprule
				 \multicolumn{2}{l}{\textbf{Umweltauswirkungen durch die Nutzung}} \vspace{0.2 cm} \\
				 \textbf{positiv} & \textbf{negativ} \\
				 \midrule
				 Nutzungsdauer\-verlängerung \vspace{0.2cm} &  Zusätzliche Transaktionen \vspace{0.2cm} \\
				 Einsatz langlebiger Produkte \vspace{0.2cm} &  zusätzlicher Ressourcenverbrauch für Langlebigkeit \vspace{0.2cm} \\
				 Verwendung verbrauchsarmer bzw. leistungsstarker Geräte \vspace{0.2cm} &  Beschleunigte Ausmusterung \vspace{0.2cm} \\
				 Maximierung der Geräteauslastung \vspace{0.2cm} & Tauschbedingter Verschleiß \vspace{0.2cm} \\
				 Berücksichtigung des technischen Fortschritts \vspace{0.2cm} & \vspace{0.2cm} \\
				 Wartung / Reparaturen \vspace{0.5cm}& \\
				 \midrule
				 \multicolumn{2}{l}{\textbf{Umweltauswirkungen durch Nachfrageänderung}} \vspace{0.2 cm} \\
				 \textbf{positiv} & \textbf{negativ} \\
				 \midrule
				 Nachfrageverringerung durch höhere Kostentransparenz \vspace{0.2cm} & Erleichterter Produktzugang \vspace{0.2cm} \\
				 Vermeidung von Fehlkäufen \vspace{0.2cm} & Wunsch nach Eigentum \vspace{0.5cm} \\
				  & Rebound-Effekt \vspace{0.2cm} \\
				 \bottomrule
			\end{tabular}
			\caption{Betrachtete Mechanismen für die Umweltauswirkungen durch gemeinschaftliche Nutzung. Quelle: verändert nach \cite{scholl_marketing_2009}.}
			\label{tab:Mechanismen}
		\end{table}
		
		Nicht alle Mechanismen sind für alle Produkte und Nutzungssysteme relevant. Welche Mechanismen für eine bestimmte Produkt-Nutzungssystem-Kombination zum Tragen kommen, hängt von spezifischen Annahmen über die Produkte und Nutzungssysteme ab, die im Laufe der Arbeit getroffen werden. Auch sei erwähnt, dass nicht alle Mechanismen ausschließlich bei gemeinschaftlichen Nutzungsformen wirken, sondern auch bei individueller Nutzung auftreten können.
		
		Um die Mechanismen und deren Auswirkungen quantifizierbar und vergleichbar zu machen, werden wir ein Modell entwickeln. Der Modellzweck ist, die Zusammenhänge zwischen den Merkmalen der Produkte und Nutzungssysteme und dem Umweltverbrauch abzubilden. Während das Modell letztlich möglichst allgemein anwendbar sein soll, lassen wir uns bei der Modellentwicklung durch zwei Fallbeispiele leiten. Sie dienen der Veranschaulichung der abstrakten Zusammenhänge und gewährleisten die praktische Anwendbarkeit des Modells. Zudem kann die Fokussierung auf konkrete Fallstudien die Modellanalyse vereinfachen, da der potentiell sehr große Parameterraum etwas eingeschränkt werden kann. Auswahlkriterien für die Fallstudien waren die Verfügbarkeit von Literatur und die Abdeckung möglichst unterschiedlicher Produkteigenschaften, damit alle potentiellen Mechanismen zur Anwendung kommen. Die Entscheidung fiel auf die zwei Produkte Waschmaschinen und Bücher.
		
		\subsection{Fallstudien}
\section{Modell}
		\subsection{Einzeleffekte}
			\subsubsection{Erhöhung der Produktauslastung} % Ursprünglich:
                                              %"Maximierung der Geräteauslastung"
      \noteS{hmmm, ich frage mich gerade, an
        welcher Stelle, wir diesen Mechanismus eigentlich im Kontext
      gemeinschaftlicher Nutzungsformen erklären}
			Unter dem Begriff \enquote{Produktauslastung} verstehen wir die
      Intensität, mit der ein Produkt während einer einzelnen Nutzung eingesetzt
      wird. Im Fall einer Waschmaschine entspricht dies dem Gewicht der Beladung
      während eines Waschgangs. Bei Büchern könnte analog die Anzahl der
      Personen, die durch Vorlesen gleichzeitig in den Genuss eines Buches
      kommen, betrachtet werden. \noteS{bin mir nicht sicher, ob man dieses
      Vorlese-Szenario so gut versteht, oder ob das eher verwirrt} Im Folgenden
      bleiben wir beim Beispiel der Waschmaschine, da die Relevanz des
      Vorlese-Szenarios als vernachlässigbar angesehen werden
      kann.
			
			Formaler betrachtet ist die absolute Produktauslastung $A$ der Service,
      der durch eine einzelne Produktnutzungseinheit erzielt wird -- im Fall der
      Waschmaschine also das Gewicht der gewaschenen Wäsche pro Waschgang. Nimmt
      man an, dass ein Produkt über eine gewisse Kapazität $K$ (die maximale
      absolute Auslastung) verfügt, so kann zur Betrachtung der relativen
      Produktauslastung $a$ übergegangen werden: $a = A/K$. Im Fall einer
      Waschmaschine ist die Kapazität das maximale Beladungsgewicht und die
      relative Produktauslastung der Anteil des tatsächlichen am maximalen
      Beladungsgewicht während eines Waschgangs.\noteS{über diese
      Satzkonstruktion bin ich gestolpert}
			
			Damit ergibt sich der folgende Zusammenhang zwischen der Quantität der Produktnutzung $N$, der relativen Produktauslastung $a$, der Kapazität $K$ und dem erzielten Service $S$:
%
			\begin{align}
				\label{eq:S(a)}
				S &= N \cdot a \cdot K \\
				\label{eq:N(a)}
				\Leftrightarrow N &= \frac{S}{a \cdot K}
			\end{align}
%
			Diese Gleichung muss der Nebenbedingung genügen, dass die abgerufene Produktnutzungs-Menge $N$ den gesamten Nutzungsvorrat $N_{\text{max}}$ des Produktnutzungssystems nicht übersteigt. Dieser setzt sich aus den Nutzungsvorräten $n_{\text{max}}^i$ der eingesetzten Produkte zusammen. Im Fall von $P$ identischen Produkten ergibt sich:
%
			\begin{equation}
				\label{eq:Nebenbedingung N_max}
				N \leq N_{\text{max}} = P \cdot n_{\text{max}}
			\end{equation}
%
			In Gleichung \ref{eq:N(a)} spiegelt sich bereits ein erster (positiver) Effekt einer Produktauslastungs-Erhöhung wieder, nämlich eine proportionale \textit{Verringerung der Produktnutzung} $N$ unter der Annahme eines gleichbleibenden Services $S$. Es können jedoch unter bestimmten Umständen -- die im Analyse-Teil ausführlich erörtert werden \todo{Referenz einfügen} -- bis zu drei weitere Effekte zum Tragen kommen: erstens ein Mehrbedarf an Inputs je Produktnutzung, zweitens eine Reduktion des Nutzungsvorrats und drittens eine Einsparung von Produkten.
			
			Der zweite (negative) Effekt -- ein \textit{Mehrbedarf an Inputs je Produktnutzung} $i_N$ -- tritt dann auf, wenn durch eine erhöhte Produktauslastung mehr Energie oder andere Betriebsmittel bei der Nutzung des Produkts eingesetzt werden müssen. So steigt bei einer Waschmaschine beispielsweise der Bedarf an elektrischer Energie und an Waschmittel, wenn die Maschine voller beladen wird. Formal kann dieser Zusammenhang durch eine funktionale Abhängigkeit der Inputs je Produktnutzung $i_N$ von der Auslastung $a$ beschrieben werden:
%
			\begin{equation*}
				i_N = i_N (a)
			\end{equation*}
%
			Wir gehen davon aus, dass eine erhöhte Auslastung mit höheren oder wenigstens gleichbleibenden Inputs je Produktnutzung einhergeht. Das heißt, der Funktionsverlauf von $i_N (a)$ kann als monoton steigend angenommen werden:
%
			\begin{equation*}
				\frac{\partial i_N}{\partial a} \geq 0
			\end{equation*}
			
			Zu dem dritten (negativen) Effekt -- einer \textit{Reduktion des Nutzungsvorrats} $N_{\text{max}}$ -- kommt es dann, wenn eine erhöhte Produktauslastung zu einer verstärkten Abnutzung des Produkts führt. So wird eine Waschmaschine beispielsweise nach einer geringeren Anzahl von Waschgängen aus der Nutzung genommen werden müssen, wenn sie im Durchschnitt voller beladen wurde. Auch dieser Zusammenhang kann durch eine funktionale Abhängigkeit abgebildet werden:
%
			\begin{align}
				\label{N_max(a)}
				n_{\text{max}} &= n_{\text{max}} (a) \\
				\Rightarrow N_{\text{max}} &= N_{\text{max}} (a) = P \cdot n_{\text{max}} (a)
			\end{align}
			\todo{Nummerierung der zweiten Gleichung entfernen}
%
			Analog zu den monoton steigenden Inputs pro Nutzungseinheit gehen wir davon aus, dass der Nutzungsvorrat durch eine erhöhte Auslastung geringer wird, oder höchstens gleich bleibt:
%
			\begin{equation*}
				\frac{\partial n_{\text{max}}}{\partial a} \leq 0
			\end{equation*}

			Der vierte (positive) Effekt -- eine \textit{Einsparung von Produkten} --
      tritt dann auf, wenn durch eine erhöhte Auslastung einzelne Produkte im
      Produktnutzungssystem nicht mehr benötigt werden und daher weder
      bereitgestellt, noch entsorgt werden müssen. Beispielsweise kann eine
      Waschmaschine eingespart werden, wenn zwei Personen, die bisher jeweils
      eine eigene Waschmaschine nutzten und nur halbvoll beluden, nun gemeinsam
      eine Maschine nutzen, die immer voll beladen wird. Während die ersten drei
      Effekte auch bei einer individuellen Nutzung auftreten können, wenn dort
      die Auslastung erhöht wird, wird der hier besprochene Effekt erst durch
      eine kollektive Nutzung ermöglicht. \noteS{geht auch individuell}
			
			Die konkrete Organisation der kollektiven Produktnutzung ist ein komplexes Problem, das von vielen Faktoren wie der zeitlichen Struktur des Bedarfs, räumlichen Gegebenheiten und organisatorischen Einschränkungen bestimmt wird. Zudem wird nicht jedes ungenutzte Produkt auch tatsächlich zeitnah aus dem Nutzungssystem genommen. Daher lässt sich aus der Produktauslastung allein nicht ableiten, wie viele Produkte durch die kollektive Nutzung eingespart werden können. Jedoch kann eine Aussage darüber getroffen werden, wie viele Produkte \textit{maximal} eingespart werden können, indem die folgende Überlegung über die Anzahl der mindestens notwendigen Produkte angestellt wird: Um den Bedarf zu decken, muss die Nebenbedingung aus Ungleichung \ref{eq:Nebenbedingung N_max} erfüllt sein. Umgestellt nach $P$ schreibt sich diese als:
%
			\begin{equation*}
				P \geq \frac{N}{n_\text{max}}
			\end{equation*}
%
			Mit Gleichung \ref{eq:N(a)} ist dies äquivalent zu:
%
			\begin{equation}
				\label{eq:P_min}
				P \geq \frac{S}{n_\text{max} \cdot a \cdot K} =: P_\text{min} (a)
			\end{equation}
%
			Dabei ist zu beachten, dass $P$ ganzzahlig sein muss, das heißt es muss gegebenenfalls aufgerundet werden. Da die Ungleichung dabei ihre Richtigkeit nicht verliert, verzichten wir auf diesen Schritt um die Analyse zu vereinfachen. Die eingesparte Menge an Produkten durch eine erhöhte Auslastung ergibt sich aus der Differenz der ursprünglich benötigten und der nach der Auslastungserhöhung eingesetzten Produkte.
			
			Nachdem nun die verschiedenen Effekte einer Erhöhung der Produktauslastung formal beschrieben sind, kann nun untersucht werden, wie sie gemeinsam auf die Zielgröße MIPS einwirken. Dazu werden im Folgenden zunächst die beiden Komponenten des MIPS, die Material-Inputs $I$ und die Service-Menge $S$, einzeln betrachtet und anschließend alle betrachteten Aspekte zu einem Modell zusammengeführt.
			
			Die Material-Inputs setzen sich aus nutzungsbedingten Inputs $I_N = N \cdot i_N(a)$, produktbezogenen Inputs $I_P = P \cdot i_P$  und konstanten Inputs $I_{\text{fix}}$ zusammen:
%
			\begin{equation*}
				I = N \cdot i_N(a) + P \cdot i_P + I_{\text{fix}}
			\end{equation*}			
%
			Die nutzungsbedingten Inputs $I_N$ sind solche, die unmittelbar bei der
      Nutzung anfallen, wie beispielsweise Energie oder andere Betriebsmittel.
      Die produktbezogenen Inputs $I_P$ fallen bei der Bereitstellung und
      Entsorgung eines Produkts an und sind daher unabhängig von der Anzahl der
      Nutzungen. Zu den konstanten Inputs $I_\text{fix}$ schließlich zählen alle Inputs, die weder von der Anzahl der Produkte, noch von der Anzahl der Nutzungen direkt abhängen. Diese fallen beispielsweise für die Bereitstellung der Infrastruktur, wie etwa der Räumlichkeiten, eines Nutzungssystems an.
			
			Auf der Seite der Service-Menge ist mit Gleichung \ref{eq:S(a)} bereits eine Bestimmungsgleichung gegeben, die beschreibt, welche Service-Menge von einem Nutzungssystem angeboten wird:
%
			\begin{equation*}
				S = N \cdot a \cdot K
			\end{equation*}
%
			Die Kapazität $K$ wird in Bezug auf $a$ als konstant angenommen. Sie ist als Kapazität eines einzelnen Produkts zu verstehen, selbst wenn das betrachtete Produktsystem mehrere Produkte gleichen Typs umfasst. Die abgerufene Produktnutzungs-Menge $N$ stellt in diesem Modell eine Zwischengröße dar, die aus anderen Größen abgeleitet werden kann, wie im Folgenden noch gezeigt wird.
			
			Neben dieser Betrachtung des Angebots lässt sich die Service-Menge auch von der Nachfrageseite her betrachten: In einem konkreten Nutzungssystem besteht eine bestimmte Nachfrage nach Service. Diese wird von der Anzahl der Personen, dem individuellen Servicebedarf und dem Betrachtungszeitraum abhängen, was in diesem Teil-Modell aber nicht weiter von Belang ist, da diese Größen in Bezug auf die Auslastung als konstant angenommen werden können. Insofern stellt die Service-Nachfrage $S_D$ hier eine Konstante dar und es ergibt sich eine zweite Bestimmungsgleichung:
%
			\begin{equation}
				\label{S_D}
				S = S_D
			\end{equation}
%
			%Damit lässt sich Gleichung \ref{eq:S(a)} umformen zu
%
			%\begin{equation}
			%	N = \frac{S_D}{a \cdot K}
			%\end{equation}			
%
			%und die Nebenbedingung \ref{eq:Nebenbedingung N_max} wird unter Anwendung von Gleichung \ref{N_max(a)} zu
%
			%\begin{equation}
			%	N_{\text{max}}(a) \geq \frac{S_D}{a \cdot K}
			%\end{equation}
%
			Nun liegen alle Voraussetzungen vor um das MIPS für eine gegebene Auslastung $a$ zu bestimmen:
%
			\begin{align*}
				\text{MIPS}_a &= \frac{I}{S} = \frac{N \cdot i_N(a) + P \cdot i_P + I_{\text{fix}}}{S} \\[10pt]
				&= \frac{N \cdot i_N(a)}{S} + \frac{P \cdot I_P}{S} + \frac{I_{\text{fix}}}{S} = \frac{N \cdot i_N(a)}{N \cdot a \cdot K} + \frac{P \cdot I_P }{S_D} + \frac{I_{\text{fix}}}{S_D} \\[10pt]
				&= \frac{i_N(a)}{a \cdot K} + \frac{P \cdot I_P }{S_D} + \frac{I_{\text{fix}}}{S_D}
			\end{align*}	
			
			Die Nebenbedingung aus Ungleichung \ref{eq:P_min} lässt sich unter Verwendung der Gleichungen \ref{N_max(a)} und \ref{S_D} folgendermaßen formulieren:
%
			\begin{equation*}
				P \cdot n_\text{max}(a) \geq \frac{S_D}{a \cdot K}
			\end{equation*}
%
			Zu beachten ist hier, dass sowohl die Produktauslastung $a$ als auch die
      Produktanzahl $P$ freie Variablen sind, die im Zusammenspiel die
      Nebenbedingung erfüllen müssen.\noteS{dieser Fakt sollte sich vielleicht
      auch in der Übersicht wiederspiegeln}
			
			Zusammenfassend lässt sich das folgende Teilmodell als Zwischenergebnis festhalten:
			\\
			
			\begin{mdframed}[frametitle={Erhöhung der Produktauslastung}, frametitlerule=true]
				Teilmodell:
				\begin{equation}
					\text{MIPS}_a = \frac{i_N(a)}{a \cdot K} + \frac{P \cdot I_P }{S_D} + \frac{I_{\text{fix}}}{S_D}
				\end{equation}
				Nebenbedingung:
				\begin{equation}
					P \cdot n_\text{max}(a) \geq \frac{S_D}{a \cdot K}
				\end{equation}
				Funktionseigenschaften:
				\begin{equation}
					\frac{\partial i_N}{\partial a} \geq 0
				\end{equation}
				\begin{equation}
					\frac{\partial n_{\text{max}}}{\partial a} \leq 0
				\end{equation}
			\end{mdframed}
			
			%Wie bereits bei der Herleitung des Teilmodells bereits erwähnt wurde, kann der Mechanismus \enquote{Erhöhung der Produktauslastung} sowohl positive als auch negative Umweltauswirkungen zur Folge haben. Im Modell spiegelt sich dies durch das mehrfache Vorkommen der Produktauslastung $a$ mit zum Teil entgegengesetzter Wirkung auf das MIPS wieder. Unter welchen Umständen die positiven Wirkungen die negativen überwiegen und damit eine Erhöhung der Produktauslastung insgesamt zu einer Verringerung des Umweltverbrauchs führt, wird Gegenstand der Analyse dieses Teilmodells sein. \todo{Referenz einfügen}
			
			
      \subsubsection{Wartung und Reparatur}
      \todo{Beispiele nennen}
      Produkte können während ihrer Nutzungsdauer mit dem Zweck, einer
      Verlängerung ihrer Lebensdauer gewartet und repariert werden. 
      Unter dem Begriff "`Wartung"' verstehen wir dabei eine Pflegetätigkeit an
      einem Produkt, welche durchgeführt wird, um dessen Funktionstüchtigkeit
      lange zu gewährleisten bzw. diese zu verbessern. Das betreffende Produkt
      ist zum Zeitpunkt der Wartung funktionstüchtig, die Wartung ist
      also ein geplanter, vorausschauender Eingriff, dessen Häufigkeit von den
      Nutzer\_innen frei festgelegt werden kann.

      Im Gegensatz dazu verwenden wir den Begriff "`Reparatur"' für
      Tätigkeiten an einem defekten Produkt, mit dem Ziel, dessen
      Funktionstüchtigkeit wieder herzustellen. In diesem Sinne sind
      Reparaturen also Handlungen, die auf Ereignisse reagieren und nicht
      geplant werden können. Es kann lediglich zum Zeitpunkt des Defekts
      entschieden werden, ob die Reparatur vollzogen oder das Produkt entsorgt
      wird.

      Sowohl für Wartungen als auch für Reparaturen werden Material-Inputs
      aufgebracht, welche die Ressourcenbilanz zunächst verschlechtern. Dieser
      Effekt wird jedoch durch eine Verlängerung der Lebensdauer begleitet.
      Bei Wartungen ist diese Verlängerung beschränkt -- Produktteile können
      selbst bei optimaler Pflege defekt werden -- während Reparaturen die
      Lebensdauer theoretisch unbegrenzt verlängern können.
      Führt die Verlängerung der Lebensdauer auch zu einer Verlängerung der
      Nutzungsdauer, so können die Wartungs- und Reparaturarbeiten insgesamt
      ökologisch vorteilhaft sein. Bei Wartungen ist noch ein dritter Effekt zu
      beobachten: da regelmäßige Wartungen zu einer Verbesserung der
      Funktionstüchtigkeit führen, kann dies unter Umständen auch zu einer
      Verringerung der Material-Inputs je Produktnutzung führen, also die
      Ressourcenbilanz zusätzlich verbessern.

      % erster Effekt: Erhöhung der Material-Inputs
      % zweiter Effekt: Erhöhung des Nutzungsvorrats
      % dritter Effekt: Verringerung der Material-Inputs je Produktnutzung

      Eine Zusammenstellung der Unterschiede und Gemeinsamkeiten zwischen
      Reparaturen und Wartungen findet sich in Tabelle
      \ref{tab:wartung_reparatur}. Insbesondere der Unterschied, dass es sich
      bei Wartungen um geplante Tätigkeiten handelt, während Reparaturen auf
      Ereignisse reagierende Handlungen sind, macht es notwendig, für die beiden
      Fälle zwei separate
      Modelle für die Abbildung der genannten Effekte auf die MIPS zu
      entwickeln.

      \begin{table}
        \centering
        \begin{tabular}{lp{4cm}p{4cm}}
          \toprule
                                   & Wartung             &     Reparatur\\
          \midrule
          \textit{Erhöhung der 
                  Lebensdauer}     & ja, aber begrenzt   & ja, theoretisch
                                                                unbegrenzt\\
          \textit{Erhöhung der 
              Material-Inputs}     & ja                  & ja\\

          \textit{Zeitpunkt,
                  Häufigkeit  }    & kann frei bestimmt 
                                     werden              & zufällig,
                                                           vorgegeben\\
          \textit{Einfluss auf 
                  Nutzungsinputs}  & verringert sie      & keiner\\
          \bottomrule
        \end{tabular}
        \caption{Unterschiede und Gemeinsamkeiten zwischen Wartung und Reparatur}
        \label{tab:wartung_reparatur}
      \end{table}

      \paragraph{Wartung}
      Um die genannten drei Effekte höherer Wartungstätigkeiten auf den
      Umweltverbrauch abzubilden, müssen wir zunächst die Wartungstätigkeit
      selber formalisieren. Wir nehmen an, dass regelmäßig, d.h. nach Abruf
      einer bestimmten Menge von Nutzungseinheiten, in dem gesamten
      Nutzungssystem eine Wartung mit durchschnittlichen Wartungsinputs $i_W$
      durchgeführt wird. Unter dieser Annahme können wir die Wartungshäufigkeit
      $w$ einführen, welche die Anzahl Wartungen je Nutzungseinheiten angibt. Da
      maximal nach jeder Produktnutzung gewartet werden kann, ist $w \in
      [0,1]$.
      Die Anzahl Wartungen $W$ ergeben sich dann als Produkt der
      Wartungshäufigkeit und den Nutzungseinheiten. 
      Der erste Effekt, die Erhöhung der Material-Inputs lässt sich nun direkt
      formulieren:
      \begin{equation}
        I_W = W \cdot i_W = N \cdot w \cdot i_W
      \end{equation}
      Hierbei bezeichnet $I_W$ die wartungsbedingten Material-Inputs, die, wie
      zu erkennen ist, direkt proportional mit der Wartungshäufigkeit wachsen.

      Wir formalisieren den zweiten Effekt -- die Verlängerung der Lebensdauer
      -- indem wir annehmen, dass der Nutzungsvorrat $\n{max}$ eines Produkts
      \todo{Annahme: alles gleiche Produkte} monoton mit der Wartungshäufigkeit
      steigt und dies als funktioneller Zusammenhang beschrieben werden kann:
      \begin{equation*}
        \n{max} = \n{max}(w) \quad \text{mit} \quad
        \frac{\partial \n{max}}{\partial w}\geq 0 \quad , \quad
      \end{equation*}
      Bei konstanter Nutzungshäufigkeit $h$ verlängert sich mit steigender
      Wartungshäufigkeit wegen $\t{tech} = \frac{\n{max}(w)}{h}$ die
      Lebensdauer. 

      Neben dem Monotonieverhalten treffen wir noch zwei weitere Annahmen über
      den Verlauf von $\n{max}(w)$:  Zum einen nehmen wir an, dass der
      Nutzungsvorrat nach oben beschränkt ist, also eine Erhöhung der
      Wartungshäufigkeit den Nutzungsvorrat und damit die Lebensdauer nicht
      unbeschränkt erhöhen kann. Zum anderen nehmen wir an,
      dass der Zugewinn am Nutzungsvorrat durch eine Erhöhung der
      Wartungshäufigkeit mit steigender Wartungshäufigkeit abnimmt. In Formeln
      ausgedrückt, können wir schreiben:\todo{schauen, ob diese Annahmen
      später überhaupt benötigt werden}

      \begin{equation*}
        \frac{\partial^2 \n{max}}{\partial w^2}\leq 0 \quad , \quad
        \max_{w \in [0,1]} \n{max}(w) < \infty
      \end{equation*}
      
      Die Verlängerung der technischen Lebensdauer $\t{tech}$ wird dann einen
      positiven Effekt auf die MIPS des betrachteten Produkts haben, wenn
      dadurch mehr der durch das Produkt bereitgestellten Nutzungseinheiten
      verwendet werden. Bei konstanter Nutzungshäufigkeit $h$ trifft das genau
      dann zu,
      wenn die Nutzungsdauer $t$ verlängert wird. Der Zusammenhang zwischen
      Nutzungsdauer und Lebensdauer wurde bereits im Abschnitt zur
      Nutzungsintensivierung \ref{sec:Nutzungsintensivierung} formalisiert (vgl.
      Gleichung \ref{eq:Nutzungsdauer}): 

      \begin{equation*}
        t = \min \left\{\frac{\n{max}(w)}{h}, \t{max}\right\}
      \end{equation*}
      Hieran lässt sich erkennen, dass eine Erhöhung der Wartungshäufigkeit
      genau dann zu einer Verlängerung der Nutzungsdauer führt, wenn $\t{tech} <
      \t{max}$ gilt, das Produkt also aufgrund des Erreichens des
      Nutzungsvorrats ausgemustert wird. Unter der Annahme identischer Produkte
      im Nutzungssystem, insbesondere also gleicher Nutzungsdauern, gilt der im
      Abschnitt zur Nutzungsintensivierung \ref{sec:Nutzungsintensivierung}
      beschriebene Zusammenhang zwischen der Produktnutzungs-Menge $N$ und der
      Produktanzahl $P$:
      \begin{equation*}
        N = h \cdot t \cdot P
      \end{equation*}
      Der ökologisch positive Effekt der Nutzungsdauerverlängerung durch eine
      Erhöhung der Wartungshäufigkeit resultiert aus einer Reduktion der
      im Betrachtungszeitraum benötigten Produktanzahl. Bei konstanten
      Material-Inputs je Produkt $i_P$ verringern sich
      somit die produktbezogenen Material-Inputs $I_P$:
      \begin{equation*}
        I_P = i_P \cdot P = i_P \cdot \frac{N}{h \cdot t} = 
        \left\{ 
          \begin{array}{r@{\quad : \quad}l}
            i_P \cdot \frac{N}{\n{max}(w)} & \t{tech} < \t{max} \\[10pt]
            i_P \cdot \frac{N}{h \cdot \t{max}} & \text{sonst}
          \end{array}
        \right.
      \end{equation*}

      Es ist zu erkennen, dass die Wartungshäufigkeit nur dann einen Effekt auf
      die produktbezogenen Material-Inputs hat, wenn die Nutzungsdauer durch
      $\t{tech}$ beschränkt ist. 

      Wie bereits erwähnt, kommt im Falle der Wartungen -- anders als bei den
      Reparaturen -- noch ein dritter Effekt hinzu: eine höhere
      Wartungshäufigkeit kann zu einer Verringerung der Material-Inputs je
      Produktnutzung führen. Wir nehmen an, dass sich die Material-Inputs je
      Produktnutzung im Allgemeinen nach folgendem Muster mit der
      Zeit verändern: Zum Zeitpunkt der Wartung, sind sie am geringsten, erhöhen
      sich dann und erreichen kurz vor der nächsten Wartung
      ihr Maximum (man denke an das Ölen irgendeiner Mechanik). Da unabhängig
      von der Wartung ein Verschleiß eintritt, werden sich die
      durchschnittlichen Material-Inputs mit der Zeit erhöhen. Vernachlässigen
      wir diesen letzten Aspekt, so können wir, 
      da in diesem Teilmodell keine Größe von den Nutzungsinputs abhängt, zu
      einer durchschnittlichen Höhe der Material-Inputs je Produktnutzung
      übergehen. Wir nehmen an, dass dieser durchschnittliche Wert monoton 
      mit der Wartungshäufigkeit fällt: 
      \begin{equation}
        i_N = i_N(w)\quad  , \quad
        \frac{\partial i_N}{\partial w} \leq 0
      \end{equation}

      Weitere angenommene Eigenschaften der Funktion $i_N$ sind, dass sie sowohl
      nach oben als auch nach unten beschränkt ist und dass die Verringerung der
      Material-Inputs mit zunehmender Wartungshäufigkeit abnimmt:

      \begin{equation*}
        0 < \min_{w \in [0,1]} i_N(w) \leq \max_{w \in [0,1]} i_N(w) < \infty
        \quad , \quad \frac{\partial^2 i_N}{\partial w^2} \geq 0
      \end{equation*}

      Wir sind nun in der Lage, eine Gleichung für die MIPS anzugeben, welche
      die Wartungshäufigkeit berücksichtigt. Zunächst betrachten wir dafür die
      Input-Seite:

      \begin{align*}
        I &= I_N + I_W + I_P + \I{fix}\\
          % &= N \cdot i_N + N \cdot w \cdot i_W  
          %   + P \cdot i_P + \I{fix}\\
          &= N \cdot i_N + N \cdot w \cdot i_W  
            + \frac{N}{h\cdot t} \cdot i_P + \I{fix}\\
      \end{align*}

      Von der Wartungshäufigkeit hängen die nutzungsbedingten, die
      produktbezogenen und die wartungsbedingten Material-Inputs ab. Alle
      anderen davon unabhängigen Material-Inputs sind in der Größe $\I{fix}$
      zusammengefasst. Unter Ausnutzung des Zusammenhangs zwischen
      Nutzungseinheiten und Serviceeinheiten ($S=N\cdot A$), sowie bei
      im Betrachtungszeitraum konstanter Nachfrage ($S=S_D$), ergibt sich für
      die MIPS:

      \begin{align*}
        \MIPS[w]&= \frac{I}{S} 
                 = \frac{I_N + I_W + I_P + \I{fix}}{S_D}\\
                &= \frac{1}{A} \cdot \left( i_N(w) + w \cdot i_W 
                 + \frac{i_P}{h\cdot t}\right) + \frac{\I{fix}}{S_D}
      \end{align*}

      Zu beachten ist, dass diese Gleichung abhängig davon, ob $\t{tech}$
      kleiner oder größer als $\t{max}$ ist, jeweils eine etwas
      unterschiedliche Gestalt annimmt. Wir geben das Ergebnis direkt in der
      Zusammenstellung der für die Abbildung der Wartung auf die MIPS relevanten
      Gleichungen an:\\

			\begin{mdframed}[frametitle={Wartung}, frametitlerule=true]
				Teilmodell:
      % \begin{equation}
      %   \MIPS[w] =\left\{
      %             \begin{array}{l@{\quad : \quad}l}
      %               \frac{1}{A} \cdot \left( i_N(w) + w \cdot i_W 
      %               + \frac{i_P}{\n{max}(w)}\right) + \frac{\I{fix}}{S_D}
      %               &
      %               \t{tech} < \t{max} \\[10pt]
      %
      %               \frac{1}{A} \cdot \left( i_N(w) + w \cdot i_W 
      %               + \frac{i_P}{h\cdot \t{max}}\right) + \frac{\I{fix}}{S_D}
      %               &
      %               \text{sonst}
      %
      %             \end{array}
      %             \right.
      % \end{equation}
      \begin{equation}
        \MIPS[w] =
                    \frac{1}{A} \cdot \left( i_N(w) + w \cdot i_W 
                    + \frac{i_P}{h\cdot \min\left(\frac{\n{max}(w)}{h} ,
                    \t{max}\right)}\right) + \frac{\I{fix}}{S_D}
      \end{equation}
				Funktionseigenschaften:
				\begin{equation}
          \frac{\partial i_N}{\partial w} \leq 0 \quad , \quad
          0 < \min_{w \in [0,1]} i_N(w) \leq \max_{w \in [0,1]} i_N(w) < \infty
          \quad , \quad \frac{\partial^2 i_N}{\partial w^2} \geq 0
				\end{equation}
        \begin{equation}
          \frac{\partial \n{max}}{\partial w}\geq 0 \quad , \quad
          \frac{\partial^2 \n{max}}{\partial w^2}\leq 0 \quad , \quad
          \sup_{w \in [0,1]} \n{max}(w) < \infty
        \end{equation}
			\end{mdframed}

      \paragraph{Reparatur}
      Laut unserer Terminologie
      entspricht der  Zeitpunkt der Funktionsstörung, dem Erreichen des
      Nutzungsvorrats.

      Fasst man alle im Nutzungssystem durchgeführten Reparaturen zusammen, so
      ergibt sich eine Vergrößerung der Material-Inputs $\Delta I = I_R$ und
      eine damit korrespondierende Vergrößerung des Nutzungsvorrats $\Delta
      \N{max}$ 

      \begin{equation}
        P = \P{sim} \cdot \frac{T}{t}
      \end{equation}

      \begin{equation}
        t = \min\{\t{technisch}, \t{fix}\} =
        \min\left\{\frac{\n{max}(w)}{h} , \t{fix}\right\}
      \end{equation}

			\begin{mdframed}[frametitle={Reparatur}, frametitlerule=true]
				Teilmodell:
      \begin{equation}
        \MIPS[R] = 
        \frac{\P{sim} \cdot T
      \cdot i_P(w)}{\min\left\{\frac{\n{max}(w)}{h} , \t{fix}\right\}
        \cdot S_D} + \frac{\I{fix}}{S_D}
      \end{equation}
				Funktionseigenschaften:
				\begin{equation}
					\frac{\partial i_P}{\partial w} \geq 0
				\end{equation}
				\begin{equation}
					\frac{\partial \n{max}}{\partial w} \geq 0
				\end{equation}
			\end{mdframed}


		\subsection{Effekt-Kopplungen}
		\subsection{Modellübersicht}

\section{Analyse}
\begin{figure}
  \centering
  % Created by tikzDevice version 0.8.1 on 2015-09-08 10:23:58
% !TEX encoding = UTF-8 Unicode
\begin{tikzpicture}[x=1pt,y=1pt]
\definecolor{fillColor}{RGB}{255,255,255}
\path[use as bounding box,fill=fillColor,fill opacity=0.00] (0,0) rectangle (325.21,216.81);
\begin{scope}
\path[clip] ( 46.80, 49.20) rectangle (324.01,185.61);
\definecolor{drawColor}{RGB}{0,0,0}

\path[draw=drawColor,line width= 0.4pt,line join=round,line cap=round] ( 57.07,180.56) --
	( 57.32,179.43) --
	( 57.58,178.32) --
	( 57.84,177.23) --
	( 58.09,176.16) --
	( 58.35,175.11) --
	( 58.61,174.08) --
	( 58.87,173.07) --
	( 59.12,172.07) --
	( 59.38,171.08) --
	( 59.64,170.12) --
	( 59.89,169.17) --
	( 60.15,168.24) --
	( 60.41,167.32) --
	( 60.66,166.41) --
	( 60.92,165.52) --
	( 61.18,164.65) --
	( 61.44,163.78) --
	( 61.69,162.93) --
	( 61.95,162.10) --
	( 62.21,161.27) --
	( 62.46,160.46) --
	( 62.72,159.67) --
	( 62.98,158.88) --
	( 63.23,158.10) --
	( 63.49,157.34) --
	( 63.75,156.59) --
	( 64.00,155.85) --
	( 64.26,155.11) --
	( 64.52,154.39) --
	( 64.78,153.68) --
	( 65.03,152.98) --
	( 65.29,152.29) --
	( 65.55,151.61) --
	( 65.80,150.94) --
	( 66.06,150.28) --
	( 66.32,149.63) --
	( 66.57,148.98) --
	( 66.83,148.35) --
	( 67.09,147.72) --
	( 67.34,147.10) --
	( 67.60,146.49) --
	( 67.86,145.89) --
	( 68.12,145.29) --
	( 68.37,144.70) --
	( 68.63,144.12) --
	( 68.89,143.55) --
	( 69.14,142.99) --
	( 69.40,142.43) --
	( 69.66,141.88) --
	( 69.91,141.33) --
	( 70.17,140.79) --
	( 70.43,140.26) --
	( 70.68,139.74) --
	( 70.94,139.22) --
	( 71.20,138.71) --
	( 71.46,138.20) --
	( 71.71,137.70) --
	( 71.97,137.21) --
	( 72.23,136.72) --
	( 72.48,136.24) --
	( 72.74,135.76) --
	( 73.00,135.29) --
	( 73.25,134.83) --
	( 73.51,134.37) --
	( 73.77,133.91) --
	( 74.03,133.46) --
	( 74.28,133.02) --
	( 74.54,132.58) --
	( 74.80,132.14) --
	( 75.05,131.71) --
	( 75.31,131.28) --
	( 75.57,130.86) --
	( 75.82,130.45) --
	( 76.08,130.04) --
	( 76.34,129.63) --
	( 76.59,129.23) --
	( 76.85,128.83) --
	( 77.11,128.43) --
	( 77.37,128.04) --
	( 77.62,127.65) --
	( 77.88,127.27) --
	( 78.14,126.89) --
	( 78.39,126.52) --
	( 78.65,126.15) --
	( 78.91,125.78) --
	( 79.16,125.42) --
	( 79.42,125.06) --
	( 79.68,124.70) --
	( 79.93,124.35) --
	( 80.19,124.00) --
	( 80.45,123.66) --
	( 80.71,123.32) --
	( 80.96,122.98) --
	( 81.22,122.64) --
	( 81.48,122.31) --
	( 81.73,121.98) --
	( 81.99,121.66) --
	( 82.25,121.33) --
	( 82.50,121.01) --
	( 82.76,120.70) --
	( 83.02,120.38) --
	( 83.27,120.07) --
	( 83.53,119.77) --
	( 83.79,119.46) --
	( 84.05,119.16) --
	( 84.30,118.86) --
	( 84.56,118.56) --
	( 84.82,118.27) --
	( 85.07,117.98) --
	( 85.33,117.69) --
	( 85.59,117.41) --
	( 85.84,117.12) --
	( 86.10,116.84) --
	( 86.36,116.56) --
	( 86.62,116.29) --
	( 86.87,116.01) --
	( 87.13,115.74) --
	( 87.39,115.47) --
	( 87.64,115.21) --
	( 87.90,114.94) --
	( 88.16,114.68) --
	( 88.41,114.42) --
	( 88.67,114.17) --
	( 88.93,113.91) --
	( 89.18,113.66) --
	( 89.44,113.41) --
	( 89.70,113.16) --
	( 89.96,112.91) --
	( 90.21,112.67) --
	( 90.47,112.43) --
	( 90.73,112.19) --
	( 90.98,111.95) --
	( 91.24,111.71) --
	( 91.50,111.48) --
	( 91.75,111.24) --
	( 92.01,111.01) --
	( 92.27,110.78) --
	( 92.52,110.56) --
	( 92.78,110.33) --
	( 93.04,110.11) --
	( 93.30,109.89) --
	( 93.55,109.67) --
	( 93.81,109.45) --
	( 94.07,109.23) --
	( 94.32,109.02) --
	( 94.58,108.80) --
	( 94.84,108.59) --
	( 95.09,108.38) --
	( 95.35,108.17) --
	( 95.61,107.97) --
	( 95.86,107.76) --
	( 96.12,107.56) --
	( 96.38,107.36) --
	( 96.64,107.16) --
	( 96.89,106.96) --
	( 97.15,106.76) --
	( 97.41,106.57) --
	( 97.66,106.37) --
	( 97.92,106.18) --
	( 98.18,105.99) --
	( 98.43,105.80) --
	( 98.69,105.61) --
	( 98.95,105.42) --
	( 99.20,105.23) --
	( 99.46,105.05) --
	( 99.72,104.87) --
	( 99.98,104.68) --
	(100.23,104.50) --
	(100.49,104.32) --
	(100.75,104.15) --
	(101.00,103.97) --
	(101.26,103.79) --
	(101.52,103.62) --
	(101.77,103.44) --
	(102.03,103.27) --
	(102.29,103.10) --
	(102.55,102.93) --
	(102.80,102.76) --
	(103.06,102.60) --
	(103.32,102.43) --
	(103.57,102.27) --
	(103.83,102.10) --
	(104.09,101.94) --
	(104.34,101.78) --
	(104.60,101.62) --
	(104.86,101.46) --
	(105.11,101.30) --
	(105.37,101.14) --
	(105.63,100.99) --
	(105.89,100.83) --
	(106.14,100.68) --
	(106.40,100.52) --
	(106.66,100.37) --
	(106.91,100.22) --
	(107.17,100.07) --
	(107.43, 99.92) --
	(107.68, 99.77) --
	(107.94, 99.62) --
	(108.20, 99.48) --
	(108.45, 99.33) --
	(108.71, 99.19) --
	(108.97, 99.04) --
	(109.23, 98.90) --
	(109.48, 98.76) --
	(109.74, 98.62) --
	(110.00, 98.48) --
	(110.25, 98.34) --
	(110.51, 98.20) --
	(110.77, 98.06) --
	(111.02, 97.93) --
	(111.28, 97.79) --
	(111.54, 97.66) --
	(111.79, 97.52) --
	(112.05, 97.39) --
	(112.31, 97.26) --
	(112.57, 97.13) --
	(112.82, 97.00) --
	(113.08, 96.87) --
	(113.34, 96.74) --
	(113.59, 96.61) --
	(113.85, 96.48) --
	(114.11, 96.35) --
	(114.36, 96.23) --
	(114.62, 96.10) --
	(114.88, 95.98) --
	(115.14, 95.85) --
	(115.39, 95.73) --
	(115.65, 95.61) --
	(115.91, 95.49) --
	(116.16, 95.37) --
	(116.42, 95.25) --
	(116.68, 95.13) --
	(116.93, 95.01) --
	(117.19, 94.89) --
	(117.45, 94.77) --
	(117.70, 94.66) --
	(117.96, 94.54) --
	(118.22, 94.42) --
	(118.48, 94.31) --
	(118.73, 94.20) --
	(118.99, 94.08) --
	(119.25, 93.97) --
	(119.50, 93.86) --
	(119.76, 93.74) --
	(120.02, 93.63) --
	(120.27, 93.52) --
	(120.53, 93.41) --
	(120.79, 93.30) --
	(121.04, 93.20) --
	(121.30, 93.09) --
	(121.56, 92.98) --
	(121.82, 92.87) --
	(122.07, 92.77) --
	(122.33, 92.66) --
	(122.59, 92.56) --
	(122.84, 92.45) --
	(123.10, 92.35) --
	(123.36, 92.25) --
	(123.61, 92.14) --
	(123.87, 92.04) --
	(124.13, 91.94) --
	(124.38, 91.84) --
	(124.64, 91.74) --
	(124.90, 91.64) --
	(125.16, 91.54) --
	(125.41, 91.44) --
	(125.67, 91.34) --
	(125.93, 91.24) --
	(126.18, 91.15) --
	(126.44, 91.05) --
	(126.70, 90.95) --
	(126.95, 90.86) --
	(127.21, 90.76) --
	(127.47, 90.67) --
	(127.73, 90.57) --
	(127.98, 90.48) --
	(128.24, 90.39) --
	(128.50, 90.29) --
	(128.75, 90.20) --
	(129.01, 90.11) --
	(129.27, 90.02) --
	(129.52, 89.93) --
	(129.78, 89.84) --
	(130.04, 89.75) --
	(130.29, 89.66) --
	(130.55, 89.57) --
	(130.81, 89.48) --
	(131.07, 89.39) --
	(131.32, 89.30) --
	(131.58, 89.21) --
	(131.84, 89.13) --
	(132.09, 89.04) --
	(132.35, 88.95) --
	(132.61, 88.87) --
	(132.86, 88.78) --
	(133.12, 88.70) --
	(133.38, 88.61) --
	(133.63, 88.53) --
	(133.89, 88.45) --
	(134.15, 88.36) --
	(134.41, 88.28) --
	(134.66, 88.20) --
	(134.92, 88.12) --
	(135.18, 88.04) --
	(135.43, 87.95) --
	(135.69, 87.87) --
	(135.95, 87.79) --
	(136.20, 87.71) --
	(136.46, 87.63) --
	(136.72, 87.55) --
	(136.97, 87.47) --
	(137.23, 87.40) --
	(137.49, 87.32) --
	(137.75, 87.24) --
	(138.00, 87.16) --
	(138.26, 87.09) --
	(138.52, 87.01) --
	(138.77, 86.93) --
	(139.03, 86.86) --
	(139.29, 86.78) --
	(139.54, 86.71) --
	(139.80, 86.63) --
	(140.06, 86.56) --
	(140.31, 86.48) --
	(140.57, 86.41) --
	(140.83, 86.33) --
	(141.09, 86.26) --
	(141.34, 86.19) --
	(141.60, 86.12) --
	(141.86, 86.04) --
	(142.11, 85.97) --
	(142.37, 85.90) --
	(142.63, 85.83) --
	(142.88, 85.83) --
	(143.14, 85.83) --
	(143.40, 85.83) --
	(143.66, 85.83) --
	(143.91, 85.83) --
	(144.17, 85.83) --
	(144.43, 85.83) --
	(144.68, 85.83) --
	(144.94, 85.83) --
	(145.20, 85.83) --
	(145.45, 85.83) --
	(145.71, 85.83) --
	(145.97, 85.83) --
	(146.22, 85.83) --
	(146.48, 85.83) --
	(146.74, 85.83) --
	(147.00, 85.83) --
	(147.25, 85.83) --
	(147.51, 85.83) --
	(147.77, 85.83) --
	(148.02, 85.83) --
	(148.28, 85.83) --
	(148.54, 85.83) --
	(148.79, 85.83) --
	(149.05, 85.83) --
	(149.31, 85.83) --
	(149.56, 85.83) --
	(149.82, 85.83) --
	(150.08, 85.83) --
	(150.34, 85.83) --
	(150.59, 85.83) --
	(150.85, 85.83) --
	(151.11, 85.83) --
	(151.36, 85.83) --
	(151.62, 85.83) --
	(151.88, 85.83) --
	(152.13, 85.83) --
	(152.39, 85.83) --
	(152.65, 85.83) --
	(152.90, 85.83) --
	(153.16, 85.83) --
	(153.42, 85.83) --
	(153.68, 85.83) --
	(153.93, 85.83) --
	(154.19, 85.83) --
	(154.45, 85.83) --
	(154.70, 85.83) --
	(154.96, 85.83) --
	(155.22, 85.83) --
	(155.47, 85.83) --
	(155.73, 85.83) --
	(155.99, 85.83) --
	(156.25, 85.83) --
	(156.50, 85.83) --
	(156.76, 85.83) --
	(157.02, 85.83) --
	(157.27, 85.83) --
	(157.53, 85.83) --
	(157.79, 85.83) --
	(158.04, 85.83) --
	(158.30, 85.83) --
	(158.56, 85.83) --
	(158.81, 85.83) --
	(159.07, 85.83) --
	(159.33, 85.83) --
	(159.59, 85.83) --
	(159.84, 85.83) --
	(160.10, 85.83) --
	(160.36, 85.83) --
	(160.61, 85.83) --
	(160.87, 85.83) --
	(161.13, 85.83) --
	(161.38, 85.83) --
	(161.64, 85.83) --
	(161.90, 85.83) --
	(162.15, 85.83) --
	(162.41, 85.83) --
	(162.67, 85.83) --
	(162.93, 85.83) --
	(163.18, 85.83) --
	(163.44, 85.83) --
	(163.70, 85.83) --
	(163.95, 85.83) --
	(164.21, 85.83) --
	(164.47, 85.83) --
	(164.72, 85.83) --
	(164.98, 85.83) --
	(165.24, 85.83) --
	(165.49, 85.83) --
	(165.75, 85.83) --
	(166.01, 85.83) --
	(166.27, 85.83) --
	(166.52, 85.83) --
	(166.78, 85.83) --
	(167.04, 85.83) --
	(167.29, 85.83) --
	(167.55, 85.83) --
	(167.81, 85.83) --
	(168.06, 85.83) --
	(168.32, 85.83) --
	(168.58, 85.83) --
	(168.84, 85.83) --
	(169.09, 85.83) --
	(169.35, 85.83) --
	(169.61, 85.83) --
	(169.86, 85.83) --
	(170.12, 85.83) --
	(170.38, 85.83) --
	(170.63, 85.83) --
	(170.89, 85.83) --
	(171.15, 85.83) --
	(171.40, 85.83) --
	(171.66, 85.83) --
	(171.92, 85.83) --
	(172.18, 85.83) --
	(172.43, 85.83) --
	(172.69, 85.83) --
	(172.95, 85.83) --
	(173.20, 85.83) --
	(173.46, 85.83) --
	(173.72, 85.83) --
	(173.97, 85.83) --
	(174.23, 85.83) --
	(174.49, 85.83) --
	(174.74, 85.83) --
	(175.00, 85.83) --
	(175.26, 85.83) --
	(175.52, 85.83) --
	(175.77, 85.83) --
	(176.03, 85.83) --
	(176.29, 85.83) --
	(176.54, 85.83) --
	(176.80, 85.83) --
	(177.06, 85.83) --
	(177.31, 85.83) --
	(177.57, 85.83) --
	(177.83, 85.83) --
	(178.08, 85.83) --
	(178.34, 85.83) --
	(178.60, 85.83) --
	(178.86, 85.83) --
	(179.11, 85.83) --
	(179.37, 85.83) --
	(179.63, 85.83) --
	(179.88, 85.83) --
	(180.14, 85.83) --
	(180.40, 85.83) --
	(180.65, 85.83) --
	(180.91, 85.83) --
	(181.17, 85.83) --
	(181.42, 85.83) --
	(181.68, 85.83) --
	(181.94, 85.83) --
	(182.20, 85.83) --
	(182.45, 85.83) --
	(182.71, 85.83) --
	(182.97, 85.83) --
	(183.22, 85.83) --
	(183.48, 85.83) --
	(183.74, 85.83) --
	(183.99, 85.83) --
	(184.25, 85.83) --
	(184.51, 85.83) --
	(184.77, 85.83) --
	(185.02, 85.83) --
	(185.28, 85.83) --
	(185.54, 85.83) --
	(185.79, 85.83) --
	(186.05, 85.83) --
	(186.31, 85.83) --
	(186.56, 85.83) --
	(186.82, 85.83) --
	(187.08, 85.83) --
	(187.33, 85.83) --
	(187.59, 85.83) --
	(187.85, 85.83) --
	(188.11, 85.83) --
	(188.36, 85.83) --
	(188.62, 85.83) --
	(188.88, 85.83) --
	(189.13, 85.83) --
	(189.39, 85.83) --
	(189.65, 85.83) --
	(189.90, 85.83) --
	(190.16, 85.83) --
	(190.42, 85.83) --
	(190.67, 85.83) --
	(190.93, 85.83) --
	(191.19, 85.83) --
	(191.45, 85.83) --
	(191.70, 85.83) --
	(191.96, 85.83) --
	(192.22, 85.83) --
	(192.47, 85.83) --
	(192.73, 85.83) --
	(192.99, 85.83) --
	(193.24, 85.83) --
	(193.50, 85.83) --
	(193.76, 85.83) --
	(194.01, 85.83) --
	(194.27, 85.83) --
	(194.53, 85.83) --
	(194.79, 85.83) --
	(195.04, 85.83) --
	(195.30, 85.83) --
	(195.56, 85.83) --
	(195.81, 85.83) --
	(196.07, 85.83) --
	(196.33, 85.83) --
	(196.58, 85.83) --
	(196.84, 85.83) --
	(197.10, 85.83) --
	(197.36, 85.83) --
	(197.61, 85.83) --
	(197.87, 85.83) --
	(198.13, 85.83) --
	(198.38, 85.83) --
	(198.64, 85.83) --
	(198.90, 85.83) --
	(199.15, 85.83) --
	(199.41, 85.83) --
	(199.67, 85.83) --
	(199.92, 85.83) --
	(200.18, 85.83) --
	(200.44, 85.83) --
	(200.70, 85.83) --
	(200.95, 85.83) --
	(201.21, 85.83) --
	(201.47, 85.83) --
	(201.72, 85.83) --
	(201.98, 85.83) --
	(202.24, 85.83) --
	(202.49, 85.83) --
	(202.75, 85.83) --
	(203.01, 85.83) --
	(203.26, 85.83) --
	(203.52, 85.83) --
	(203.78, 85.83) --
	(204.04, 85.83) --
	(204.29, 85.83) --
	(204.55, 85.83) --
	(204.81, 85.83) --
	(205.06, 85.83) --
	(205.32, 85.83) --
	(205.58, 85.83) --
	(205.83, 85.83) --
	(206.09, 85.83) --
	(206.35, 85.83) --
	(206.60, 85.83) --
	(206.86, 85.83) --
	(207.12, 85.83) --
	(207.38, 85.83) --
	(207.63, 85.83) --
	(207.89, 85.83) --
	(208.15, 85.83) --
	(208.40, 85.83) --
	(208.66, 85.83) --
	(208.92, 85.83) --
	(209.17, 85.83) --
	(209.43, 85.83) --
	(209.69, 85.83) --
	(209.95, 85.83) --
	(210.20, 85.83) --
	(210.46, 85.83) --
	(210.72, 85.83) --
	(210.97, 85.83) --
	(211.23, 85.83) --
	(211.49, 85.83) --
	(211.74, 85.83) --
	(212.00, 85.83) --
	(212.26, 85.83) --
	(212.51, 85.83) --
	(212.77, 85.83) --
	(213.03, 85.83) --
	(213.29, 85.83) --
	(213.54, 85.83) --
	(213.80, 85.83) --
	(214.06, 85.83) --
	(214.31, 85.83) --
	(214.57, 85.83) --
	(214.83, 85.83) --
	(215.08, 85.83) --
	(215.34, 85.83) --
	(215.60, 85.83) --
	(215.85, 85.83) --
	(216.11, 85.83) --
	(216.37, 85.83) --
	(216.63, 85.83) --
	(216.88, 85.83) --
	(217.14, 85.83) --
	(217.40, 85.83) --
	(217.65, 85.83) --
	(217.91, 85.83) --
	(218.17, 85.83) --
	(218.42, 85.83) --
	(218.68, 85.83) --
	(218.94, 85.83) --
	(219.19, 85.83) --
	(219.45, 85.83) --
	(219.71, 85.83) --
	(219.97, 85.83) --
	(220.22, 85.83) --
	(220.48, 85.83) --
	(220.74, 85.83) --
	(220.99, 85.83) --
	(221.25, 85.83) --
	(221.51, 85.83) --
	(221.76, 85.83) --
	(222.02, 85.83) --
	(222.28, 85.83) --
	(222.53, 85.83) --
	(222.79, 85.83) --
	(223.05, 85.83) --
	(223.31, 85.83) --
	(223.56, 85.83) --
	(223.82, 85.83) --
	(224.08, 85.83) --
	(224.33, 85.83) --
	(224.59, 85.83) --
	(224.85, 85.83) --
	(225.10, 85.83) --
	(225.36, 85.83) --
	(225.62, 85.83) --
	(225.88, 85.83) --
	(226.13, 85.83) --
	(226.39, 85.83) --
	(226.65, 85.83) --
	(226.90, 85.83) --
	(227.16, 85.83) --
	(227.42, 85.83) --
	(227.67, 85.83) --
	(227.93, 85.83) --
	(228.19, 85.83) --
	(228.44, 85.83) --
	(228.70, 85.83) --
	(228.96, 85.83) --
	(229.22, 85.83) --
	(229.47, 85.83) --
	(229.73, 85.83) --
	(229.99, 85.83) --
	(230.24, 85.83) --
	(230.50, 85.83) --
	(230.76, 85.83) --
	(231.01, 85.83) --
	(231.27, 85.83) --
	(231.53, 85.83) --
	(231.78, 85.83) --
	(232.04, 85.83) --
	(232.30, 85.83) --
	(232.56, 85.83) --
	(232.81, 85.83) --
	(233.07, 85.83) --
	(233.33, 85.83) --
	(233.58, 85.83) --
	(233.84, 85.83) --
	(234.10, 85.83) --
	(234.35, 85.83) --
	(234.61, 85.83) --
	(234.87, 85.83) --
	(235.12, 85.83) --
	(235.38, 85.83) --
	(235.64, 85.83) --
	(235.90, 85.83) --
	(236.15, 85.83) --
	(236.41, 85.83) --
	(236.67, 85.83) --
	(236.92, 85.83) --
	(237.18, 85.83) --
	(237.44, 85.83) --
	(237.69, 85.83) --
	(237.95, 85.83) --
	(238.21, 85.83) --
	(238.47, 85.83) --
	(238.72, 85.83) --
	(238.98, 85.83) --
	(239.24, 85.83) --
	(239.49, 85.83) --
	(239.75, 85.83) --
	(240.01, 85.83) --
	(240.26, 85.83) --
	(240.52, 85.83) --
	(240.78, 85.83) --
	(241.03, 85.83) --
	(241.29, 85.83) --
	(241.55, 85.83) --
	(241.81, 85.83) --
	(242.06, 85.83) --
	(242.32, 85.83) --
	(242.58, 85.83) --
	(242.83, 85.83) --
	(243.09, 85.83) --
	(243.35, 85.83) --
	(243.60, 85.83) --
	(243.86, 85.83) --
	(244.12, 85.83) --
	(244.37, 85.83) --
	(244.63, 85.83) --
	(244.89, 85.83) --
	(245.15, 85.83) --
	(245.40, 85.83) --
	(245.66, 85.83) --
	(245.92, 85.83) --
	(246.17, 85.83) --
	(246.43, 85.83) --
	(246.69, 85.83) --
	(246.94, 85.83) --
	(247.20, 85.83) --
	(247.46, 85.83) --
	(247.71, 85.83) --
	(247.97, 85.83) --
	(248.23, 85.83) --
	(248.49, 85.83) --
	(248.74, 85.83) --
	(249.00, 85.83) --
	(249.26, 85.83) --
	(249.51, 85.83) --
	(249.77, 85.83) --
	(250.03, 85.83) --
	(250.28, 85.83) --
	(250.54, 85.83) --
	(250.80, 85.83) --
	(251.06, 85.83) --
	(251.31, 85.83) --
	(251.57, 85.83) --
	(251.83, 85.83) --
	(252.08, 85.83) --
	(252.34, 85.83) --
	(252.60, 85.83) --
	(252.85, 85.83) --
	(253.11, 85.83) --
	(253.37, 85.83) --
	(253.62, 85.83) --
	(253.88, 85.83) --
	(254.14, 85.83) --
	(254.40, 85.83) --
	(254.65, 85.83) --
	(254.91, 85.83) --
	(255.17, 85.83) --
	(255.42, 85.83) --
	(255.68, 85.83) --
	(255.94, 85.83) --
	(256.19, 85.83) --
	(256.45, 85.83) --
	(256.71, 85.83) --
	(256.96, 85.83) --
	(257.22, 85.83) --
	(257.48, 85.83) --
	(257.74, 85.83) --
	(257.99, 85.83) --
	(258.25, 85.83) --
	(258.51, 85.83) --
	(258.76, 85.83) --
	(259.02, 85.83) --
	(259.28, 85.83) --
	(259.53, 85.83) --
	(259.79, 85.83) --
	(260.05, 85.83) --
	(260.30, 85.83) --
	(260.56, 85.83) --
	(260.82, 85.83) --
	(261.08, 85.83) --
	(261.33, 85.83) --
	(261.59, 85.83) --
	(261.85, 85.83) --
	(262.10, 85.83) --
	(262.36, 85.83) --
	(262.62, 85.83) --
	(262.87, 85.83) --
	(263.13, 85.83) --
	(263.39, 85.83) --
	(263.64, 85.83) --
	(263.90, 85.83) --
	(264.16, 85.83) --
	(264.42, 85.83) --
	(264.67, 85.83) --
	(264.93, 85.83) --
	(265.19, 85.83) --
	(265.44, 85.83) --
	(265.70, 85.83) --
	(265.96, 85.83) --
	(266.21, 85.83) --
	(266.47, 85.83) --
	(266.73, 85.83) --
	(266.99, 85.83) --
	(267.24, 85.83) --
	(267.50, 85.83) --
	(267.76, 85.83) --
	(268.01, 85.83) --
	(268.27, 85.83) --
	(268.53, 85.83) --
	(268.78, 85.83) --
	(269.04, 85.83) --
	(269.30, 85.83) --
	(269.55, 85.83) --
	(269.81, 85.83) --
	(270.07, 85.83) --
	(270.33, 85.83) --
	(270.58, 85.83) --
	(270.84, 85.83) --
	(271.10, 85.83) --
	(271.35, 85.83) --
	(271.61, 85.83) --
	(271.87, 85.83) --
	(272.12, 85.83) --
	(272.38, 85.83) --
	(272.64, 85.83) --
	(272.89, 85.83) --
	(273.15, 85.83) --
	(273.41, 85.83) --
	(273.67, 85.83) --
	(273.92, 85.83) --
	(274.18, 85.83) --
	(274.44, 85.83) --
	(274.69, 85.83) --
	(274.95, 85.83) --
	(275.21, 85.83) --
	(275.46, 85.83) --
	(275.72, 85.83) --
	(275.98, 85.83) --
	(276.23, 85.83) --
	(276.49, 85.83) --
	(276.75, 85.83) --
	(277.01, 85.83) --
	(277.26, 85.83) --
	(277.52, 85.83) --
	(277.78, 85.83) --
	(278.03, 85.83) --
	(278.29, 85.83) --
	(278.55, 85.83) --
	(278.80, 85.83) --
	(279.06, 85.83) --
	(279.32, 85.83) --
	(279.58, 85.83) --
	(279.83, 85.83) --
	(280.09, 85.83) --
	(280.35, 85.83) --
	(280.60, 85.83) --
	(280.86, 85.83) --
	(281.12, 85.83) --
	(281.37, 85.83) --
	(281.63, 85.83) --
	(281.89, 85.83) --
	(282.14, 85.83) --
	(282.40, 85.83) --
	(282.66, 85.83) --
	(282.92, 85.83) --
	(283.17, 85.83) --
	(283.43, 85.83) --
	(283.69, 85.83) --
	(283.94, 85.83) --
	(284.20, 85.83) --
	(284.46, 85.83) --
	(284.71, 85.83) --
	(284.97, 85.83) --
	(285.23, 85.83) --
	(285.48, 85.83) --
	(285.74, 85.83) --
	(286.00, 85.83) --
	(286.26, 85.83) --
	(286.51, 85.83) --
	(286.77, 85.83) --
	(287.03, 85.83) --
	(287.28, 85.83) --
	(287.54, 85.83) --
	(287.80, 85.83) --
	(288.05, 85.83) --
	(288.31, 85.83) --
	(288.57, 85.83) --
	(288.82, 85.83) --
	(289.08, 85.83) --
	(289.34, 85.83) --
	(289.60, 85.83) --
	(289.85, 85.83) --
	(290.11, 85.83) --
	(290.37, 85.83) --
	(290.62, 85.83) --
	(290.88, 85.83) --
	(291.14, 85.83) --
	(291.39, 85.83) --
	(291.65, 85.83) --
	(291.91, 85.83) --
	(292.17, 85.83) --
	(292.42, 85.83) --
	(292.68, 85.83) --
	(292.94, 85.83) --
	(293.19, 85.83) --
	(293.45, 85.83) --
	(293.71, 85.83) --
	(293.96, 85.83) --
	(294.22, 85.83) --
	(294.48, 85.83) --
	(294.73, 85.83) --
	(294.99, 85.83) --
	(295.25, 85.83) --
	(295.51, 85.83) --
	(295.76, 85.83) --
	(296.02, 85.83) --
	(296.28, 85.83) --
	(296.53, 85.83) --
	(296.79, 85.83) --
	(297.05, 85.83) --
	(297.30, 85.83) --
	(297.56, 85.83) --
	(297.82, 85.83) --
	(298.07, 85.83) --
	(298.33, 85.83) --
	(298.59, 85.83) --
	(298.85, 85.83) --
	(299.10, 85.83) --
	(299.36, 85.83) --
	(299.62, 85.83) --
	(299.87, 85.83) --
	(300.13, 85.83) --
	(300.39, 85.83) --
	(300.64, 85.83) --
	(300.90, 85.83) --
	(301.16, 85.83) --
	(301.41, 85.83) --
	(301.67, 85.83) --
	(301.93, 85.83) --
	(302.19, 85.83) --
	(302.44, 85.83) --
	(302.70, 85.83) --
	(302.96, 85.83) --
	(303.21, 85.83) --
	(303.47, 85.83) --
	(303.73, 85.83) --
	(303.98, 85.83) --
	(304.24, 85.83) --
	(304.50, 85.83) --
	(304.75, 85.83) --
	(305.01, 85.83) --
	(305.27, 85.83) --
	(305.53, 85.83) --
	(305.78, 85.83) --
	(306.04, 85.83) --
	(306.30, 85.83) --
	(306.55, 85.83) --
	(306.81, 85.83) --
	(307.07, 85.83) --
	(307.32, 85.83) --
	(307.58, 85.83) --
	(307.84, 85.83) --
	(308.10, 85.83) --
	(308.35, 85.83) --
	(308.61, 85.83) --
	(308.87, 85.83) --
	(309.12, 85.83) --
	(309.38, 85.83) --
	(309.64, 85.83) --
	(309.89, 85.83) --
	(310.15, 85.83) --
	(310.41, 85.83) --
	(310.66, 85.83) --
	(310.92, 85.83) --
	(311.18, 85.83) --
	(311.44, 85.83) --
	(311.69, 85.83) --
	(311.95, 85.83) --
	(312.21, 85.83) --
	(312.46, 85.83) --
	(312.72, 85.83) --
	(312.98, 85.83) --
	(313.23, 85.83) --
	(313.49, 85.83) --
	(313.75, 85.83);
\end{scope}
\begin{scope}
\path[clip] (  0.00,  0.00) rectangle (325.21,216.81);
\definecolor{drawColor}{RGB}{0,0,0}

\path[draw=drawColor,line width= 0.4pt,line join=round,line cap=round] ( 46.80, 54.25) -- ( 46.80,180.56);

\path[draw=drawColor,line width= 0.4pt,line join=round,line cap=round] ( 46.80, 54.25) -- ( 40.80, 54.25);

\path[draw=drawColor,line width= 0.4pt,line join=round,line cap=round] ( 46.80, 85.83) -- ( 40.80, 85.83);

\path[draw=drawColor,line width= 0.4pt,line join=round,line cap=round] ( 46.80,117.41) -- ( 40.80,117.41);

\path[draw=drawColor,line width= 0.4pt,line join=round,line cap=round] ( 46.80,148.98) -- ( 40.80,148.98);

\path[draw=drawColor,line width= 0.4pt,line join=round,line cap=round] ( 46.80,180.56) -- ( 40.80,180.56);

\node[text=drawColor,rotate= 90.00,anchor=base,inner sep=0pt, outer sep=0pt, scale=  1.00] at ( 32.40, 54.25) {0};

\node[text=drawColor,rotate= 90.00,anchor=base,inner sep=0pt, outer sep=0pt, scale=  1.00] at ( 32.40, 85.83) {2};

\node[text=drawColor,rotate= 90.00,anchor=base,inner sep=0pt, outer sep=0pt, scale=  1.00] at ( 32.40,117.41) {4};

\node[text=drawColor,rotate= 90.00,anchor=base,inner sep=0pt, outer sep=0pt, scale=  1.00] at ( 32.40,148.98) {6};

\node[text=drawColor,rotate= 90.00,anchor=base,inner sep=0pt, outer sep=0pt, scale=  1.00] at ( 32.40,180.56) {8};

\path[draw=drawColor,line width= 0.4pt,line join=round,line cap=round] ( 46.80, 49.20) --
	(324.01, 49.20) --
	(324.01,185.61) --
	( 46.80,185.61) --
	( 46.80, 49.20);
\end{scope}
\begin{scope}
\path[clip] (  0.00,  0.00) rectangle (325.21,216.81);
\definecolor{drawColor}{RGB}{0,0,0}

\node[text=drawColor,anchor=base,inner sep=0pt, outer sep=0pt, scale=  1.00] at (185.41,  3.60) {relative Produktauslastung $a$};

\node[text=drawColor,rotate= 90.00,anchor=base,inner sep=0pt, outer sep=0pt, scale=  1.00] at (  8.40,117.41) {effektive Produktanzahl $P$};
\end{scope}
\begin{scope}
\path[clip] ( 46.80, 49.20) rectangle (324.01,185.61);
\definecolor{drawColor}{RGB}{0,0,0}

\path[draw=drawColor,line width= 0.4pt,dash pattern=on 1pt off 3pt ,line join=round,line cap=round] (142.63, 49.20) -- (142.63,185.61);
\end{scope}
\begin{scope}
\path[clip] (  0.00,  0.00) rectangle (325.21,216.81);
\definecolor{drawColor}{RGB}{0,0,0}

\node[text=drawColor,anchor=base,inner sep=0pt, outer sep=0pt, scale=  1.20] at (185.41,203.61) {\bfseries effektive Produktanzahl $P(a)$};
\end{scope}
\begin{scope}
\path[clip] (  0.00,  0.00) rectangle (325.21,216.81);
\definecolor{drawColor}{RGB}{0,0,0}

\path[draw=drawColor,line width= 0.4pt,line join=round,line cap=round] ( 85.59, 49.20) -- (313.75, 49.20);

\path[draw=drawColor,line width= 0.4pt,line join=round,line cap=round] ( 85.59, 49.20) -- ( 85.59, 43.20);

\path[draw=drawColor,line width= 0.4pt,line join=round,line cap=round] (142.63, 49.20) -- (142.63, 43.20);

\path[draw=drawColor,line width= 0.4pt,line join=round,line cap=round] (199.67, 49.20) -- (199.67, 43.20);

\path[draw=drawColor,line width= 0.4pt,line join=round,line cap=round] (256.71, 49.20) -- (256.71, 43.20);

\path[draw=drawColor,line width= 0.4pt,line join=round,line cap=round] (313.75, 49.20) -- (313.75, 43.20);

\node[text=drawColor,anchor=base,inner sep=0pt, outer sep=0pt, scale=  1.00] at ( 85.59, 27.60) {0.2};

\node[text=drawColor,anchor=base,inner sep=0pt, outer sep=0pt, scale=  1.00] at (142.63, 27.60) {$a^*$};

\node[text=drawColor,anchor=base,inner sep=0pt, outer sep=0pt, scale=  1.00] at (199.67, 27.60) {0.6};

\node[text=drawColor,anchor=base,inner sep=0pt, outer sep=0pt, scale=  1.00] at (256.71, 27.60) {0.8};

\node[text=drawColor,anchor=base,inner sep=0pt, outer sep=0pt, scale=  1.00] at (313.75, 27.60) {1};
\end{scope}
\end{tikzpicture}

  \caption{}
\end{figure}
\begin{equation}
  \frac{\partial \MIPS[a]}{\partial a} = \frac{i_N'(a)\cdot a - i_N(a)}{a^2 \cdot K}
  \label{eqn:ana-Auslastung}
\end{equation}
Wir können nun eine Bedingung angeben, unter der eine (marginale) Erhöhung der
Produktauslastung zu einer (marginalen) Verringerung führt:
\begin{equation}
  \frac{\partial \MIPS[a]}{\partial a} < 0 
  \qquad \Leftrightarrow \qquad 
  \epsilon_{i_N, a} = i'_N(a) \cdot  \frac{a}{i_N(a)} < 1
  \label{eqn:ana-Auslastung-Kriterium}
\end{equation}

Es ergibt sich also, dass eine Verringerung des MIPS durch eine Erhöhung der
Produktauslastung genau dann erreicht wird, wenn der Input pro Nutzungseinheit
unelastisch\footnote{zum Begriff der Elastizität vgl. Abschnitt \ref{sec:Elastizität}} auf
eine Änderung der relativen Auslastung reagiert.
\section{Synthese}
\section{Ausblick/Diskussion}
\section{Literaturverzeichnis}
\printbibliography
\section{Anhang}
\subsection{Elastizität}\label{sec:Elastizität}
\begin{equation}
  \text{Elastizität} \quad \epsilon_{y,x} := \frac{dy(x)}{dx} \cdot
  \frac{x}{y}
  \label{eqn:def-Elastizität}
\end{equation}
\subsection{Übersicht über die in den Modellen verwendeten Größen}
				\renewcommand{\arraystretch}{1.25}
			\begin{longtable}{p{1cm} p{3,5cm} p{3,5cm} p{\textwidth - 9.5 cm}}
				% \begin{tabular}{p{1cm} p{3,5cm} p{3,5cm} p{\textwidth - 8 cm}}
					\toprule
					Symbol & Bezeichnung & Kurzbezeichnung & Definition \\
					\midrule
          \endhead
          \bottomrule
				\caption{Übersicht über die in den Modellen verwendeten Größen.
        (Fortsetzung auf der nächsten Seite)}
          \endfoot
          \bottomrule
				\caption{Übersicht über die in den Modellen verwendeten Größen.}
          \endlastfoot
					MIPS & Material-Inputs pro Serviceeinheit & - &  $\text{MIPS} = I / S$\\
					$I$ & Material-Inputs & Inputs & Summe aller Materialströme, die zur Bereitstellung, Nutzung und Entsorgung der betrachteten Produkte sowie der Infrastruktur des Nutzungssystems notwendig sind \\
					$I_N$ & nutzungsbedingte Material-Inputs & nutzungsbedingte Inputs & Material-Inputs, die unmittelbar bei der Nutzung anfallen \\
					$I_P$ & produktbezogene Material-Inputs & produktbezogene Inputs & Material-Inputs, die zur Bereitstellung und Entsorgung der betrachteten Produkte notwendig sind \\
          $I_W$ & wartungsbedingte Material-Inputs & wartungsbedingte Inputs &
          Material-Inputs, die für Wartungstätigkeiten im Nutzungssystem
          aufgebracht werden\\
					$\I{fix}$ & konstante Material-Inputs & konstante Inputs & Material-Inputs, die weder von der Produktanzahl, noch von der Produktnutzung abhängen (Nutzungssystem-Infrastruktur) \\
					$i_N$ & Material-Inputs je Produktnutzung & Inputs je Produktnutzung & Material-Inputs, die für eine Produktnutzungseinheit notwendig sind \\
					$i_P$ & Material-Inputs je Produkt & Inputs je Produkt & Material-Inputs, die für die Bereitstellung und Entsorgung eines Produkts notwendig sind \\
          $i_W$ & Material-Inputs je Wartung & Inputs je Wartung &
          Material-Inputs die für eine Wartungstätigkeit im gesamten
          Nutzungssystem aufgebracht werden müssen\\
					$P$ & Produktanzahl & - & Anzahl der betrachteten Produkte im Produktnutzungssystem \\
					$P_\text{min}$ & Mindestproduktanzahl & - & Anzahl mindestens notwendiger Produkte im Produktnutzungssystem \\
          $W$ & Anzahl Wartungen & - & Anzahl an Wartungen im Nutzungssystem
          während des Betrachtungszeitraums\\
					$S$ & Service-Menge & Service & Gesamtmenge des Nutzens der betrachteten Produkte \\
					$S_D$ & nachgefragte Service-Menge & Service-Nachfrage & Gesamtnachfrage nach Service im betrachteten Nutzungssystem \\
					$N$ & Produktnutzungs-Menge & Produktnutzung & Anzahl der Nutzungen oder Gesamtnutzungsdauer der betrachteten Produkte im Produktnutzungssystem \\
					$N_{\text{max}}$ & Nutzungsvorrat & - & maximale Produktnutzungs-Menge im Produktnutzungssystem \\
					$n_{\text{max}}$ & Einzelnutzungsvorrat & - & maximale Produktnutzungs-Menge eines Produkts \\
					$A$ & absolute Produktauslastung & absolute Auslastung & erzielte Service-Menge je Produktnutzungseinheit \\ 
					$K$ & Kapazität & - & maximale absolute Auslastung \\
					$a$ & relative Produktauslastung & Auslastung & $a = A / K$\\
          $w$ & Wartungshäufigkeit & - & Anzahl Wartungen je Nutzungseinheiten\\
					% \bottomrule
				% \end{tabular}
			\end{longtable}
			\todo{Zelleninhalte linksbündig ausrichten)}





\end{document}


