%        File: Expose.tex
%     Created: Di Mrz 04 02:00  2014 Mitteleuropäische Z
% Last Change: Di Mrz 04 02:00  2014 Mitteleuropäische Z
%
% !TEX TS-program = pdflatex
% !TEX encoding = UTF-8 Unicode

\documentclass[11pt, titlepage=true]{scrartcl} % use larger type; default would be 10pt

\usepackage[utf8]{inputenc} % set input encoding (not needed with XeLaTeX)
\usepackage[ngerman]{babel} 
\usepackage[babel,german=quotes]{csquotes}
\usepackage[T1]{fontenc} % for enabling non-standard characters in
                         % hyphenation-list and smallcaps-boldfont

\usepackage[bibstyle=authortitle, citestyle=authoryear]{biblatex}
\bibliography{expose}

% \DeclareUnicodeCharacter{00A0}{~} % for avoiding problems with no-break spaces

%%% PAGE DIMENSIONS
\usepackage{geometry} % to change the page dimensions
\geometry{a4paper} % or letterpaper (US) or a5paper or....
% \geometry{margin=2in} % for example, change the margins to 2 inches all round
% \geometry{landscape} % set up the page for landscape
%   read geometry.pdf for detailed page layout information

\usepackage{graphicx} % support the \includegraphics command and options

\graphicspath{{./Bilder/}}

\usepackage[parfill]{parskip} % Activate to begin paragraphs with an empty line rather than an indent

%%% PACKAGES
\usepackage{booktabs} % for much better looking tables
\usepackage{array} % for better arrays (eg matrices) in maths
%\usepackage{paralist} % very flexible & customisable lists (eg. enumerate/itemize, etc.)
\usepackage{verbatim} % adds environment for commenting out blocks of text & for better verbatim
\usepackage{subfig} % make it possible to include more than one captioned figure/table in a single float
% These packages are all incorporated in the memoir class to one degree or another...

%%% HEADERS & FOOTERS
\usepackage{fancyhdr} % This should be set AFTER setting up the page geometry
\pagestyle{fancy} % options: empty , plain , fancy
\renewcommand{\headrulewidth}{0pt} % customise the layout...
\lhead{}\chead{}\rhead{}
\lfoot{}\cfoot{\thepage}\rfoot{}

%%% SECTION TITLE APPEARANCE
\usepackage{sectsty}
\allsectionsfont{\sffamily\mdseries\upshape} % (See the fntguide.pdf for font help)
% (This matches ConTeXt defaults)

%%% ToC (table of contents) APPEARANCE
\usepackage[nottoc,notlof,notlot]{tocbibind} % Put the bibliography in the ToC
\usepackage[titles,subfigure]{tocloft} % Alter the style of the Table of Contents
\renewcommand{\cftsecfont}{\rmfamily\mdseries\upshape}
\renewcommand{\cftsecpagefont}{\rmfamily\mdseries\upshape} % No bold!

\usepackage[hidelinks]{hyperref}
\usepackage{ragged2e} %für links und rechtsbündige Umgebungen in Tabellen
\usepackage{todonotes}

%%% Eigene Befehle %%%
\newcommand{\was}[1]{\small\textit{#1}}

%%% Allgemeine Daten %%%
\newcommand{\autorOne}{Alexander Müller}
\newcommand{\matrikelnrOne}{940597}
\newcommand{\adresseOne}{Gartlager Weg 37, \\49086 Osnabrück}
\newcommand{\telOne}{01578 1901578}
\newcommand{\emailOne}{alemuell@uos.de}

\newcommand{\autorTwo}{János Sebestyén}
\newcommand{\matrikelnrTwo}{939525}
\newcommand{\adresseTwo}{Gartlager Weg 37, \\49086 Osnabrück}
\newcommand{\telTwo}{}
\newcommand{\emailTwo}{jsebesty@uos.de}

%%% Titlepage Expose %%%
\newcommand{\titel}{Ökologische Nachhaltigkeit durch \enquote{Nutzen statt Besitzen}?}
\newcommand{\stitle}{}
\newcommand{\art}{Exposé}
\newcommand{\fachgebiet}{Angewandte Systemwissenschaft}
\newcommand{\betreuung}{Katrin Bienge und Prof. Dr. Claudia Pahl-Wostl}
\newcommand{\institut}{Wuppertal Institut für Klima, Umwelt, Energie und Institut für Umweltsystemforschung}
\newcommand{\ort}{Osnabrück}

%%% END Article customizations

\begin{document}
\begin{titlepage}
  \begin{center}
    \includegraphics[scale=0.5]{Logo-Uni-Osnabrueck.jpg}\\[3ex]

    \vfill
    \LARGE{\textbf{\art}}\\[1.5ex]
    \huge{\textbf{\textsc{\titel}}}\\[1.0ex]
    \LARGE{\textbf{\stitle}}\\[6ex]
    \vfill

    \normalsize
    vorgelegt von:\\[12pt]

      \begin{minipage}[]{0.30\textwidth}
        \textbf{\autorOne}\\
        Matrikelnummer: \matrikelnrOne\\
        \emailOne\\
        \adresseOne\\
      \end{minipage}
      \hspace{0.15\textwidth}
      \begin{minipage}[]{0.3\textwidth}
        \textbf{\autorTwo}\\
        Matrikelnummer: \matrikelnrTwo\\
        \emailTwo\\
        \adresseTwo\\
      \end{minipage}

      \vfill
\todo{schön machen}
      \begin{minipage}[t]{0.45\textwidth}
        \begin{flushright}
          Datum:              \\[1.2ex]
          Ort:                \\[1.2ex]
          Betreuende Person:  \\[1.2ex]
          Institut:           \\[3ex]
        \end{flushright}
      \end{minipage}
      \hspace{0.01\textwidth}
      \begin{minipage}[t]{0.45\textwidth}
        \begin{flushleft}
          \today\\[1.2ex]
          \ort\\[1.2ex]
          \betreuung\\[1.2ex]
          \institut \\[3ex]
        \end{flushleft}
      \end{minipage}
  \end{center}
\end{titlepage}
\tableofcontents
\section{Thema der Arbeit}
\was{Kurze allgemeine Einführung in unser Thema. Was wird behandelt?}
- Problembeschreibung: Konsumkultur, nicht nachhaltiger Ressourcenverbrauch. Ein Teilproblem: Produkte werden nicht voll ausgenutzt (Lebensdauer/Nutzungsvorrat, Auslastung).
- Betrachtung von alternativen Konsumformen, welche diese Probleme durch gemeinschaftliche Nutzung von Produkten des alltäglichen Bedarfs möglicherweise verringern
- heutzutage: Nutzen statt Besitzen, häufig internetbasiert, collaborative consumption, Sharing Economy
- aber auch traditionell: second hand
- positive Umweltwirkungen: ...
- aber auch negative Umweltauswirkugen: ...
- insbesondere Nachfrageeeffekte relevant, da mit diesen Nutzungsformen häufig finanzielle Einspareffekte einhergehen (gleichzeitig Anreiz, aber eben auch Rebound)
- Vielzahl von verschiedenen Produkten und Nutzungssystemen, mit je spezifischen Charakteristika, bspw. aktive und passive Produkte, von denen die jeweiligen zu erwartenden Umweltauswirkungen abhängen

\section{Abriss des aktuellen Forschungsstandes}
\was{An welchen derzeitigen Forschungsstand knüpfen wir mit unserem Thema an? Hier führen wir relevante Forschung und aktuelle Studien an.}
- einiges an empirischer Forschung zu speziellen Produkten (Werkzeuge, textiles Waschen, ...) und Nutzungssystemen (Gebrauchtwarenhandel übers Internet, Car-Sharing)
- Umweltwirkungen stellen immer einen Teilaspekt dar, häufig noch Betrachtung von Handlungsempfehlungen, Strategien zur Förderung, Hemmnisse, Marktanalysen, Anschlussfähigkeit, Typisierung von Konsument*innen, Motivation für solche Nutzungsformen,...
- konzeptionelle Arbeiten/theoretischer Rahmen: Nutzungsdauerverlängerung, -intensivierung, Klassifizierung von Nutzungssystemen
- 


Fragen für die Literaturarbeit:
\begin{itemize}
	\item Welche Zielgrößen gibt es, z.B. MIPS, CO2-Bilanz, ...? Insbesondere auch für Nachfrageeffekte.
	\item Welche Bezugsrahmen gibt es und wie lassen sie sich miteinander vergleichen?
	\item Was sind die konzeptionellen Grundlagen des MAIA/MIPS-Ansatzes?
	
	\item Was wurde zum Bereich Ökobilanz-Modellierung der Nutzungsphase bereits publiziert?
	\item Spezieller: Was gibt es schon zum Bereich Ökobilanz-Modellierung von \enquote{Nutzen statt Besitzen}?
	
	\item Gibt es bereits Klassifizierungen für Güter/Nutzungssysteme?
	\item Welche Beispiele für Güter/Nutzungssysteme gibt es?
	
	\item Welche Prozesse (positive und negative Auswirkungen) von \enquote{Nutzen statt Besitzen} gibt es?
	\item Welche Prozesse werden als relevant erachtet?
\end{itemize}

 \section{(Identifizierung einer Forschungslücke)}
 \was{Gibt es eine Forschungslücke im derzeitigen Forschungsstand? Wie werden wir
 diese mit unserer Arbeit schließen?
 Gegebenenfalls mit vorigem Abschnitt zusammenfassen zu \enquote{Forschungsstand und Forschungslücke}.}
 - immer nur detaillierte Betrachtung einzelner Produkte
 - kein einheitliches methodisches Vorgehen: Bezugsgröße variiert
 - Einzelfallberechnungen, keine Analyse der Parameter, Potentiale
 - nicht allgemein
 - keine konsequente Betrachtung der Ressourceninputs
 
 
\section{Forschungsziel}
\was{Was wollen wir mit unserer Arbeit erreichen?}

- systematische/allgemeine Betrachtung der Umweltwirkungen bei gemeinschaftlicher Nutzung
- einheitlicher Methodenrahmen für verschiedene Produkte und Nutzungssysteme
- Entwicklung von Kriterien für die Nachhaltigkeit kollektiver Nutzungssysteme. Ableitung einer Klassifikation für Güter und Nutzungssysteme.

\section{Zentrale Fragestellungen}
\was{Was ist/sind die zentrale(n) Fragestellung(en)? Wenn wir mit Hypothesen
arbeiten, geben wir diese hier an.
Gegebenenfalls mit vorigem Abschnitt zusammenfassen zu "`Forschungsziel
und Fragestellung"'.}
- Wovon hängt es ab, wie ressourceneffizient eine bestimmte Form der gemeinsamen Nutzung eines bestimmten Gutes ist?
	- Was sind die Mechanismen, die dazu führen, dass durch gemeinsame Nutzung Ressourcen eingespart werden können?
	- Von welchen Eigenschaften (bzw. herkömmlichen Nutzungsformen) der Güter hängt das Ressourceneinsparpotential durch gemeinsame Nutzung ab?
	- Wie müssen gemeinsame Nutzungsformen beschaffen sein, um dieses Potential auszuschöpfen? (in Abhängigkeit vom betrachteten Gut)

\section{Methodische Vorgehensweise}
\was{Welche Methoden werden wir verwenden? Wie gehen wir methodisch vor?}

\was{Auswahlkriterien sowie Umfang, Erschließung und Aufbereitung unseres
Untersuchungsmaterials.}

Operationalisierung von ökologischer Nachhaltigkeit: Inputs, Emissionen. Genaue betrachtete Größen (z.B. MIPS-Konzept, CO2-Emissionen) Betrachtungsrahmen (z.B. Produktsicht, individuelle Sicht, gesellschaftliche Sicht) je betrachtetem Effekt. 

Input-Variablen: Güter, Nutzungssysteme. Genauere Spezifikationen der Charakteristischen Variablen folgen im Projektverlauf.

Liste an Prozessen, die prinzipiell berücksichtigt werden sollen

Liste an Gütern und Nutzungssystemen, die als Beispiele dienen und für die Modellentwicklung und Analyse richtungsweisend sind. Sie können später gegebenenfalls auch als Repräsentanten der identifizierten Klassen dienen.

Vorgehen:
\begin{itemize}
	\item Wirkungsgraphen für die einzelnen Prozesse
	\item mathematische Modelle für die einzelnen Prozesse
	\item Sensitivitätsanalysen der einzelnen Prozesse zur Abschätzung der Wirkungen (Größenordnungen und Heterogenität bzgl. der Input-Variablen)
	\item Auswahl der entscheidenden Prozesse anhand der vorangegangenen Analyse
	\item Modellkopplung mehrerer wechselwirkender Prozesse
	\item Ausführliche Modellanalyse zur Ableitung von Kriterien zur Nachhaltigkeit
	\item Erarbeitung einer Klassifizierung anhand der abgeleiteten Kriterien. Interpretation der Ergebnisse.
	\item Anwendung des Gesamtmodells auf Repräsentanten der identifizierten Klassen.
	\item Vergleich der Ergebnisse mit ausgewählten Suffizienz- und Effizienzstrategien bei individueller Nutzung (z.B. Angabe der Äquivalente)
\end{itemize}


\section{Zu erwartende Ergebnisse}
\was{Welchen Beitrag zum Wissensstand der Forschungsgemeinschaft leisten wir mit
unserer Arbeit?
Oder: Welche Ergebnisse sind zu erwarten?}


\section{Aufbau der Arbeit}
\was{Ein erstes Inhaltsverzeichnis}
Einleitung

Modell:
Einzelmodelle
Gekoppelte Modelle

Modellanalyse:
Einzelmodelle
Gekoppelte Modelle

Anwendungen

Ausblick
Literaturverzeichnis
Anhang

\section{Zeitplan und einzelne Arbeitsschritte}
\was{Ein erster grober Zeitplan, der die wichtigsten Arbeitsschritte enthält.}

\section{Offene Fragen}

\section{Auswahlbibliographie}
\was{Relevante Literatur (erste Auswahl/Standardwerke, aktuelle Fachartikel)}

% \printbibliography
\end{document}


