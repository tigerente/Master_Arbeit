%        File: Expose.tex
%     Created: Di Mrz 04 02:00  2014 Mitteleuropäische Z
% Last Change: Di Mrz 04 02:00  2014 Mitteleuropäische Z
%
% !TEX TS-program = pdflatex
% !TEX encoding = UTF-8 Unicode

\documentclass[11pt, titlepage=true]{scrartcl} % use larger type; default would be 10pt

\usepackage[utf8]{inputenc} % set input encoding (not needed with XeLaTeX)
\usepackage[ngerman]{babel} 
\usepackage[babel,german=quotes]{csquotes}
\usepackage[T1]{fontenc} % for enabling non-standard characters in
                         % hyphenation-list and smallcaps-boldfont

\usepackage[bibstyle=authortitle, citestyle=authoryear]{biblatex}
\bibliography{expose}

% \DeclareUnicodeCharacter{00A0}{~} % for avoiding problems with no-break spaces

%%% PAGE DIMENSIONS
\usepackage{geometry} % to change the page dimensions
\geometry{a4paper} % or letterpaper (US) or a5paper or....
% \geometry{margin=2in} % for example, change the margins to 2 inches all round
% \geometry{landscape} % set up the page for landscape
%   read geometry.pdf for detailed page layout information

\usepackage{graphicx} % support the \includegraphics command and options

\graphicspath{{./Bilder/}}

\usepackage[parfill]{parskip} % Activate to begin paragraphs with an empty line rather than an indent

%%% PACKAGES
\usepackage{booktabs} % for much better looking tables
\usepackage{array} % for better arrays (eg matrices) in maths
%\usepackage{paralist} % very flexible & customisable lists (eg. enumerate/itemize, etc.)
\usepackage{verbatim} % adds environment for commenting out blocks of text & for better verbatim
\usepackage{subfig} % make it possible to include more than one captioned figure/table in a single float
% These packages are all incorporated in the memoir class to one degree or another...

%%% HEADERS & FOOTERS
\usepackage{fancyhdr} % This should be set AFTER setting up the page geometry
\pagestyle{fancy} % options: empty , plain , fancy
\renewcommand{\headrulewidth}{0pt} % customise the layout...
\lhead{}\chead{}\rhead{}
\lfoot{}\cfoot{\thepage}\rfoot{}

%%% SECTION TITLE APPEARANCE
\usepackage{sectsty}
\allsectionsfont{\sffamily\mdseries\upshape} % (See the fntguide.pdf for font help)
% (This matches ConTeXt defaults)

%%% ToC (table of contents) APPEARANCE
\usepackage[nottoc,notlof,notlot]{tocbibind} % Put the bibliography in the ToC
\usepackage[titles,subfigure]{tocloft} % Alter the style of the Table of Contents
\renewcommand{\cftsecfont}{\rmfamily\mdseries\upshape}
\renewcommand{\cftsecpagefont}{\rmfamily\mdseries\upshape} % No bold!

\usepackage[hidelinks]{hyperref}
\usepackage{ragged2e} %für links und rechtsbündige Umgebungen in Tabellen
\usepackage{todonotes}
\usepackage[normalem]{ulem} % ermöglicht es, Text mittels \sout und \xout
                            % durchzustreichen

%%% Eigene Befehle %%%
\newcommand{\was}[1]{\small\textit{#1}}

%%% Allgemeine Daten %%%
\newcommand{\autorOne}{Alexander Müller}
\newcommand{\matrikelnrOne}{940597}
\newcommand{\adresseOne}{Gartlager Weg 37, \\49086 Osnabrück}
\newcommand{\telOne}{01578 1901578}
\newcommand{\emailOne}{alemuell@uos.de}

\newcommand{\autorTwo}{János Sebestyén}
\newcommand{\matrikelnrTwo}{939525}
\newcommand{\adresseTwo}{Gartlager Weg 37, \\49086 Osnabrück}
\newcommand{\telTwo}{}
\newcommand{\emailTwo}{jsebesty@uos.de}

%%% Titlepage Expose %%%
\newcommand{\titel}{Ökologische Nachhaltigkeit durch \enquote{Nutzen statt Besitzen}?}
\newcommand{\stitle}{}
\newcommand{\art}{Exposé}
\newcommand{\fachgebiet}{Angewandte Systemwissenschaft}
\newcommand{\betreuung}{Katrin Bienge und Prof. Dr. Claudia Pahl-Wostl}
\newcommand{\institut}{Wuppertal Institut für Klima, Umwelt, Energie und Institut für Umweltsystemforschung}
\newcommand{\ort}{Osnabrück}

%%% END Article customizations

\begin{document}
\begin{titlepage}
  \begin{center}
    \includegraphics[scale=0.5]{Logo-Uni-Osnabrueck.jpg}\\[3ex]

    \vfill
    \LARGE{\textbf{\art}}\\[1.5ex]
    \huge{\textbf{\textsc{\titel}}}\\[1.0ex]
    \LARGE{\textbf{\stitle}}\\[6ex]
    \vfill

    \normalsize
    vorgelegt von:\\[12pt]

      \begin{minipage}[]{0.30\textwidth}
        \textbf{\autorOne}\\
        Matrikelnummer: \matrikelnrOne\\
        \emailOne\\
        \adresseOne\\
      \end{minipage}
      \hspace{0.15\textwidth}
      \begin{minipage}[]{0.3\textwidth}
        \textbf{\autorTwo}\\
        Matrikelnummer: \matrikelnrTwo\\
        \emailTwo\\
        \adresseTwo\\
      \end{minipage}

      \vfill
\todo{schön machen}
      \begin{minipage}[t]{0.45\textwidth}
        \begin{flushright}
          Datum:              \\[1.2ex]
          Ort:                \\[1.2ex]
          Betreuende Person:  \\[1.2ex]
          Institut:           \\[3ex]
        \end{flushright}
      \end{minipage}
      \hspace{0.01\textwidth}
      \begin{minipage}[t]{0.45\textwidth}
        \begin{flushleft}
          \today\\[1.2ex]
          \ort\\[1.2ex]
          \betreuung\\[1.2ex]
          \institut \\[3ex]
        \end{flushleft}
      \end{minipage}
  \end{center}
\end{titlepage}
\tableofcontents
\section{Thema der Arbeit}
\was{Kurze allgemeine Einführung in unser Thema. Was wird behandelt?}

\begin{itemize}
  \item \sout{Problembeschreibung: Konsumkultur, nicht nachhaltiger
        Ressourcenverbrauch. }
  \item verschiedene Lösungsstrategien: weniger verbrauchen (Suffizienz),
    Effizienz, Konsistenz 
  \item Richtungen der effizienz: Produktion, Nutzung
  \item \sout{Ein Teilproblem: Produkte werden nicht voll ausgenutzt
    (Lebensdauer/Nutzungsvorrat, Auslastung).  }
  \item \sout{Betrachtung von alternativen Konsumformen, welche diese
      Probleme durch
      gemeinschaftliche Nutzung von Produkten des alltäglichen Bedarfs
    möglicherweise verringern}
  \item \sout{heutzutage: Nutzen statt Besitzen, häufig internetbasiert,
      collaborative
      consumption, Sharing Economy, Eco Services (Behrendt 2000, S. 7) ,
    product service system }
  \item \sout{aber auch traditionell: second hand}
  \item \sout{positive Umweltwirkungen: ...  }
  \item \sout{aber auch negative Umweltauswirkugen: ...  }
  \item \sout{insbesondere Nachfrageeffekte relevant, da mit
      diesen Nutzungsformen häufig finanzielle Einspareffekte einhergehen
    (gleichzeitig Anreiz, aber eben auch Rebound)}
  \item \sout{Vielzahl von verschiedenen Produkten und Nutzungssystemen,
      mit je spezifischen Charakteristika, bspw. aktive und passive
      Produkte, von denen die jeweiligen zu erwartenden Umweltauswirkungen
    abhängen}
      \end{itemize}
\hrule

Die heutigen industrialisierten Kulturen sind durch einen umfrangreichen
individuellen Bedarf an Produkten und Dienstleistungen gekennzeichnet. Jeder
Haushalt, jede Person verfügt über zahlreiche, in der Summe höchst
ressourcenintensive Güter, was gesamtgesellschaftlich zu einer hohen nicht mehr
tragfähigen Umweltbelastung führt.

Ein Problem heutiger Konsumkulturen, an dem Nachhaltigkeitsstrategien ansetzen
können, ist die ineffiziente Nutzung der produzierten Gütermenge: so wurde eine
Bohrmaschine bei ihrer Entsorgung lediglich 50 Stunden genutzt, obwohl sie 250
Stunden hätte genutzt werden können \todo{Quelle und genauen Zahlen}. Und ein
Auto welches von Osnabrück nach Hamburg fährt, transportiert im Schnitt 1,1
Personen, während es eigentlich Platz für vier Menschen hätte \todo{Quelle und
genauen Zahlen}. Formen gemeinschaftlicher Nutzung \todo{explizit auf Thema der Arbeit hinweisen} können diese für die einzelne
Person nicht ausschöpfbaren Effizienzpotentiale heben und dadurch zu einer
Reduktion der Umweltbelastung führen.

Zahlreiche dieser Nutzungsformen, die sich über alle Produktgruppen erstrecken,
sind bereits seit geraumer Zeit etabliert. Zum
einen finden sich hier alle Formen des Gebrauchtwarenmarkts: von Second Hand
Läden für Kleidung über Geschäfte des An- und Verkaufs bis hin zu Flohmärkten.
Zum anderen existiert eine breite Palette an Miet- und Verleihsystemen, wie zum
Beispiel Bibliotheken, Videotheken oder Autovermietungen. Ergänzend lassen sich
vielfältige direkte Formen der gemeinschaftlichen Nutzung entdecken, wie
beispielsweise die gemeinsame Waschmaschine, Rasenmäher oder Kühlschrank in
Wohnanlagen und Wohngemeinschaften, oder die Bildung von Fahrgemeinschaften.

Mit dem Aufkommen der heutigen Informations- und Kommunikationstechnologien
erhalten viele dieser Formen eine neue, ungeahnte Dimension: das Internet
vergrößert die räumlichen Entfernungen und die Anzahl beteiligter
Teilnehmer*innen der herkömmlichen Gebraucht-, Miet- und Verleihmärkte um ein
Vielfaches. Hinzu kommen durch den einfachen, ortsungebundenen Zugang zum
Internet und die Vernetzung von Menschen, mit ähnlichen Bedürfnissen, ganz neue
Formen der gemeinschaflichen Nutzung: so kann ich mit Hilfe entsprechender
Internetplattformen anstatt in einem Hotel zu übernachten,
auf meiner Reise ganz bequem bei mir fremden Menschen unterkommen oder
ich kann mit Hilfe eines Smartphones ganz einfach das nächste Auto eine Floating
Car Sharing Systems \todo{richtigen Begriff rausfinden} finden.

Hinter diesen neuen Formen werden unter Verwendung der Begriffe
\enquote{collaborative consumption} oder \enquote{Kokonsum} soziale Innovationen
gesehen, die zu mehr Nachhaltigkeit bei gleichbleidendem oder sogar gesteigerten
Wohlstand führen. Und aus Unternehmensperspektive stellen sich diese neuen
Nutzungsformen als wichtiger neuer
ökonomischer Wachstumsmarkt -- die \enquote{Sharing Economy} -- dar.

In der Forschung wurde und wird dieses Thema unter den Begriffen \enquote{Nutzen
statt Besitzen}, \enquote{eco services} und \enquote{product-service systems}
diskutiert. \todo{Begriffe definieren}

Insbesondere im Vergleich zur Euphorie der Sharing Economy zeigt sich hier
hinsichtlich der Umweltwirkungen gemeinschaftlicher Nutzungssysteme ein
differenzierteres Bild: zum einen werden die positiven Primäreffekte der
Nutzungsdauerverlängerung und -intensivierung um weitere sekundäre
Ressourceneinsparmöglichkeiten ergänzt. So kann es beispielsweise bei der
gemeinschaftlichen Nutzung einer Waschmaschine sinnvoll sein, eine
halbgewerbliche Maschine mit entsprechend längerer Lebensdauer und höherer
Effizienz zu verwenden. Zum anderen werden die durch gemeinschaftliche Nutzung
ermöglichten Ressourceneinsparungen jedoch durch negative Umweltwirkungen
begleitet: so werden beispielsweise für die Koordination der gemeinschaftlichen
Nutzung und den Transport der genutzten Produkte zwischen den Nutzer*innen
Ressourcen aufgewendet. Und auch sekundär positive Effekte, wie die Anschaffung
einer halbgewerblichen Waschmaschine, geht mit einem gegenüber der Nutzung einer
normalen Waschmaschine höheren Ressourcenverbrauch während der Produktion der
genutzen Produkte einher. 

Nicht zuletzt wirken sich die aus der Perspektive der Sharing Economy positiven
finaziellen Einsparungen beim kollaborativen Konsum auf die Nachfrageseite der
Konsumenten aus: das eingesparte Geld kann in weiteren Konsum und damit
zusätzlichen Ressourcenverbrauch investiert werden (Rebound-Effekt). Außerdem
bleiben in einer solchermaßen veränderten Ökonomie, welche den Zugang und nicht
den Besitz in den Vordergrund stellt, die Bedürfnisstrukturen nicht unberührt:
die Möglichkeit, auf viele (Luxus-)Güter zuzugreifen, könnte mit einem
verstärkten Wunsch einhergehen, dies auch tatsächlich zu tun.

Welche Netto-Umweltwirkungen schließlich durch die gemeinschaftliche Nutzung zu
erwarten sind hängt in hohem Maße von dem betrachteten Produkt und dem konkreten
Nutzungssystem ab. \todo{Abschnitt etwas ausbauen, evtl. noch Beispiele (Bücher und Waschmaschinen)} Diese Zusammenhänge zu
analysieren ist Gegenstand unserer Arbeit.

\section{Abriss des aktuellen Forschungsstandes}
\was{An welchen derzeitigen Forschungsstand knüpfen wir mit unserem Thema an? Hier führen wir relevante Forschung und aktuelle Studien an.}
- einiges an empirischer Forschung zu speziellen Produkten (Werkzeuge, textiles Waschen, ...) und Nutzungssystemen (Gebrauchtwarenhandel übers Internet, Car-Sharing)
- Umweltwirkungen stellen immer einen Teilaspekt dar, häufig noch Betrachtung von Handlungsempfehlungen, Strategien zur Förderung, Hemmnisse, Marktanalysen, Anschlussfähigkeit, Typisierung von Konsument*innen, Motivation für solche Nutzungsformen,...
- konzeptionelle Arbeiten/theoretischer Rahmen: Nutzungsdauerverlängerung, -intensivierung, Klassifizierung von Nutzungssystemen
- 


Fragen für die Literaturarbeit:
\begin{itemize}
	\item Welche Zielgrößen gibt es, z.B. MIPS, CO2-Bilanz, ...? Insbesondere auch für Nachfrageeffekte.
	\item Welche Bezugsrahmen gibt es und wie lassen sie sich miteinander vergleichen?
	\item Was sind die konzeptionellen Grundlagen des MAIA/MIPS-Ansatzes?
	
	\item Was wurde zum Bereich Ökobilanz-Modellierung der Nutzungsphase bereits publiziert?
	\item Spezieller: Was gibt es schon zum Bereich Ökobilanz-Modellierung von \enquote{Nutzen statt Besitzen}?
	
	\item Gibt es bereits Klassifizierungen für Güter/Nutzungssysteme?
	\item Welche Beispiele für Güter/Nutzungssysteme gibt es?
	
	\item Welche Prozesse (positive und negative Auswirkungen) von \enquote{Nutzen statt Besitzen} gibt es?
	\item Welche Prozesse werden als relevant erachtet?
\end{itemize}

\section{(Identifizierung einer Forschungslücke)}
 \was{Gibt es eine Forschungslücke im derzeitigen Forschungsstand? Wie werden wir
 diese mit unserer Arbeit schließen?
 Gegebenenfalls mit vorigem Abschnitt zusammenfassen zu \enquote{Forschungsstand und Forschungslücke}.}
 - immer nur detaillierte Betrachtung einzelner Produkte
 - kein einheitliches methodisches Vorgehen: Bezugsgröße variiert
 - Einzelfallberechnungen, keine Analyse der Parameter, Potentiale
 - nicht allgemein
 - keine konsequente Betrachtung der Ressourceninputs
 
 
\section{Forschungsziel}
\was{Was wollen wir mit unserer Arbeit erreichen?}

- systematische/allgemeine Betrachtung der Umweltwirkungen bei gemeinschaftlicher Nutzung
- einheitlicher Methodenrahmen für verschiedene Produkte und Nutzungssysteme
- Entwicklung von Kriterien für die Nachhaltigkeit kollektiver Nutzungssysteme. Ableitung einer Klassifikation für Güter und Nutzungssysteme.

\section{Zentrale Fragestellungen}
\was{Was ist/sind die zentrale(n) Fragestellung(en)? Wenn wir mit Hypothesen
arbeiten, geben wir diese hier an.
Gegebenenfalls mit vorigem Abschnitt zusammenfassen zu "`Forschungsziel
und Fragestellung"'.}
- Wovon hängt es ab, wie ressourceneffizient eine bestimmte Form der gemeinsamen Nutzung eines bestimmten Gutes ist?
	- Was sind die Mechanismen, die dazu führen, dass durch gemeinsame Nutzung Ressourcen eingespart werden können?
	- Von welchen Eigenschaften (bzw. herkömmlichen Nutzungsformen) der Güter hängt das Ressourceneinsparpotential durch gemeinsame Nutzung ab?
	- Wie müssen gemeinsame Nutzungsformen beschaffen sein, um dieses Potential auszuschöpfen? (in Abhängigkeit vom betrachteten Gut)

\section{Methodische Vorgehensweise}
\was{Welche Methoden werden wir verwenden? Wie gehen wir methodisch vor?}

Annahme: Nachhaltigkeitseffekt hängt ab von betrachtetem Produkt (einschließlich des Produktionssystems) und Nutzungssystem.

Ziel dieser Untersuchung: Herausfinden, von welchen Eigenschaften der Produkte und Nutzungssysteme der Nachhaltigkeitseffekt abhängt.

Operationalisierung von ökologischer Nachhaltigkeit: Inputs, Emissionen. Genaue betrachtete Größen (z.B. MIPS-Konzept, CO2-Emissionen) Betrachtungsrahmen (z.B. Produktsicht, individuelle Sicht, gesellschaftliche Sicht) je betrachtetem Effekt.

Liste an Prozessen, die prinzipiell berücksichtigt werden sollen

Modell entwickeln: Prozesse quantifizierbar und vergleichbar machen.  Modellzweck: Abhängigkeit der Nachhaltigkeit von Eigenschaften der Produkte und Nutzungssysteme abbilden.

Fallstudien: Leiten Modellentwicklung/gewährleisten Anwendbarkeit; Illustration, Gesamtmodell-Auswertung.
Auswahlkriterien für die Fallstudien. Entscheidung für Waschmaschinen und Bücher.

Vorgehen:
\begin{itemize}
	\item Modellentwicklung für Einzelprozesse
	\item Analyse der Einzelmodelle
	\item Modellkopplung mehrerer wechselwirkender Prozesse
	\item Analyse der gekoppelten Modelle
	\item Anwendung der Einzelmodelle und gekoppelten Modelle auf Fallstudien
\end{itemize}


\section{Zu erwartende Ergebnisse}
\was{Welchen Beitrag zum Wissensstand der Forschungsgemeinschaft leisten wir mit
unserer Arbeit?
Oder: Welche Ergebnisse sind zu erwarten?}

\newpage
\section{Aufbau der Arbeit}
grobes Inhaltsverzeichnis:

\begin{enumerate}

\item Einleitung

\item abstraktes Modell
	\begin{enumerate}
		\item Einzeleffekte
		\item Effekt-Kopplungen
		\item Gesamtmodell
	\end{enumerate}

\item Anwendungen
	\begin{enumerate}
		\item Beispiel Waschmaschine
		\begin{enumerate}
			\item Vorstellen der Fallstudie (Literatur)
			\item Anwendung des Modells
			\begin{enumerate}
				\item Bestimmung der Modellparameter
				\item Einzeleffekte (je Nutzungssystem)
				\item Ausgewählte Modellkopplungen (je Nutzungssystem)
				\item Gesamtmodell (je Nutzungssystem)
			\end{enumerate}
		\end{enumerate}
		\item Beispiel Buch
		\begin{enumerate}
		\item \dots
		\end{enumerate}
	\end{enumerate}

\item Ausblick
\item Literaturverzeichnis
\item Anhang
\end{enumerate}

\section{Zeitplan und einzelne Arbeitsschritte}
\was{Ein erster grober Zeitplan, der die wichtigsten Arbeitsschritte enthält.}

\section{Offene Fragen}

\section{Auswahlbibliographie}
\was{Relevante Literatur (erste Auswahl/Standardwerke, aktuelle Fachartikel)}

% \printbibliography
\end{document}


